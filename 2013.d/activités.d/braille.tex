%%%%%%%%%%%%%%%%%%%%%%%%%%%%%%%%%%%%%%%%%
% LaTeX Template
% http://www.LaTeXTemplates.com
%
% Original author:
% Linux and Unix Users Group at Virginia Tech Wiki 
% (https://vtluug.org/wiki/Example_LaTeX_chem_lab_report)
%
% License:
% CC BY-NC-SA 3.0 (http://creativecommons.org/licenses/by-nc-sa/3.0/)
%
%%%%%%%%%%%%%%%%%%%%%%%%%%%%%%%%%%%%%%%%%

%----------------------------------------------------------------------------------------
%	PACKAGES AND DOCUMENT CONFIGURATIONS
%----------------------------------------------------------------------------------------

\documentclass[10pt]{article}
\usepackage{geometry} % Pour passer au format A4
%\geometry{a4paper} % 
\geometry{hmargin=1cm, vmargin=1cm} % 
\usepackage{graphicx} % Required for including pictures
\usepackage{float} % 

%Français
\usepackage[T1]{fontenc} 
\usepackage[english,francais]{babel}
\usepackage[utf8]{inputenc}
\usepackage{lmodern}
\usepackage{url}
\usepackage{multicol}
\usepackage{cancel}

%Maths
\usepackage{amsmath,amsfonts,amssymb,amsthm}
%\usepackage[linesnumbered, ruled, vlined]{algorithm2e}
%\SetAlFnt{\small\sffamily}

%Autres
\linespread{1} % Line spacing
\setlength\parindent{0pt} % Removes all indentation from paragraphs

\renewcommand{\labelenumi}{\alph{enumi}.} % 
\pagestyle{empty}
%----------------------------------------------------------------------------------------
%	DOCUMENT INFORMATION
%----------------------------------------------------------------------------------------

\begin{document}

%----------------------------------------------------------------------------------------
%	SECTION 1
%----------------------------------------------------------------------------------------

\section{Histoire}
\subsection{Inventeur}
\begin{enumerate}
\item Qui est l'inventeur du braille ?
\item Quand et où est-il né ?
\item Est-il aveugle de naissance ?
\end{enumerate}
\subsection{Cases}
\begin{enumerate}
\item Combien y-a-t-il de points par case ?
\item Au maximum combien de points sont noircis dans une seule case ?
\item Au minimum combien de points sont noircis dans une seule case ?
\item À quelle lettre correspond cette case ?
\end{enumerate}
\subsection{Calcul en braille}
Écrire en braille les opérations suivantes ainsi que leurs résultats:\\
\textit{Bien penser au signe égale en braille.}
\begin{enumerate}
\item $10 + 2 = $
\item $5 * 4 = $
\item $14 - 7 = $
\item $12 / 3= $
\end{enumerate}
\subsection{Écrire en braille}
Écrire en braille les phrases suivantes:\\
\textit{Mettre des points sur à l'intersection de carreaux consécutifs. Sauter une intersection de carreaux entre les lettres d'un même mot et deux entre les mots.}
\begin{enumerate}
\item finalement, j'ai de la chance.
\item Les mathématiques c'est fantastique.
\end{enumerate}
\subsection{Court instant de réflexion}
\textit{Suite à la lecture en français et non en braille de ce court texte, réfléchir aux difficultés que va rencontrer un aveugle lors de ses études, son apprentissage et de la vie de tous les jours.}
\begin{enumerate}
\item Proposer quelque chose / une idée / un projet susceptible d'améliorer le quotidien d'un élève aveugle dans l'enceinte de l'établissement.
\end{enumerate}
\end{document}








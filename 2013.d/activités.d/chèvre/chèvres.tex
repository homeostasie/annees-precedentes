%%%%%%%%%%%%%%%%%%%%%%%%%%%%%%%%%%%%%%%%%
% Short Sectioned Assignment
% LaTeX Template
% Version 1.0 (5/5/12)
%
% This template has been downloaded from:
% http://www.LaTeXTemplates.com
%
% Original author:
% Frits Wenneker (http://www.howtotex.com)
%
% License:
% CC BY-NC-SA 3.0 (http://creativecommons.org/licenses/by-nc-sa/3.0/)
%
%%%%%%%%%%%%%%%%%%%%%%%%%%%%%%%%%%%%%%%%%

%----------------------------------------------------------------------------------------
%	PACKAGES AND OTHER DOCUMENT CONFIGURATIONS
%----------------------------------------------------------------------------------------
\documentclass[11pt]{article}
\usepackage{geometry} % Pour passer au format A4
\geometry{hmargin=1cm, vmargin=1cm} % 


\usepackage[T1]{fontenc} % Use 8-bit encoding that has 256 glyphs
\usepackage[english,francais]{babel} % Français et anglais
\usepackage[utf8]{inputenc} 

\usepackage{amsmath,amsfonts,amsthm} % Math packages

\usepackage{lmodern}
\usepackage{url}
\usepackage{eurosym} % signe Euros
\usepackage{geometry} % Pour passer au format A4
\geometry{a4paper} % 
\usepackage{graphicx} % Required for including pictures
\usepackage{float} % Allows putting an [H] in \begin{figure} to specify the exact location of the figure

\usepackage{multicol}
\usepackage{sectsty} % Allows customizing section commands
\allsectionsfont{\centering \normalfont\scshape} % Make all sections centered, the default font and small caps

%----------------------------------------------------------------------------------------
%	Pied de Page
%----------------------------------------------------------------------------------------

\setlength\parindent{0pt} % Removes all indentation from paragraphs - comment this line for an assignment with lots of text


%----------------------------------------------------------------------------------------
%	Titre
%----------------------------------------------------------------------------------------

\newcommand{\horrule}[1]{\rule{\linewidth}{#1}} % Create horizontal rule command with 1 argument of height


%----------------------------------------------------------------------------------------
%	Début du document
%----------------------------------------------------------------------------------------



\begin{document}


%----------------------------------------------------------------------------------------
%	SECTION 1
%----------------------------------------------------------------------------------------
\section*{$Bêh^2$ - Ex 104 p167} % Title
\vspace{1cm}

\subsection*{Données de l'énoncés}

\begin{itemize}
\item Les poteaux $P_1$ et $P_2$ sont distants de 8 mètres.
\item La corde attachant la première chèvre au poteau $P_1$ mesure 5 mètres.
\end{itemize}

\setlength{\columnseprule}{1pt}
\begin{multicols}{2}


\begin{enumerate}
\item La corde attachant la deuxième chèvre au poteau $P_1$ mesure 2 mètres.
\begin{figure}[H]
  \centering
  \includegraphics[width=\linewidth]{chevres-1.pdf}
\end{figure}

\item La corde attachant la deuxième chèvre au poteau $P_1$ mesure 13 mètres.
\begin{figure}[H]
  \centering
  \includegraphics[width=\linewidth]{chevres-2.pdf}
\end{figure}

\item À partir des deux questions précédentes, nous pouvons faire deux observations.

\begin{itemize}
\item Si la corde mesure seulement 2m, les deux chèvres ne peuvent se rencontrer.
\item Si la corde mesure au moins 13m, la première chèvre ne peut s'isoler.
\end{itemize}
On en déduit que la longueur de corde attachant la deuxième chèvre doit être plus petite que 13m tout en étant suffisament grande pour que les chèvres puissent se rencontrer.\\
Cette longueur est : $8m - 5m = 3m$\\
\textbf{La longueur de corde doit être comprise entre 3m et 13m.}
\begin{figure}[H]
  \centering
  \includegraphics[width=\linewidth]{chevres-3.pdf}
\end{figure}
\end{enumerate}
\end{multicols}

\end{document}






%%%%%%%%%%%%%%%%%%%%%%%%%%%%%%%%%%%%%%%%%
% Short Sectioned Assignment
% LaTeX Template
% Version 1.0 (5/5/12)
%
% This template has been downloaded from:
% http://www.LaTeXTemplates.com
%
% Original author:
% Frits Wenneker (http://www.howtotex.com)
%
% License:
% CC BY-NC-SA 3.0 (http://creativecommons.org/licenses/by-nc-sa/3.0/)
%
%%%%%%%%%%%%%%%%%%%%%%%%%%%%%%%%%%%%%%%%%

%----------------------------------------------------------------------------------------
%	PACKAGES AND OTHER DOCUMENT CONFIGURATIONS
%----------------------------------------------------------------------------------------
\documentclass[11pt]{article}
\usepackage{geometry} % Pour passer au format A4
\geometry{hmargin=1cm, vmargin=1cm} % 


\usepackage[T1]{fontenc} % Use 8-bit encoding that has 256 glyphs
\usepackage[english,francais]{babel} % Français et anglais
\usepackage[utf8]{inputenc} 

\usepackage{amsmath,amsfonts,amsthm} % Math packages

\usepackage{lmodern}
\usepackage{url}
\usepackage{eurosym} % signe Euros
\usepackage{geometry} % Pour passer au format A4
\geometry{a4paper} % 
\usepackage{graphicx} % Required for including pictures
\usepackage{float} % Allows putting an [H] in \begin{figure} to specify the exact location of the figure

\usepackage{multicol}
\usepackage{sectsty} % Allows customizing section commands
\allsectionsfont{\centering \normalfont\scshape} % Make all sections centered, the default font and small caps

%----------------------------------------------------------------------------------------
%	Pied de Page
%----------------------------------------------------------------------------------------

\setlength\parindent{0pt} % Removes all indentation from paragraphs - comment this line for an assignment with lots of text


%----------------------------------------------------------------------------------------
%	Titre
%----------------------------------------------------------------------------------------

\newcommand{\horrule}[1]{\rule{\linewidth}{#1}} % Create horizontal rule command with 1 argument of height


%----------------------------------------------------------------------------------------
%	Début du document
%----------------------------------------------------------------------------------------

\title{Problème Déco  \copyright Valérie Damidot} % Title
\author{$5^e 1$}
%\date{3 Avril 2014} % Date for the report

\begin{document}

%----------------------------------------------------------------------------------------
%	Titre
%----------------------------------------------------------------------------------------

\maketitle % Print the title

%-----------------------------------------------------------------
%-------- 1. Rappel sur les aires
\section{L'aire d'un rectangle}

La surface au sol, la surface recouvrante d'un mur représente un calcul d'aire, de superficie. Son unité est le $m^2$.\\
Pour un rectangle de longueur l et de largeur L, la formule d'aire est le produit de la longueur et de la largeur :
$$\mathcal{A}_{rec} = l \times L$$.
%-- fin partie 1

\begin{figure}[H]
  \centering
  \includegraphics[width=.6\linewidth]{sources/ie/valdami-sol.pdf}
  \caption{Plan du sol}
  \label{fig:ch1-plan}
\end{figure}

%-------- 2. Superficie au sol et béton
\section{Refaire le sol en béton}

%-------- 2.1 Superficie au sol
\subsection{Superficie au sol}

La superficie au sol est un rectangle. Pour connaître sa valeur, on calcul le produit de la longueur et de la largeur de la pièce.
$$\mathcal{A}_{A} = l_A \times L_A$$
\begin{eqnarray*}
  \mathcal{A}_{A} & = & 10 \times 3)  \text{// On termine le calcul avec la multiplication.}\\
  & = & 30 m^2\\
\end{eqnarray*}
$$\mathcal{A}_{B} = l_B \times L_B$$
\begin{eqnarray*}
  \mathcal{A}_{B} & = & 10 \times (14 - 3) \text{//  Le calcul dans la parenthèse est prioritaire.}\\
  & = & 10 \times 11 \text {// On termine le calcul avec la multiplication.}\\
  & = & 110 m^2\\
\end{eqnarray*}
$$\mathcal{A}_{C} = l_C \times L_C$$
\begin{eqnarray*}
  \mathcal{A}_{C} & = & 3 \times 5 \text{// On termine le calcul avec la multiplication.}\\
  & = & 15 m^2
\end{eqnarray*}

On a trouvé l'aire de chacune des pièces : A, B et C. La superficie totale de la maison est la somme de la superficie de chacune des pièces.\\
$$ \mathcal{A}_{M} = \mathcal{A}_{A} + \mathcal{A}_{B} \mathcal{A}_{C} = 30 + 110 + 15$$\\
$$ \mathcal{A}_{M} = 155 m^2$$\\

La superficie totale de la maison est $155m^2$.

%-- fin partie 2.1 super sol
%-------- 2.2 Béton
\subsection{Coulage de la dalle de béton}


\begin{enumerate}
  %-------- 2.2.1 nombre de bétonnières
\item On sait qu'une bétonière recouvre $\mathcal{A}_{béton} = 31m^2$ de superficie.\\
  La superficie totale de la maison est $\mathcal{A}_{M} = 155 m^2$.\\
  On note $n_b$ le nombre de bétonnière nécessaire.\\
  Pour trouver le nombre nécessaire de bétonnière ($n_b$) pour recouvrir le sol, on effectue un calcul de proportionalité :\\
  
  $$n_b = \frac{\mathcal{A}_M}{\mathcal{A}_{béton}} = \frac{155}{31} = 5$$\\
  
  Pour recouvrir l'intégralité du sol de la maison, il est nécessaire d'utiliser \textbf{5 bétonnières}.

  %-------- 2.2.2 quantité d'ingrédient
\item On sait d'après la question précédente qu'on a besoin de 5 bétonnières. Pour réaliser 5 bétonnières, on multiplie par 5 les quantités nécessaires à la réalisation d'une seule bétonnière :
  \begin{itemize}
  \item $2 \times 5 = 10$ sacs de sable.
  \item $1 \times 5 = 5$ sacs de graviers.
  \item $15 \times 5 = 75$L d'eau.
  \end{itemize}
  
  Pour recouvrir l'intégralité du sol de la maison, il est nécéssaire d'avoir \textbf{10 sacs de sable, 5 sacs de graviers et 75L d'eau}.

  %-------- 2.2.3 prix du sol
\item Possédant la quantité d'ingrédient et leur prix, on cherche la somme de leurs coûts. \\
  $$P_{sol} = 10 \times P_s + 5 \times P_g + 75 \times P_e$$
  \begin{eqnarray*}
    P_{sol} & = & 10 \times 2 + 5 \times 4 + 75 \times 0\\
    & = & 20 + 20 + 0\\
    & = & 40
  \end{eqnarray*}
  Pour recouvrir l'intégralité du sol de la maison en béton, il est nécessaire de dépenser \textbf{40 \euro}.
\end{enumerate} %-- fin 2.2          
%-- fin partie 2 béton au sol

%-------- 3 repeindre les murs
\section{Repeindre les murs}

\begin{figure}[H]
  \centering
  \includegraphics[width=.4\linewidth]{sources/ie/valdami-sol-col.pdf}
  \caption{Plan du sol}
  \label{fig:ch1-plan}
\end{figure}


%-------- 3.1 nombre de pots
\subsection{Pièce en rose}
\subsubsection{Calcul de la surface totale à repeindre pour la pièce A}
Le sol de la pièce A est un rectangle de périmètre $ \mathcal{P}_{A} = 10 \times 2 + 3 \times 2 = 20 + 6 = 26m$.\\
La surface a repeindre est $\mathcal{P}_{A} \times h = 26 \times 3 = 78m^2$.\\

\subsubsection{Calcul du nombre de pots de peinture nécéssaire}
Un pot de peinture recouvre $10m^2$. Il nous faut recouvrir $78m^2$.\\
7 pots ne recouvrent que $7 \times 10 = 70m^2$ alors que 8 pots recouvrent $8 \times 10 = 80m^2$.\\
Il faut \textbf{8 pots} de peinture rose.  

\subsection{Pièce en bleue}
\subsubsection{Calcul de la surface totale à repeindre pour les pièces B et C}
Le sol de la pièce B est un rectangle de périmètre $ \mathcal{P}_{B} = 10 \times 2 + 11 \times 2 = 20 + 22 = 42m$.\\
Le sol de la pièce C est un rectangle de périmètre $ \mathcal{P}_{C} = 5 \times 2 + 3 \times 2 = 10 + 6 = 16m$.\\  

La surface totale est $(\mathcal{P}_{B} + \mathcal{P}_{C}) \times h = (42 + 16) \times 3 = 58 \times 3 = 174m^2$.\\

\subsubsection{Calcul du nombre de pots de peinture nécéssaire}
Un pot de peinture recouvre $10m^2$. Il nous faut recouvrir $174m^2$.\\
17 pots ne recouvrent que $17 \times 10 = 170m^2$ alors que 18 pots recouvrent $18 \times 10 = 180m^2$.\\
Il faut \textbf{18 pots} de peinture bleue.

%-- fin exo

\end{document}

%%%%%%%%%%%%%%%%%%%%%%%%%%%%%%%%%%%%%%%%%
% Short Sectioned Assignment
% LaTeX Template
% Version 1.0 (5/5/12)
%
% This template has been downloaded from:
% http://www.LaTeXTemplates.com
%
% Original author:
% Frits Wenneker (http://www.howtotex.com)
%
% License:
% CC BY-NC-SA 3.0 (http://creativecommons.org/licenses/by-nc-sa/3.0/)
%
%%%%%%%%%%%%%%%%%%%%%%%%%%%%%%%%%%%%%%%%%

%----------------------------------------------------------------------------------------
%	PACKAGES AND OTHER DOCUMENT CONFIGURATIONS
%----------------------------------------------------------------------------------------
\documentclass[11pt]{article}
\usepackage{geometry} % Pour passer au format A4
\geometry{hmargin=1cm, vmargin=1cm} % 


\usepackage[T1]{fontenc} % Use 8-bit encoding that has 256 glyphs
\usepackage[english,francais]{babel} % Français et anglais
\usepackage[utf8]{inputenc} 

\usepackage{amsmath,amsfonts,amsthm} % Math packages

\usepackage{lmodern}
\usepackage{url}
\usepackage{eurosym} % signe Euros
\usepackage{geometry} % Pour passer au format A4
\geometry{a4paper} % 
\usepackage{graphicx} % Required for including pictures
\usepackage{float} % Allows putting an [H] in \begin{figure} to specify the exact location of the figure

\usepackage{multicol}
\usepackage{sectsty} % Allows customizing section commands
\allsectionsfont{\centering \normalfont\scshape} % Make all sections centered, the default font and small caps

%----------------------------------------------------------------------------------------
%	Pied de Page
%----------------------------------------------------------------------------------------

\setlength\parindent{0pt} % Removes all indentation from paragraphs - comment this line for an assignment with lots of text


%----------------------------------------------------------------------------------------
%	Titre
%----------------------------------------------------------------------------------------

\newcommand{\horrule}[1]{\rule{\linewidth}{#1}} % Create horizontal rule command with 1 argument of height


%----------------------------------------------------------------------------------------
%	Début du document
%----------------------------------------------------------------------------------------

\title{Problème Déco  \copyright Valérie Damidot} % Title
\author{$5^e 1$}
%\date{3 Avril 2014} % Date for the report

\begin{document}

%----------------------------------------------------------------------------------------
%	Titre
%----------------------------------------------------------------------------------------

\maketitle % Print the title

%-----------------------------------------------------------------
%-------- 1. Rappel sur les aires


\section{L'aire d'un rectangle}
Rappeller la formule de l'aire d'un rectangle en fonction de la longueur l et de largeur L. \\
Préciser l'unité de l'aire.

\begin{figure}[H]
  \centering
  \includegraphics[width=.6\linewidth]{sources/ie/valdami-sol.pdf}
  \caption{Plan du sol}
  \label{fig:ch1-plan}
\end{figure}

%-- fin partie 1

%-------- 2. Superficie au sol et béton
\section{Refaire le sol en béton}

%-------- 2.1 Superficie au sol
\subsection{Superficie au sol}
Calculer la superficie des trois pièces A, B et C et donner la superficie totale de la maison.\\

%-------- 2.2 Béton
\subsection{Coulage de la dalle de béton}
Une bétonnière recouvre $31m^2$ de superficie. Pour en remplir une il faut réunir et mélanger les ingrédients suivants :

\begin{itemize}
\item 2 sacs de sable.
\item 1 sac de graviers.
\item 15 L d'eau.
\end{itemize}

1 sac de sable coûte : $P_s = 2$\euro, 1 sac de gravier coûte : $P_g = 4$  \euro, on suppose que l'eau est gratuite : $P_e = 0$ \euro(évidemment faux pour l'eau de ville, pensez à fermer les robinets !!!.

\begin{enumerate}
  %-------- 2.2.1 nombre de bétonnières
\item Calculer le nombre nécessaire de bétonnière utilisé pour recouvrir l'ensemble de la maison.\\
  
  %-------- 2.2.2 quantité d'ingrédient
\item Calculer la quantité nécessaire de sacs de sable et de sac de graviers et d'eau en litre néssaire pour la réalisation du sol0.\\

  %-------- 2.2.3 prix du sol
\item Estimer le prix de cette opération.\\

\end{enumerate} %-- fin 2.2          


%-------- 3 repeindre les murs
\section{Repeindre les murs}
Maintenant que le sol est bien bétonné, on s'intéresse à refaire les murs. On souhaite repeindre la pièce A en rose et les pièces B et C en bleue. On ne repeint ni le plafond, ni le sol, on suppose que les pièces n'ont pas de porte. Tous les murs de la maison ont une hauteur de $h=3m$.\\
Chaque pot de peinture recouvre une surface maximale de de $10m^2$. 

\begin{figure}[H]
  \centering
  \includegraphics[width=.4\linewidth]{sources/ie/valdami-sol-col.pdf}
  \caption{Plan du sol}
  \label{fig:ch1-plan}
\end{figure}

\begin{enumerate}
  %-------- 3.1 nombre de pots
\item Calculer la surface totale à repeindre pour la pièce A.\\
  En déduire le nombre de pots de peinture rose nécessaire.\\
\item Calculer la surface totale à repeindre pour la pièce B et pour la pièce C\\
  En déduire le nombre de pots de peinture rose nécessaire.\\
\end{enumerate}


\end{document}

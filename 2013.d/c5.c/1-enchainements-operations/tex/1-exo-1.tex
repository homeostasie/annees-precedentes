%%%%%%%%%%%%%%%%%%%%%%%%%%%%%%%%%%%%%%%%%
% Short Sectioned Assignment
% LaTeX Template
% Version 1.0 (5/5/12)
%
% This template has been downloaded from:
% http://www.LaTeXTemplates.com
%
% Original author:
% Frits Wenneker (http://www.howtotex.com)
%
% License:
% CC BY-NC-SA 3.0 (http://creativecommons.org/licenses/by-nc-sa/3.0/)
%
%%%%%%%%%%%%%%%%%%%%%%%%%%%%%%%%%%%%%%%%%

%----------------------------------------------------------------------------------------
%	PACKAGES AND OTHER DOCUMENT CONFIGURATIONS
%----------------------------------------------------------------------------------------
\documentclass[11pt]{article}
\usepackage{geometry} % Pour passer au format A4
\geometry{hmargin=1cm, vmargin=1cm} % 


\usepackage[T1]{fontenc} % Use 8-bit encoding that has 256 glyphs
\usepackage[english,francais]{babel} % Français et anglais
\usepackage[utf8]{inputenc} 

\usepackage{amsmath,amsfonts,amsthm} % Math packages

\usepackage{lmodern}
\usepackage{url}
\usepackage{eurosym} % signe Euros
\usepackage{geometry} % Pour passer au format A4
\geometry{a4paper} % 
\usepackage{graphicx} % Required for including pictures
\usepackage{float} % Allows putting an [H] in \begin{figure} to specify the exact location of the figure

\usepackage{multicol}
\usepackage{sectsty} % Allows customizing section commands
\allsectionsfont{\centering \normalfont\scshape} % Make all sections centered, the default font and small caps

%----------------------------------------------------------------------------------------
%	Pied de Page
%----------------------------------------------------------------------------------------

\setlength\parindent{0pt} % Removes all indentation from paragraphs - comment this line for an assignment with lots of text


%----------------------------------------------------------------------------------------
%	Titre
%----------------------------------------------------------------------------------------

\newcommand{\horrule}[1]{\rule{\linewidth}{#1}} % Create horizontal rule command with 1 argument of height


%----------------------------------------------------------------------------------------
%	Début du document
%----------------------------------------------------------------------------------------


\title{Enchaînements d'opérations} % Title
\author{$5^e 1$}
\begin{document}

\maketitle % Insert the title, author and date


%------------------------------------------------
\section{Exercices}
%------------------------------------------------

%-----------------------------------111111111111111111111111111111
\subsection{Connaître et utiliser le vocabulaire}
%-----------------------------------------------------------------

\textbf{EXO : 2 - p11} - \underline{J'utilise un vocabulaire précis}
\begin{enumerate}
\item
  \begin{enumerate}
  \item Le résultat d'une addition est une \textbf{somme}.
  \item Le résultat d'une soustraction est une \textbf{différence}.
  \item Les nombres que l'on ajoute ou que l'on soustrait s'appellent des \textbf{termes}.
  \end{enumerate}
\item
  \begin{enumerate}
  \item Le résultat d'une multiplication est un \textbf{produit}.
  \item Les nombres que l'on multiplient s'appellent des \textbf{facteurs}.
  \end{enumerate}

\item Le résultat d'une division est un \textbf{quotient}.
\end{enumerate}


%-----------------------------------222222222222222222222222222222
\subsection{Connaître et utiliser les règles de priorités opératoires}
%-----------------------------------------------------------------

%-----------------------------------------------------------------
\subsubsection{Priorités de calcul sans les parenthèses}

\textbf{EXO : 34 - p19} - \underline{Recopier et calculer en détaillant les étapes}

\begin{enumerate}
\item $$A = 5 + 4 \times 7 - 3 $$
  \begin{enumerate}
  \item $A = 5 + \underline{4 \times 7} - 3 $
    On remarque, une multiplication : $4 \times 7$. On effectue en premier son calcul.\\
    $4 \times 7 = 28$\\
    Le produit de 4 et 7 est égale à 28.
  \item On remplace la multiplication par la valeur du produit dans A.\\
    $A = 5 + 4 \times 7 - 3 = 5 + 28 - 3$\\
  \item Il n'y a plus que des opérations de mêmes priorités opératoires. On effectue maintenant les opérations de gauche à droite, dans le sens de la lecture.
    $A = \underline{5 + 28} - 3$\\
    La somme de 5 et 28 est égale à 33.\\
    $5 + 28 = 33$\\
  \item On remplace l'addition par la valeur de la somme dans A.\\
    $A = 5 + 28 - 3 = 33 - 3$\\
    On poursuit le calcul avec la dernière opération : 33 - 3.\\
    $33 - 3 = 30$\\
  \item \textbf{A = 30}\\
  \end{enumerate}


\item $$B = 30 - 25 /5 + 11 $$
  $B = 30 -  \underline{25 /5} + 11 $ // On commence par la division.\\
  $25 / 5 = 5$\\
  $B = \underline{30 - 5} + 11$ // On effectue les opérations dans le sens de la lecture.\\
  $30 - 5 = 25$\\
  $B = 25 + 11$ // On finit avec l'addition.\\
  $25 + 11 = 36$\\
  \textbf{B = 36}\\
  
\item $$C = 24 + 42 /6 - 30 $$
  $C = 24 +  \underline{42 /6} - 30 $ // On commence par la division.\\
  $42 / 6 = 7$\\
  $C =  \underline{24 + 7} - 30 $ // On effectue les opérations dans le sens de la lecture.\\
  $24 + 7 = 33$\\
  $C = 33 - 30$ // On finit avec la soustraction.\\
  $33 - 30$\\
  \textbf{C = 30}\\
  

\item $$D = 51 - 12 \times 4 + 33 $$
  $D = 51 - \underline{12 \times 4} + 33 $ // On commence par la multiplication.\\
  $12 \times 4 = 48$\\
  $D = \underline{51 - 48} + 33 $ // On effectue les opérations dans le sens de la lecture.\\
  $51 - 48 = 3$\\
  $D = 3 + 33$ // On finit avec l'addition.\\
  $3 + 33$\\
  \textbf{D = 36}\\
\end{enumerate}

\textbf{EXO : 35 - p19} - \underline{Recopier et calculer en détaillant les étapes}
\begin{enumerate}

\item $$A = 4 \times 3 + 45 / 5$$
  $A =  \underline{4 \times 3} + 45 / 5$ // On commence par la multiplication.\\
  $4 \times 3 = 12$\\
  $A =  12 + \underline{45 / 5}$ // On poursuit avec la division.\\
  $45 / 5 = 9$\\
  $A = 12 + 9$ // On finit avec l'addition.\\
  $12 + 9 = 21$\\ 
  \textbf{A = 21}\\
  
\item $$B = 88/4 - 6 \times 3$$
  $B = \underline{88/4} - 6 \times 3$ // On commence avec la division.\\
  $88/4 = 22$\\
  $B = 22 - \underline{6 \times 3}$ // On poursuit avec la multiplication.\\
  $6 \times 3 = 18$\\
  $B = 22 - 18$  // On finit avec la soustraction.\\
  $22 - 18 = 4$\\
  \textbf{B = 4}\\
  
\item $$C = 17 \times 2 - 81/9$$

  $C =  \underline{17 \times 2} - 81/9$ // On commence par la multiplication.\\
  $17 \times 2 = 34$\\
  $C =  34 - \underline{81/9}$ // On poursuit avec la division.\\
  $81/9 = 9$\\
  $C =  34 - 9$ // On finit avec la soutraction.\\
  $34 - 9 = 25$\\
  \textbf{C = 25}\\

\item $$D = 56/7 + 8 \times 4$$

  $D = \underline{56/7} + 8 \times 4$ // On commence avec la division.\\
  $56/7 = 8$\\
  $D = 8 + \underline{8 \times 4}$ // On poursuit avec la multiplication.\\
  $8 \times 4 = 32$\\
  $D = 8 + 32$ // On finit avec l'addition.\\
  $8 + 32 = 40$\\
  \textbf{D = 40}\\
\end{enumerate}
%-----------------------------------------------------------------
\subsubsection{Priorités de calcul avec des parenthèses}

\textbf{Rédaction détaillée}

$$A = 1 + (6 + 2 \times (5 - 3))/5 $$

Le calcul est composé des quatre opérations et de deux couples de parenthèses. Un calcul à l'intérieur d'une parenthèse est prioritaire face aux autres opérations. On commence par le calcul de l'intérieur de la première parenthèse.
\begin{enumerate}
\item Le calcul intermédiaire de  $6 + 2 \times (5 - 3)$.\\
  Le calcul est composé de plusieurs opérations et d'un couple de parenthèses. Le calcul à l'intérieur de la parenthèse est prioritaire.
  
  \begin{enumerate}
  \item Le calcul intermédiaire de  $5 - 3$.\\
    $5 - 3 = 2$\\
  \item On revient au calcul $6 + 2 \times 2$.
    On effectue en priorité la multiplication :\\
    $2 \times 2 = 4$\\
    Puis, on calcule ensuite l'addition :\\
    $6 + 4 = 10$
  \end{enumerate}
  L'intérieur de la première parenthèse est égale à 10.

\item On revient au calcul de A : $A =  1 + 10/5$.\\
  La division est prioritaire sur l'addition. Le premier calcul a effectuer est la division de 10 par 5 et le deuxième l'addition.\\
  $10/5 = 2$\\
  $1 + 2 = 3$
\end{enumerate}
Le résultat du calcul est A = 3.\\

\paragraph{Placement des parenthèses}~~\\

\textbf{EXO : 40 - p19} - \underline{Ajouter les parenthèses pour que les égalités soit vraies}
\begin{enumerate}
\item $$8 + 5 \times 2 = 26 $$
  $8 + 5 \times 2 = 8 + 10 = 18 $ $\bigotimes$\\  
  $(8 + 5) \times 2 =  13 \times 2 = 26$ $\surd$

\item $$9 \times 7 - 4 = 27$$
  $9 \times 7 - 4 = 63 - 4 = 59$ $\bigotimes$\\
  $9 \times (7 - 4) = 9 \times 3 = 27$  $\surd$

\item $$12 - 9 + 3 = 0$$
  $12 - 9 + 3 = 3 + 3 = 6$ $\bigotimes$\\
  $12 - (9 + 3) = 12 - 12 = 0$ $\surd$

\item $$3 \times 4 + 2 \times 5 = 90$$
  $3 \times 4 + 2 \times 5 = 12 + 2 \times 5 = 12 + 10 = 22$  $\bigotimes$ \\
  $3 \times (4 + 2 \times 5) = 3 \times (4 + 10) = 3 \times 14 = 42$  $\bigotimes$\\
  $(3 \times 4 + 2) \times 5 = (12 + 2) \times 5 = 14 \times 5 = 70$ $\bigotimes$\\
  $3 \times (4 + 2) \times 5 = 3 \times 6 \times 5 = 18 \times 5 = 90$  $\surd$

\end{enumerate}

\paragraph{Calculs ``simples'' avec des parenthèses}~~\\

\textbf{EXO : 41 - p19} - \underline{Calculer en détaillant les étapes}

\begin{enumerate}
\item $$A = 17 + (13 - 4 + 2)$$
  (remarques : la parenthèse n'était pas nécessaire)\\
  $A = 17 + \underline{(13 - 4 + 2)}$ // On commence par l'intérieur de la parenthèse.\\
  $\underline{13 - 4} + 2$ // On calcul dans le sens de lecture.\\
  $13 - 4 = 9$\\
  $9 + 2 = 11$\\
  $A = 17 + 11$ // On finit par l'addition.\\
  $17 + 11 = 28$\\
  \textbf{A = 28}\\

\item $$B = 39 / (25 - 14 + 2)$$
  $B = 39 / \underline{(25 - 14 + 2)}$ // On commence par l'intérieur de la parenthèse.\\
  $\underline{25 - 14} + 2$ // On calcul dans le sens de lecture.\\
  $25 - 14 = 11$\\
  $11 + 2 = 13$\\
  $B = 39/13$ // On finit avec la division.\\
  $39/13 = 3$\\
  \textbf{B = 3}\\

\item $$C = (31 + 15)/(7 + 16)$$
  $C =\underline{(31 + 15)}/(7 + 16)$// On commence par l'intérieur de la parenthèse.\\
  $31 + 15 = 46$\\ 
  $C =31/ \underline{(7 + 16)}$// On continue par l'intérieur de la parenthèse.\\
  $7 + 16 = 23$\\
  $C = 46 / 23$ // On finit avec la division.\\
  $46 / 23 = 2$\\
  \textbf{C = 2}

\item $$D = 8 \times (19 - 16 + 3)$$

  $D = 8 \times \underline{(19 - 16 + 3)}$ // On commence par l'intérieur de la parenthèse.\\
  $\underline{19 - 16} + 3$ // On effectue les calculs dans le sens de lecture.\\
  $19 - 16 = 3$\\
  $3 + 3 = 6$\\
  $D = 8 \times 6$\\
  $8 \times 6 = 48$ //On finit le calcul avec la multiplication. \\
  \textbf{D = 48}\\
  
\end{enumerate}

\paragraph{Calculs avec des parenthèses}~~\\

\textbf{EXO : 39 - p19} - \underline{Supprimer les parenthèses en trop et calculer.}
\begin{enumerate}
\item $$A = $$
\item $$B = $$
\item $$C = $$
\item $$D = $$
\end{enumerate}


\end{document}

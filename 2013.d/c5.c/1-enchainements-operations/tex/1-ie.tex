%%%%%%%%%%%%%%%%%%%%%%%%%%%%%%%%%%%%%%%%%
% Short Sectioned Assignment
% LaTeX Template
% Version 1.0 (5/5/12)
%
% This template has been downloaded from:
% http://www.LaTeXTemplates.com
%
% Original author:
% Frits Wenneker (http://www.howtotex.com)
%
% License:
% CC BY-NC-SA 3.0 (http://creativecommons.org/licenses/by-nc-sa/3.0/)
%
%%%%%%%%%%%%%%%%%%%%%%%%%%%%%%%%%%%%%%%%%

%----------------------------------------------------------------------------------------
%	PACKAGES AND OTHER DOCUMENT CONFIGURATIONS
%----------------------------------------------------------------------------------------
\documentclass[11pt]{article}
\usepackage{geometry} % Pour passer au format A4
\geometry{hmargin=1cm, vmargin=1cm} % 


\usepackage[T1]{fontenc} % Use 8-bit encoding that has 256 glyphs
\usepackage[english,francais]{babel} % Français et anglais
\usepackage[utf8]{inputenc} 

\usepackage{amsmath,amsfonts,amsthm} % Math packages

\usepackage{lmodern}
\usepackage{url}
\usepackage{eurosym} % signe Euros
\usepackage{geometry} % Pour passer au format A4
\geometry{a4paper} % 
\usepackage{graphicx} % Required for including pictures
\usepackage{float} % Allows putting an [H] in \begin{figure} to specify the exact location of the figure

\usepackage{multicol}
\usepackage{sectsty} % Allows customizing section commands
\allsectionsfont{\centering \normalfont\scshape} % Make all sections centered, the default font and small caps

%----------------------------------------------------------------------------------------
%	Pied de Page
%----------------------------------------------------------------------------------------

\setlength\parindent{0pt} % Removes all indentation from paragraphs - comment this line for an assignment with lots of text


%----------------------------------------------------------------------------------------
%	Titre
%----------------------------------------------------------------------------------------

\newcommand{\horrule}[1]{\rule{\linewidth}{#1}} % Create horizontal rule command with 1 argument of height


%----------------------------------------------------------------------------------------
%	Début du document
%----------------------------------------------------------------------------------------


\title{Enchaînements d'opérations} % Title
\author{$5^e 1$}
\date{17 Septembre 2013} % Date for the report
\begin{document}

\maketitle % Insert the title, author and date

\begin{center}
\textsf{--}\\
\textsf{La réalité, c'est ce qui refuse de disparaître quand on cesse d'y croire.}\\
\texttt{Philip K. Dick}\\
\textsf{--}
\end{center}
%----------------------------------------------------------------------------------------
%	SECTION 1
%----------------------------------------------------------------------------------------

\thispagestyle{empty}

\section{Leçon}
\textbf{Remplir le texte à trous.}

\begin{enumerate}
\item Le résultat de \underline{\phantom{l'addition11111}} de deux \underline{\phantom{termes1111}} est une somme.\\
\item Le résultat de la soustraction de deux \underline{\phantom{termes1111}} est une \underline{\phantom{différence1111}}.\\
\item Le résultat de la multiplication de deux \underline{\phantom{facteurs1111}} est un \underline{\phantom{produit1111}}.\\
\item Le \underline{\phantom{quotient1111}} de a par b est la résultat de la division de a par b.
\end{enumerate}


%----------------------------------------------------------------------------------------
%	SECTION 2
%----------------------------------------------------------------------------------------
\vspace{1cm}
\section{Calcul}
\textbf{Effectuer les calculs en détaillant les étapes.}

\begin{enumerate}
\item $A = 3 + 7 \times 3 $\\
\item $B = 5 - (5 \div 2  + 2)$\\
\item $C = 2 + 2 \times 3 \times 5 \div 3 - (2 \times 2)$
\end{enumerate}

\end{document}

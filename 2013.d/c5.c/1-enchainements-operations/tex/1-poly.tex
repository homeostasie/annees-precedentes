%%%%%%%%%%%%%%%%%%%%%%%%%%%%%%%%%%%%%%%%%
% Short Sectioned Assignment
% LaTeX Template
% Version 1.0 (5/5/12)
%
% This template has been downloaded from:
% http://www.LaTeXTemplates.com
%
% Original author:
% Frits Wenneker (http://www.howtotex.com)
%
% License:
% CC BY-NC-SA 3.0 (http://creativecommons.org/licenses/by-nc-sa/3.0/)
%
%%%%%%%%%%%%%%%%%%%%%%%%%%%%%%%%%%%%%%%%%

%----------------------------------------------------------------------------------------
%	PACKAGES AND OTHER DOCUMENT CONFIGURATIONS
%----------------------------------------------------------------------------------------

\documentclass[paper=a4, fontsize=10pt]{scrartcl} % A4 paper and 11pt font size


\usepackage[T1]{fontenc} % Use 8-bit encoding that has 256 glyphs
\usepackage[english,francais]{babel} % Français et anglais
\usepackage[utf8]{inputenc} 

\usepackage{amsmath,amsfonts,amsthm} % Math packages

\usepackage{lmodern}
\usepackage{url}
\usepackage{eurosym} % signe Euros
\usepackage{geometry} % Pour passer au format A4
\geometry{a4paper} % 
\usepackage{graphicx} % Required for including pictures
\usepackage{float} % Allows putting an [H] in \begin{figure} to specify the exact location of the figure

\usepackage{multicol}
\usepackage{sectsty} % Allows customizing section commands
\allsectionsfont{\centering \normalfont\scshape} % Make all sections centered, the default font and small caps

%----------------------------------------------------------------------------------------
%	Pied de Page
%----------------------------------------------------------------------------------------


\usepackage{fancyhdr} % Custom headers and footers
\pagestyle{fancyplain} % Makes all pages in the document conform to the custom headers and footers
\fancyhead{} % No page header - if you want one, create it in the same way as the footers below
\fancyfoot[L]{$5^{e}1$} % Empty left footer
\fancyfoot[C]{Chapitre 1 - Enchaînements d'opérations} % Empty center footer
\fancyfoot[R]{\thepage} % Page numbering for right footer

\renewcommand{\headrulewidth}{0pt} % Remove header underlines
\renewcommand{\footrulewidth}{0pt} % Remove footer underlines

\setlength{\headheight}{13.6pt} % Customize the height of the header


\setlength\parindent{0pt} % Removes all indentation from paragraphs - comment this line for an assignment with lots of text


%----------------------------------------------------------------------------------------
%	Titre
%----------------------------------------------------------------------------------------

\newcommand{\horrule}[1]{\rule{\linewidth}{#1}} % Create horizontal rule command with 1 argument of height


\title{	
\vspace{-10ex}
\horrule{0.5pt} \\[0.4cm] % Thin top horizontal rule
\huge Chapitre 1 - Enchaînements d'opérations\\ % The assignment title
\horrule{2pt} \\[0.5cm] % Thick bottom horizontal rule
}

\author{}
\date{\vspace{-10ex}} % Today's date or a custom date

%----------------------------------------------------------------------------------------
%	Début du document
%----------------------------------------------------------------------------------------

\begin{document}

%----------------------------------------------------------------------------------------
% RE-DEFINITION
%----------------------------------------------------------------------------------------
% MATHS
%-----------

\newtheorem{Definition}{Définition}
\newtheorem{Theorem}{Théorème}
\newtheorem{Proposition}{Proposition}

% MATHS
%-----------
\renewcommand{\labelitemi}{$\bullet$}
\renewcommand{\labelitemii}{$\circ$}
%----------------------------------------------------------------------------------------
%	Titre
%----------------------------------------------------------------------------------------

\maketitle

%----------------------------------------------------------111111111111111111111111111111
\section{Connaître et utiliser le vocabulaire}
%----------------------------------------------------------------------------------------

\begin{description}
\item[$+$] : Le résultat de l'\textbf{addition} de deux \textbf{termes} est une \textbf{somme}.
\item[$-$] : Le résultat de la \textbf{soustraction} de deux \textbf{termes} est une \textbf{différence}.
\item[$*,\times$] : Le résultat du \textbf{produit} de deux \textbf{facteurs} est un \textbf{produit}.
\item[$/,:,\div$] :  Le résultat du \textbf{quotient} du \textbf{numérateur} (en haut) par le \textbf{dénominateur} est un \textbf{quotient}.
\end{description}

%----------------------------------------------------------222222222222222222222222222222
\section{Connaître et utiliser les règles de priorités opératoires}
%----------------------------------------------------------------------------------------

Les calculs s’effectuent dans le sens de la lecture, \textbf{de gauche à droite} tout en respectant \textbf{des règles de priorité}. 

%----------------------------------------------------------------------------------------
\subsection{Règles de priorité}

Pour un même calcul, on effectue en premier les calculs à \textbf{l'intérieur des parenthèses}. Lorsque les parenthèses ne sont plus utiles, on calcule \textbf{les multiplications et les divisions}. Et enfin, on calcule \textbf{les additions et les soustractions}.\\ 

\subsubsection*{Résumé}
On effectue en premier : 
\begin{enumerate}
\item Les calculs à l'intérieur des parenthèses.
\item Les multiplications et les divisions.
\item Les additions et les soustractions.
\end{enumerate}

\subsubsection{Exemples}
\begin{itemize}
\item $8 + \underline{3 \times 2} = 8 + \textbf{6} = 14$
\item $\underline{8 \times 4} - \underline{6 \times 2} = \textbf{24} - \textbf{12} = 12$
\item $8 \times \underline{(2 + 3)} \times 10 = 8 \times \textbf{5} \times 10 = 40 \times 10 = 400$
\end{itemize}
%-----------------------------------------------------------333333333333333333333333333333
\section{Résoudre des problèmes à l'aides des enchaînements d'opérations}
%-----------------------------------------------------------------------------------------

Il est très courant de faire appel aux Mathématiques pour rechercher une solution à un problème posé. On appelle cette démarche de la modélisation.

\end{document}

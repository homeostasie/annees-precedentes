%%%%%%%%%%%%%%%%%%%%%%%%%%%%%%%%%%%%%%%%%
% LaTeX Template
% http://www.LaTeXTemplates.com
%
% Original author:
% Linux and Unix Users Group at Virginia Tech Wiki 
% (https://vtluug.org/wiki/Example_LaTeX_chem_lab_report)
%
% License:
% CC BY-NC-SA 3.0 (http://creativecommons.org/licenses/by-nc-sa/3.0/)
%
%%%%%%%%%%%%%%%%%%%%%%%%%%%%%%%%%%%%%%%%%

%----------------------------------------------------------------------------------------
%	PACKAGES AND DOCUMENT CONFIGURATIONS
%----------------------------------------------------------------------------------------

\documentclass[11pt]{article}
\usepackage{geometry} % Pour passer au format A4
\geometry{hmargin=1cm, vmargin=1cm} % 

\usepackage{graphicx} % Required for including pictures
\usepackage{float} % 

%Français
\usepackage[T1]{fontenc} 
\usepackage[english,francais]{babel}
\usepackage[utf8]{inputenc}
\usepackage{eurosym}
\usepackage{lmodern}
\usepackage{url}
\usepackage{multicol}
\usepackage{fancybox} 

%Maths
\usepackage{amsmath,amsfonts,amssymb,amsthm}
%\usepackage[linesnumbered, ruled, vlined]{algorithm2e}
%\SetAlFnt{\small\sffamily}

%Autres
\linespread{1} % Line spacing
\setlength\parindent{0pt} % Removes all indentation from paragraphs

\renewcommand{\labelenumi}{\alph{enumi}.} % 
\pagestyle{empty}
%----------------------------------------------------------------------------------------
%	DOCUMENT INFORMATION
%----------------------------------------------------------------------------------------
\begin{document}

\cornersize{1}
%\maketitle % Insert the title, author and date

\begin{minipage}[t]{\textwidth}
  \raggedright
      {\bfseries Série : \textbf{A}}\\
      {\bfseries $5^{e}1$}\\[.35ex]
      \vspace*{-1cm}
      \raggedleft
          {\bfseries Opérations relatifs}\\[.35ex]
          {\bfseries 27 Mai 2014}\\[.35ex]
\end{minipage}\\[1em]

\begin{center}
  \textsf{Bien souvent, les réponses les plus simples dans la vie sont celles qui ne nous viennent pas aussitôt à l'esprit.}\\
  \texttt{Stephen King}
\end{center}
  \begin{eqnarray*}
    Samia &=& \ovalbox{-3} \; \ovalbox{15} \; \ovalbox{-20} \; \ovalbox{0} \; \ovalbox{5} \; \ovalbox{-7} \; \ovalbox{6} \; \ovalbox{8} \; \ovalbox{-12} \; \ovalbox{2}\\
    Kévin &=& \ovalbox{1} \; \ovalbox{-4} \; \ovalbox{-10} \; \ovalbox{-7} \; \ovalbox{3} \; \ovalbox{0} \; \ovalbox{1} \; \ovalbox{15} \; \ovalbox{-7} \; \ovalbox{3}\\
    Caroline &=& \ovalbox{0} \; \ovalbox{1} \; \ovalbox{2} \; \ovalbox{-3} \; \ovalbox{-10} \; \ovalbox{11} \; \ovalbox{-7} \; \ovalbox{9} \; \ovalbox{4} \; \ovalbox{14}
  \end{eqnarray*}


    \begin{enumerate}
    \item[1]
          \begin{enumerate}
          \item Écrire une ligne de calcul permettant de calculer le score de Samia.
          \item Calculer le score.
          \end{enumerate}
    \item[2] 
          \begin{enumerate}
          \item Écrire une ligne de calcul permettant de calculer le score de Kévin.
          \item Calculer le score.
          \end{enumerate}
    \item[3] 
          \begin{enumerate}
          \item Écrire une ligne de calcul permettant de calculer le score de Caroline.
          \item Calculer le score.
          \end{enumerate}
    \item[4] Classer les scores des trois joueurs dans l'ordre croissant.
    \item[5]   
          \begin{enumerate}
          \item Enlever un jeton à Samia pour que son score augmente.
          \item Enlever un jeton à Kévin pour que son score diminue.
          \item Enlever un jeton à Caroline pour que son score reste inchangé.
          \end{enumerate}
    \end{enumerate}


\end{document}

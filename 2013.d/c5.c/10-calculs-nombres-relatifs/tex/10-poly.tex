%%%%%%%%%%%%%%%%%%%%%%%%%%%%%%%%%%%%%%%%%
% Short Sectioned Assignment
% LaTeX Template
% Version 1.0 (5/5/12)
%
% This template has been downloaded from:
% http://www.LaTeXTemplates.com
%
% Original author:
% Frits Wenneker (http://www.howtotex.com)
%
% License:
% CC BY-NC-SA 3.0 (http://creativecommons.org/licenses/by-nc-sa/3.0/)
%
%%%%%%%%%%%%%%%%%%%%%%%%%%%%%%%%%%%%%%%%%

%----------------------------------------------------------------------------------------
%	PACKAGES AND OTHER DOCUMENT CONFIGURATIONS
%----------------------------------------------------------------------------------------

\documentclass[paper=a4, fontsize=10pt]{scrartcl} % A4 paper and 11pt font size


\usepackage[T1]{fontenc} % Use 8-bit encoding that has 256 glyphs
\usepackage[english,francais]{babel} % Français et anglais
\usepackage[utf8]{inputenc} 

\usepackage{amsmath,amsfonts,amsthm} % Math packages

\usepackage{enumitem}
\usepackage{lmodern}
\usepackage{url}
\usepackage{eurosym} % signe Euros
\usepackage{geometry} % Pour passer au format A4
\geometry{a4paper} % 
\usepackage{graphicx} % Required for including pictures
\usepackage{float} % Allows putting an [H] in \begin{figure} to specify the exact location of the figure

\usepackage{multicol}
\usepackage{sectsty} % Allows customizing section commands
\allsectionsfont{\centering \normalfont\scshape} % Make all sections centered, the default font and small caps

%----------------------------------------------------------------------------------------
%	Pied de Page
%----------------------------------------------------------------------------------------


\usepackage{fancyhdr} % Custom headers and footers
\pagestyle{fancyplain} % Makes all pages in the document conform to the custom headers and footers
\fancyhead{} % No page header - if you want one, create it in the same way as the footers below
\fancyfoot[L]{$5^{e}1$} % Empty left footer
\fancyfoot[C]{Chapitre 10 - opérations sur les nombres relatifs} % Empty center footer
\fancyfoot[R]{\thepage} % Page numbering for right footer

\renewcommand{\headrulewidth}{0pt} % Remove header underlines
\renewcommand{\footrulewidth}{0pt} % Remove footer underlines

\setlength{\headheight}{13.6pt} % Customize the height of the header


\setlength\parindent{0pt} % Removes all indentation from paragraphs - comment this line for an assignment with lots of text


%----------------------------------------------------------------------------------------
%	Titre
%----------------------------------------------------------------------------------------

\newcommand{\horrule}[1]{\rule{\linewidth}{#1}} % Create horizontal rule command with 1 argument of height


\title{	
  \vspace{-10ex}
  \horrule{0.5pt} \\[0.4cm] % Thin top horizontal rule
  \huge Chapitre 10 - opérations sur les nombres relatifs\\ % The assignment title
  \horrule{2pt} \\[0.5cm] % Thick bottom horizontal rule
}

\author{}
\date{\vspace{-10ex}} % Today's date or a custom date

%----------------------------------------------------------------------------------------
%	Début du document
%----------------------------------------------------------------------------------------

\begin{document}

%----------------------------------------------------------------------------------------
% RE-DEFINITION
%----------------------------------------------------------------------------------------
% MATHS
%-----------

\newtheorem{Definition}{Définition}
\newtheorem{Theorem}{Théorème}
\newtheorem{Proposition}{Proposition}

% MATHS
%-----------
\renewcommand{\labelitemi}{$\bullet$}
\renewcommand{\labelitemii}{$\circ$}
%----------------------------------------------------------------------------------------
%	Titre
%----------------------------------------------------------------------------------------

\maketitle % Print the title


%-----------------------------------111111111111111111111111111111111111
\section{Addition et soustraction de nombres relatifs}
%-----------------------------------------------------------------------
\begin{multicols}{2}
  %-----------------------------------------------------------------------
  \subsection{Addition de nombres relatifs de même signe}

  \begin{eqnarray*}
    2 + 3 &=& 5\\
    -4 + (-5) &=& -9
  \end{eqnarray*}

  \paragraph{Remarques: } 
  \begin{itemize}
  \item On met des parenthèses entre deux signes consécutifs.
  \item Le signe est conservé
  \end{itemize}

  %-----------------------------------------------------------------------
  \subsection{Addition de nombres relatifs de signes différents}

  \begin{itemize}[label=$\Diamond$]
  \item $2 + (-5) = -3$
  \item -$7 + 13 = 5$
  \end{itemize}

  \paragraph{Remarques: }On soustrait le plus grand nombre \textit{sans son signe} par le plus petit nombre \textit{sans son signe}. On conserve le signe du plus grand.
\end{multicols}

%-----------------------------------------------------------------------
\subsection{Soustraction de nombres relatifs}

\begin{Definition}{}
  Pour soustraire un nombre relatif, on ajoute son opposé.
\end{Definition}

\begin{itemize}[label=$\Diamond$]
\item $ 10 - 2    = 10 + (-2) =  8$
\item $ 10 - (-2) = 10 + 2    = 12$
\item $-10 - 2    =-10 + (-2) =-12$
\item $-10 - (-2) =-10 + 2    =-8$
\end{itemize}


%-----------------------------------222222222222222222222222222222222222
\section{Distance}
%-----------------------------------------------------------------------

\begin{Definition}{Distance séparant 2 points sur un axe gradué}\\
  La distance entre le point $A$ et le point $B$ est noté AB.\\
  $AB = $ Abscisse la plus grande - Abscisse la plus petite.
\end{Definition}

\begin{figure}[H]
  \centering
  \includegraphics[width=.8\linewidth]{sources/cours/dist-axe.pdf}
\end{figure}

\begin{itemize}[label=$\Diamond$]
\item $AB = -2.7 - (-5.1) = -2.7 + 5.1 = 2.4$
\item $DE = 6.4 - 1.5 = 4.9$
\item $CD = 1.5 - (-0.1) = 1.5 + 0.1 = 1.6$
\end{itemize}

\paragraph{Remarques: } 
\begin{itemize}
\item AB = BA
\item La distance entre deux points est toujours positive.
\end{itemize}



\end{document}

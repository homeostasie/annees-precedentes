\documentclass{beamer}

\usepackage{geometry} % Pour passer au format A4
\usepackage{graphicx} % Required for including pictures
\usepackage{float} % 

\usepackage{amsmath,amsfonts,amssymb,amsthm}
\usepackage[T1]{fontenc} 
\usepackage[english,francais]{babel}
\usepackage[utf8]{inputenc}
\usepackage{lmodern}

\usetheme{Warsaw}

\title{Opérations sur les nombres relatifs}
\author{$5^{e}1$}

\begin{document}

\frame{\titlepage}

\section{Addition et soustraction de nombres relatifs}

\subsection{Addition de nombres relatifs de même signe}

\begin{frame}
  \frametitle{Addition de nombre relatif de même signe}
  \begin{exampleblock}{Exemples}	
    \begin{eqnarray*}
      2 + 3 &=& 5\\
      -4 + (-5) &=& -9
    \end{eqnarray*}
  \end{exampleblock}

  \begin{block}{Remarques}
    \begin{itemize}
    \item On met des parenthèses entre deux signes consécutifs.
    \item Le signe est conservé
    \end{itemize}
  \end{block}
\end{frame}

\subsection{Addition de nombres relatifs de signes différents}

\begin{frame}
  \frametitle{Addition de nombres relatifs de signes différents}
  \begin{exampleblock}{Exemples}	
    \begin{eqnarray*}
      2 + (-5) &=& -3\\
      -7 + 13 &=& -5
    \end{eqnarray*}
  \end{exampleblock}

  \begin{block}{Remarque}
    On soustrait le plus grand nombre \textit{sans son signe} par le plus petit nombre \textit{sans son signe}. On conserve le signe du plus grand.
  \end{block}
\end{frame}

%-----------------------------------------------------------------------
\subsection{Soustraction de nombres relatifs}
\begin{frame}
  \frametitle{Soustraction de nombres relatifs}
  
  \begin{alertblock}{Définition}	
    Pour soustraire un nombre relatif, on ajoute son opposé.
  \end{alertblock}
  
  \begin{exampleblock}{Exemples}	
    \begin{eqnarray*}
      10 - 2    = 10 + (-2) &=&  8\\
      10 - (-2) = 10 + 2    &=& 12\\
      -10 - 2    =-10 + (-2) &=&-12\\
      -10 - (-2) =-10 + 2    &=&-8
    \end{eqnarray*}
  \end{exampleblock}
  
\end{frame}

\end{document}

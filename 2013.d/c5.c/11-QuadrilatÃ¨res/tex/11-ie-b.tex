%%%%%%%%%%%%%%%%%%%%%%%%%%%%%%%%%%%%%%%%%
% LaTeX Template
% http://www.LaTeXTemplates.com
%
% Original author:
% Linux and Unix Users Group at Virginia Tech Wiki 
% (https://vtluug.org/wiki/Example_LaTeX_chem_lab_report)
%
% License:
% CC BY-NC-SA 3.0 (http://creativecommons.org/licenses/by-nc-sa/3.0/)
%
%%%%%%%%%%%%%%%%%%%%%%%%%%%%%%%%%%%%%%%%%

%----------------------------------------------------------------------------------------
%	PACKAGES AND DOCUMENT CONFIGURATIONS
%----------------------------------------------------------------------------------------

\documentclass[11pt]{article}
\usepackage{geometry} % Pour passer au format A4
\geometry{hmargin=1cm, vmargin=1cm} % 

\usepackage{graphicx} % Required for including pictures
\usepackage{float} % 

%Français
\usepackage[T1]{fontenc} 
\usepackage[english,francais]{babel}
\usepackage[utf8]{inputenc}
\usepackage{eurosym}
\usepackage{lmodern}
\usepackage{url}
\usepackage{multicol}
\usepackage{fancybox} 

%Maths
\usepackage{amsmath,amsfonts,amssymb,amsthm}
%\usepackage[linesnumbered, ruled, vlined]{algorithm2e}
%\SetAlFnt{\small\sffamily}

%Autres
\linespread{1} % Line spacing
\setlength\parindent{0pt} % Removes all indentation from paragraphs

\renewcommand{\labelenumi}{\alph{enumi}.} % 
\pagestyle{empty}
%----------------------------------------------------------------------------------------
%	DOCUMENT INFORMATION
%----------------------------------------------------------------------------------------
\begin{document}

\cornersize{1}
%\maketitle % Insert the title, author and date

\begin{minipage}[t]{\textwidth}
  \raggedright
      {\bfseries Série : \textbf{B}}\\
      {\bfseries $5^{e}1$}\\[.35ex]
      \vspace*{-1cm}
      \raggedleft
          {\bfseries Opérations relatifs}\\[.35ex]
          {\bfseries 27 Mai 2014}\\[.35ex]
\end{minipage}\\[1em]

\begin{center}
  \textsf{Un peuple prêt à sacrifier un peu de liberté pour un peu de sécurité ne mérite ni l'une ni l'autre, et finit par perdre les deux.}\\
  \texttt{Benjamin Franklin}
\end{center}

\subsection*{1 - Tracer}
\begin{enumerate}
\item Tracer un parallélogramme ABCD tel que [AB] = 6cm, [AD] = 3cm et [DB] = 8cm.
\item Tracer un parallélogramme EFGH tel que [HF] = 10cm, [EG] = 8cm et [EH] = 4cm.
\end{enumerate}

\subsection*{2 - Reconnaître}

Donner la nature : quadrilatère, parallélogramme, losange, rectangle ou carré. \textbf{JUSTIFIER} avec la propriété correspondant au codage.
\begin{multicols}{2}
  \begin{enumerate}
  \item 
    \begin{figure}[H]
      \centering
      \includegraphics[width=0.8\linewidth]{sources/ie/qu-5.pdf}
    \end{figure}
  \item
    \begin{figure}[H]
      \centering
      \includegraphics[width=0.8\linewidth]{sources/ie/qu-6.pdf}
    \end{figure}
  \item 
    \begin{figure}[H]
      \centering
      \includegraphics[width=0.8\linewidth]{sources/ie/qu-7.pdf}
    \end{figure}
  \item  
    \begin{figure}[H]
      \centering
      \includegraphics[width=0.8\linewidth]{sources/ie/qu-8.pdf}
    \end{figure}
  \end{enumerate}
\end{multicols}
\end{document}

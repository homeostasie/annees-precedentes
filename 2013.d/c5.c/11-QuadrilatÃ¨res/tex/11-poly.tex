%%%%%%%%%%%%%%%%%%%%%%%%%%%%%%%%%%%%%%%%%
% Short Sectioned Assignment
% LaTeX Template
% Version 1.0 (5/5/12)
%
% This template has been downloaded from:
% http://www.LaTeXTemplates.com
%
% Original author:
% Frits Wenneker (http://www.howtotex.com)
%
% License:
% CC BY-NC-SA 3.0 (http://creativecommons.org/licenses/by-nc-sa/3.0/)
%
%%%%%%%%%%%%%%%%%%%%%%%%%%%%%%%%%%%%%%%%%

%----------------------------------------------------------------------------------------
%	PACKAGES AND OTHER DOCUMENT CONFIGURATIONS
%----------------------------------------------------------------------------------------

\documentclass[paper=a4, fontsize=9pt]{scrartcl} % A4 paper and 11pt font size


\usepackage[T1]{fontenc} % Use 8-bit encoding that has 256 glyphs
\usepackage[english,francais]{babel} % Français et anglais
\usepackage[utf8]{inputenc} 

\usepackage{amsmath,amsfonts,amsthm} % Math packages

\usepackage{enumitem}
\usepackage{lmodern}
\usepackage{url}
\usepackage{eurosym} % signe Euros
\usepackage{geometry} % Pour passer au format A4
\geometry{a4paper} % 
\usepackage{graphicx} % Required for including pictures
\usepackage{float} % Allows putting an [H] in \begin{figure} to specify the exact location of the figure

\usepackage{multicol}
\usepackage{sectsty} % Allows customizing section commands
\allsectionsfont{\centering \normalfont\scshape} % Make all sections centered, the default font and small caps

%----------------------------------------------------------------------------------------
%	Pied de Page
%----------------------------------------------------------------------------------------


\usepackage{fancyhdr} % Custom headers and footers
\pagestyle{fancyplain} % Makes all pages in the document conform to the custom headers and footers
\fancyhead{} % No page header - if you want one, create it in the same way as the footers below
\fancyfoot[L]{$5^{e}1$} % Empty left footer
\fancyfoot[C]{Chapitre 11 - Quadrilatères} % Empty center footer
\fancyfoot[R]{\thepage} % Page numbering for right footer

\renewcommand{\headrulewidth}{0pt} % Remove header underlines
\renewcommand{\footrulewidth}{0pt} % Remove footer underlines

\setlength{\headheight}{13.6pt} % Customize the height of the header


\setlength\parindent{0pt} % Removes all indentation from paragraphs - comment this line for an assignment with lots of text


%----------------------------------------------------------------------------------------
%	Titre
%----------------------------------------------------------------------------------------

\newcommand{\horrule}[1]{\rule{\linewidth}{#1}} % Create horizontal rule command with 1 argument of height


\title{	
  \vspace{-10ex}
  \horrule{0.5pt} \\[0.4cm] % Thin top horizontal rule
  \huge Chapitre 11 - Quadrilatères\\ % The assignment title
  \horrule{2pt} \\[0.5cm] % Thick bottom horizontal rule
}

\author{}
\date{\vspace{-10ex}} % Today's date or a custom date

%----------------------------------------------------------------------------------------
%	Début du document
%----------------------------------------------------------------------------------------

\begin{document}

%----------------------------------------------------------------------------------------
% RE-DEFINITION
%----------------------------------------------------------------------------------------
% MATHS
%-----------

\newtheorem{Definition}{Définition}
\newtheorem{Theorem}{Théorème}
\newtheorem{Proposition}{Propriété}

% MATHS
%-----------
\renewcommand{\labelitemi}{$\bullet$}
\renewcommand{\labelitemii}{$\circ$}
%----------------------------------------------------------------------------------------
%	Titre
%----------------------------------------------------------------------------------------

\maketitle % Print the title

%-----------------------------------111111111111111111111111111111111111
\section{Parallélogramme}
%-----------------------------------------------------------------------

\begin{multicols}{2}
  \begin{figure}[H]
    \centering
    \includegraphics[width=0.5\linewidth]{sources/cours/para-0.pdf}
  \end{figure}

  \begin{Definition}
    Un parallélogramme est un quadrilatère dont les côtés opposés sont parallèles.
  \end{Definition}

  \paragraph{Remarques}
  Un quadrilatère est une figure avec quatre côtés.
\end{multicols}

\subsection{Propriétés du parallélogramme}
%-----------------------------------------------------------------------
\setlength{\columnseprule}{1pt}
\begin{multicols}{2}
  \begin{Proposition}
    Un quadrilatère avec ses côtés opposés de même longueur est un parallélogramme.
  \end{Proposition}

  \begin{Proposition}
    Un quadrilatère avec deux côtés opposés parallèles et de même longueur est un parallélogramme.
  \end{Proposition}

  \begin{Proposition}
    Un quadrilatère dont les diagonales se coupent en leur milieu est un parallélogramme.
  \end{Proposition}

  \begin{Proposition}
    Un quadrilatère dont les angles opposés ont une même mesure.
  \end{Proposition}
  \subsection{Aire du parallélogramme}

  \begin{figure}[H]
    \centering
    \includegraphics[width=0.7\linewidth]{sources/cours/aire-0.pdf}
  \end{figure}

  $$\mathcal{A} = base \times hauteur$$
\end{multicols}
%-----------------------------------------------------------------------

%-----------------------------------111111111111111111111111111111111111
\section{Parallélogrammes particuliers}
%-----------------------------------------------------------------------

\begin{Proposition}
  Le rectangle, le losange et le carré sont des parallélogrammes.
\end{Proposition}

\begin{multicols}{3}
  \subsection{Le rectangle}
  %-----------------------------------------------------------------------

  \begin{figure}[H]
    \centering
    \includegraphics[width=0.5\linewidth]{sources/cours/rec-0.pdf}
  \end{figure}

  \begin{Definition}
    Un rectangle est un quadrilatère avec quatre angles droits.
  \end{Definition}

  \paragraph{Propriétés}
  \begin{itemize}[label=$\Diamond$]
  \item Les diagonnales d'un rectangle sont de même longueur.
  \item Un rectangle a deux axes de symétrie : les médiatrices de ses côtés.
  \end{itemize}


  \subsection{Le losange}
  %-----------------------------------------------------------------------

  \begin{figure}[H]
    \centering
    \includegraphics[width=0.4\linewidth]{sources/cours/los-0.pdf}
  \end{figure}

  \begin{Definition}
    Un losange est un quadrilatère avec quatre côtés de même longueur.
  \end{Definition}

  \paragraph{Propriétés}
  \begin{itemize}[label=$\Diamond$]
  \item Les diagonnales d'un losange sont perpendiculaires.
  \item Un losange a deux axes de symétrie : ses diagonales.
  \end{itemize}

  \subsection{Le carré}
  %-----------------------------------------------------------------------
  \begin{figure}[H]
    \centering
    \includegraphics[width=0.3\linewidth]{sources/cours/car-0.pdf}
  \end{figure}
  
  \begin{Definition}
    Un carré est un quadrilatère avec quatre angles droits et quatre côtés de même longueur.
  \end{Definition}

  \paragraph{Propriétés}
  \begin{itemize}[label=$\Diamond$]
  \item Les diagonnales d'un rectangle sont perpendiculaires et de même longueur.
  \item Un rectangle a quatre axes de symétrie : les médiatrices de ses côtés et ses diagonales.
  \end{itemize}
\end{multicols}
\end{document}

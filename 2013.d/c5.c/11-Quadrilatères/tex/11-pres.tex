\documentclass{beamer}

\usepackage{geometry} % Pour passer au format A4
\usepackage{graphicx} % Required for including pictures
\usepackage{float} % 

\usepackage{amsmath,amsfonts,amssymb,amsthm}
\usepackage[T1]{fontenc} 
\usepackage[english,francais]{babel}
\usepackage[utf8]{inputenc}
\usepackage{lmodern}

\usetheme{Warsaw}

\title{Quadrilatères}
\author{$5^{e}1$}

\begin{document}

\frame{\titlepage}

\section{Parallélogramme}

\begin{frame}
  \frametitle{Le parallélogramme}
  \begin{alertblock}{Définition}	
    Un parallélogramme est un quadrilatère dont les côtés opposés sont parallèles. 

    \begin{columns}[t]
      \begin{column}{6cm}
        \begin{figure}[H]
          \centering
          \includegraphics[width=\linewidth]{sources/cours/para-2.pdf}
        \end{figure}
      \end{column}
      \begin{column}{4cm}
        \vspace{1cm}
        \[
        \left \{
        \begin{array}{r c l}
          (AB) & \slash \slash & (CD) \\
          (AD) & \slash \slash & (BC)\\
        \end{array}
        \right .
        \]
      \end{column}
    \end{columns} 
  \end{alertblock}

  \begin{exampleblock}{Remarques}
    Un quadrilatère est une figure avec quatre côtés.
  \end{exampleblock}

\end{frame}

\subsection{Propriétés du parallélogramme}

\begin{frame}
  \frametitle{Propriétés du parallélogramme}
  \begin{block}{ABCD est un parallélogramme si ABCD est :}	

    \begin{columns}[t]
      \begin{column}{4cm}
        \begin{figure}[H]
          \centering
          \includegraphics[width=\linewidth]{sources/cours/para-1.pdf}
        \end{figure}
      \end{column}
      \begin{column}{6cm}
        \begin{enumerate}
        \item<1-> Un quadrilatère avec ses quatre côtés opposés de même longueur.
        \item<2-> Un quadrilatère avec ses deux côtés opposés parallèles et de même longueur.
        \item<3-> Un quadrilatère dont les diagonales se coupent en leur milieu.
        \item<4> Un quadrilatère dont les angles opposés sont de même mesure.     
        \end{enumerate}
      \end{column}
    \end{columns} 
  \end{block}
\end{frame}

\begin{frame}
  \frametitle{L'aire du parallélogramme}

  \begin{columns}[t]
    \begin{column}{6cm}
      \begin{block}{}	
        \begin{figure}[H]
          \centering
          \includegraphics[height=2.3cm]{sources/cours/aire-0.pdf} \\
          \includegraphics[height=2.3cm]{sources/cours/aire-1.pdf} \\
          \includegraphics[height=2.3cm]{sources/cours/aire-2.pdf}       
        \end{figure}
      \end{block}
    \end{column}
    \begin{column}{5.5cm}
      \vspace{2.5cm}
      \begin{alertblock}{Aire du parallélogramme}
        $$\mathcal{A} = base \times hauteur$$
      \end{alertblock}
      
    \end{column}
  \end{columns} 

\end{frame}

\section{Parallélogrammes particuliers}

\subsection{Le rectangle}

\begin{frame}
  \frametitle{Le rectangle}
  
  \begin{columns}[t]
    \begin{column}{4cm}
      \begin{block}{}	
        \begin{figure}[H]
          \centering
          \includegraphics[width=\linewidth]{sources/cours/rec-0.pdf}
        \end{figure}
      \end{block} 
    \end{column}
    \begin{column}{6cm}
      \begin{alertblock}{Définition}	
	Un rectangle est un quadrilatère avec quatre angles droits.
      \end{alertblock}
    \end{column}
  \end{columns} 
  
  \begin{block}{Propriétés du rectangle}	
    \begin{itemize}
    \item Les diagonales d'un rectangle sont de même longueur et se coupent en leur milieu.
    \item Un rectangle a deux axes de symétrie : les médiatrices de ses côtés.
    \end{itemize}
  \end{block} 
  
\end{frame}

\subsection{Le losange}

\begin{frame}
  \frametitle{Le losange}
  
  \begin{columns}[t]
    \begin{column}{4cm}
      \begin{block}{}	
        \begin{figure}[H]
          \centering
          \includegraphics[width=\linewidth]{sources/cours/los-0.pdf}
        \end{figure}
      \end{block} 
    \end{column}
    \begin{column}{6cm}
      \begin{alertblock}{Définition}	
	Un losange est un quadrilatère avec quatre côtés de même longueur.
      \end{alertblock}
    \end{column}
  \end{columns} 
  
  \begin{block}{Propriétés du losange}	
    \begin{itemize}
    \item Les diagonales d'un losange se coupent en leur milieu perpendiculairement.
    \item Un losange a deux axes de symétrie : ses diagonales.
    \end{itemize}
  \end{block} 
  
\end{frame}

\subsection{Le carré}

\begin{frame}
  \frametitle{Le carré}
  
  \begin{columns}[t]
    \begin{column}{4cm}
      \begin{block}{}	
        \begin{figure}[H]
          \centering
          \includegraphics[width=0.8\linewidth]{sources/cours/car-0.pdf}
        \end{figure}
      \end{block} 
    \end{column}
    \begin{column}{6cm}
      \begin{alertblock}{Définition}	
	Un carré est un quadrilatère avec quatre angles droits et quatre côtés de même longueur.
      \end{alertblock}
    \end{column}
  \end{columns} 
  
  \begin{block}{Propriétés du carré}	
    \begin{itemize}
    \item Les diagonales d'un carré sont perpendiculaires et sont de même longueur.
    \item Un carré a quatre axes de symétrie : ses diagonales et les médiatrices de ses côtés.
    \end{itemize}
  \end{block} 
\end{frame}

\end{document}

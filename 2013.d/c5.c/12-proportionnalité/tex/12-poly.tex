%%%%%%%%%%%%%%%%%%%%%%%%%%%%%%%%%%%%%%%%%
% Short Sectioned Assignment
% LaTeX Template
% Version 1.0 (5/5/12)
%
% This template has been downloaded from:
% http://www.LaTeXTemplates.com
%
% Original author:
% Frits Wenneker (http://www.howtotex.com)
%
% License:
% CC BY-NC-SA 3.0 (http://creativecommons.org/licenses/by-nc-sa/3.0/)
%
%%%%%%%%%%%%%%%%%%%%%%%%%%%%%%%%%%%%%%%%%

%----------------------------------------------------------------------------------------
%	PACKAGES AND OTHER DOCUMENT CONFIGURATIONS
%----------------------------------------------------------------------------------------

\documentclass[paper=a4, fontsize=9pt]{scrartcl} % A4 paper and 11pt font size


\usepackage[T1]{fontenc} % Use 8-bit encoding that has 256 glyphs
\usepackage[english,francais]{babel} % Français et anglais
\usepackage[utf8]{inputenc} 

\usepackage{amsmath,amsfonts,amsthm} % Math packages

\usepackage{enumitem}
\usepackage{lmodern}
\usepackage{url}
\usepackage{eurosym} % signe Euros
\usepackage{geometry} % Pour passer au format A4
\geometry{a4paper} % 
\usepackage{graphicx} % Required for including pictures
\usepackage{float} % Allows putting an [H] in \begin{figure} to specify the exact location of the figure

\usepackage{multicol}
\usepackage{sectsty} % Allows customizing section commands
\allsectionsfont{\centering \normalfont\scshape} % Make all sections centered, the default font and small caps

%----------------------------------------------------------------------------------------
%	Pied de Page
%----------------------------------------------------------------------------------------


\usepackage{fancyhdr} % Custom headers and footers
\pagestyle{fancyplain} % Makes all pages in the document conform to the custom headers and footers
\fancyhead{} % No page header - if you want one, create it in the same way as the footers below
\fancyfoot[L]{$5^{e}1$} % Empty left footer
\fancyfoot[C]{Chapitre 12 - Proportionnalité} % Empty center footer
\fancyfoot[R]{\thepage} % Page numbering for right footer

\renewcommand{\headrulewidth}{0pt} % Remove header underlines
\renewcommand{\footrulewidth}{0pt} % Remove footer underlines

\setlength{\headheight}{13.6pt} % Customize the height of the header


\setlength\parindent{0pt} % Removes all indentation from paragraphs - comment this line for an assignment with lots of text


%----------------------------------------------------------------------------------------
%	Titre
%----------------------------------------------------------------------------------------

\newcommand{\horrule}[1]{\rule{\linewidth}{#1}} % Create horizontal rule command with 1 argument of height


\title{	
  \vspace{-10ex}
  \horrule{0.5pt} \\[0.4cm] % Thin top horizontal rule
  \huge Chapitre 12 - Proportionnalité\\ % The assignment title
  \horrule{2pt} \\[0.5cm] % Thick bottom horizontal rule
}

\author{}
\date{\vspace{-10ex}} % Today's date or a custom date

%----------------------------------------------------------------------------------------
%	Début du document
%----------------------------------------------------------------------------------------

\begin{document}

%----------------------------------------------------------------------------------------
% RE-DEFINITION
%----------------------------------------------------------------------------------------
% MATHS
%-----------

\newtheorem{Definition}{Définition}
\newtheorem{Theorem}{Théorème}
\newtheorem{Proposition}{Propriété}

% MATHS
%-----------
\renewcommand{\labelitemi}{$\bullet$}
\renewcommand{\labelitemii}{$\circ$}
%----------------------------------------------------------------------------------------
%	Titre
%----------------------------------------------------------------------------------------

\maketitle % Print the title

%-----------------------------------111111111111111111111111111111111111
\section{Tableau de proportionnalité}
%-----------------------------------------------------------------------
\begin{multicols}{2}
  \subsection{Reconnaître une situation de proportionnalité}
  %-----------------------------------------------------------------------

  \begin{Definition}
    Deux quantités sont proportionnelles si on peut passer de l'une à l'autre en multipliant par un même nombre non nul : \textbf{le coefficient de proportionnalité}.
  \end{Definition}

  \subsection{Exploiter une situation de proportionnalité}
  %-----------------------------------------------------------------------

  \begin{Proposition}
    Dans un tableau de proportionnalité, on peut additionner des colonnes et multiplier une colonne par un nombre.
  \end{Proposition}

  \begin{Proposition}
    La représentation graphique de deux quantités proportionnelles est une droite passant par le point 0 de coordonnée (0,0).
  \end{Proposition}

  \subsection{Croissance des bambous}
  %-----------------------------------------------------------------------

  La croissance des bambous est de 7cm par jours. La taille d'un bambou et son âge sont proportionnels.
  \begin{center}
    \begin{tabular}{| l || c | c | c | c | c |}
      \hline			
      Âge du bambou (en jours) & 1 &  2 &  3 & 10 &  25\\
      \hline  
      Taille du bambou (en cm) & 7 & 14 & 21 & 70 & 175\\
      \hline  
    \end{tabular}
  \end{center}

  \begin{figure}[H]
    \centering
    \includegraphics[width=0.7\linewidth]{sources/cours/bambou.pdf}
  \end{figure}
\end{multicols}
%-----------------------------------222222222222222222222222222222222222
\section{Pourcentages}
%-----------------------------------------------------------------------
\begin{multicols}{2}
  \subsection{Appliquer un pourcentage}
  %-----------------------------------------------------------------------

  \begin{Definition}
    Un pourcentage est une fraction avec 100 comme dénominateur. Elle traduit une situation de proportionnalité.
  \end{Definition}

  \paragraph{Exemple}~~\\
  Un magasin de vêtement propose 20\% de réduction sur un article de 72\euro.\\
  $20\% \text{ de } 72 = \dfrac{20}{100} \times 72 = \dfrac{1440}{100} = 14.4$\euro.\\
  Le prix de l'article après la réduction est : 72 - 14.4 = 57.6\euro.

  \subsection{Rechercher un pourcentage}
  %-----------------------------------------------------------------------

  \begin{Definition}
    Pour exprimer une proportion sous forme de pourcentage, on écrit la proportion comme une fraction avec 100 comme dénominateur.
  \end{Definition}

  \paragraph{Exemple}~~\\
  Dans la classe de $5^{e}1$, il y a 25 élèves dont 13 filles et 12 garçons.\\
  $\dfrac{13}{25} = \dfrac{52}{100} = 52 \text{\% de filles et } \dfrac{12}{25} = \dfrac{48}{100} = 48$\% de garçon.
\end{multicols}

\begin{multicols}{2}

  %-----------------------------------333333333333333333333333333333333333
  \section{Echelles et temps}
  %-----------------------------------------------------------------------

  \subsection{Temps}
  %-----------------------------------------------------------------------

  durée en jours $\stackrel{\times 24}{=}$ durée en heures $\stackrel{\times 60}{=}$ durée en minute $\stackrel{\times 60}{=}$ durée en seconde.
  \subsection{Echelles}
  %-----------------------------------------------------------------------

  \begin{Definition}
    L'échelle d'un plan est le quotient à la même unité d'une longueur sur le plan par la longueur réelle qu'elle représente.
  \end{Definition}

  \begin{Proposition}
    Sur un plan, les longueurs mesurés sont proportionnelles au longueurs qu'elles représentent.
  \end{Proposition}

\end{multicols}
\end{document}

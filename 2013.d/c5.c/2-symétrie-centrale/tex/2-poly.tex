%%%%%%%%%%%%%%%%%%%%%%%%%%%%%%%%%%%%%%%%%
% Short Sectioned Assignment
% LaTeX Template
% Version 1.0 (5/5/12)
%
% This template has been downloaded from:
% http://www.LaTeXTemplates.com
%
% Original author:
% Frits Wenneker (http://www.howtotex.com)
%
% License:
% CC BY-NC-SA 3.0 (http://creativecommons.org/licenses/by-nc-sa/3.0/)
%
%%%%%%%%%%%%%%%%%%%%%%%%%%%%%%%%%%%%%%%%%

%----------------------------------------------------------------------------------------
%	PACKAGES AND OTHER DOCUMENT CONFIGURATIONS
%----------------------------------------------------------------------------------------

\documentclass[paper=a4, fontsize=10pt]{scrartcl} % A4 paper and 11pt font size


\usepackage[T1]{fontenc} % Use 8-bit encoding that has 256 glyphs
\usepackage[english,francais]{babel} % Français et anglais
\usepackage[utf8]{inputenc} 

\usepackage{amsmath,amsfonts,amsthm} % Math packages

\usepackage{lmodern}
\usepackage{url}
\usepackage{eurosym} % signe Euros
\usepackage{geometry} % Pour passer au format A4
\geometry{a4paper} % 
\usepackage{graphicx} % Required for including pictures
\usepackage{float} % Allows putting an [H] in \begin{figure} to specify the exact location of the figure

\usepackage{multicol}
\usepackage{sectsty} % Allows customizing section commands
\allsectionsfont{\centering \normalfont\scshape} % Make all sections centered, the default font and small caps

%----------------------------------------------------------------------------------------
%	Pied de Page
%----------------------------------------------------------------------------------------


\usepackage{fancyhdr} % Custom headers and footers
\pagestyle{fancyplain} % Makes all pages in the document conform to the custom headers and footers
\fancyhead{} % No page header - if you want one, create it in the same way as the footers below
\fancyfoot[L]{$5^{e}1$} % Empty left footer
\fancyfoot[C]{Chapitre 1 - Enchaînements d'opérations} % Empty center footer
\fancyfoot[R]{\thepage} % Page numbering for right footer

\renewcommand{\headrulewidth}{0pt} % Remove header underlines
\renewcommand{\footrulewidth}{0pt} % Remove footer underlines

\setlength{\headheight}{13.6pt} % Customize the height of the header


\setlength\parindent{0pt} % Removes all indentation from paragraphs - comment this line for an assignment with lots of text


%----------------------------------------------------------------------------------------
%	Titre
%----------------------------------------------------------------------------------------

\newcommand{\horrule}[1]{\rule{\linewidth}{#1}} % Create horizontal rule command with 1 argument of height


\title{	
  \vspace{-10ex}
  \horrule{0.5pt} \\[0.4cm] % Thin top horizontal rule
  \huge Chapitre 2 - Symétrie centrale\\ % The assignment title
  \horrule{2pt} \\[0.5cm] % Thick bottom horizontal rule
}

\author{}
\date{\vspace{-10ex}} % Today's date or a custom date

%----------------------------------------------------------------------------------------
%	Début du document
%----------------------------------------------------------------------------------------

\begin{document}

%----------------------------------------------------------------------------------------
% RE-DEFINITION
%----------------------------------------------------------------------------------------
% MATHS
%-----------

\newtheorem{Definition}{Définition}
\newtheorem{Theorem}{Théorème}
\newtheorem{Proposition}{Proposition}

% MATHS
%-----------
\renewcommand{\labelitemi}{$\bullet$}
\renewcommand{\labelitemii}{$\circ$}
%----------------------------------------------------------------------------------------
%	Titre
%----------------------------------------------------------------------------------------

\maketitle % Print the title

%-----------------------------------111111111111111111111111111111111111
\section{Le symétrique d'un point}
%----------------------------------------------------------------------

\begin{Definition}{Symétrique d'un point}\\
  \label{def:ch2-1.sympt}
  $M^{'}$ est le symétrique du point M par rapport au centre O est équivalent à O est le milieu de de [M $M^{'}$].
\end{Definition}

%------------------------------------22222222222222222222222222222222222
\subsection{Construction}
%----------------------------------------------------------------------

\subsubsection{À l'aide d'un quadrillage}

\begin{figure}[H]
  \centering
  \includegraphics[width=.5\linewidth]{sources/cours/quad.pdf}
  \label{fig:ch2-quad}
\end{figure}

Pour construire le symétrique d'un point à l'aide d'un quadrillage, on se déplace d'autant de carreaux horizontalement (violet) et verticalement (bleu) par rapport au centre.

\subsubsection{Sans l'aide d'un quadriallge}

\begin{figure}[H]
  \centering
  \includegraphics[width=.5\linewidth]{sources/cours/sansquad.pdf}
  \label{fig:ch2-sansquad}
\end{figure}

Pour construire le symétrique d'un point sans l'aide d'un quadrillage, on s'arme d'une règle et d'un compas. 
\begin{enumerate}
\item On trace la droite (OM) (bleue) car les points O, M et M' sont alignés.
\item On marque au compas la longueur OM avec un arc de cercle. 
\end{enumerate}
Le point M' est à l'intersection de la droite et de l'arc de cercle.

%-----------------------------------222222222222222222222222222222
\section{Le symétrique d'objets simples}
%-----------------------------------------------------------------

%-----------------------------------------------------------------
\subsection{Le symétrique d'une droite}
Le symétrique d'une droite par rapport à un point est une autre droite. Elles sont parallèles.
\begin{figure}[H]
  \centering
  \includegraphics[width=.6\linewidth]{sources/cours/co_sym_d.pdf}
  \caption{Symétrique de la droite (AB) par rapport à O}
  \label{fig:ch2-symd}
\end{figure}


%-----------------------------------------------------------------
\subsection{Le symétrique d'un segment}
Le symétrique d'un segment par rapport à un point est un autre segment. Ils sont parallèles et de même longueurs.

\begin{figure}[H]
  \centering
  \includegraphics[width=.6\linewidth]{sources/cours/co_sym_s.pdf}
  \caption{Symétrique du segment [AB] par rapport à O}
  \label{fig:ch2-syms}
\end{figure}

%-----------------------------------------------------------------
\subsection{Le symétrique d'un cercle}
Le symétrique d'un cercle par rapport à un point est un autre cercle. Le centre du cercle dont les centres sont symétriques. Ils ont même rayon.

\begin{figure}[H]
  \centering
  \includegraphics[width=.6\linewidth]{sources/cours/co_sym_c.pdf}
  \caption{Symétrique du cercle de centre C et de rayon [CA] par rapport à O}
  \label{fig:ch2-symc}
\end{figure}

%-----------------------------------222222222222222222222222222222
\section{Propriétés}
%-----------------------------------------------------------------

\begin{enumerate}
\item Deux figures symétriques sont superposables.\\
  Il s'agit d'une rotation à 180\degre suivant le centre de symétrie.\\
\item La symétrie centrale conserve les longueurs et les mesures d'angles.\\
\item La symétrie centrale conserve les périmètres, les aires et les alignements.\\
\end{enumerate}

\end{document}

%%%%%%%%%%%%%%%%%%%%%%%%%%%%%%%%%%%%%%%%%
% Short Sectioned Assignment
% LaTeX Template
% Version 1.0 (5/5/12)
%
% This template has been downloaded from:
% http://www.LaTeXTemplates.com
%
% Original author:
% Frits Wenneker (http://www.howtotex.com)
%
% License:
% CC BY-NC-SA 3.0 (http://creativecommons.org/licenses/by-nc-sa/3.0/)
%
%%%%%%%%%%%%%%%%%%%%%%%%%%%%%%%%%%%%%%%%%

%----------------------------------------------------------------------------------------
%	PACKAGES AND OTHER DOCUMENT CONFIGURATIONS
%----------------------------------------------------------------------------------------
\documentclass[11pt]{article}
\usepackage{geometry} % Pour passer au format A4
\geometry{hmargin=1cm, vmargin=1cm} % 


\usepackage[T1]{fontenc} % Use 8-bit encoding that has 256 glyphs
\usepackage[english,francais]{babel} % Français et anglais
\usepackage[utf8]{inputenc} 

\usepackage{amsmath,amsfonts,amsthm} % Math packages

\usepackage{lmodern}
\usepackage{url}
\usepackage{eurosym} % signe Euros
\usepackage{geometry} % Pour passer au format A4
\geometry{a4paper} % 
\usepackage{graphicx} % Required for including pictures
\usepackage{float} % Allows putting an [H] in \begin{figure} to specify the exact location of the figure

\usepackage{multicol}
\usepackage{sectsty} % Allows customizing section commands
\allsectionsfont{\centering \normalfont\scshape} % Make all sections centered, the default font and small caps

%----------------------------------------------------------------------------------------
%	Pied de Page
%----------------------------------------------------------------------------------------

\setlength\parindent{0pt} % Removes all indentation from paragraphs - comment this line for an assignment with lots of text


%----------------------------------------------------------------------------------------
%	Titre
%----------------------------------------------------------------------------------------

\newcommand{\horrule}[1]{\rule{\linewidth}{#1}} % Create horizontal rule command with 1 argument of height


%----------------------------------------------------------------------------------------
%	Début du document
%----------------------------------------------------------------------------------------


\title{Enchaînements d'opérations} % Title
\author{$5^e 1$}
\date{17 Septembre 2013} % Date for the report
\begin{document}

%\maketitle % Insert the title, author and date

\begin{center}
  \vspace*{-2cm}
  \textsf{--}\\
  \textsf{Démocratie : l’oppression du peuple par le peuple pour le peuple.}
  \texttt{Oscar Wilde}\\
  \textsf{--}
\end{center}
%----------------------------------------------------------------------------------------
%	SECTION 1
%----------------------------------------------------------------------------------------

\thispagestyle{empty}

\section{Leçon}
\textbf{Écrire la définition du symétrique d'un point.}

%----------------------------------------------------------------------------------------
%	SECTION 2
%----------------------------------------------------------------------------------------

\section{Symétrique d'un point}
\textbf{Remplir les trous.}

\begin{figure}[H]
  \centering
  \includegraphics[width=0.8\linewidth]{sources/ie/ie_1.pdf}
\end{figure}

\begin{enumerate}
\item Le point \textbf{J} est le symétrique du point \textbf{U} par rapport à \underline{\phantom{123456}}.
\item Le point \textbf{V} est le symétrique du point \underline{\phantom{123456}} par rapport à  \textbf{Y}.
\item Le point \underline{\phantom{123456}} est le symétrique du point \textbf{K} par rapport à  \textbf{V}.
\end{enumerate}


%----------------------------------------------------------------------------------------
%	SECTION 3
%----------------------------------------------------------------------------------------
\section{Symétrique d'une figure}
\begin{multicols}{2}
  \begin{enumerate}
  \item \textbf{Entourer la lettre des figures possédant un centre de symétrie.}

    \begin{figure}[H]
      \centering
      \includegraphics[width=0.8\linewidth]{sources/ie/ie_2.pdf}
    \end{figure}

  \item \textbf{Compléter la figure au minimum en noircissant les cases au crayon à papier pour que la figure possède O comme centre de symétrie.}

    \begin{figure}[H]
      \centering
      \includegraphics[width=\linewidth]{sources/ie/ie_3.pdf}
    \end{figure}

  \item \textbf{Tracer à l'aide du quadrillage le symétrique de la figure par rapport au point $O_1$.  Puis tracer le symétrique de la figure par rapport à $O_2$.}

    \begin{figure}[H]
      \centering
      \includegraphics[width=\linewidth]{sources/ie/ie_4.pdf}
    \end{figure}
  \end{enumerate}
\end{multicols}
\end{document}







\documentclass{beamer}

\usepackage{geometry} % Pour passer au format A4
\usepackage{graphicx} % Required for including pictures
\usepackage{float} % 

\usepackage{amsmath,amsfonts,amssymb,amsthm}
\usepackage[T1]{fontenc} 
\usepackage[english,francais]{babel}
\usepackage[utf8]{inputenc}
\usepackage{lmodern}

\usetheme{Warsaw}

\title{Découverte des nombres relatifs}
\author{$5^{e}1$}

\begin{document}

\frame{\titlepage}

\begin{frame}
\frametitle{77p90 - 1)}
  \begin{figure}[H]
    \centering
    \includegraphics[width=.8\linewidth]{sources/ie/77-0.pdf}
  \end{figure}
\end{frame}

\begin{frame}
\frametitle{77p90 - 2)}
  \begin{figure}[H]
    \centering
    \includegraphics[width=.8\linewidth]{sources/ie/77-1.pdf}
  \end{figure}
\end{frame}

\begin{frame}
\frametitle{77p90 - 3)}
  \begin{figure}[H]
    \centering
    \includegraphics[width=.8\linewidth]{sources/ie/77-2.pdf}
  \end{figure}
\end{frame}

\begin{frame}
\frametitle{77p90 - 4)}
  \begin{figure}[H]
    \centering
    \includegraphics[width=.8\linewidth]{sources/ie/77-3.pdf}
  \end{figure}
\end{frame}

\begin{frame}
\frametitle{77p90 - 5)}
  \begin{figure}[H]
    \centering
    \includegraphics[width=.8\linewidth]{sources/ie/77-4.pdf}
  \end{figure}
\end{frame}
\end{document}

%%%%%%%%%%%%%%%%%%%%%%%%%%%%%%%%%%%%%%%%%
% Short Sectioned Assignment
% LaTeX Template
% Version 1.0 (5/5/12)
%
% This template has been downloaded from:
% http://www.LaTeXTemplates.com
%
% Original author:
% Frits Wenneker (http://www.howtotex.com)
%
% License:
% CC BY-NC-SA 3.0 (http://creativecommons.org/licenses/by-nc-sa/3.0/)
%
%%%%%%%%%%%%%%%%%%%%%%%%%%%%%%%%%%%%%%%%%

%----------------------------------------------------------------------------------------
%	PACKAGES AND OTHER DOCUMENT CONFIGURATIONS
%----------------------------------------------------------------------------------------
\documentclass[11pt]{article}
\usepackage{geometry} % Pour passer au format A4
\geometry{hmargin=1cm, vmargin=0.8cm} % 


\usepackage[T1]{fontenc} % Use 8-bit encoding that has 256 glyphs
\usepackage[english,francais]{babel} % Français et anglais
\usepackage[utf8]{inputenc} 

\usepackage{amsmath,amsfonts,amsthm} % Math packages

\usepackage{lmodern}
\usepackage{url}
\usepackage{eurosym} % signe Euros
\usepackage{geometry} % Pour passer au format A4
\geometry{a4paper} % 
\usepackage{graphicx} % Required for including pictures
\usepackage{float} % Allows putting an [H] in \begin{figure} to specify the exact location of the figure

\usepackage{multicol}
\usepackage{sectsty} % Allows customizing section commands
\allsectionsfont{\centering \normalfont\scshape} % Make all sections centered, the default font and small caps

%----------------------------------------------------------------------------------------
%	Pied de Page
%----------------------------------------------------------------------------------------

\setlength\parindent{0pt} % Removes all indentation from paragraphs - comment this line for an assignment with lots of text


%----------------------------------------------------------------------------------------
%	Titre
%----------------------------------------------------------------------------------------

\newcommand{\horrule}[1]{\rule{\linewidth}{#1}} % Create horizontal rule command with 1 argument of height


%----------------------------------------------------------------------------------------
%	Début du document
%----------------------------------------------------------------------------------------

\title{Nombres relatifs} % Title
\author{$5^e 1$}
\date{8 Avril 2014} % Date for the report
\begin{document}

%\maketitle % Insert the title, author and date

%----------------------------------------------------------------------------------------
%	SECTION 1
%----------------------------------------------------------------------------------------
\thispagestyle{empty}

\setlength{\columnseprule}{1pt}
\begin{multicols}{4}

  \begin{figure}[H]
    \centering
    \includegraphics[width=.4\linewidth]{sources/exo/thermo-1.pdf}
    \caption{T = }
  \end{figure}

  \begin{figure}[H]
    \centering
    \includegraphics[width=.4\linewidth]{sources/exo/thermo-2.pdf}
    \caption{T = }
  \end{figure}

  \begin{figure}[H]
    \centering
    \includegraphics[width=.4\linewidth]{sources/exo/thermo-0.pdf}
    \caption{T = 22 \char6 C}  
  \end{figure}

  \begin{figure}[H]
    \centering
    \includegraphics[width=.4\linewidth]{sources/exo/thermo-0.pdf}
    \caption{T = -4 \char6 C}  
  \end{figure}
\end{multicols}

\begin{multicols}{4}
  \begin{figure}[H]
    \centering
    \includegraphics[width=.4\linewidth]{sources/exo/thermo-1.pdf}
    \caption{T = }
  \end{figure}

  \begin{figure}[H]
    \centering
    \includegraphics[width=.4\linewidth]{sources/exo/thermo-2.pdf}
    \caption{T = }
  \end{figure}

  \begin{figure}[H]
    \centering
    \includegraphics[width=.4\linewidth]{sources/exo/thermo-0.pdf}
    \caption{T = 22 \char6 C}  
  \end{figure}

  \begin{figure}[H]
    \centering
    \includegraphics[width=.4\linewidth]{sources/exo/thermo-0.pdf}
    \caption{T = -4 \char6 C}  
  \end{figure}
\end{multicols}

\begin{multicols}{4}
  \begin{figure}[H]
    \centering
    \includegraphics[width=.4\linewidth]{sources/exo/thermo-3.pdf}
    \caption{T = }
  \end{figure}

  \begin{figure}[H]
    \centering
    \includegraphics[width=.4\linewidth]{sources/exo/thermo-4.pdf}
    \caption{T = }
  \end{figure}

  \begin{figure}[H]
    \centering
    \includegraphics[width=.4\linewidth]{sources/exo/thermo-0.pdf}
    \caption{T = 22 \char6 C}  
  \end{figure}

  \begin{figure}[H]
    \centering
    \includegraphics[width=.4\linewidth]{sources/exo/thermo-0.pdf}
    \caption{T = -4 \char6 C}  
  \end{figure}

\end{multicols}

\end{document}

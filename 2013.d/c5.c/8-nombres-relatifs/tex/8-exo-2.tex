%%%%%%%%%%%%%%%%%%%%%%%%%%%%%%%%%%%%%%%%%
% LaTeX Template
% http://www.LaTeXTemplates.com
%
% Original author:
% Linux and Unix Users Group at Virginia Tech Wiki 
% (https://vtluug.org/wiki/Example_LaTeX_chem_lab_report)
%
% License:
% CC BY-NC-SA 3.0 (http://creativecommons.org/licenses/by-nc-sa/3.0/)
%
%%%%%%%%%%%%%%%%%%%%%%%%%%%%%%%%%%%%%%%%%

%----------------------------------------------------------------------------------------
%	PACKAGES AND DOCUMENT CONFIGURATIONS
%----------------------------------------------------------------------------------------

\documentclass[11pt]{article}
\usepackage{geometry} % Pour passer au format A4
\geometry{hmargin=1cm, vmargin=1cm} % 

\usepackage{graphicx} % Required for including pictures
\usepackage{float} % 

%Français
\usepackage[T1]{fontenc} 
\usepackage[english,francais]{babel}
\usepackage[utf8]{inputenc}
\usepackage{eurosym}
\usepackage{lmodern}
\usepackage{url}
\usepackage{multicol}

%Maths
\usepackage{amsmath,amsfonts,amssymb,amsthm}
%\usepackage[linesnumbered, ruled, vlined]{algorithm2e}
%\SetAlFnt{\small\sffamily}

%Autres
\linespread{1} % Line spacing
\setlength\parindent{0pt} % Removes all indentation from paragraphs

\renewcommand{\labelenumi}{\alph{enumi}.} % 
\pagestyle{empty}
%----------------------------------------------------------------------------------------
%	DOCUMENT INFORMATION
%----------------------------------------------------------------------------------------
\begin{document}

%\maketitle % Insert the title, author and date

\begin{minipage}[t]{\textwidth}
\raggedright
{\bfseries Nom, Prénom : \underline{\phantom{123456789123456789}}}\\
{\bfseries $5^{e}1$}\\[.35ex]
\vspace*{-1cm}
\raggedleft
{\bfseries Découverte des relatifs}\\[.35ex]
{\bfseries 22 avril 2014}\\[.35ex]
\end{minipage}\\[1em]

\begin{center}
  \textsf{--}\\
  \textsf{Les mathématiques consistent à prouver des choses évidentes par des moyens complexes.}
  \texttt{George Polya}\\
  \textsf{--}
\end{center}

\section{Toucher - Couler}

\setlength{\columnseprule}{1pt}
\begin{multicols}{2}
\begin{figure}[H]
  \centering
  \includegraphics[width=\linewidth]{sources/exo/tc-0.pdf}
  \caption{Votre plateau}
\end{figure}

\begin{figure}[H]
  \centering
  \includegraphics[width=\linewidth]{sources/exo/tc-1.pdf}
  \caption{Plateau de l'adversaire}
\end{figure}
\end{multicols}  

\subsection{Mise en place}
\begin{enumerate}
\item Placer horizontalement ou verticalement sur votre plateau  :
\begin{itemize}
\item Un porte-avion de taille 5.
\item Un zodiac de taille 2.
\item Deux frégates de taille 3.
\end{itemize}

\item Donner les coordonnées de toutes les cases composants chacun de vos navires.
\end{enumerate}

\subsection{Destruction}
\begin{enumerate}
\item Donner les coordonnées de 4 tirs permettant de toucher une fois les 4 bateaux.
\item Donner les coordonnées de 8 tirs permettant de couler le deux frégates.
\end{enumerate}






\end{document}

%%%%%%%%%%%%%%%%%%%%%%%%%%%%%%%%%%%%%%%%%
% LaTeX Template
% http://www.LaTeXTemplates.com
%
% Original author:
% Linux and Unix Users Group at Virginia Tech Wiki 
% (https://vtluug.org/wiki/Example_LaTeX_chem_lab_report)
%
% License:
% CC BY-NC-SA 3.0 (http://creativecommons.org/licenses/by-nc-sa/3.0/)
%
%%%%%%%%%%%%%%%%%%%%%%%%%%%%%%%%%%%%%%%%%

%----------------------------------------------------------------------------------------
%	PACKAGES AND DOCUMENT CONFIGURATIONS
%----------------------------------------------------------------------------------------

\documentclass[11pt]{article}
\usepackage{geometry} % Pour passer au format A4
\geometry{hmargin=1cm, vmargin=1cm} % 

\usepackage{graphicx} % Required for including pictures
\usepackage{float} % 

%Français
\usepackage[T1]{fontenc} 
\usepackage[english,francais]{babel}
\usepackage[utf8]{inputenc}
\usepackage{eurosym}
\usepackage{lmodern}
\usepackage{url}
\usepackage{multicol}

%Maths
\usepackage{amsmath,amsfonts,amssymb,amsthm}
%\usepackage[linesnumbered, ruled, vlined]{algorithm2e}
%\SetAlFnt{\small\sffamily}

%Autres
\linespread{1} % Line spacing
\setlength\parindent{0pt} % Removes all indentation from paragraphs

\renewcommand{\labelenumi}{\alph{enumi}.} % 
\pagestyle{empty}
%----------------------------------------------------------------------------------------
%	DOCUMENT INFORMATION
%----------------------------------------------------------------------------------------
\begin{document}

%\maketitle % Insert the title, author and date

\begin{minipage}[t]{\textwidth}
\raggedright
{\bfseries Nom, Prénom : \underline{\phantom{123456789123456789}}}\\
{\bfseries $5^{e}1$}\\[.35ex]
\vspace*{-1cm}
\raggedleft
{\bfseries Découverte des relatifs}\\[.35ex]
{\bfseries 22 avril 2014}\\[.35ex]
\end{minipage}\\[1em]

\begin{center}
  \textsf{Les mathématiques consistent à prouver des choses évidentes par des moyens complexes.}
  \texttt{George Polya}\\
\end{center}

\vspace{-1cm}
\subsection*{1 - It's time}


\begin{figure}[H]
  \centering
  \includegraphics[width=\linewidth]{sources/ie/doctor.pdf}
\end{figure}

\begin{enumerate}
\item[1] Graduer la frise de -3000 ans avant JV à 3000 ans après JC. \textbf{Placer une graduation tous les 500 ans}.
\item[2] Placer les différents évènements sur la frise chronologique :
\begin{multicols}{2}
  \begin{itemize}
  \item D : \textbf{476} : Dispartion de l'empire Romain.
  \item C : \textbf{-2300} : Civilisation Olmèque au Mexique.
  \item R : \textbf{-1279} : Ramsès II monte sur le Trône.	  
  \item V : \textbf{-52} : Vercingétorix est vaincu à Alésia par Jules César.
  \item M : \textbf{2014} : Maintenant, vous profitez d'un superbe sours de Mathématiques.
  \item N : \textbf{1687} : Isaac Newton prend une pomme sur la tête.
  \end{itemize}
\end{multicols}  
\end{enumerate}

\subsection*{2 - Toucher / Couler}

\setlength{\columnseprule}{1pt}
\begin{multicols}{2}
\begin{figure}[H]
  \centering
  \includegraphics[width=0.8\linewidth]{sources/exo/tc-0.pdf}
  \caption{Votre plateau}
\end{figure}

\begin{figure}[H]
  \centering
  \includegraphics[width=0.8\linewidth]{sources/exo/tc-1.pdf}
  \caption{Plateau de l'adversaire}
\end{figure}
\end{multicols}  

\subsubsection*{1 - Mise en place : une graduation = 10m}
\begin{enumerate}
\item Placer horizontalement ou verticalement sur votre plateau : un porte-avion de taille \textbf{5}, un zodiac de taille \textbf{2} et \textbf{2} frégates de taille \textbf{3}.
\item Donner les coordonnées de toutes les cases composant chacun de vos navires.
\end{enumerate}

\subsubsection*{2 - Destruction  : une graduation = 10m}
Grâce à Turing et sa machine enigma vous avez réussi à décoder les plans adverses. Sur le plateau de l'adversaire : 
\begin{enumerate}
\item Donner les coordonnées de 4 tirs permettant de toucher chacun des 4 bateaux.
\item Donner les coordonnées de 6 tirs permettant de couler (toucher toutes les cases) les deux frégates.
\end{enumerate}

\end{document}

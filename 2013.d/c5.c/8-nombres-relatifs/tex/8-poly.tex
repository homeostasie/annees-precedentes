%%%%%%%%%%%%%%%%%%%%%%%%%%%%%%%%%%%%%%%%%
% Short Sectioned Assignment
% LaTeX Template
% Version 1.0 (5/5/12)
%
% This template has been downloaded from:
% http://www.LaTeXTemplates.com
%
% Original author:
% Frits Wenneker (http://www.howtotex.com)
%
% License:
% CC BY-NC-SA 3.0 (http://creativecommons.org/licenses/by-nc-sa/3.0/)
%
%%%%%%%%%%%%%%%%%%%%%%%%%%%%%%%%%%%%%%%%%

%----------------------------------------------------------------------------------------
%	PACKAGES AND OTHER DOCUMENT CONFIGURATIONS
%----------------------------------------------------------------------------------------

\documentclass[paper=a4, fontsize=10pt]{scrartcl} % A4 paper and 11pt font size


\usepackage[T1]{fontenc} % Use 8-bit encoding that has 256 glyphs
\usepackage[english,francais]{babel} % Français et anglais
\usepackage[utf8]{inputenc} 

\usepackage{amsmath,amsfonts,amsthm} % Math packages

\usepackage{lmodern}
\usepackage{url}
\usepackage{eurosym} % signe Euros
\usepackage{geometry} % Pour passer au format A4
\geometry{a4paper} % 
\usepackage{graphicx} % Required for including pictures
\usepackage{float} % Allows putting an [H] in \begin{figure} to specify the exact location of the figure

\usepackage{multicol}
\usepackage{sectsty} % Allows customizing section commands
\allsectionsfont{\centering \normalfont\scshape} % Make all sections centered, the default font and small caps

%----------------------------------------------------------------------------------------
%	Pied de Page
%----------------------------------------------------------------------------------------


\usepackage{fancyhdr} % Custom headers and footers
\pagestyle{fancyplain} % Makes all pages in the document conform to the custom headers and footers
\fancyhead{} % No page header - if you want one, create it in the same way as the footers below
\fancyfoot[L]{$5^{e}1$} % Empty left footer
\fancyfoot[C]{Chapitre 8 - Découverte des nombres relatifs} % Empty center footer
\fancyfoot[R]{\thepage} % Page numbering for right footer

\renewcommand{\headrulewidth}{0pt} % Remove header underlines
\renewcommand{\footrulewidth}{0pt} % Remove footer underlines

\setlength{\headheight}{13.6pt} % Customize the height of the header


\setlength\parindent{0pt} % Removes all indentation from paragraphs - comment this line for an assignment with lots of text


%----------------------------------------------------------------------------------------
%	Titre
%----------------------------------------------------------------------------------------

\newcommand{\horrule}[1]{\rule{\linewidth}{#1}} % Create horizontal rule command with 1 argument of height


\title{	
  \vspace{-10ex}
  \horrule{0.5pt} \\[0.4cm] % Thin top horizontal rule
  \huge Chapitre 8 - Découverte des nombres relatifs\\ % The assignment title
  \horrule{2pt} \\[0.5cm] % Thick bottom horizontal rule
}

\author{}
\date{\vspace{-10ex}} % Today's date or a custom date

%----------------------------------------------------------------------------------------
%	Début du document
%----------------------------------------------------------------------------------------

\begin{document}

%----------------------------------------------------------------------------------------
% RE-DEFINITION
%----------------------------------------------------------------------------------------
% MATHS
%-----------

\newtheorem{Definition}{Définition}
\newtheorem{Theorem}{Théorème}
\newtheorem{Proposition}{Proposition}

% MATHS
%-----------
\renewcommand{\labelitemi}{$\bullet$}
\renewcommand{\labelitemii}{$\circ$}
%----------------------------------------------------------------------------------------
%	Titre
%----------------------------------------------------------------------------------------

\maketitle % Print the title



%-----------------------------------111111111111111111111111111111111111
\section{Notations et Vocabulaires}
%----------------------------------------------------------------------
\begin{multicols}{2}
  \begin{figure}[H]
    \centering
    \includegraphics[width=.8\linewidth]{sources/cours/axe-g.pdf}
  \end{figure}

  \begin{itemize}
  \item Un nombre supérieur à 0 est un nombre positif :  
  \item Un nombre inférieur à 0 est un nombre négatif :\\
    On le note avec le signe \textbf{-} devant.
  \item 0 est à la fois négatif et positif.
  \end{itemize}

  \paragraph{Remarque: }L'ensemble des nombres positifs et des nombres relatifs forme l'ensemble des nombres relatifs.


  \begin{Definition}{L'opposé d'un nombre}\\
    Deux nombres sont opposés s'ils ne diffèrent que du signe \textbf{-}.
  \end{Definition}

  \paragraph{Exemples : }L'opposé de 102 est -102 et l'opposé de -42 est 42.
\end{multicols}
%-----------------------------------222222222222222222222222222222222222
\section{Repérer sur une droite graduée}
%----------------------------------------------------------------------
\begin{multicols}{2}
  \begin{figure}[H]
    \centering
    \includegraphics[width=.8\linewidth]{sources/cours/axe-g-1.pdf}
  \end{figure}

  \begin{Definition}{Abscisse d'un nombre}\\

    On repère un point sur un axe horizontal à l'aide du nombre situé au niveau de sa graduation. Ce nombre représente son \textbf{abscisse}.
  \end{Definition}

  \paragraph{Remarques: }
  \begin{itemize}
  \item On dit que A est le point d'abscisse 70.
  \item Sur un axe gradué, on appelle 0 l'\textbf{origine}.
  \end{itemize}
\end{multicols}

%-----------------------------------333333333333333333333333333333333333
\section{Comparer}
%----------------------------------------------------------------------

\begin{Definition}{Comparer deux nombres}\\
  Comparer deux nombres revient à dire lequel des deux est le plus grand.\\
  \begin{itemize}
  \item Plus grand $>$ Plus petit
  \item Plus petit $<$ Plus grand
  \end{itemize}
\end{Definition}
\begin{Proposition}
  Sur un axe gradué de la gauche vers la droite, les nombres lus dans ce même sens sont classés du plus petit au plus grand.\\
\end{Proposition}

\newpage
\begin{figure}[H]
  \centering
  \includegraphics[width=0.8\linewidth]{sources/cours/axe-g-2.pdf}
\end{figure}
... $< -6 < -4.5 < -3 < -1.5 < 0 < 1.5 < 3 < 4.5 < 6 < 7.5 < 9 < 10.5 < 12 < 13.5 < 15 <$ ... \\
\hrule
\begin{multicols}{2}
  \begin{figure}[H]
    \centering
    \includegraphics[width=.6\linewidth]{sources/cours/axe-g-3.pdf}
  \end{figure}
  $-7.2 < -7.1 < -7.01 < 7 < -6.99 < -6.9$
\end{multicols} 
%-----------------------------------444444444444444444444444444444444444
\section{Repérer dans le plan}
%----------------------------------------------------------------------
\begin{multicols}{2}
  \begin{figure}[H]
    \centering
    \includegraphics[width=\linewidth]{sources/cours/repere.pdf}
  \end{figure}

  L'abscisse du point O est 2 et son ordonnée 0.\\
  L'abscisse du point A est 2 et son ordonnée 1.\\
  L'abscisse du point B est -1 et son ordonnée 3.\\
  L'abscisse du point C est 2.5 et son ordonnée -1.\\
  L'abscisse du point D est 2 et son ordonnée -2.\\
\end{multicols}

\begin{Definition}{Repère dans le plan}\\
  On définit un repère dans un plan à partir d'\textbf{un point d'origine} et de \textbf{deux axes} :
  \begin{itemize}
  \item L'axe horizontal est l'axe des \textbf{abscisses}.
  \item L'axe vertical est l'axe des \textbf{ordonnées}.
  \item Le point d'intersection des deux axes s'appelle l'\textbf{origine}.
  \end{itemize}
\end{Definition}

\begin{Definition}{Point dans un repère du plan}\\
  Pour repérer un point dans le plan, on utilise des coordonnées.
  \begin{itemize}
  \item L'\textbf{abscisse} est la première coordonnée situe le point sur l'axe horizontal.
  \item L'\textbf{ordonnée} est la deuxième coordonnée situe le point sur l'axe vertical.
  \end{itemize}

\end{Definition}

\end{document}

\documentclass{beamer}

\usepackage{geometry} % Pour passer au format A4
\usepackage{graphicx} % Required for including pictures
\usepackage{float} % 

\usepackage{amsmath,amsfonts,amssymb,amsthm}
\usepackage[T1]{fontenc} 
\usepackage[english,francais]{babel}
\usepackage[utf8]{inputenc}
\usepackage{lmodern}

\usetheme{Warsaw}

\title{Découverte des nombres relatifs}
\author{$5^{e}1$}

\begin{document}

\frame{\titlepage}

%-----------------------------------111111111111111111111111111111111111
\section{Notations et Vocabulaires}
%----------------------------------------------------------------------


\frame{\tableofcontents[sectionstyle=show/shaded, subsectionstyle=show/shaded]}

\begin{frame}
  1. Notations et Vocabulaires

  \begin{figure}[H]
    \centering
    \includegraphics[width=.8\linewidth]{sources/cours/axe-g.pdf}
  \end{figure}

  \begin{alertblock}{Définition :}	
    \begin{itemize}
    \item Un nombre supérieur à 0 est un nombre positif :  
    \item Un nombre inférieur à 0 est un nombre négatif :\\
      On le note avec le signe \textbf{-} devant.
    \item 0 est à la fois négatif et positif.
    \end{itemize}
  \end{alertblock}
  
  \begin{exampleblock}{Remarque}
    L'ensemble des nombres positifs et des nombres relatifs forme l'ensemble des nombres relatifs.
  \end{exampleblock}
\end{frame}

\begin{frame}
  \only<1->
      {
        \begin{block}{Utilisation :}	
          On peut trouver des nombres négatifs dans les situations suivantes :\\
          Température l'hiver, frise chronologique, étage d'un ascenseur...
        \end{block}
      }
      \only<2>
          {
            \begin{alertblock}{Définition : l'opposé d'un nombre}	
              Deux nombres sont opposés s'ils ne diffèrent que du signe \textbf{-}.
            \end{alertblock}

            \begin{exampleblock}{Exemples}
	      \begin{itemize}
              \item L'opposé de 102 est -102.
	      \item L'opposé de -42 est 42.
	      \end{itemize}
            \end{exampleblock}
          }
\end{frame}

%-----------------------------------111111111111111111111111111111111111
\section{Repérer sur une droite graduée}
%----------------------------------------------------------------------
\frame{\tableofcontents[sectionstyle=show/shaded, subsectionstyle=show/shaded]}

\begin{frame}
  2. Repérer sur une droite graduée

  \begin{figure}[h!]
    \centering
    \includegraphics[width=.6\linewidth]{sources/cours/axe-g-2.pdf}
  \end{figure}

  \begin{alertblock}{Définition : Abscisse d'un nombre}	
    On repère un point sur un axe horizontal à l'aide du nombre situé au niveau de sa graduation. Ce nombre représente son \textbf{abscisse}.
  \end{alertblock}

  \begin{exampleblock}{Remarques :}	
    \begin{itemize}
    \item A est le point d'abscisse 6.
    \item A est le point d'abscisse -1.5.
    \item Sur un axe gradué, on appelle 0 l'\textbf{origine}.
    \end{itemize}
  \end{exampleblock}
\end{frame}

%-----------------------------------111111111111111111111111111111111111
\section{Comparer}
%----------------------------------------------------------------------
\frame{\tableofcontents[sectionstyle=show/shaded, subsectionstyle=show/shaded]}

\begin{frame}
  3. Comparer
  \only<1>
      {
        \begin{block}{Rappel : Comparaison}	
          Comparer deux nombres revient à dire lequel des deux est le plus grand.\\
          \begin{itemize}
          \item Plus grand $>$ Plus petit
          \item Plus petit $<$ Plus grand
          \end{itemize}
        \end{block}
      }
      \begin{alertblock}{Proposition}	
        Sur un axe gradué de la gauche vers la droite, les nombres lus dans le même sens sont classés du plus petit au plus grand.
      \end{alertblock}
      \only<2>{
        \begin{block}{Exemples}	
          \begin{figure}[h!]
            \centering
            \includegraphics[width=.6\linewidth]{sources/cours/axe-g-2.pdf}
          \end{figure}
          $$-6 < -4.5 < -3 < -1.5 < 0 < 1.5 < 3 < 4.5 < 6 < 7.5 < 9$$
          
          \begin{columns}[t]
            \begin{column}{4cm}
              \begin{figure}[h!]
                \centering
                \includegraphics[width=\linewidth]{sources/cours/axe-g-3.pdf}
              \end{figure}
            \end{column}
            \begin{column}{6cm}
              $$-7.2 < -7.1 < -7.01 < 7 < -6.99$$
            \end{column}
          \end{columns} 

          
        \end{block}
      }
\end{frame}


%-----------------------------------111111111111111111111111111111111111
\section{Repérer dans le plan}
%----------------------------------------------------------------------
\frame{\tableofcontents[sectionstyle=show/shaded, subsectionstyle=show/shaded]}

\begin{frame}
  4. Repérer dans le plan
  \begin{alertblock}{Définition : Repère dans le plan}	
    On définit un repère dans un plan à partir d'\textbf{un point d'origine} et de \textbf{deux axes} :
    \begin{itemize}
    \item L'axe horizontal est l'axe des \textbf{abscisses}.
    \item L'axe vertical est l'axe des \textbf{ordonnées}.
    \item Le point d'intersection des deux axes s'appelle l'\textbf{origine}.
    \end{itemize}
  \end{alertblock}
  
  \begin{alertblock}{Définition : Point dans un repère du plan}	
    Pour repérer un point dans le plan, on utilise des coordonnées :
    \begin{itemize}
    \item L'\textbf{abscisse} est la première coordonnée situe le point sur l'axe horizontal.
    \item L'\textbf{ordonnée} est la deuxième coordonnée situe le point sur l'axe vertical.
    \end{itemize}
  \end{alertblock}
\end{frame}

\begin{frame}
  \begin{columns}[t]
    \begin{column}{5cm}
      \begin{figure}[h!]
        \centering
        \includegraphics[width=\linewidth]{sources/cours/repere.pdf}
      \end{figure}
    \end{column}
    \begin{column}{5cm}
      \begin{block}{Exemples}	
        L'abscisse du point O est 0 et son ordonnée 0.\\
        L'abscisse du point A est 2 et son ordonnée 1.\\
        L'abscisse du point B est -1 et son ordonnée 3.\\
        L'abscisse du point C est 2.5 et son ordonnée -1.\\
        L'abscisse du point D est 2 et son ordonnée -2.\\

      \end{block}
    \end{column}
  \end{columns} 


\end{frame}

\end{document}

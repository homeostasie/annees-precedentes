%%%%%%%%%%%%%%%%%%%%%%%%%%%%%%%%%%%%%%%%%
% Short Sectioned Assignment
% LaTeX Template
% Version 1.0 (5/5/12)
%
% This template has been downloaded from:
% http://www.LaTeXTemplates.com
%
% Original author:
% Frits Wenneker (http://www.howtotex.com)
%
% License:
% CC BY-NC-SA 3.0 (http://creativecommons.org/licenses/by-nc-sa/3.0/)
%
%%%%%%%%%%%%%%%%%%%%%%%%%%%%%%%%%%%%%%%%%

%----------------------------------------------------------------------------------------
%	PACKAGES AND OTHER DOCUMENT CONFIGURATIONS
%----------------------------------------------------------------------------------------
\documentclass[11pt]{article}
\usepackage{geometry} % Pour passer au format A4
\geometry{hmargin=1cm, vmargin=1cm} % 


\usepackage[T1]{fontenc} % Use 8-bit encoding that has 256 glyphs
\usepackage[english,francais]{babel} % Français et anglais
\usepackage[utf8]{inputenc} 

\usepackage{amsmath,amsfonts,amsthm} % Math packages

\usepackage{lmodern}
\usepackage{url}
\usepackage{eurosym} % signe Euros
\usepackage{geometry} % Pour passer au format A4
\geometry{a4paper} % 
\usepackage{graphicx} % Required for including pictures
\usepackage{float} % Allows putting an [H] in \begin{figure} to specify the exact location of the figure

\usepackage{multicol}
\usepackage{sectsty} % Allows customizing section commands
\allsectionsfont{\centering \normalfont\scshape} % Make all sections centered, the default font and small caps

%----------------------------------------------------------------------------------------
%	Pied de Page
%----------------------------------------------------------------------------------------

\setlength\parindent{0pt} % Removes all indentation from paragraphs - comment this line for an assignment with lots of text


%----------------------------------------------------------------------------------------
%	Titre
%----------------------------------------------------------------------------------------

\newcommand{\horrule}[1]{\rule{\linewidth}{#1}} % Create horizontal rule command with 1 argument of height


%----------------------------------------------------------------------------------------
%	Début du document
%----------------------------------------------------------------------------------------



\begin{document}

\section{Geogebra}

\subsection{Construction du triangle ABC}
\begin{enumerate}
\item Choisir la longueur de AB = \\
À l'aide de l'outil : \textbf{Segment de longueur donnée / Segment crée par un point et une longueur.}\\
Tracer le segment [AB].
\item Choisir la longueur de AC = \\
À l'aide de l'outil : \textbf{Cercle (Centre-Rayon)}.\\
Tracer le cercle de centre A et de rayon AC.
\item Choisir la longueur de AC = \\
À l'aide de l'outil : \textbf{Cercle (Centre-Rayon)}.\\
Tracer le cercle de centre A et de rayon AC.
\item À l'aide de l'outil : \textbf{Intersection entre deux objets}.\\
Placer un point d'intersection entre les deux cercles. Point qui deviendra notre futur sommet du triangle ABC.
\item Tracer le triangle.
\item À l'aide de l'outil : \textbf{Polygône}.\\
Cliquer successivement sur les trois points puis à nouveau sur le premier.
\item À l'aide de l'outil : \textbf{Distance ou longueur}.\\
Cliquer successivement sur les trois côtés du triangle pour afficher la longueur du côté.
\end{enumerate}

\subsection{Construction des Médiatrices}
\begin{enumerate}
\item Tracer les médiatrices à l'aide de l'outil éponyme (de même nom).
\item Placer un point \textbf{I} à l'intersection des trois médiatrices.
\item Tracer le cercle circonscrit.
\item À l'aide de geogebra peut-on tracer ce cercle sans construire le cercle circonscrit ? Si oui, quel outil employer ?
\end{enumerate}

\subsection{Construction des Médianes}
\begin{enumerate}
\item Trouver les milieux des côtés du triangle ABC.
À l'aide de l'outil : \textbf{Milieu ou centre}\\
\item Tracer les médianes.
\item Placer le point d'intersection des médianes \textbf{J}.
\end{enumerate}

\subsection{Construction des Hateurs}
\begin{enumerate}
\item Tracer les hauteurs.
À l'aide de l'outil : \textbf{Perpendiculaire}\\
\item Placer le point d'intersection des hauteurs \textbf{H}.
\end{enumerate}

\subsection{Conclusion}
\begin{enumerate}
\item Trace la droite (IJ).
\item À l'aide de l'outil : \textbf{angle}, cliquer les trois points I,J et K.\\
\end{enumerate}

\subsubsection*{Que peut-on dire des points I, J, K ?}

\end{document}






%%%%%%%%%%%%%%%%%%%%%%%%%%%%%%%%%%%%%%%%%
% LaTeX Template
% http://www.LaTeXTemplates.com
%
% Original author:
% Linux and Unix Users Group at Virginia Tech Wiki 
% (https://vtluug.org/wiki/Example_LaTeX_chem_lab_report)
%
% License:
% CC BY-NC-SA 3.0 (http://creativecommons.org/licenses/by-nc-sa/3.0/)
%
%%%%%%%%%%%%%%%%%%%%%%%%%%%%%%%%%%%%%%%%%

%----------------------------------------------------------------------------------------
%	PACKAGES AND DOCUMENT CONFIGURATIONS
%----------------------------------------------------------------------------------------

\documentclass[11pt]{article}
\usepackage{geometry} % Pour passer au format A4
\geometry{hmargin=1cm, vmargin=1cm} % 

\usepackage{graphicx} % Required for including pictures
\usepackage{float} % 

%Français
\usepackage[T1]{fontenc} 
\usepackage[english,francais]{babel}
\usepackage[utf8]{inputenc}
\usepackage{eurosym}
\usepackage{lmodern}
\usepackage{url}
\usepackage{multicol}

%Maths
\usepackage{amsmath,amsfonts,amssymb,amsthm}
%\usepackage[linesnumbered, ruled, vlined]{algorithm2e}
%\SetAlFnt{\small\sffamily}

%Autres
\linespread{1} % Line spacing
\setlength\parindent{0pt} % Removes all indentation from paragraphs

\renewcommand{\labelenumi}{\alph{enumi}.} % 
\pagestyle{empty}
%----------------------------------------------------------------------------------------
%	DOCUMENT INFORMATION
%----------------------------------------------------------------------------------------

\title{Résolution} % Title
\author{$6^e 6$}
\date{4 Février 2014} % Date for the report
\begin{document}

%\maketitle % Insert the title, author and date

\begin{center}
  \textsf{--}\\
  \textsf{Un peuple prêt à sacrifier un peu de liberté pour un peu de sécurité ne mérite ni l'une ni l'autre, et finit par perdre les deux.}
  \texttt{Benjamin Franklin, 1785–1788}\\
  \textsf{--}
\end{center}

\vspace{-0.5cm}

\setlength{\columnseprule}{1pt}
\begin{multicols}{2}


\begin{enumerate}
\item[1] Exercice 1 - Alice
  \begin{enumerate}
  \item Alice trouve 5 \euro~ dans la rue.\\
    Ne retrouvant pas la personne qui les a perdu, elle décide de les ajouter aux 25 \euro~ qu’elle possède déjà dans son porte-monnaie.\\
    De quelle somme dispose-t-elle désormais?\\
    On ajoute 5\euro~ au 25\euro~ possédés : $5 + 25 = 30$.\\
    Alice dispose de 30\euro~.

  \item Alice rentre chez le pâtissier avec 25 \euro~ dans son porte-monnaie. Elle achète un gâteau à 5 \euro.\\
    Combien lui reste-t-il en sortant de la pâtisserie ?\\
    On enlève 5\euro~ à 25\euro~ : $25 - 5 = 20$\\
    Il reste 20\euro~ à Alice.
  \item La maman d’Alice lui donne 25 \euro~ par mois d’argent de poche. \\
    Si Alice ne dépense pas cet argent, de quelle somme disposera-t-elle dans 5 mois ?
    On multiplie 25\euro~ par les 5 mois : $25 \times 5 = 125$.\\
    Alice disposera de 125\euro~.
  \item Alice décide de mettre dans sa tirelire 5 \euro~ par semaine pendant 25 semaines. Ses 5 soeurs décident de faire la même chose. \\
    Sachant que la maman donnera à chaque fillette 25 \euro~ à la fin de cette période, calculer la somme dont disposeraient Alice et ses soeurs en mettant en commun toutes leurs économies.\\
Avec 25 semaines d'économie : $25 \times 5 = 125$. Alice met de côté 125\euro~ de côté. Chacune de 5 soeurs fait la même chose. La mère ajoute 25\euro~ à la cagnote : $125 + 25 = 150$. Chaque soeurs a donc emmagasiné 150\euro~. 
Alice a 5 soeurs, cela fait donc 6 soeurs dans la fraterie : $6 \times 150 = 900$. 
Les six soeurs ont mis de côté 900\euro~.

 

  \end{enumerate}
  \rule{\linewidth}{0.5pt}
\item[2] Exercice 2 - Boxeur\\
  Un boxeur pèse 97,3 kg à 4 mois d’un combat. Il fait un régime qui lui permet de perdre 3 kg par mois jusqu’au jour du combat.
  \begin{enumerate}
  \item Quel poids va-t-il perdre pendant son régime ?
  \item Combien pèsera t-il le jour du combat ?
  \end{enumerate}
\rule{\linewidth}{0.5pt}
\item[3] Exercice 3 - Équipement de football \\
  L’entraîneur d’une équipe de football doit acheter dans un magasin de sport des équipements pour ses 16 joueurs.\\
  Chaque équipement est composé d’un maillot à 27 \euro~, d’un short à 15 \euro~ et d’une paire de bas à 6 \euro~.\\
  \begin{enumerate}
  \item Quel est le prix d’un équipement complet ?
  \item Quelle somme faut-il dépenser pour acheter des équipements pour l’ensemble de l’équipe ?
  \item Un sponsor donne au club 150 \euro~. Le magasin de sport accorde une réduction de 60 \euro~ sur l’achat. \\
 Combien l’entraîneur doit-il alors débourser pour acheter des équipements pour l’ensemble de l’équipe ?
  \end{enumerate}
\rule{\linewidth}{0.5pt}
\item[4] Exercice 4 - Budget football\\
  Un club de foot a un budget de 100 M\euro~ (Millions d’Euros).\\
  Le club vend 2 de ses joueurs à 10 M\euro~ chacun, et en achète 4 autres à 15 M\euro~ chacun.\\
  Combien d’argent reste t-il dans la caisse du club ?\\
\rule{\linewidth}{0.5pt}
\item[5] Exercice 5 - Pokemon\\
  Benoît s’est offert le jeu vidéo « Pekomen » version BLEUE qui contient 87 monstres.\\
  Il possède déjà la version ROUGE qui en contient 79.
 \begin{enumerate}
  \item Quand il aura complété les deux jeux, combien possédera-t-il de Pekomen ?
  \item Benoît a déjà trouvé 33 monstres sur la version BLEUE et 65 dans la version rouge.
Combien de Pekomen lui reste-t-il à trouver ?
  \item Benoît a besoin d’environ 30 minutes pour trouver chaque personnage. \\
Pendant combien de temps doit-il encore jouer pour terminer les deux jeux ?
  \end{enumerate}
\rule{\linewidth}{0.5pt}
\item[6] Exercice 6 - Cinéma\\
  3 filles et 5 garçons vont au cinéma. Chacun d’eux paye sa place 6 \euro~, s’achète un soda à 1,5 \euro~ et une glace à 2 \euro~.
  \begin{enumerate}
  \item Combien paye chaque enfant ?
  \item Quelle a été la somme dépensée par le groupe dans son ensemble ?\\
    Dans un autre cinéma, la place ne coûte que 5,50 \euro~, mais le soda et la glace coûtent 0,25 \euro~ de plus que dans le premier cinéma. En allant dans ce cinéma...
  \item combien aurait payé chaque enfant ?
  \item quelle aurait été la somme dépensée par le groupe dans son ensemble ?
  \end{enumerate}
\rule{\linewidth}{0.5pt}
\item[7] Exercice 7 - Téléphones\\
  Une boutique qui vend des téléphones mobiles propose les tarifs suivants sur son dernier modèle :\\
  TARIF A : Le téléphone à 49 \euro~ avec un abonnement à 26 \euro~/mois pendant 24 mois.\\
  TARIF B : Le téléphone à 149 \euro~ avec un abonnement à 37 \euro~/mois pendant 12 mois.\\
  TARIF C : Le téléphone sans abonnement à 399 \euro~.\\
  
  Calculer le prix de revient du téléphone pour chaque tarif. Lequel est le plus intéressant ?\\
\rule{\linewidth}{0.5pt}
\item[8] Exercice 8 - Tarifs cinéma\\
  Un cinéma propose des tarifs « dégressifs » afin de fidéliser sa clientèle :\\
  Du 1er au 10ème ticket acheté : 6 \euro~ par séance.\\
  Du 11ème au 20ème ticket acheté : 5,50 \euro~ par séance.\\
  A partir du 21ème ticket acheté : 5 \euro~ par séance.
  \begin{enumerate}
  \item Combien aura dépensé une personne qui va 5 fois au cinéma ?
  \item Combien aura dépensé une personne qui va 18 fois au cinéma ?
  \item Combien aura dépensé une personne qui va 32 fois au cinéma ?
  \end{enumerate}
\rule{\linewidth}{0.5pt}
\item[9] Exercice - Temps\\
  \begin{enumerate}
  \item Jean doit prendre deux trains pour se rendre au camping où il va passer l'été. Le premier trajet en train dure 2h15 et le second 1h37. \\
    Quelle est la durée totale de son trajet ?
  \item Bella doit prendre deux trains pour se rendre chez ses grands-parents pour les vacances. Le premier trajet en train dure 3h38 et le second 2h49. \\
    Quelle est la durée totale de son trajet ?
  \item Elsa part en voyage en car. Le départ en car a lieu à 6h35. La durée du trajet est de 2h40. On prévoit en plus un arrêt de 30 min. \\
    À quelle heure le car arrivera-t-il à destination ? 
  \item mélie part de Paris à 20h54 pour Rouen. Le trajet dure 1h12. \\
    À quelle heure arrivera -t-elle ?
  \item Sonia a quitté le collège à 16h35. Elle est ensuite allée chez une camarade qui habite à 10 min du collège. Elles sont restées ensemble une heure et quart puis Sonia est rentrée chez elle, ce qui lui a pris 25 minutes. \\
    À quelle heure est-elle arrivée à son domicile ?
  \item Pierre prend le train pour aller de Lyon à Marseille. Il part de Lyon à 15h20 et arrive à Marseille à 17h58. \\
    Combien de temps le trajet a-t-il duré ?
  \item Un film débute à 15h27 et finit à 18h14. \\
    Quelle est la durée du film ? 
  \item Un film débute à 19h34 et finit à 21h58. \\
    Quelle est la durée du film ?
  \item Un concert commence à 21h05 et se termine à 23h51. \\
    Quelle est la durée du concert ?
  \item Le vendredi 25 mars 2011 en France, le soleil s'est levé à 06h43 pour se coucher à 19h10. \\
    Combien de temps la journée a-t-elle duré ?\\
    Ce même jour, la lune s'est levée à 1h43 pour se coucher à 9h58. \\
    Combien de temps la lune a-t-elle été visible ce jour-là ?
  \item Monsieur Kiroul se rend dans un garage pour faire réparer son véhicule.\\
    La réparation débute à 8h15 et s'achève à 10h25. Quelle est la durée de la réparation ?\\
    Dans ce garage, Paul, l'employé, doit travailler 7 heures par jour. Il a travaillé 4h15 le matin. \\
    Combien doit-il travailler l'après-midi ?
  \item Au cinéma Le Palace, la première séance du soir a commencé à 19h45 et s'est terminée à 21h33. \\
    Quelle est la durée de cette séance ?
  \item Un avion a atterri à l'aéroport de New York à 18h55 (heure de Paris) après une durée de vol de 8h20. \\
    À quelle avait-il décollé de l'aéroport de Paris ?
  \item Le train que Marc a pris est parti à 11h17 et le voyage a duré 3h23.\\
    A quelle heure Marc est-il arrivé à destination ?\\
    Le père de Marc, qui devait venir le chercher à la gare, est arrivé en retard à 15h05.\\
    Combien de temps Marc a-t-il dû attendre son père ?
  \end{enumerate}
\rule{\linewidth}{0.5pt}
\end{enumerate}
\end{multicols}
\end{document}







\documentclass{beamer}

\usepackage{geometry} % Pour passer au format A4
\usepackage{graphicx} % Required for including pictures
\usepackage{float} % 

\usepackage{amsmath,amsfonts,amssymb,amsthm}
\usepackage[T1]{fontenc} 
\usepackage[english,francais]{babel}
\usepackage[utf8]{inputenc}
\usepackage{lmodern}

\usetheme{Warsaw}

\title{Nombres}
\author{$6^{e}6$}

\begin{document}

\frame{\titlepage}


\section{Chiffres Mystères}

\frame{\tableofcontents[sectionstyle=show/shaded]}

\begin{frame}
  \begin{block}{Nombre 1}
      \begin{itemize}
            \item Ma partie entière est 89.
            \item Mon chiffre des millièmes est 7. 
            \item Mon chiffre des dixièmes est 5.
      \end{itemize}
      Le résultat est : 
  \end{block} 
\end{frame}

\begin{frame}
  \begin{block}{Nombre 2}
      \begin{itemize}
            \item Ma partie entière est 124.
            \item Mon chiffre des millièmes est 2. 
            \item Mon chiffre des dixièmes est le double de celui des millièmes.
            \item Mon chiffre des centièmes est la moitié de celui des millièmes.
      \end{itemize}
      Le résultat est : 
  \end{block} 
\end{frame}

\begin{frame}
  \begin{block}{Nombre 3}
      \begin{itemize}
            \item Ma partie entière est 333.
            \item Ma partie décimale a 4 chiffres.
            \item Mon chiffre des centièmes est 4. 
            \item Mon chiffre des dix-millièmes est le quart de celui des centièmes.
            \item Mon chiffre des millièmes est la moitié de celui des centièmes.
            \item La sommme totale des chiffres de la partie décimal est 7
      \end{itemize}
      Le résultat est : 
  \end{block} 
\end{frame}

\begin{frame}
  \begin{block}{Nombre 4}
      \begin{itemize}
            \item Je suis un nombre inférieur à 1000.
            \item Ma partie décimale a 2 chiffres.
            \item Mon chiffre des centaines est la moitié du chiffre des centièmes. 
            \item Mon chiffre des unités est 4.
            \item Mon chiffre des dizaines est la somme de celui des centaines et de celui des unités.
            \item La sommme de mes chiffres est 17. 
      \end{itemize}
      Le résultat est : 
  \end{block} 
\end{frame}

\begin{frame}
  \begin{block}{Nombre 5}
      \begin{itemize}
            \item Je suis un nombre inférieur à 1000.
            \item Ma partie décimale a 2 chiffres.
            \item Mon chiffre des centièmes est 1.
            \item Mon chiffre des centaines est le double du chiffre des centièmes. 
            \item Mon chiffre des dizaines est 3.
            \item Mon chiffre des unités est la somme de celui des centaines et de celui des dizaines.
            \item Mon chiffre des dixièmes est le double du chiffre des dizaines. 
      \end{itemize}
      Le résultat est : 
  \end{block} 
\end{frame}


\section{Que de nombres !}

\frame{\tableofcontents[sectionstyle=show/shaded]}

\begin{frame}
  \begin{block}{cent, vingt, quatre et mille}
 Ecrire tous les chiffres en utilisant \alert{une fois et une seule} chacun des mots \textbf{cent}, \textbf{vingt}, \textbf{quatre} et \textbf{mille}.
  \end{block} 
\end{frame}





\end{document}

%%%%%%%%%%%%%%%%%%%%%%%%%%%%%%%%%%%%%%%%%
% LaTeX Template
% http://www.LaTeXTemplates.com
%
% Original author:
% Linux and Unix Users Group at Virginia Tech Wiki 
% (https://vtluug.org/wiki/Example_LaTeX_chem_lab_report)
%
% License:
% CC BY-NC-SA 3.0 (http://creativecommons.org/licenses/by-nc-sa/3.0/)
%
%%%%%%%%%%%%%%%%%%%%%%%%%%%%%%%%%%%%%%%%%

%----------------------------------------------------------------------------------------
%	PACKAGES AND DOCUMENT CONFIGURATIONS
%----------------------------------------------------------------------------------------

\documentclass[11pt]{article}
\usepackage{geometry} % Pour passer au format A4
\geometry{hmargin=1cm, vmargin=1cm} % 

\usepackage{graphicx} % Required for including pictures
\usepackage{float} % 

%Français
\usepackage[T1]{fontenc} 
\usepackage[english,francais]{babel}
\usepackage[utf8]{inputenc}
\usepackage{eurosym}
\usepackage{lmodern}
\usepackage{url}
\usepackage{multicol}

%Maths
\usepackage{amsmath,amsfonts,amssymb,amsthm}
%\usepackage[linesnumbered, ruled, vlined]{algorithm2e}
%\SetAlFnt{\small\sffamily}

%Autres
\linespread{1} % Line spacing
\setlength\parindent{0pt} % Removes all indentation from paragraphs

\renewcommand{\labelenumi}{\alph{enumi}.} % 
\pagestyle{empty}

%----------------------------------------------------------------------------------------
%	DOCUMENT INFORMATION
%----------------------------------------------------------------------------------------
\begin{document}

%\maketitle % Insert the title, author and date

\begin{minipage}[t]{\textwidth}
  \raggedright
      {\bfseries $BTS STS ET$}\\[.35ex]
      \vspace*{-1cm}
      \raggedleft
          {\bfseries ie Complexes 2}\\[.35ex]
          {\bfseries 17 Octobre 2014}\\[.35ex]
\end{minipage}\\[1em]

\begin{center}
  \textsf{Une méthode est un truc qui a été utilisé plusieurs fois. - George Polya}\\
\end{center}


\setlength{\columnseprule}{1pt}

\begin{multicols}{2}

% ------ Exercice 1 ------
\section*{1 - Forme algébrique}
Mettre sous la forme algébrique : $a+jb$ avec $j^2 = -1$ et $a,b \in \mathbb{R}$.

\begin{eqnarray*}
	z_1 &=& -j \times (j -1) \times (2j+4) \\
	z_2 &=& 1 + j + j^2 + j^3 + j^4 \\
	z_3 &=& (2 - 3j)^2 \\
	z_4 &=& \dfrac{j+4}{j+5} 
\end{eqnarray*}

% ------ Exercice 2 ------
\section*{2 - Forme exponentielle}
Mettre sous la forme exponentielle : $r \times e^{j\theta}$ avec $r \in \mathbb{R^{+}}$ et $\theta \in \mathbb{R}$.

\begin{eqnarray*}
	z_5 &=& 2 + 2j \\
	z_6 &=& -1 + j\sqrt{3} \\
	z_7 &=& -\sqrt{6} + j\sqrt{2} \\
	z_8 &=& -12 j
\end{eqnarray*}


% ------ Exercice 3 ------
\section*{3 - Par la forme trigonométrique}
Mettre sous la forme algébrique : $a+jb$ avec $j^2 = -1$ et $a,b \in \mathbb{R}$.

\begin{eqnarray*}
	z_9 &=& 2 e^{j \frac{\pi}{3}} \\
	z_{10} &=& 5 e^{-j \frac{\pi}{6}} \\
	z_{11} &=& 2 e^{j \frac{3\pi}{2}}\\
	z_{12} &=& 12 e^{j \pi}
\end{eqnarray*}

% ------ Exercice 4 ------
\section*{4 - Dans un repère}
Placer les points dans un repère orthonormal.

\begin{eqnarray*}
	z_{13} &=& 1 + 2j \\
	z_{14} &=& -2 \\
	z_{15} &=&  4j \\
	z_{16} &=&  -2 + 3j
\end{eqnarray*}
\end{multicols}
\end{document}

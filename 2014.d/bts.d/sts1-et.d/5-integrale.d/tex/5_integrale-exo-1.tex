%%%%%%%%%%%%%%%%%%%%%%%%%%%%%%%%%%%%%%%%%
% LaTeX Template
% http://www.LaTeXTemplates.com
%
% Original author:
% Linux and Unix Users Group at Virginia Tech Wiki
% (https://vtluug.org/wiki/Example_LaTeX_chem_lab_report)
%
% License:
% CC BY-NC-SA 3.0 (http://creativecommons.org/licenses/by-nc-sa/3.0/)
%
%%%%%%%%%%%%%%%%%%%%%%%%%%%%%%%%%%%%%%%%%

%----------------------------------------------------------------------------------------
%	PACKAGES AND DOCUMENT CONFIGURATIONS
%----------------------------------------------------------------------------------------

\documentclass[11pt]{article}
\usepackage{geometry} % Pour passer au format A4
\geometry{hmargin=1cm, vmargin=1cm} %

\usepackage{graphicx} % Required for including pictures
\usepackage{float} %

%Français
\usepackage[T1]{fontenc}
\usepackage[english,francais]{babel}
\usepackage[utf8]{inputenc}
\usepackage{eurosym}
\usepackage{lmodern}
\usepackage{url}
\usepackage{multicol}

%Maths
\usepackage{amsmath,amsfonts,amssymb,amsthm}
%\usepackage[linesnumbered, ruled, vlined]{algorithm2e}
%\SetAlFnt{\small\sffamily}

%Autres
\linespread{1} % Line spacing
\setlength\parindent{0pt} % Removes all indentation from paragraphs


\renewcommand{\labelenumi}{\alph{enumi}.} %

%----------------------------------------------------------------------------------------
%	DOCUMENT INFORMATION
%----------------------------------------------------------------------------------------
\begin{document}

\setlength{\columnseprule}{1pt}

\begin{multicols}{2}

  \section*{EXERCICE 1 - Primitive - 1}
  Pour chaque fonction donner une primitive.

  \begin{enumerate}
  \item $f_{1}(x) = 0$
  \item $f_{2}(x) = \sqrt(2) + \pi$
  \item $f_{3}(x) = x + 5x^3 + 9 x^5$
  \item $f_{4}(x) = 12 + \dfrac{1}{10}x + \dfrac{1}{5}x^{4} \dfrac{1}{2}x^{8}$
  \item $f_{5}(x) = 3x + \dfrac{4}{x}$ 
  \item $f_{6}(x) = 2x + e^(x)$
  \item $f_{7}(x) = 4\cos(x) + 4\sin(x)$
  \item $f_{8}(x) = \dfrac{4}{x^2} - \dfrac{7}{x^3} + \dfrac{1}{2x^5}$
  \end{enumerate}

  \section*{EXERCICE 2 - Primitive - 2}
  Pour chaque fonction donner une primitive.

  \begin{enumerate}
  \item $g_{1}(x) = (2x +1)^3$
  \item $g_{2}(x) =  2 \cos(3x + \pi)$
  \item $g_{3}(x) = -4 \sin(2x - \pi)$
  \item $g_{4}(x) = \sin(x) \cos^{2}(x)$
  \item $g_{5}(x) = 8 e^{2x + 1}$ 
  \item $g_{6}(x) = -x e^{x^2 + 1}$
  \item $g_{7}(x) = \dfrac{1}{2x + 9}$
  \item $g_{8}(x) = \dfrac{2x}{x^2 - 8}$
  \end{enumerate}

\end{multicols}

\rule{\textwidth}{1pt}

\begin{multicols}{2}

  \section*{EXERCICE 3 - Propriétés}
  Répondre par \textbf{vrai} ou par \textbf{faux}.

  \begin{enumerate}
  \item $ \int_{1}^{10} f(x) \, \mathrm dx = \int_{1}^{6} f(x) \, \mathrm dx + \int_{6}^{10} f(x) \, \mathrm dx$
  \item $ \int_{-2}^{12} f(x) \, \mathrm dx = \int_{-2}^{-1} f(x) \, \mathrm dx + \int_{1}^{12} f(x) \, \mathrm dx$
  \item $f$ est une fonction paire. \\
$ \int_{-10}^{20} f(x) \, \mathrm dx = 3\int_{0}^{20} f(x) \, \mathrm dx $ 
  \item $f$ est une fonction paire. \\
$ \int_{-10}^{20} f(x) \, \mathrm dx = 2\int_{0}^{10} f(x) \, \mathrm dx + \int_{10}^{20} f(x) \, \mathrm dx $ 
  \item $f$ est une fonction impaire. \\
$ \int_{-10}^{20} f(x) \, \mathrm dx = \int_{10}^{20} f(x) \, \mathrm dx$ 
  \item $ \int_{0}^{10} 12f(x) \, \mathrm dx = 12 \int_{0}^{10} f(x) \, \mathrm dx$
  \item $ \int_{0}^{0} -2f(x) \, \mathrm dx = -2$
  \end{enumerate}

\end{multicols}

\rule{\textwidth}{1pt}

\section*{EXERCICE 4 -  Graphiques}
Calculer les intégrales suivantes.

\begin{figure}[H]
  \centering
  \fbox{\includegraphics[width=0.6\textwidth]{sources/ie/integrale-1.pdf}}
\end{figure}

\begin{multicols}{3}

  \begin{enumerate}
  \item $ \int_{-3}^{4} f_{1}(x) \, \mathrm dx $
  \item $ \int_{-4}^{2} f_2(x) \, \mathrm dx $
  \item $ \int_{-3}^{2} f_3(x) \, \mathrm dx $
  \item $ \int_{-3}^{3} f_4(x) \, \mathrm dx $
  \item $ \int_{-4}^{4} f_5(x) \, \mathrm dx $
  \item $ \int_{-2}^{4} f_6(x) \, \mathrm dx $
  \item $ \int_{-4}^{4} f_2(x) \, \mathrm dx $
  \item $ \int_{-2}^{4} f_5(x) \, \mathrm dx $
  \end{enumerate}
\end{multicols}

\end{document}

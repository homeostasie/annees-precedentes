%%%%%%%%%%%%%%%%%%%%%%%%%%%%%%%%%%%%%%%%%
% LaTeX Template
% http://www.LaTeXTemplates.com
%
% Original author:
% Linux and Unix Users Group at Virginia Tech Wiki
% (https://vtluug.org/wiki/Example_LaTeX_chem_lab_report)
%
% License:
% CC BY-NC-SA 3.0 (http://creativecommons.org/licenses/by-nc-sa/3.0/)
%
%%%%%%%%%%%%%%%%%%%%%%%%%%%%%%%%%%%%%%%%%

%----------------------------------------------------------------------------------------
%	PACKAGES AND DOCUMENT CONFIGURATIONS
%----------------------------------------------------------------------------------------

\documentclass[11pt]{article}
\usepackage{geometry} % Pour passer au format A4
\geometry{hmargin=1cm, vmargin=1cm} %

\usepackage{graphicx} % Required for including pictures
\usepackage{float} %

%Français
\usepackage[T1]{fontenc}
\usepackage[english,francais]{babel}
\usepackage[utf8]{inputenc}
\usepackage{eurosym}
\usepackage{lmodern}
\usepackage{url}
\usepackage{multicol}

%Maths
\usepackage{amsmath,amsfonts,amssymb,amsthm}
%\usepackage[linesnumbered, ruled, vlined]{algorithm2e}
%\SetAlFnt{\small\sffamily}

%Autres
\linespread{1} % Line spacing
\setlength\parindent{0pt} % Removes all indentation from paragraphs


\renewcommand{\labelenumi}{\alph{enumi}.} %

%----------------------------------------------------------------------------------------
%	DOCUMENT INFORMATION
%----------------------------------------------------------------------------------------
\begin{document}

\begin{minipage}[t]{\textwidth}
  \raggedright
      {\bfseries $BTS STS ET 1$}\\[.35ex]
      \vspace*{-1cm}
      \raggedleft
          {\bfseries Intégration}\\[.35ex]
          {\bfseries 17 Mars 2014}\\[.35ex]
\end{minipage}\\[1em]

\begin{center}
  \textsf{A une époque de supercherie universelle, dire la vérité est un acte révolutionnaire. - George Orwell}
\end{center}


\setlength{\columnseprule}{1pt}

  \section*{EXERCICE 1 - Primitive}
  \textit{Pour chaque fonction donner une primitive.}

\begin{multicols}{2}


  \begin{enumerate}
  \item $f_{1}(x) = \sqrt(2) + \pi$
  \item $f_{2}(x) = x + 5x^3 + 9 x^5$
  \item $f_{3}(x) = 12 + \dfrac{1}{10}x + \dfrac{1}{5}x^{4} \dfrac{1}{2}x^{8} + \dfrac{4}{x} $
  \item $f_{4}(x) = 2x + 4\cos(x) + 4\sin(x) e^{(x)}$
  \item $f_{5}(x) = \dfrac{4}{x^2} - \dfrac{7}{x^3} + \dfrac{1}{2x^5}$
  \end{enumerate}

\end{multicols}

\rule{\textwidth}{1pt}

\section*{EXERCICE 2 -  Intégrales graphiques}
\textit{Calculer les intégrales suivantes.}

\begin{figure}[H]
  \centering
  \fbox{\includegraphics[width=0.6\textwidth]{sources/ie/integrale-1.pdf}}
\end{figure}

\begin{multicols}{3}

  \begin{enumerate}
  \item $ \int_{-4}^{4} f_{1}(x) \, \mathrm dx $
  \item $ \int_{-4}^{4} f_2(x) \, \mathrm dx $
  \item $ \int_{-4}^{4} f_3(x) \, \mathrm dx $
  \item $ \int_{-4}^{4} f_4(x) \, \mathrm dx $
  \item $ \int_{-4}^{4} f_5(x) \, \mathrm dx $
  \item $ \int_{-2}^{2} f_6(x) \, \mathrm dx $
  \end{enumerate}
\end{multicols}

  \section*{EXERCICE 3 - Intégrale}
  \textit{Pour chaque fonction calculer la valeur de l'intégrale.}

  \begin{enumerate}
  \item $f_{1}(x) = \int_{- \pi}^{4\pi} x^3 - 2x           \, \mathrm dx $
  \item $f_{2}(x) = \int_{- \pi}^{4\pi} \cos(x) + 2\sin(x) \, \mathrm dx $
  \item $f_{3}(x) = \int_{-1}^{2}      e^{x} - 3          \, \mathrm dx $
  \item $f_{4}(x) = \int_{- \pi}^{2\pi} 8\cos(3x)          \, \mathrm dx $
  \end{enumerate}


  \section*{EXERCICE 4 - Intégrale - 2}
  \textit{À l'aide d'une intégration par partie, calculer la valeur de l'intégrale.}

$f(x) = \int_{0}^{5} 3x e^{-x} \, \mathrm dx $

\end{document}

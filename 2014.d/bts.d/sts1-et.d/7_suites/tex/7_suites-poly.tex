%%%%%%%%%%%%%%%%%%%%%%%%%%%%%%%%%%%%%%%%%
% Short Sectioned Assignment
% LaTeX Template
% Version 1.0 (5/5/12)
%
% This template has been downloaded from:
% http://www.LaTeXTemplates.com
%
% Original author:
% Frits Wenneker (http://www.howtotex.com)
%
% License:
% CC BY-NC-SA 3.0 (http://creativecommons.org/licenses/by-nc-sa/3.0/)
%
%%%%%%%%%%%%%%%%%%%%%%%%%%%%%%%%%%%%%%%%%

%----------------------------------------------------------------------------------------
%	PACKAGES AND OTHER DOCUMENT CONFIGURATIONS
%----------------------------------------------------------------------------------------

\documentclass[11pt]{article}%{scrartcl} % A4 paper and 11pt font size
\usepackage{geometry} 
\geometry{hmargin=1cm, vmargin=1cm} %

\usepackage[T1]{fontenc} % Use 8-bit encoding that has 256 glyphs
\usepackage[english,francais]{babel} % Français et anglais
\usepackage[utf8]{inputenc} 

\usepackage{amsmath,amsfonts,amsthm} % Math packages

\usepackage{enumitem}
\usepackage{lmodern}
\usepackage{url}
\usepackage{eurosym} % signe Euros
\usepackage{geometry} % Pour passer au format A4
\geometry{a4paper} % 
\usepackage{graphicx} % Required for including pictures
\usepackage{float} % Allows putting an [H] in \begin{figure} to specify the exact location of the figure

\usepackage{multicol}
\usepackage{caption}
\usepackage{verbatim}
\usepackage{pst-node,pst-tree}

\usepackage{sectsty} % Allows customizing section commands
\allsectionsfont{\centering \normalfont\scshape} % Make all sections centered, the default font and small caps

%----------------------------------------------------------------------------------------
%	Pied de Page
%----------------------------------------------------------------------------------------


\setlength{\headheight}{13.6pt} % Customize the height of the header


\setlength\parindent{0pt} % Removes all indentation from paragraphs - comment this line for an assignment with lots of text


%----------------------------------------------------------------------------------------
%	Titre
%----------------------------------------------------------------------------------------

\newcommand{\horrule}[1]{\rule{\linewidth}{#1}} % Create horizontal rule command with 1 argument of height


\title{	
  \vspace{-10ex}
  \horrule{0.5pt} \\[0.4cm] % Thin top horizontal rule
  \huge Suites Numériques\\ % The assignment title
  \horrule{2pt} \\[0.5cm] % Thick bottom horizontal rule
}

\author{}
\date{\vspace{-10ex}} % Today's date or a custom date

%----------------------------------------------------------------------------------------
%	Début du document
%----------------------------------------------------------------------------------------

\begin{document}

%----------------------------------------------------------------------------------------
% RE-DEFINITION
%----------------------------------------------------------------------------------------
% MATHS
%-----------

\newtheorem{Definition}{Définition}
\newtheorem{Theorem}{Théorème}
\newtheorem{Proposition}{Propriété}

% MATHS
%-----------
\renewcommand{\labelitemi}{$\bullet$}
\renewcommand{\labelitemii}{$\circ$}
%----------------------------------------------------------------------------------------
%	Titre
%----------------------------------------------------------------------------------------

\maketitle % Print the title
\setlength{\columnseprule}{1pt}
\begin{multicols}{2}
\section{Introduction}

\begin{Definition}{Suites Numériques}\\
Une suites numériques est une suite indexée de nombre. Elle a un premier terme, un second terme...
\end{Definition}

Une suite numérique est une fonction qui a tout entier naturel associe un nombre noté $u(n)$ ou $u_n$. La suite est notée dans sa globalité $(u_n)$.

\subsection{Générations d'une suite}

\subsubsection{Formule implicite}

On définit la suite $(u_n)$ à l'aide d'une fonction de $n$ : 
 
$$u_n = f(n) \text{ où } f \text{ est une fonction}$$ 

\paragraph{Exemple : }
$u_n = -n^2 + n - 2$\\
Premiers termes : $u_0 = 2 , u_1 = -2, ...$

\subsubsection{Formule de récurrence}

On définit la suite $(u_n)$ à partir d'un premier terme et d'une relation de recurrence nous permettant de passer d'un terme à l'autre.


\begin{equation*}
  \left\lbrace
  \begin{array}{ccc}
    v_0 \text{ est donnée }\\
    v_{n+1} &=& f(v_n)
  \end{array}\right.
\end{equation*}

\paragraph{Exemple : }

\begin{equation*}
  \left\lbrace
  \begin{array}{ccc}
    v_0    &=& 5\\
    v_{n+1} &=& \dfrac{v_n + 3}{2}
  \end{array}\right.
\end{equation*}

Premiers termes : $v_1 = 4, v_2 = \dfrac{7}{2},...$

\paragraph{Remarques}

\begin{itemize}
\item On dit également utiliser un procédé : Prendre le terme précédent, lui rajouter 3 et diviser par 2.
\item Pour calculer le 10-ième terme, il nous faut calculer tous les termes précédents.
\end{itemize}

\end{multicols}

\hrulefill

\begin{multicols}{2}
\section{Suites Arithmétiques}

\begin{Definition}{Suite Arithmétique}\\
  Une suite $(u_n)$ est arithmétique si on passe d'un terme au suivant en additionnant toujours par le même nombre.
  $$u_n = u_n + r$$
  On appele le nombre $r$ la raison de la suite $u_n$.
\end{Definition}

\begin{Proposition}{Démonstration}\\
 Une suite $(u_n)$ est arithmétique si et seulement si la différence $u_{n+1} - u_n$ est constante pour tout n.
\end{Proposition}

\begin{Proposition}{}
  Une suite $(u_n)$ est arithmétique si on peut l'écrire : $u_{n} = u_0 + nr$.
\end{Proposition}

\paragraph{Exemples : }
\begin{itemize}
\item $1, 6, 11, 16, 21, ... $
\item $0, 1, 2, 3, 4, 5, 6, 7, ...$
\item 
\begin{equation*}
  \left\lbrace
  \begin{array}{ccc}
    u_0 = 2\\
    r = 2
  \end{array}\right.
\end{equation*}

\item $u_n  = 3n - 2$
\end{itemize}

\section{Suites Géométriques}

\begin{Definition}{Suite Géométrique}\\
  Une suite $(u_n)$ est géométrique si on passe d'un terme au suivant en multipliant toujours par le même nombre.
  $$u_n = q \times u_n$$
  On appele le nombre $q$ la raison de la suite $u_n$.
\end{Definition}

\begin{Proposition}{Démonstration}\\
 Une suite $(u_n)$ est géométrique si et seulement si le quotient $\dfrac{u_{n+1}}{u_n}$ est constant pour tout n.
\end{Proposition}

\begin{Proposition}{}
  Une suite $(u_n)$ est géométrique si on peut l'écrire : $u_{n} = u_0 \times q^n$.
\end{Proposition}

\paragraph{Exemples : }
\begin{itemize}
\item $1, 2, 4, 8, 16, ... $
\item $1, -1, 1, - 1 , 1 , ...$
\item 
\begin{equation*}
  \left\lbrace
  \begin{array}{ccc}
    u_0 = 5\\
    q = 10
  \end{array}\right.
\end{equation*}

\item $u_n  = 5 \times 3^{n+1}$
\end{itemize}
\end{multicols}

\end{document}

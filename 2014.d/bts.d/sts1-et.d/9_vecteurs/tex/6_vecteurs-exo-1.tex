%%%%%%%%%%%%%%%%%%%%%%%%%%%%%%%%%%%%%%%%%
% LaTeX Template
% http://www.LaTeXTemplates.com
%
% Original author:
% Linux and Unix Users Group at Virginia Tech Wiki
% (https://vtluug.org/wiki/Example_LaTeX_chem_lab_report)
%
% License:
% CC BY-NC-SA 3.0 (http://creativecommons.org/licenses/by-nc-sa/3.0/)
%
%%%%%%%%%%%%%%%%%%%%%%%%%%%%%%%%%%%%%%%%%

%----------------------------------------------------------------------------------------
%	PACKAGES AND DOCUMENT CONFIGURATIONS
%----------------------------------------------------------------------------------------

\documentclass[10pt]{article}
\usepackage{geometry} % Pour passer au format A4
\geometry{hmargin=1cm, vmargin=1cm} %

\usepackage{graphicx} % Required for including pictures
\usepackage{float} %

%Français
\usepackage[T1]{fontenc}
\usepackage[english,francais]{babel}
\usepackage[utf8]{inputenc}
\usepackage{eurosym}
\usepackage{lmodern}
\usepackage{url}
\usepackage{multicol}

%Maths
\usepackage{amsmath,amsfonts,amssymb,amsthm}
%\usepackage[linesnumbered, ruled, vlined]{algorithm2e}
%\SetAlFnt{\small\sffamily}

%Autres
\linespread{1} % Line spacing
\setlength\parindent{0pt} % Removes all indentation from paragraphs


\renewcommand{\labelenumi}{\alph{enumi}.} %

%----------------------------------------------------------------------------------------
%	DOCUMENT INFORMATION
%----------------------------------------------------------------------------------------
\begin{document}

\setlength{\columnseprule}{1pt}

\subsubsection*{EXERCICE 1}

Donner les coordonnées du vecteur $\overrightarrow{u}$.

\begin{figure}[H]
  \centering
  \fbox{\includegraphics[width=0.6\textwidth]{sources/exo/vecteur-1.pdf}}
\end{figure}

\subsubsection*{EXERCICE 2}

Tracer un représentant du vecteur $\overrightarrow{u}$ et un représentant du vecteur $\overrightarrow{v}$.

\begin{figure}[H]
  \centering
  \fbox{\includegraphics[width=0.6\textwidth]{sources/exo/vecteur-2.pdf}}
\end{figure}

\begin{multicols}{2}
  \begin{enumerate}
  \item $\overrightarrow{u} = ( 2 ;  3)$ et $\overrightarrow{v} = (-1 :  2)$
  \item $\overrightarrow{u} = (1 : - 2)$ et $\overrightarrow{v} = (-1 :  -3)$
  \item $\overrightarrow{u} = ( 6 ;  1)$ et $\overrightarrow{v} = ( 2 ;  0)$
  \item $\overrightarrow{u} = ( 2.5 ;  0)$ et $\overrightarrow{v} = ( 3.5 ;  0)$
  \item $\overrightarrow{u} = ( 2 ; -5)$ et $\overrightarrow{v} = ( 0 ;  7)$
  \item $\overrightarrow{u} = ( -2 ; -7)$ et $\overrightarrow{v} = ( -5 ;  4)$
  \end{enumerate}
\end{multicols}

\subsubsection*{EXERCICE 3}
Soient $\overrightarrow{AB}$, $\overrightarrow{BC}$ et $\overrightarrow{AC}$ trois vecteurs du plan tel que $\overrightarrow{AB} = (10 ; 2)$, $\overrightarrow{BC} = (-3 ; 4)$ et $\overrightarrow{AC} = (4 ; -7)$   . Calculer les vecteurs suivants.

\begin{multicols}{2}

\begin{enumerate}
\item $\overrightarrow{AB}$ + $\overrightarrow{BC}$ ; $\overrightarrow{BC}$ + $\overrightarrow{CA}$ et  $\overrightarrow{AB}$.\\
Comparer avec  $\overrightarrow{AC}$.
\item  $\overrightarrow{u_1} =  4\overrightarrow{AB}$
\item  $\overrightarrow{u_2} = -2\overrightarrow{AB} + \overrightarrow{AB} $
\item  $\overrightarrow{u_3} =  7\overrightarrow{AB} + \overrightarrow{AC} +  3\overrightarrow{BC}$
\item  $\overrightarrow{u_4} = 10\overrightarrow{AB} - \overrightarrow{AC} - 2\overrightarrow{AB}$
\end{enumerate}

\end{multicols}

\end{document}

%%%%%%%%%%%%%%%%%%%%%%%%%%%%%%%%%%%%%%%%%
% LaTeX Template
% http://www.LaTeXTemplates.com
%
% Original author:
% Linux and Unix Users Group at Virginia Tech Wiki 
% (https://vtluug.org/wiki/Example_LaTeX_chem_lab_report)
%
% License:
% CC BY-NC-SA 3.0 (http://creativecommons.org/licenses/by-nc-sa/3.0/)
%
%%%%%%%%%%%%%%%%%%%%%%%%%%%%%%%%%%%%%%%%%

%----------------------------------------------------------------------------------------
%	PACKAGES AND DOCUMENT CONFIGURATIONS
%----------------------------------------------------------------------------------------

\documentclass[11pt]{article}
\usepackage{geometry} % Pour passer au format A4
\geometry{hmargin=1cm, vmargin=1cm} % 

\usepackage{graphicx} % Required for including pictures
\usepackage{float} % 

%Français
\usepackage[T1]{fontenc} 
\usepackage[english,francais]{babel}
\usepackage[utf8]{inputenc}
\usepackage{eurosym}
\usepackage{lmodern}
%\usepackage{xltxtra}

\usepackage{url}
\usepackage{multicol}

%Maths
\usepackage{amsmath,amsfonts,amssymb,amsthm}
%\usepackage[linesnumbered, ruled, vlined]{algorithm2e}
%\SetAlFnt{\small\sffamily}

\usepackage{pst-node,pst-tree}

%Autres
\linespread{1} % Line spacing
\setlength\parindent{0pt} % Removes all indentation from paragraphs

\renewcommand{\labelenumi}{\alph{enumi}.} % 
\pagestyle{empty}
%----------------------------------------------------------------------------------------
%	DOCUMENT INFORMATION
%----------------------------------------------------------------------------------------
\begin{document}

%\maketitle % Insert the title, author and date

\begin{minipage}[t]{\textwidth}
  \raggedright
      {\bfseries 1 STI 2D 2}\\[.35ex]
      \vspace*{-1cm}
      \raggedleft
          {\bfseries Probabilité}\\[.35ex]
          {\bfseries 9 Octobre 2014}\\[.35ex]
\end{minipage}\\[1em]

\begin{center}
  \textsf{Les mathématiques ne sont une moindre immensité que la mer. - Victor Hugo}
\end{center}

\setlength{\columnseprule}{1pt}
\section*{QCM}
\subsection*{Données}

Soit un examen avec comme exerice un QCM (Questionnaire à choix multiples).
\begin{itemize}
\item Pour chaque question, il y a 4 réponses possibles. 
\item Une seule et uniquement une seule est bonne. 
\item L'exercice est composé de 3 questions indépendantes.
\item Jean-Heude n'ayant vraiment pas travaillé décide de répondre vraiment au hasard.
\end{itemize}

\subsection*{Questions et réponses}

  \begin{enumerate}
  \item Quelle est la probabilité pour Jean-Heude de répondre correctement à la première question ?\\
    Il y 4 réponses possibles mais une seule est correcte.
    La probabilité de répondre correctement à la première question est $p = \dfrac{1}{4}$.

  \item Décrire pourquoi la situation peut être représentée par une loi binomiale dont on précisera les paramètres.\\

    \begin{itemize}
    \item Chaque épreuve a 2 issues possibles : Tomber sur la bonne réponse ou non.
    \item Les 3 questions sont indépendantes.
    \item La probabilité d'un succès est la même à chaque question.
    \end{itemize}

    Soit $X$ la variable aléatoire comptabilisant le nombre de bonne réponse par exercice. $X$ suit une loi binomiale de paramètre $n = 3$ et $p = \frac{1}{4}$. On note $X \sim \mathcal{B}(n,p)$ .

\item Réaliser un arbre représentant cette situation.\\
  \pstree[treemode=R,nodesep=5pt,levelsep=4cm]{\Tp}
         {
           \pstree{\TR{$S$}^{0,25}} {
             \pstree{\TR{$S$}^{0,25}} {
               \TR{$S : (\# 3) $}^{0,25}
               \TR{$\overline{S} : (\# 2) $}_{0,75}
             }
             \pstree{\TR{$\overline{S}$}^{0,75}} {
               \TR{$S : (\# 2) $}^{0,25}
               \TR{$\overline{S} : (\# 1) $}_{0,75}
             }
           }
           \pstree{\TR{$\overline{S}$}^{0,75}} {
             \pstree{\TR{$S$}^{0,25}} {
               \TR{$S : (\# 2)$}^{0,25}
               \TR{$\overline{S} : (\# 1)$}_{0,75}
             }
             \pstree{\TR{$\overline{S}$}^{0,75}} {
               \TR{$S : (\# 1)$}^{0,25}
               \TR{$\overline{S} : (\# 0)$}_{0,75}
             }
           }
         }

  \item Quelle est la probabilté que Jean-Heude réponde correctement à toutes les questions ?\\
    Pour cela, il Jean-Heude doit répondre correctement aux trois questions.\\
    $$p(X = 3) = \left( \dfrac{1}{4} \right)^{3} = \dfrac{1}{64} = 0.25 ^{4} = 0.0156$$
    La probabilité que Jean-Heude est la note maximale en répondant au hasard est de $0.0156$.
  \item Quelle est la probabilité que Jean-Heude est la moyenne en répondant correctement à plus de deux questions ?\\
    Jean-Heude a la moyenne s'il répond correctement à au moins deux questions. Il doit répondre à deux questions correctement ou trois.
    \subsubsection*{1 - Probabilité que Jean-Heude réponde correctement à exactement deux questions.}
    Trois chemins possibles nous mènent au résultat d'exactement deux réponses correctes.
    $$p(X = 2) = 3 \times \left( \dfrac{1}{4} \right)^{2} \times \left( \dfrac{3}{4} \right) = \dfrac{9}{64}$$

    \subsubsection*{2 - Probabilité que Jean-Heude est la moyenne.}
    On additionne les probabilités qu'il réponde correctement à toutes les questions, et celle qu'il réponde exactement à deux questions.
    $$p(X \leq 2) = p(X = 2) + p(X = 3) = \dfrac{9}{64} + \dfrac{1}{64} = \dfrac{10}{64} = 0.156$$
    La probabilité que Jean-Heude est la moyenne est répondant au hasard est $1.56$.
  \end{enumerate}

\end{document}

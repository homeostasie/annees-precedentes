%%%%%%%%%%%%%%%%%%%%%%%%%%%%%%%%%%%%%%%%%
% Short Sectioned Assignment
% LaTeX Template
% Version 1.0 (5/5/12)
%
% This template has been downloaded from:
% http://www.LaTeXTemplates.com
%
% Original author:
% Frits Wenneker (http://www.howtotex.com)
%
% License:
% CC BY-NC-SA 3.0 (http://creativecommons.org/licenses/by-nc-sa/3.0/)
%
%%%%%%%%%%%%%%%%%%%%%%%%%%%%%%%%%%%%%%%%%

%----------------------------------------------------------------------------------------
%	PACKAGES AND OTHER DOCUMENT CONFIGURATIONS
%----------------------------------------------------------------------------------------

\documentclass[paper=a4, fontsize=9pt]{scrartcl} % A4 paper and 11pt font size


\usepackage[T1]{fontenc} % Use 8-bit encoding that has 256 glyphs
\usepackage[english,francais]{babel} % Français et anglais
\usepackage[utf8]{inputenc} 

\usepackage{amsmath,amsfonts,amsthm} % Math packages

\usepackage{enumitem}
\usepackage{lmodern}
\usepackage{url}
\usepackage{eurosym} % signe Euros
\usepackage{geometry} % Pour passer au format A4
\geometry{a4paper} % 
\usepackage{graphicx} % Required for including pictures
\usepackage{float} % Allows putting an [H] in \begin{figure} to specify the exact location of the figure

\usepackage{multicol}
\usepackage{caption}
\usepackage{verbatim}
\usepackage{pst-node,pst-tree}

\usepackage{sectsty} % Allows customizing section commands
\allsectionsfont{\centering \normalfont\scshape} % Make all sections centered, the default font and small caps

%----------------------------------------------------------------------------------------
%	Pied de Page
%----------------------------------------------------------------------------------------


\usepackage{fancyhdr} % Custom headers and footers
\pagestyle{fancyplain} % Makes all pages in the document conform to the custom headers and footers
\fancyhead{} % No page header - if you want one, create it in the same way as the footers below
\fancyfoot[L]{$1^{ère}STI2D 2$} % Empty left footer
\fancyfoot[C]{Chapitre 2 - Probabilité} % Empty center footer
\fancyfoot[R]{\thepage} % Page numbering for right footer

\renewcommand{\headrulewidth}{0pt} % Remove header underlines
\renewcommand{\footrulewidth}{0pt} % Remove footer underlines

\setlength{\headheight}{13.6pt} % Customize the height of the header


\setlength\parindent{0pt} % Removes all indentation from paragraphs - comment this line for an assignment with lots of text


%----------------------------------------------------------------------------------------
%	Titre
%----------------------------------------------------------------------------------------

\newcommand{\horrule}[1]{\rule{\linewidth}{#1}} % Create horizontal rule command with 1 argument of height


\title{	
  \vspace{-10ex}
  \horrule{0.5pt} \\[0.4cm] % Thin top horizontal rule
  \huge Chapitre 2 - Probabilité\\ % The assignment title
  \horrule{2pt} \\[0.5cm] % Thick bottom horizontal rule
}

\author{}
\date{\vspace{-10ex}} % Today's date or a custom date

%----------------------------------------------------------------------------------------
%	Début du document
%----------------------------------------------------------------------------------------

\begin{document}

%----------------------------------------------------------------------------------------
% RE-DEFINITION
%----------------------------------------------------------------------------------------
% MATHS
%-----------

\newtheorem{Definition}{Définition}
\newtheorem{Theorem}{Théorème}
\newtheorem{Proposition}{Propriété}

% MATHS
%-----------
\renewcommand{\labelitemi}{$\bullet$}
\renewcommand{\labelitemii}{$\circ$}
%----------------------------------------------------------------------------------------
%	Titre
%----------------------------------------------------------------------------------------

\maketitle % Print the title
\setlength{\columnseprule}{1pt}

\section{Expérience de Bernoulli}

\subsection{Expérience de Milgram}

Stanley Milgram a commencé une série d'expérience en 1963 pour estimer quelle partie de la population est obéissante à une autorité et prête à infliger de sévère choc électrique à un étranger. Plusieurs recherches sur le sujet suggèrent que le pourcentage de la population obéissante est constant dans le temps et à travers les différentes communautés.

Chaque personne testée lors de cette expérience représente une \textbf{épreuve}. Chaque épreuve n'a que \textbf{deux issues possibles}. On parle de \textbf{succès}, si une peronne refuse d'infliger le choc le plus sévère en refusant d'obéir. On parle au contraire d'\textbf{échec} quand la personne obéit en accpetant d'infliger le choc le plus violent. De manière expérimentale, 35\% des personnes soumises à l'expérience refusent d'administrer le choc. Ce pourcentage représente la probabilité pour chaque épreuve qu'un candidat accepte d'administrer un choc. On note $p$ cette probabilité $p = 0.35$.

\subsection{Épreuve de Bernoulli}

Jacques Bernoulli est un mathématicien et physicien Suisse né en 1654 et mort en 1705. Il est principalement connu pour ses travaux sur les probabilités : \textit{Ars Conjectandi} et pour ses travaux sur les intérêts composés liés à la découverte du nombre $e$.

\begin{Definition}{Épreuve de Bernoulli}\\
  Une épreuve de Bernoulli n'a que deux issues possibles.
  \begin{itemize}
  \item Le succès $S$ de probabilité $p$.
  \item L'échec $E$ ou $\overline{S}$ de probabilité $1-p$.
  \end{itemize}
\end{Definition}

Soit $X$ une variable aléatoire suivant une loi de Bernoulli. X prend la valeur 1 pour un succès avec une probabilité p et la valeur 0 pour un échec.\\

$\mathbb{P}(X = 1) = p$ signifie que la probabilité pour une variable aléatoire suivant une loi de Bernoulli de prendre la valeur $1$ est $p$. C'est la probabilité d'obtenir un succès. De manière similaire, on définit $\mathbb{P}(X = 0) = 1-p$ la probabilité d'obtenir un échec.

\begin{Proposition}{Caractérisation}
  \begin{enumerate}
  \item L'espérance ou moyenne est $\mathbb{E}(X) = p$. Elle est également noté $\mu$.
  \item La variance est $V(X)  = p (1-p)$.
  \item L'écart-type est $\sigma(X) = \sqrt{V(X)} = \sqrt{p(1-p)}$.
  \end{enumerate}
\end{Proposition}

\newpage
\section{Loi Binomiale}

\subsection{Dans le contexte de l'expérience de Milgram}

On choisit $n$ personnes au hasard et on les soumet à l'expérience de Milgram. On s'intéresse alors \textbf{au nombre de succès}. On appelle cette série d'épreuves \textbf{un schéma de Bernoulli}.

\begin{Definition}{Schéma de Bernoulli de paramètre $n$ et $p$}\\
  Un schéma de Bernoulli possède les caractéristiques suivantes.
  \begin{enumerate}
  \item Il y a nombre fini $n$ d'épreuves.
  \item Chaque épreuve est indépendante des autres.
  \item Chaque épreuve ne possède deux issues qui peuvent être rangées comme un succès ou un échec.
  \item La probabilité d'un succès est toujours la même : $p$ à chaque épreuve.
  \end{enumerate}
\end{Definition}

Soit $X$ une variable aléatoire suivant un schéma de Bernoulli de paramètre $n$ et $p$. On dit également que $X$ suit une loi binomiale. On note $X \sim \mathcal{B}(n,p)$. X représente le nombre de succès à l'issue des $n$ tirages. $\mathbb{P}(X = 0)$ représente la probabilité d'obtenir zéro succès lors des $n$ tirages. La représentation graphique d'une telle variable aléatoire est un \textbf{histogramme}.\\


\begin{Proposition}{Caractérisation}
  \begin{enumerate}
  \item L'espérance est $\mathbb{E}(X) = np$.
  \item La variance est $V(X)  = n p (1-p)$.
  \item L'écart-type est $\sigma(X) = \sqrt{V(X)} = \sqrt{np(1-p)}$.
  \end{enumerate}
\end{Proposition}

Pour un nombre de tirage suffisamment petit $(n \leq 4)$, on peut calculer la probabilité à l'aide d'un arbre.

\subsection{Arbre}

\begin{itemize}
\item Sur chaque nœud, il n'y a que deux embranchements possibles : le succès noté S et l'échec noté E. 
\item On réalise autant de niveau d'embranchement qu'il y a de tirage.
\end{itemize}

Afin de calculer les probabilités avec un arbre.
\begin{enumerate}
\item Chaque branche représentant un succès est de probabilité $p$ et chaque branche représentant un échec est de probabilité $1-p$.
\item À la fin de chaque branche, on compte le nombre de succès qu'on peut noter avec un \#.
\item La probabilité d'arriver à la fin d'une branche est \textbf{le produit} des probabilités de chaque branche la composant.
\item Afin de calculer la probabilité d'un nombre $k$ de succès parmi les $n$ tirages, on \textbf{additionne} toutes les probabilités conduisant à ce même nombre $\#k$.
\end{enumerate}

\subsubsection{Dans le contexte de l'expérience de Milgram avec $n=3$}

On choisit trois personnes au hasard pour subir l'expérience de Milgram. La probabilité d'un succès est $p = 0.35$. On s'intéresse aux nombres de succès.\\

Cette épreuve est un schéma de Bernoulli : Il y a $n = 3$ épreuves indépendantes. À chaque épreuve, il n'y a que deux issues de probabilités $p=0.35$ pour un succès et de probabilité $1-p = 0.65$ pour un échec. On note $X \sim \mathcal{B}(3,0.35)$. 

\paragraph{Arbre}~~\\

\pstree[treemode=R,nodesep=5pt,levelsep=5cm]{\Tp}{
  \pstree{\TR{$S$}^{0,35}} {
    \pstree{\TR{$S$}^{0,35}} {
      \TR{$S : (\# 3) - \mathbb{P}(SSS)$}^{0,35}
      \TR{$\overline{S} : (\# 2) -  \mathbb{P}(SS\overline{S})$}_{0,65}
    }
    \pstree{\TR{$\overline{S}$}^{0,65}} {
      \TR{$S : (\# 2) - \mathbb{P}(S\overline{S}S)$}^{0,35}
      \TR{$\overline{S} : (\# 1)  - \mathbb{P}(S\overline{S}\overline{S})$}_{0,65}
    }
  }
  \pstree{\TR{$\overline{S}$}^{0,65}} {
    \pstree{\TR{$S$}^{0,35}} {
      \TR{$S : (\# 2) - \mathbb{P}(S\overline{S}S)$}^{0,35}
      \TR{$\overline{S} : (\# 1) - \mathbb{P}(\overline{S}S\overline{S}$}_{0,65}
    }
    \pstree{\TR{$\overline{S}$}^{0,65}} {
      \TR{$S : (\# 1)  - \mathbb{P}(\overline{S}\overline{S}S) $}^{0,35}
      \TR{$\overline{S} : (\# 0)   - \mathbb{P}(\overline{S}\overline{S}\overline{S})$}_{0,65}
    }
  }
}

\paragraph{Remarque}~~\\
Les probabilités menant au même nombre de succès sont égales. On peut donc remplacer : 
\begin{enumerate}
\item[4.] On \textbf{additionne} toutes les probabilités conduisant à ce même nombre $\#k$ \hspace{1cm} \textit{par} :
\item[4'.] On multiplie la probabilité d'arrivée au bout d'une branche par le nombre de chemin conduisant à ce même nombre de succès.
\end{enumerate}

\paragraph{Résultats}~~\\
\begin{enumerate}
\item[3.] $\mathbb{P}(X=3) = 1 \times 0.35^3  \simeq 0.043$
\item[2.] $\mathbb{P}(X=2) = 3 \times 0.35^2 \times 0.65 \simeq 0.24$ 
\item[1.] $\mathbb{P}(X=1) = 3 \times 0.35 \times 0.65^2 \simeq 0.44$ 
\item[2.] $\mathbb{P}(X=0) = 1 \times 0.65^3 \simeq 0.27$  
\end{enumerate}

\paragraph{Vocabulaire}~~\\
\begin{itemize}
\item La probabilité d'obtenir moins de deux succès :\\
$\mathbb{P}(X \leq 2) = \mathbb{P}(X = 0) + \mathbb{P}(X = 1) + \mathbb{P}(X = 2)$.
\item  La probabilité d'obtenir au moins deux succès est la probabilité d'obtenir plus de deux succès :\\
$\mathbb{P}(X \geq 2) = \mathbb{P}(X = 2) + \mathbb{P}(X = 3) = 1 - \mathbb{P}(X \leq 1).$
\end{itemize}


\newpage
\section{calcul numérique}

À partir d'un certain nombre d'épreuve les calculs à l'aide d'arbre deviennent fastidieux. Il est alors préférable d'utiliser les outils de calcul numérique disponibles. Pour un échantillon $n=35$ personnes sur qui on s'éssayerait à l'expérience de Milgram avec une probabilité $p=0.35$ de succès.

\subsection{Casio}
\begin{itemize}
\item La probrabilité que 14 personnes sur 35 réussissent l'éxpérience de Milgram\\
$\mathbb{P}(X = 14) = binompd(35, 0.35, 14)$.
\item La probrabilité que moins de 22 personnes sur 35 réussissent l'éxpérience de Milgram\\
$\mathbb{P}(X \leq 22) = binomcd(35, 0.35, 22)$.
\item La probrabilité qu'au moins 8 personnes sur 35 réussissent l'éxpérience de Milgram\\
$\mathbb{P}(X \leq 22) = 1 - binomcd(35, 0.35, 7)$.
\end{itemize}



\subsection{Texas Instrument}
\begin{itemize}
\item La probrabilité que 14 personnes sur 35 réussissent l'éxpérience de Milgram\\
$\mathbb{P}(X = 14) = binompdf(35, 0.35, 14)$.
\item La probrabilité que moins de 22 personnes sur 35 réussissent l'éxpérience de Milgram\\
$\mathbb{P}(X \leq 22) = binomcdf(35, 0.35, 22)$.
\item La probrabilité qu'au moins 8 personnes sur 35 réussissent l'éxpérience de Milgram\\
$\mathbb{P}(X \leq 22) = 1 - binomcd(35, 0.35, 7)$.
\end{itemize}

\subsection{R}

\begin{itemize}
\item La probrabilité que 14 personnes sur 35 réussissent l'éxpérience de Milgram\\
$\mathbb{P}(X = 14) = dbinom(14, 35,  0.35)$.
\item La probrabilité que moins de 22 personnes sur 35 réussissent l'éxpérience de Milgram\\
$\mathbb{P}(X \leq 22) = sum(binom(0:22,35, 0.35))$.
\item La probrabilité qu'au moins 8 personnes sur 35 réussissent l'éxpérience de Milgram\\
$\mathbb{P}(X \leq 22) = sum(binom(8:35, 35, 0.35))$.
\end{itemize}

\end{document}

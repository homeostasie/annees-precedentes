%%%%%%%%%%%%%%%%%%%%%%%%%%%%%%%%%%%%%%%%%
% LaTeX Template
% http://www.LaTeXTemplates.com
%
% Original author:
% Linux and Unix Users Group at Virginia Tech Wiki
% (https://vtluug.org/wiki/Example_LaTeX_chem_lab_report)
%
% License:
% CC BY-NC-SA 3.0 (http://creativecommons.org/licenses/by-nc-sa/3.0/)
%
%%%%%%%%%%%%%%%%%%%%%%%%%%%%%%%%%%%%%%%%%

%----------------------------------------------------------------------------------------
%	PACKAGES AND DOCUMENT CONFIGURATIONS
%----------------------------------------------------------------------------------------

\documentclass[11pt]{article}
\usepackage{geometry} % Pour passer au format A4
\geometry{hmargin=1cm, vmargin=1cm} %

\usepackage{graphicx} % Required for including pictures
\usepackage{float} %

%Français
\usepackage[T1]{fontenc}
\usepackage[english,francais]{babel}
\usepackage[utf8]{inputenc}
\usepackage{eurosym}
\usepackage{lmodern}
\usepackage{url}
\usepackage{multicol}

%Maths
\usepackage{amsmath,amsfonts,amssymb,amsthm}
%\usepackage[linesnumbered, ruled, vlined]{algorithm2e}
%\SetAlFnt{\small\sffamily}

%Autres
\linespread{1} % Line spacing
\setlength\parindent{0pt} % Removes all indentation from paragraphs

\renewcommand{\labelenumi}{\alph{enumi}.} %
\pagestyle{empty}
%----------------------------------------------------------------------------------------
%	DOCUMENT INFORMATION
%----------------------------------------------------------------------------------------
\begin{document}

%\maketitle % Insert the title, author and date

\begin{minipage}[t]{\textwidth}
  \raggedright
      {\bfseries 1 STI 2D 2}\\[.35ex]
      \vspace*{-1cm}
      \raggedleft
          {\bfseries Fonctions associées}\\[.35ex]
          {\bfseries 13 Novembre 2014}\\[.35ex]
\end{minipage}\\[1em]

\begin{center}
  \textsf{L'atterrisseur Philae de la sonde Rosetta de l'Agence spatiale européenne s'est posé sur la comète 67P-Churyumov-Gerasimenko après un voyage de dix ans. Le 12 novembre 2014 à 16h34}
\end{center}



\setlength{\columnseprule}{1pt}
\begin{multicols}{2}
\section*{I - Ça translate !}
Soient $f$, $g$ et $h$ les fonctions définies sur $[-6, 2]$ par :
\begin{itemize}
  \item $f(x) = x^2$
  \item $g(x) = f(x + 2)$
  \item $h(x) = g(x) - 4$
\end{itemize}

\begin{enumerate}
  \item[1.] Vérifier que $h(x) = x^2 + 4x$.
  \item[2.] Soit $\mathcal{C}_f$, $\mathcal{C}_g$ et $\mathcal{C}_h$ les courbes représentatives des fonctions $f$, $g$ et $h$. \\
  \begin{enumerate}
    \item[a)] Par quelle transformation géométrique passe-t-on de $\mathcal{C}_f$ à $\mathcal{C}_g$ ?
    \item[b)] Par quelle transformation géométrique passe-t-on de $\mathcal{C}_g$ à $\mathcal{C}_h$ ?
    \item[c)] Par quelle transformation géométrique passe-t-on de $\mathcal{C}_f$ à $\mathcal{C}_h$ ?
  \end{enumerate}
  \item[3.] Tracer les courbes $\mathcal{C}_f$, $\mathcal{C}_g$, $\mathcal{C}_h$ et $\mathcal{C}_u$ sur l'intervalle $[-6, 2]$.
\end{enumerate}

\section*{II - Ça bsolut !}
Soient $f$ la fonction définie sur $[-2, 6]$ par $f(x) = |x - 2|$ et $\mathcal{C}_f$ sa courbe représentative.
\begin{enumerate}
  \item[1.] Tracer $\mathcal{C}_f$ sur $[-2, 6]$.
  \item[2.] Déterminer graphiquement quand $f(x) = 2$.
  \item[3.] Résoudre algébriquement $f(x) = 2$.
\end{enumerate}
\end{multicols}
\end{document}

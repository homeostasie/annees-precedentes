%%%%%%%%%%%%%%%%%%%%%%%%%%%%%%%%%%%%%%%%%
% LaTeX Template
% http://www.LaTeXTemplates.com
%
% Original author:
% Linux and Unix Users Group at Virginia Tech Wiki 
% (https://vtluug.org/wiki/Example_LaTeX_chem_lab_report)
%
% License:
% CC BY-NC-SA 3.0 (http://creativecommons.org/licenses/by-nc-sa/3.0/)
%
%%%%%%%%%%%%%%%%%%%%%%%%%%%%%%%%%%%%%%%%%

%----------------------------------------------------------------------------------------
%	PACKAGES AND DOCUMENT CONFIGURATIONS
%----------------------------------------------------------------------------------------

\documentclass[11pt]{article}
\usepackage{geometry} % Pour passer au format A4
\geometry{hmargin=1cm, vmargin=1cm} % 

\usepackage{graphicx} % Required for including pictures
\usepackage{float} % 

%Français
\usepackage[T1]{fontenc} 
\usepackage[english,francais]{babel}
\usepackage[utf8]{inputenc}
\usepackage{eurosym}
\usepackage{lmodern}
\usepackage{url}
\usepackage{multicol}

%Maths
\usepackage{amsmath,amsfonts,amssymb,amsthm}
%\usepackage[linesnumbered, ruled, vlined]{algorithm2e}
%\SetAlFnt{\small\sffamily}

%Autres
\linespread{1} % Line spacing
\setlength\parindent{0pt} % Removes all indentation from paragraphs

\renewcommand{\labelenumi}{\alph{enumi}.} % 
\pagestyle{empty}
%----------------------------------------------------------------------------------------
%	DOCUMENT INFORMATION
%----------------------------------------------------------------------------------------
\begin{document}

%\maketitle % Insert the title, author and date

\begin{minipage}[t]{\textwidth}
  \raggedright
      {\bfseries 1 STI 2D 2}\\[.35ex]
      \vspace*{-1cm}
      \raggedleft
          {\bfseries Probabilité}\\[.35ex]
          {\bfseries 9 Octobre 2014}\\[.35ex]
\end{minipage}\\[1em]

\begin{center}
  \textsf{Les mathématiques ne sont une moindre immensité que la mer. - Victor Hugo}
\end{center}



\setlength{\columnseprule}{1pt}

\section*{QCM}
\subsection*{Données}

Soit un examen avec comme exerice un QCM (Questionnaire à choix multiples).
\begin{itemize}
\item Pour chaque question, il y a 4 réponses possibles. 
\item Une seule et uniquement une seule est bonne. 
\item L'exercice est composé de 3 questions indépendantes.
\item Jean-Heudes n'ayant vraiment pas travaillé décide de répondre vraiment au hasard.
\end{itemize}

\subsection*{Questions}
\begin{enumerate}
\item Quelle est la probabilité pour Jean-Heude de répondre correctement à la première question ?
\item Décrire pourquoi la situation peut être représentée par une loi binomiale dont on précisera les paramètres.
\item Réaliser un arbre représentant cette situation.
\item Quelle est la probabilté que Jean-Heude réponde correctement à toutes les questions ?
\item Quelle est la probabilité que Jean-Heude est la moyenne en répondant correctement à plus de deux questions ?
\end{enumerate}



\end{document}

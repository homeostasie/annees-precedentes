%%%%%%%%%%%%%%%%%%%%%%%%%%%%%%%%%%%%%%%%%
% LaTeX Template
% http://www.LaTeXTemplates.com
%
% Original author:
% Linux and Unix Users Group at Virginia Tech Wiki 
% (https://vtluug.org/wiki/Example_LaTeX_chem_lab_report)
%
% License:
% CC BY-NC-SA 3.0 (http://creativecommons.org/licenses/by-nc-sa/3.0/)
%
%%%%%%%%%%%%%%%%%%%%%%%%%%%%%%%%%%%%%%%%%

%----------------------------------------------------------------------------------------
%	PACKAGES AND DOCUMENT CONFIGURATIONS
%----------------------------------------------------------------------------------------

\documentclass[12pt]{article}
\usepackage{geometry} % Pour passer au format A4
\geometry{hmargin=1cm, vmargin=1cm} % 

\usepackage{graphicx} % Required for including pictures
\usepackage{float} % 

%Français
\usepackage[T1]{fontenc} 
\usepackage[english,francais]{babel}
\usepackage[utf8]{inputenc}
\usepackage{eurosym}
\usepackage{lmodern}
\usepackage{url}
\usepackage{multicol}

%Maths
\usepackage{amsmath,amsfonts,amssymb,amsthm}
%\usepackage[linesnumbered, ruled, vlined]{algorithm2e}
%\SetAlFnt{\small\sffamily}

%Autres
\linespread{1} % Line spacing
\setlength\parindent{0pt} % Removes all indentation from paragraphs

\renewcommand{\labelenumi}{\alph{enumi}.} % 
\pagestyle{empty}
%----------------------------------------------------------------------------------------
%	DOCUMENT INFORMATION
%----------------------------------------------------------------------------------------
\begin{document}

%\maketitle % Insert the title, author and date

\begin{minipage}[t]{\textwidth}
  \raggedright
      {\bfseries Série A}\\[.35ex]
      {\bfseries 1 STI 2D 2}\\[.35ex]
      {\bfseries Nom : }\\[.35ex]
      \vspace*{-1cm}
      \raggedleft
          {\bfseries Suites}\\[.35ex]
          {\bfseries 31 Mars 2015}\\[.35ex]
\end{minipage}\\[1em]

\begin{center}
  \textsf{La normalité est une route pavée : on y marche aisément mais les fleurs n'y poussent pas. - Vincent Van Gogh}\\
\end{center}

\setlength{\columnseprule}{1pt}

\begin{multicols}{2}
\subsection*{1 - Fibonacci}

Leonardo Fibonacci mathématicien italien du $XIII^e$ siècle, proposa le problème suivant.

`` Un homme met un couple de lapins dans un lieu isolé. Combien de couples obtient-on en un an si chaque couple engendre tous les mois un nouveau couple à compter du troisième mois de son existence ?''\\

Si on note $C_n$ le nombre de couples de lapins le n-ième mois, on a alors : $C_1 = 1$, $C_2 = 1$, $C_3 = 2$, $C_4 = 1 + 2 = 3$, $C_5 = 2 + 3 = 5$...\\
D'une manière plus générale : le $n-ième$ mois.
$$C_n = C_{n-2} + C_{n-1}$$.

\begin{enumerate}
\item[1.] Calculer les valeurs de $C_6$, $C_7$ et $C_8$.
\item[2.] Combien de couples de lapin y-a-t'il au bout d'un an ?
\item[3.] Ce modèle est-il réaliste ? quelle critique peut-on émettre ?
\end{enumerate}

\noindent\hrulefill

\subsection*{2 - Abonnés}

Le service abonnement du monde diplomatique constate pour chaque année les éléments suivants.

\begin{itemize}
\item En 2015, $4000$ personnes sont abonnées.
\item 100 nouveaux abonnés sont enregistrés.
\item la moitié des abonnées de l’année précédente ne renouvellent pas leur abonnement.
\end{itemize}
On note $a_n$ le nombre d’abonnées de l’année $2015 + n$, ainsi, $a_0 = 4000$.

\begin{enumerate}
\item[1.] Déterminer le nombre d’abonnées en 2016 et 2017.
\item[2.] Calculer $a_1$,  $a_2$,  $a_3$ et  $a_4$.
\item[3.] Exprimer le nombre d’abonnées $a_{n+1}$ en fonction du nombre d’abonnées $a_n$.
\item[4.] La suite $(a_n)_n$ est-elle géométrique ? 
\end{enumerate}

\subsection*{3 - Habitants}

Dans Un village du Jura.
\begin{itemize}
\item En 2015, le village compte 2000 habitants
\item Chaque année sa population baisse de 5 \%.
\end{itemize}
Soit $P_n$ sa population l’année $2015 + n$, ainsi $P_0 = 2000$

\begin{enumerate}
\item[1.] Calculer $P_1$ , $P_2$ et $P_3$.
\item[2.] Quelle est la nature de la suite $(P_n)_n$ ?
\item[3.] Donner l’expression de $P_n$ en fonction de $n$.
\item[4.] La suite $(P_n)_n$ admet-elle une limite ? si oui, laquelle ?
\end{enumerate}

\noindent\hrulefill

\subsection*{4 - Échecs}

En Inde, un roi, à qui un mathématicien venait de présenter le jeu d'échec, fut si émerveillé qu'il lui proposa de choisir lui-même sa récompense. \\
Le mathématicien demanda au roi de le récompenser en grains de blé de la façon suivante :

\begin{itemize}
\item sur la $1^{ère}$ case de l'échiquier : 1 grain de blé.
\item sur la $2^{ème}$ case de l'échiquier : 2 grain de blé.
\item sur la $3^{ème}$ case de l'échiquier : 4 grain de blé.
\item et ainsi de suite, en déposant sur chaque nouvelle case le double de grains de blé de celui de la case précédente.
\end{itemize}

Un  échiquier comporte 64 cases. On note $u_1$ le nombre de grains de blé sur la $1^{ère}$ case, $u_2$ sur la $2^{ème}$ case...

\begin{enumerate}
\item[1.] Quelle est la nature de la suite $(u_n)_n$ ?
\item[2.] Quelle est le nombre de grain de riz sur la dernière case ?
\item[3.] Bonus : On note S la somme des grains de blé sur l'échiquier.
Écrire un algorithme qui permet de calculer et d’afficher S - en français, ti ou casio.
Programmer sur calculatrice cet algorithme, et donner la valeur de S. Si un grain de riz pèse 0,02g, donner, en tonnes, le poids de blé sur l'échiquier.
\end{enumerate}

\end{multicols}
\end{document}

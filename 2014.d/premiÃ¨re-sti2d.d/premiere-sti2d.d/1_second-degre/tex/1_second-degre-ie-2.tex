%%%%%%%%%%%%%%%%%%%%%%%%%%%%%%%%%%%%%%%%%
% LaTeX Template
% http://www.LaTeXTemplates.com
%
% Original author:
% Linux and Unix Users Group at Virginia Tech Wiki 
% (https://vtluug.org/wiki/Example_LaTeX_chem_lab_report)
%
% License:
% CC BY-NC-SA 3.0 (http://creativecommons.org/licenses/by-nc-sa/3.0/)
%
%%%%%%%%%%%%%%%%%%%%%%%%%%%%%%%%%%%%%%%%%

%----------------------------------------------------------------------------------------
%	PACKAGES AND DOCUMENT CONFIGURATIONS
%----------------------------------------------------------------------------------------

\documentclass[11pt]{article}
\usepackage{geometry} % Pour passer au format A4
\geometry{hmargin=1cm, vmargin=1cm} % 

\usepackage{graphicx} % Required for including pictures
\usepackage{float} % 

%Français
\usepackage[T1]{fontenc} 
\usepackage[english,francais]{babel}
\usepackage[utf8]{inputenc}
\usepackage{eurosym}
\usepackage{lmodern}
\usepackage{url}
\usepackage{multicol}

%Maths
\usepackage{amsmath,amsfonts,amssymb,amsthm}
%\usepackage[linesnumbered, ruled, vlined]{algorithm2e}
%\SetAlFnt{\small\sffamily}

%Autres
\linespread{1} % Line spacing
\setlength\parindent{0pt} % Removes all indentation from paragraphs

\renewcommand{\labelenumi}{\alph{enumi}.} % 
\pagestyle{empty}
%----------------------------------------------------------------------------------------
%	DOCUMENT INFORMATION
%----------------------------------------------------------------------------------------
\begin{document}

%\maketitle % Insert the title, author and date

\begin{minipage}[t]{\textwidth}
  \raggedright
      {\bfseries Série B}\\[.35ex]
      {\bfseries 1 STI 2D 2}\\[.35ex]
      \vspace*{-1cm}
      \raggedleft
          {\bfseries Second degré}\\[.35ex]
          {\bfseries 16 Septembre 2014}\\[.35ex]
\end{minipage}\\[1em]

\begin{center}
  \textsf{Yes, we’re being bought by Microsoft - Mojang - 2.5 Millards \$}\\
  \textsf{I’m leaving Mojang - It’s not about the money. It’s about my sanity. - Notch}
\end{center}

\setlength{\columnseprule}{1pt}
\begin{multicols}{2}
  \subsection*{Exercice 1 - Schéma}
  \begin{figure}[H]
    \centering
    \includegraphics[width=0.6\linewidth]{sources/ie/ie-porte.pdf}
  \end{figure}
  \subsection*{Données}
  Soit ABCD une porte rectangulaire dont on cherche les dimensions. On essaye de faire passer un bâton dont on ne connaît la longueur et on prend trois mesures : 
  \begin{itemize}
  \item La diagonale de la porte fait la taille exacte du bâton.
  \item En position verticale, le bâton mesure $20cm$ de plus que la porte.
  \item En position horizontale, le bâton mesure $40cm$ de plus que la porte.
  \end{itemize}

  \subsection*{Objectif}
  \textbf{Trouver la hauteur et la largeur de la porte.}
  \textit{On se propose de poser $x$ la taille du bâton.}
\end{multicols}

\subsection*{Exercice 2 - Résoudre dans $\mathbb{R}$ }
\begin{multicols}{2}
  \begin{enumerate}
  \item $2x^2 + x - \dfrac{1}{2} = 0$
  \item $3x^2 + 2x + \dfrac{1}{3} = 0$
  \end{enumerate}
\end{multicols}
\end{document}

%%%%%%%%%%%%%%%%%%%%%%%%%%%%%%%%%%%%%%%%%
% Short Sectioned Assignment
% LaTeX Template
% Version 1.0 (5/5/12)
%
% This template has been downloaded from:
% http://www.LaTeXTemplates.com
%
% Original author:
% Frits Wenneker (http://www.howtotex.com)
%
% License:
% CC BY-NC-SA 3.0 (http://creativecommons.org/licenses/by-nc-sa/3.0/)
%
%%%%%%%%%%%%%%%%%%%%%%%%%%%%%%%%%%%%%%%%%

%----------------------------------------------------------------------------------------
%	PACKAGES AND OTHER DOCUMENT CONFIGURATIONS
%----------------------------------------------------------------------------------------

\documentclass[paper=a4, fontsize=9pt]{scrartcl} % A4 paper and 11pt font size


\usepackage[T1]{fontenc} % Use 8-bit encoding that has 256 glyphs
\usepackage[english,francais]{babel} % Français et anglais
\usepackage[utf8]{inputenc} 

\usepackage{amsmath,amsfonts,amsthm} % Math packages

\usepackage{enumitem}
\usepackage{lmodern}
\usepackage{url}
\usepackage{eurosym} % signe Euros
\usepackage{geometry} % Pour passer au format A4
\geometry{a4paper} % 
\usepackage{graphicx} % Required for including pictures
\usepackage{float} % Allows putting an [H] in \begin{figure} to specify the exact location of the figure

\usepackage{multicol}
\usepackage{caption}
\usepackage{verbatim}

\usepackage{sectsty} % Allows customizing section commands
\allsectionsfont{\centering \normalfont\scshape} % Make all sections centered, the default font and small caps

%----------------------------------------------------------------------------------------
%	Pied de Page
%----------------------------------------------------------------------------------------


\usepackage{fancyhdr} % Custom headers and footers
\pagestyle{fancyplain} % Makes all pages in the document conform to the custom headers and footers
\fancyhead{} % No page header - if you want one, create it in the same way as the footers below
\fancyfoot[L]{$2^{nd}1$} % Empty left footer
\fancyfoot[C]{Chapitre 1 - Second degré} % Empty center footer
\fancyfoot[R]{\thepage} % Page numbering for right footer

\renewcommand{\headrulewidth}{0pt} % Remove header underlines
\renewcommand{\footrulewidth}{0pt} % Remove footer underlines

\setlength{\headheight}{13.6pt} % Customize the height of the header


\setlength\parindent{0pt} % Removes all indentation from paragraphs - comment this line for an assignment with lots of text


%----------------------------------------------------------------------------------------
%	Titre
%----------------------------------------------------------------------------------------

\newcommand{\horrule}[1]{\rule{\linewidth}{#1}} % Create horizontal rule command with 1 argument of height


\title{	
  \vspace{-10ex}
  \horrule{0.5pt} \\[0.4cm] % Thin top horizontal rule
  \huge Chapitre 1 - Second degré\\ % The assignment title
  \horrule{2pt} \\[0.5cm] % Thick bottom horizontal rule
}

\author{}
\date{\vspace{-10ex}} % Today's date or a custom date

%----------------------------------------------------------------------------------------
%	Début du document
%----------------------------------------------------------------------------------------

\begin{document}

%----------------------------------------------------------------------------------------
% RE-DEFINITION
%----------------------------------------------------------------------------------------
% MATHS
%-----------

\newtheorem{Definition}{Définition}
\newtheorem{Theorem}{Théorème}
\newtheorem{Proposition}{Propriété}

% MATHS
%-----------
\renewcommand{\labelitemi}{$\bullet$}
\renewcommand{\labelitemii}{$\circ$}
%----------------------------------------------------------------------------------------
%	Titre
%----------------------------------------------------------------------------------------

\maketitle % Print the title
\setlength{\columnseprule}{1pt}
\begin{multicols}{2}

  %-----------------------------------111111111111111111111111111111111111
  \section{Fonction du second degré}
  %-----------------------------------------------------------------------

  Le polynôme du second degré $a x^2 + b x + c$ est appelé un \textbf{trinôme du second degré} si $a \neq 0$ et $a, b, c \in \mathbb{R}$. La fonction est définie sur $\mathbb{R}$ tout entier. On appelle \textbf{parabole} la courbe représentative $\mathcal{C}_f$ de tous les polynômes du second degré.

  \begin{figure}[H]
    \centering
    \includegraphics[width=0.9\linewidth]{sources/cours/para.pdf}
    \caption{$f(x) = 2x^2 -4x + 1$}
  \end{figure}

  \section{Équation du second degré}

  L'équation $a x^2 + b x + c = 0$ consiste à trouver l'intersection entre la parabole et l'axe des abscisses a été originellement résolu par \textbf{Al-Khwarizmi} - \textit{activité 1}. Il est maintenant possible de trouver les deux solutions à l'aide du calcul du \textbf{discriminant}.

  \subsection{Calcul du discriminant}

  Soit $\Delta$ le discriminant. En fonction de son signe, on en déduit les solutions de l'éqution $a x^2 + b x + c = 0$.

  $$\Delta = b^2 - 4ac$$
\end{multicols}

\subsection{Résolution de l'équation}

\subsubsection{Cas : $\Delta < 0$}

Lorsque $\Delta < 0$, l'équation $a x^2 + b x + c = 0$ n'admet pas de solution dans $\mathbb{R}$.

\subsubsection{Cas : $\Delta = 0$}

Lorsque $\Delta = 0$, l'équation $a x^2 + b x + c = 0$ admet \textbf{une solution double.} On la note $x$.
$$x = -\dfrac{b}{2a}$$


\subsubsection{Cas : $\Delta > 0$}

Lorsque $\Delta > 0$, l'équation $a x^2 + b x + c = 0$ admet \textbf{deux solutions.} On les note $x_1$ et $x_2$.

$$x_1 = \dfrac{-b - \sqrt{\Delta}}{2a} \text{ et } x_2 = \dfrac{-b + \sqrt{\Delta}}{2a}$$
\newpage
\section{Étude des paraboles}

\subsection{Sens de variation}

Le sens de variation de la parabole représentative de la fonction $f(x) = ax^2 + bx + c$ \textbf{dépend uniquement du signe de a}.

\subsubsection{Cas $a > 0$}
Si $a > 0$, alors la parabole est décroissante puis croissante. Son minimum est atteint  en $-\frac{b}{2a}$ et prend la valeur $f( -\frac{b}{2a} ) $.

\begin{multicols}{3}
  \begin{figure}[H]
    \centering
    \includegraphics[height = 4cm]{sources/cours/para-ap-dneg.pdf}
    \caption*{Cas : $a > 0$ et $\Delta < 0$}
  \end{figure}

  \begin{figure}[H]
    \centering
    \includegraphics[height = 4cm]{sources/cours/para-ap-dnul.pdf}
    \caption*{Cas : $a > 0$ et $\Delta = 0$}
  \end{figure}

  \begin{figure}[H]
    \centering
    \includegraphics[height = 4cm]{sources/cours/para-ap-dp.pdf}
    \caption*{Cas : $a > 0$ et $\Delta > 0$}
  \end{figure}
\end{multicols}

\subsubsection{Cas $a < 0$}
Si $a < 0$, alors la parabole est croissante puis décroissante. Son maximum est atteint  en $-\frac{b}{2a}$ et prend la valeur $f( -\frac{b}{2a} )$.

\begin{multicols}{3}
  \begin{figure}[H]
    \centering
    \includegraphics[height = 4cm]{sources/cours/para-an-dneg.pdf}
    \caption*{Cas : $a < 0$ et $\Delta < 0$}
  \end{figure}

  \begin{figure}[H]
    \centering
    \includegraphics[height = 4cm]{sources/cours/para-an-dnul.pdf}
    \caption*{Cas : $a < 0$ et $\Delta = 0$}
  \end{figure}

  \begin{figure}[H]
    \centering
    \includegraphics[height = 4cm]{sources/cours/para-an-dp.pdf}
    \caption*{Cas : $a < 0$ et $\Delta > 0$}
  \end{figure}
\end{multicols}

\subsection{Étude de signe}
Le trinôme $ax^2 + bx + c$ est \textbf{du signe de $a$ à l'extérieur des racines}.

\end{document}

%%%%%%%%%%%%%%%%%%%%%%%%%%%%%%%%%%%%%%%%%
% LaTeX Template
% http://www.LaTeXTemplates.com
%
% Original author:
% Linux and Unix Users Group at Virginia Tech Wiki 
% (https://vtluug.org/wiki/Example_LaTeX_chem_lab_report)
%
% License:
% CC BY-NC-SA 3.0 (http://creativecommons.org/licenses/by-nc-sa/3.0/)
%
%%%%%%%%%%%%%%%%%%%%%%%%%%%%%%%%%%%%%%%%%

%----------------------------------------------------------------------------------------
%	PACKAGES AND DOCUMENT CONFIGURATIONS
%----------------------------------------------------------------------------------------

\documentclass[12pt]{article}
\usepackage{geometry} % Pour passer au format A4
\geometry{hmargin=1cm, vmargin=1cm} % 

\usepackage{graphicx} % Required for including pictures
\usepackage{float} % 

%Français
\usepackage[T1]{fontenc} 
\usepackage[english,francais]{babel}
\usepackage[utf8]{inputenc}
\usepackage{eurosym}
\usepackage{lmodern}
\usepackage{url}
\usepackage{multicol}

%Maths
\usepackage{amsmath,amsfonts,amssymb,amsthm}
%\usepackage[linesnumbered, ruled, vlined]{algorithm2e}
%\SetAlFnt{\small\sffamily}

%Autres
\linespread{1} % Line spacing
\setlength\parindent{0pt} % Removes all indentation from paragraphs

\renewcommand{\labelenumi}{\alph{enumi}.} % 
\pagestyle{empty}
%----------------------------------------------------------------------------------------
%	DOCUMENT INFORMATION
%----------------------------------------------------------------------------------------
\begin{document}

%\maketitle % Insert the title, author and date

\begin{minipage}[t]{\textwidth}
  \raggedright
      {\bfseries 1 STI 2D 2}\\[.35ex]
      {\bfseries Nom : }\\[.35ex]
      \vspace*{-1cm}
      \raggedleft
          {\bfseries Statistiques}\\[.35ex]
          {\bfseries 19 Mai}\\[.35ex]
\end{minipage}\\[1em]

\begin{center}
  \textsf{La normalité est une route pavée : on y marche aisément mais les fleurs n'y poussent pas. - Vincent Van Gogh}\\
\end{center}

\setlength{\columnseprule}{1pt}

\begin{multicols}{2}

  \subsection*{1 - Marché des portables}

  \begin{center}
    \begin{tabular}{| c || c | c | c | c | c |}
      \hline
      Mois    & 1  & 2  & 3  &  4 & 5 \\
      \hline
      Apple   & 7  & 9  & 12 & 14 & 15\\
      \hline
      Xiaomi  & 8  & 10 & 12 & 13 & 14\\
      \hline
      Samsung & 20 & 18 & 15 & 12 & 9 \\
      \hline
    \end{tabular}
  \end{center}


  `` On s'intéresse au part de marché en Chine de 3 grandes grandes marques de téléphones portables : Apple, Samsung et Xiaomi. L'unité est le million.''\\

  \begin{enumerate}
  \item[1.] Écrire la formule de la moyenne. Puis pour \textbf{chacune} des marques, calculer la moyenne du nombre de portables vendus. 
  \item[2.] Proposer un graphique représentant de manière pertinente l'évolution de ces ventes au cours de ses dernièrs mois.
  \item[3.] En deux, trois lignes, proposer un commentaire sur l'évolution des ventes.
  \end{enumerate}

\end{multicols}

\subsection*{2 - Température dans le monde}

\subsubsection*{Donnée}

\begin{center}
  \begin{tabular}{| c || c | c | c | c | c | c | c | c | c | c | c | c |}
    \hline
    2014 - Mois          & 01 & 02 & 03 & 04 & 05 & 06 & 07 & 08 & 09 & 10 & 11 & 12 \\
    \hline
    Concentration & 150 & 182 & 115 & 118 & 120 & 124 & 129 & 132 & 128 & 122 & 115 & 193 \\
    \hline
  \end{tabular}
\end{center}

\begin{center}
  \begin{tabular}{| c || c | c | c | c | c | c | c | c | c |}
    \hline
    Année         & 1800 & 1900 & 1950 & 1960 & 1970 & 1980 & 1990 & 2000 & 2010\\
    \hline
    Concentration &  0.4 &  0.5  & 0.9 & 2.6  & 12   & 19   & 32   & 82   & 98  \\
    \hline
  \end{tabular}
\end{center}

\begin{multicols}{2}

  \subsubsection*{Année 2014}

  `` On suit l'évolution de la concentration des gaz à éffets de serre dans l'atmosphère au cours du temps en Antartique durant l'année 2014.''\\

  \begin{enumerate}
  \item[1.] Quelle est la température moyenne en Europe en 2014 ? 
  \item[2.] Quelle est la médiane de cette série. À l'aide d'une phrase expliciter le sens de la médiane. 
  \item[3.] Calculer les quartiles Q1 et Q3.
  \item[4.] Tracer une boite à moustache (ou boxplot) regroupant ses informations. 
  \end{enumerate}

  \subsubsection*{Au fil du temps}

  `` On se propose maintenant d'étudier la concentration moyenne par an de ces mêmes gaz à effet de serre mais au fil des derniers siècles.''

  \begin{enumerate}
  \item[1.] Calculer la moyenne de cette série. 
  \item[2.] Calculer la médiane de cette série. 
  \item[3.] Calculer la variance et l'écart type de cette série.
  \item[4.] En deux, trois lignes, proposer une raison à cette augmentation de concentration des gaz à éffet de serre au cours du temps.
  \end{enumerate}

\end{multicols}

\begin{multicols}{2}

\subsection*{3 - Temps de jeu}

\begin{center}
  \begin{tabular}{| c | c | c |}
    \hline
    Temps & Centre & Effectif \\
    \hline
    $[ 0 ;  10[$ & \phantom{azertyuiop} & 3 \\ 
        \hline
        $[10 ;  30[$ & \phantom{azertyuiop} & 5 \\ 
            \hline
            $[30 ;  60[$ & \phantom{azertyuiop} & 8 \\ 
                \hline
                $[60 ;  90[$ & \phantom{azertyuiop} & 10 \\ 
                    \hline
                    $[90 ; 120[$ & \phantom{azertyuiop} & 6 \\ 
                        \hline
                        $[120; 180[$ & \phantom{azertyuiop} & 7 \\ 
                            \hline
                            $[180; 360[$ & \phantom{azertyuiop} & 9 \\ 
                                \hline
  \end{tabular}
\end{center}

`` On regarde le temps passé (en minute) sur un pc par une promotion d'élèves d'une classe de sti2D.''


\begin{enumerate}
\item[1.] Remplir les centres de chaque intervalle dans le tableau. 
\item[2.] Calculer la moyenne de cette série.
\item[3.] Calculer la médiane de cette série.
\item[4.] Après avoir donner votre temps passer sur un pc. Comment l'ajout de votre temps dans la série l'influence-t-elle celle-ci en terme de moyenne et de médiance ?
\end{enumerate}

\end{multicols}
\end{document}

%%%%%%%%%%%%%%%%%%%%%%%%%%%%%%%%%%%%%%%%%
% LaTeX Template
% http://www.LaTeXTemplates.com
%
% Original author:
% Linux and Unix Users Group at Virginia Tech Wiki 
% (https://vtluug.org/wiki/Example_LaTeX_chem_lab_report)
%
% License:
% CC BY-NC-SA 3.0 (http://creativecommons.org/licenses/by-nc-sa/3.0/)
%
%%%%%%%%%%%%%%%%%%%%%%%%%%%%%%%%%%%%%%%%%

%----------------------------------------------------------------------------------------
%	PACKAGES AND DOCUMENT CONFIGURATIONS
%----------------------------------------------------------------------------------------

\documentclass[11pt]{article}
\usepackage{geometry} % Pour passer au format A4
\geometry{hmargin=1cm, vmargin=1cm} % 

\usepackage{graphicx} % Required for including pictures
\usepackage{float} % 

%Français
\usepackage[T1]{fontenc} 
\usepackage[english,francais]{babel}
\usepackage[utf8]{inputenc}
\usepackage{eurosym}
\usepackage{lmodern}
\usepackage{url}
\usepackage{multicol}

%Maths
\usepackage{amsmath,amsfonts,amssymb,amsthm}
%\usepackage[linesnumbered, ruled, vlined]{algorithm2e}
%\SetAlFnt{\small\sffamily}

%Autres
\linespread{1} % Line spacing
\setlength\parindent{0pt} % Removes all indentation from paragraphs

\renewcommand{\labelenumi}{\alph{enumi}.} % 
\pagestyle{empty}
%----------------------------------------------------------------------------------------
%	DOCUMENT INFORMATION
%----------------------------------------------------------------------------------------
\begin{document}

%\maketitle % Insert the title, author and date

\begin{minipage}[t]{\textwidth}
  \raggedright
      {\bfseries 1 STI 2D 2}\\[.35ex]
      \vspace*{-1cm}
      \raggedleft
          {\bfseries Second degré}\\[.35ex]
          {\bfseries 16 Septembre 2014}\\[.35ex]
\end{minipage}\\[1em]

\begin{center}
  \textsf{Yes, we’re being bought by Microsoft - Mojang - 2.5 Millards \$}\\
  \textsf{I’m leaving Mojang - It’s not about the money. It’s about my sanity. - Notch}
\end{center}



\setlength{\columnseprule}{1pt}
\section{Exercice 1 - La porte}
\begin{multicols}{2}

  \subsection*{Shéma}

  \begin{figure}[H]
    \centering
    \includegraphics[width=0.6\linewidth]{sources/ie/ie-porte.pdf}
  \end{figure}
  \subsection*{Données}
  Soit ABCD une porte rectangulaire dont on cherche les dimensions. On essaye de faire passer un bâton dont on ne connaît la longueur et on prend trois mesures : 
  \begin{itemize}
  \item La diagonale de la porte fait la taille exacte du bâton.
  \item En position verticale, le bâton mesure $20cm$ de plus que la porte.
  \item En position horizontale, le bâton mesure $40cm$ de plus que la porte.
  \end{itemize}

  \subsection*{Objectif}

  \textbf{Trouver la hauteur et la largeur de la porte.}\\
  \textit{On se propose de poser $x$ la taille du bâton.}
\end{multicols}

\noindent\hrulefill

\begin{multicols}{2}

  \subsection{Mise en équation du problème}
  
  \begin{itemize}
  \item Hauteur de la porte : $x - 20$.
  \item Largeur de la porte : $x - 40$.
  \item Diagonale de la porte : $x$.
  \end{itemize}
  
  Dans le triangle BCD, rectangle en C, on utilise le théorème de Pythagore :\\

  \begin{eqnarray*}
    DC^2 + CB^2 & = & BD^2\\
    (x-20)^2 + (x - 40)^2 & = & x^2
  \end{eqnarray*}



  \subsection{Réduction de l'équation}

  \begin{eqnarray*}
    (x^2 - 2*20*x + 20^2) + (x^2 - 2*40*x + 40^2) &=& x^2\\
    (x^2 + x^2 - x^2) + x(40 + 80) + (400 + 1600) &=& 0\\
    x^2 - 120x + 2000 &=& 0
  \end{eqnarray*}



  \subsection{Résolution de l'équation : $ x^2 - 120x + 2000 = 0$}
  Pour résoudre une équation du second degré, on calule le \textbf{discriminant}.

  $$\Delta = 120^2 - 4*(1)(2000) = 14400 - 8000 = 6400 = 80^2$$
  
  Comme $\Delta > 0$, l'équation possède deux solutions.
  \begin{eqnarray*}
    \left\{
    \begin{aligned}
      x_1 &=& \dfrac{120 - 80}{2*1} = \dfrac{40}{2} = 20\\
      x_2 &=& \dfrac{120 + 80}{2*1} = \dfrac{200}{2} = 100
    \end{aligned}
    \right.
  \end{eqnarray*}

  La taille du bâton est  $x_1 = 20cm$ ou  $x_2 = 100cm$. On en déduit les dimensions de la porte. \\
  \begin{eqnarray*}
    \left\{
    \begin{aligned}
      h_1 &=& 20 - 20 = 0\\
      l_1 &=& 20-40 = -20
    \end{aligned}
    \right.
  \end{eqnarray*}

  \textit{Cette solution $h_1 , l_1$ est incompatible avec les dimensions d'une porte.}

  \begin{eqnarray*}
    \left\{
    \begin{aligned}
      h_2 &=& 100 - 20 = 80\\
      l_2 &=& 100 - 40 = 60
    \end{aligned}
    \right.
  \end{eqnarray*}
  
  \textbf{La porte mesure $80cm$ de haut et $60cm$ de large.}

\end{multicols}

\newpage


\section{Exercice 2 - Résoudre dans $\mathbb{R}$ }

\begin{multicols}{2}
\subsection{$x^2 + 2x - 3 = 0$} 
  Pour résoudre une équation du second degré, on calule le \textbf{discriminant}.
  $$\Delta = 2^2 - 4(1)(-3) = 4 + 12 = 16 = 4^2$$

  Comme $\Delta > 0$, l'équation possède deux solutions.


  \begin{eqnarray*}
    \left\{
    \begin{aligned}
      x_1 &=& \dfrac{-2 - 4}{2*1} = \dfrac{-6}{2} = -3\\
      x_2 &=& \dfrac{-2 + 4}{2*1} = \dfrac{2}{2} = 1
    \end{aligned}
    \right.
  \end{eqnarray*}

  \textbf{Les solutions sont $x_1 = -3$ et $x_2 = 1$.}

  \begin{figure}[H]
    \centering
    \includegraphics[width=0.6\linewidth]{sources/cor/2-1.pdf}
  \end{figure}

\end{multicols}

\begin{multicols}{2}
\subsection{$x^2 + 6x + 9 = 0$}

  Pour résoudre une équation du second degré, on calule le \textbf{discriminant}.
  $$\Delta = 6^2 - 4(1)(9) = 36 - 36 = 0$$

  $\Delta = 0$. L'équation possède une solution double.

  $$x = \dfrac{-6}{2*1} = \dfrac{-6}{2} = -3$$
  
  \textbf{La solution est $x = -3$.}

  \begin{figure}[H]
    \centering
    \includegraphics[width=0.6\linewidth]{sources/cor/2-2.pdf}
  \end{figure}

\end{multicols}

\noindent\hrulefill

\begin{multicols}{2}
\subsection{$2x^2 + x - \dfrac{1}{2} = 0$}

  Pour résoudre une équation du second degré, on calule le \textbf{discriminant}.
  $$\Delta = 1^2 + 4(2)(\dfrac{1}{2}) = 1 + 4 = 5$$

 Comme $\Delta > 0$, l'équation possède deux solutions.


  \begin{eqnarray*}
    \left\{
    \begin{aligned}
      x_1 &=& \dfrac{-1 - \sqrt(5)}{2*2} = \dfrac{-1 - \sqrt(5)}{4}\\
     x_1 &=& \dfrac{-1 + \sqrt(5)}{2*2} = \dfrac{-1 + \sqrt(5)}{4}
    \end{aligned}
    \right.
  \end{eqnarray*}

  \textbf{Les solutions sont $x_1 = -\dfrac{1 - \sqrt(5)}{4}$ et $x_2 = \dfrac{-1 - \sqrt(5)}{4} $.}


  \begin{figure}[H]
    \centering
    \includegraphics[width=0.6\linewidth]{sources/cor/2-3.pdf}
  \end{figure}

\end{multicols}


\begin{multicols}{2}
\subsection{$3x^2 + 2x + \dfrac{1}{3} = 0$}
  Pour résoudre une équation du second degré, on calule le \textbf{discriminant}.
  $$\Delta = 2^2 - 4 * 3 * \frac{1}{3} = 4 - 4 = 0.$$

  Comme $\Delta = 0$, l'équation possède une solution.

  $$x = \dfrac{-2}{2*3} = \dfrac{-2}{6} = -\dfrac{1}{3}$$
  \textbf{La solution est $-\dfrac{1}{3}$.}

  \begin{figure}[H]
    \centering
    \includegraphics[width=0.6\linewidth]{sources/cor/2-4.pdf}
  \end{figure}

\end{multicols}


\end{document}

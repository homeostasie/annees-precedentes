%%%%%%%%%%%%%%%%%%%%%%%%%%%%%%%%%%%%%%%%%
% LaTeX Template
% http://www.LaTeXTemplates.com
%
% Original author:
% Linux and Unix Users Group at Virginia Tech Wiki 
% (https://vtluug.org/wiki/Example_LaTeX_chem_lab_report)
%
% License:
% CC BY-NC-SA 3.0 (http://creativecommons.org/licenses/by-nc-sa/3.0/)
%
%%%%%%%%%%%%%%%%%%%%%%%%%%%%%%%%%%%%%%%%%

%----------------------------------------------------------------------------------------
%	PACKAGES AND DOCUMENT CONFIGURATIONS
%----------------------------------------------------------------------------------------

\documentclass[12pt]{article}
\usepackage{geometry} % Pour passer au format A4
\geometry{hmargin=1cm, vmargin=1cm} % 

\usepackage{graphicx} % Required for including pictures
\usepackage{float} % 

%Français
\usepackage[T1]{fontenc} 
\usepackage[english,francais]{babel}
\usepackage[utf8]{inputenc}
\usepackage{eurosym}
\usepackage{lmodern}
\usepackage{url}
\usepackage{multicol}

%Maths
\usepackage{amsmath,amsfonts,amssymb,amsthm}
%\usepackage[linesnumbered, ruled, vlined]{algorithm2e}
%\SetAlFnt{\small\sffamily}

%Autres
\linespread{1} % Line spacing
\setlength\parindent{0pt} % Removes all indentation from paragraphs

\renewcommand{\labelenumi}{\alph{enumi}.} % 
\pagestyle{empty}
%----------------------------------------------------------------------------------------
%	DOCUMENT INFORMATION
%----------------------------------------------------------------------------------------
\begin{document}

%\maketitle % Insert the title, author and date

\begin{minipage}[t]{\textwidth}
  \raggedright
      {\bfseries Série B}\\[.35ex]
      {\bfseries 1 STI 2D 2}\\[.35ex]
      {\bfseries Nom : }\\[.35ex]
      \vspace*{-1cm}
      \raggedleft
          {\bfseries Trigonométrie}\\[.35ex]
          {\bfseries 16 Décembre 2014}\\[.35ex]
\end{minipage}\\[1em]

\begin{center}
  \textsf{À une époque de supercherie universelle, dire la vérité est un acte révolutionnaire. - George Orwell}\\
\end{center}

\setlength{\columnseprule}{0pt}

\subsection*{I - Une histoire d'angle}

\begin{enumerate}
\item[1.] Convertir les angles suivants en radian. Mettre les résultats sous la forme $\frac{a}{b}\pi$ avec $a$ et $b$ irréductible.

  \begin{multicols}{3}
    \begin{enumerate}
    \item[a)] $74\char23$
    \item[b)] $182\char23$
    \item[c)] $-364\char23$
    \end{enumerate}
  \end{multicols}

\item[2.] Mettre les résultats suivants sous la forme $\alpha + k\times 2\pi$ avec $\alpha \in [0 ; 2\pi]$ et $k \in \mathbb{Z}$, un entier relatif. 

  \begin{multicols}{4}
    \begin{enumerate}
    \item[a)] $331 \pi$
    \item[b)] $\dfrac{183 \pi}{6} $
    \item[c)] $- \dfrac{203 \pi}{7}$ 
    \item[d)] $48$
    \end{enumerate}
  \end{multicols}

\end{enumerate}

\subsection*{II - De l'angle à la trigonométrie}
\setlength{\columnseprule}{1pt}
\begin{multicols}{2}

  \begin{enumerate}
  \item[1.] Calculer. Mettre le résultat sous la forme $\frac{a}{b}\pi$ avec $a$ et $b$ irréductible.

    \begin{enumerate}
    \item[a)] $\dfrac{\pi}{3} + \dfrac{\pi}{2}$
    \item[b)] $\dfrac{\pi}{4} + \pi$
    \item[c)] $\dfrac{\pi}{6} + \dfrac{3\pi}{2}$
    \end{enumerate}

  \item[2.] Placer les angles précédents sur le cercle.
  \item[3.] Donner les résultats exacts suivants. 
    \setlength{\columnseprule}{0pt}
    \begin{multicols}{2}
      \begin{enumerate}
      \item[a)] $\cos(0)$
      \item[b)] $\cos(\dfrac{\pi}{4})$
      \item[c)] $\cos(\dfrac{5\pi}{6})$
      \item[d)] $\sin(\dfrac{\pi}{6})$
      \item[e)] $\sin(\dfrac{-\pi}{4})$
      \item[f)] $\sin(\dfrac{5\pi}{4})$
      \end{enumerate}
    \end{multicols}

  \end{enumerate}

  \begin{figure}[H]
    \centering
    \includegraphics[width=\linewidth]{sources/ie/cercle.pdf}
  \end{figure}
\end{multicols}
\subsection*{III - Équations trigonométriques}

\begin{enumerate}
\item[1.] Résoudre les équations suivantes et donner \textbf{toutes} les solutions possibles.
  \begin{multicols}{3}
    \begin{enumerate}
    \item[a)] $\cos(x) = \cos(\dfrac{\pi}{6})$
    \item[b)] $\cos(x) = \dfrac{\sqrt{2}}{2}$
    \item[c)] $\sin(x + \dfrac{\pi}{6}) = \frac{1}{2}$ 
    \end{enumerate}
  \end{multicols}
\item[2.] Donner \textbf{une} solution de chaque équation précédente appartenant à l'intervalle $[- \pi ; \pi]$.
\end{enumerate}

\end{document}

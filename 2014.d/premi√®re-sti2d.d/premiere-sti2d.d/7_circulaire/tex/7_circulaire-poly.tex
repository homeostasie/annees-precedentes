%%%%%%%%%%%%%%%%%%%%%%%%%%%%%%%%%%%%%%%%%
% Short Sectioned Assignment
% LaTeX Template
% Version 1.0 (5/5/12)
%
% This template has been downloaded from:
% http://www.LaTeXTemplates.com
%
% Original author:
% Frits Wenneker (http://www.howtotex.com)
%
% License:
% CC BY-NC-SA 3.0 (http://creativecommons.org/licenses/by-nc-sa/3.0/)
%
%%%%%%%%%%%%%%%%%%%%%%%%%%%%%%%%%%%%%%%%%

%----------------------------------------------------------------------------------------
%	PACKAGES AND OTHER DOCUMENT CONFIGURATIONS
%----------------------------------------------------------------------------------------

\documentclass[paper=a4, fontsize=11pt]{scrartcl} % A4 paper and 11pt font size


\usepackage[T1]{fontenc} % Use 8-bit encoding that has 256 glyphs
\usepackage[english,francais]{babel} % Français et anglais
\usepackage[utf8]{inputenc}

\usepackage{amsmath,amsfonts,amsthm} % Math packages

\usepackage{enumitem}
\usepackage{lmodern}
\usepackage{url}
\usepackage{eurosym} % signe Euros
\usepackage{geometry} % Pour passer au format A4
\geometry{a4paper} %
\usepackage{graphicx} % Required for including pictures
\usepackage{float} % Allows putting an [H] in \begin{figure} to specify the exact location of the figure

\usepackage{multicol}

\usepackage{verbatim}

\usepackage{sectsty} % Allows customizing section commands
\allsectionsfont{\centering \normalfont\scshape} % Make all sections centered, the default font and small caps

%----------------------------------------------------------------------------------------
%	Pied de Page
%----------------------------------------------------------------------------------------

\usepackage{fancyhdr} % Custom headers and footers
\pagestyle{fancyplain} % Makes all pages in the document conform to the custom headers and footers
\fancyhead{} % No page header - if you want one, create it in the same way as the footers below
\fancyfoot[L]{$1^{ère}STI2D 2$} % Empty left footer
\fancyfoot[C]{Chapitre 4 - Fonctions circulaires} % Empty center footer
\fancyfoot[R]{\thepage} % Page numbering for right footer

\renewcommand{\headrulewidth}{0pt} % Remove header underlines
\renewcommand{\footrulewidth}{0pt} % Remove footer underlines

\setlength{\headheight}{13.6pt} % Customize the height of the header


\setlength\parindent{0pt} % Removes all indentation from paragraphs - comment this line for an assignment with lots of text


%----------------------------------------------------------------------------------------
%	Titre
%----------------------------------------------------------------------------------------

\newcommand{\horrule}[1]{\rule{\linewidth}{#1}} % Create horizontal rule command with 1 argument of height


\title{	
  \vspace{-10ex}
  \horrule{0.5pt} \\[0.4cm] % Thin top horizontal rule
  \huge Chapitre 4 - Fonctions circulaires\\ % The assignment title
  \horrule{2pt} \\[0.5cm] % Thick bottom horizontal rule
}

\author{}
\date{\vspace{-10ex}} % Today's date or a custom date

%----------------------------------------------------------------------------------------
%	Début du document
%----------------------------------------------------------------------------------------

\begin{document}

%----------------------------------------------------------------------------------------
% RE-DEFINITION
%----------------------------------------------------------------------------------------
% MATHS
%-----------

\newtheorem{Definition}{Définition}
\newtheorem{Theorem}{Théorème}
\newtheorem{Proposition}{Propriété}

% MATHS
%-----------
\renewcommand{\labelitemi}{$\bullet$}
\renewcommand{\labelitemii}{$\circ$}
%----------------------------------------------------------------------------------------
%	Titre
%----------------------------------------------------------------------------------------

\maketitle % Print the title
\setlength{\columnseprule}{1pt}

\section{Rappel}

\subsection{Cercle}

Soit $\mathbb{C}(O,R)$ un cercle de centre O et de rayon R.

%include image

\begin{Definition}
  On note $M(x,y)$ un point du plan et on calcule la distance $OM$ par $OM = \sqrt{x^2 + y^2}$. \\
  Le cercle de centre O et de rayon R est l'ensemble des points M(x,y) distant de O d'un rayon R : $\sqrt{x^2 + y^2} = R$.
\end{Definition}

\begin{Proposition} Soit $\mathbb{C}$ un cercle de rayon R.
  \begin{itemize}
  \item Périmètre du cercle : $\mathbb{P}_{\mathbb{C}} = 2 \pi R$. 
  \item Aire du disque : $\mathbb{A}_{\mathbb{C}} = \pi R ^2$.
  \end{itemize} 
\end{Proposition}

\subsection{Angle}




En particulier, AOA' = BOB' et COC' > AOA'

\subsection{Trigonométrie dans le triangle rectangle}

Dans le triangle rectangle, on connaît les relations de trigonométrie suivante.

$$cos(\alpha) = \dfrac{ajdacent}{hypothénus}$$

$$sin(\alpha) = \dfrac{opposé}{hypothénus}$$

\section{Mesure d'angle}

\subsection{Radian}

\begin{Definition}
  Soit $\mathbb{C}(O,1)$ un cercle de centre $O$ et de rayon $1$. La mesure d'un angle en radian correspond à la longueur algébrique de l'arc intercepté.
\end{Definition}

\begin{itemize}
\item La relation entre un angle en radian et en degré est une relation de proportionnelle.
\item Une relation est évidente : $360 = 2\pi$.
\end{itemize}

\subsubsection{Passage de degré à radian}

Il est possible de convertir un angle en degrée en radian en s'appuyant sur la proportionnalité. On souhaite convertir un angle de $72$.

\begin{eqnarray*}
  \dfrac{x}{72 } &=& \dfrac{2\pi}{360}\\
  x &=& \dfrac{2 \pi \times 72 }{360 }\\
  x &=& \dfrac{2 \pi}{5} \approx 1.26
\end{eqnarray*}


\subsubsection{Passage de radian à degré}

Il est possible de convertir un angle en radian en degrée en radian en s'appuyant sur la même méthode. On souhaite convertir un angle de $\frac{\pi}{6}$.

\begin{eqnarray*}
  \dfrac{x}{ \frac{\pi}{6} } &=& \dfrac{360}{2 \pi}\\
  x &=& \dfrac{360  \times \frac{\pi}{6}}{2 \pi}\\
  x &=& 30 
\end{eqnarray*}

\subsection{Angles orientés}

Un peu à l'image d'un phare, selon le sens dans lequel, on tourne un quart de tour vers la droite ou vers la gauche n'éclaire pas dans la même direction. Pour cela, on a besoin de définir un sens.

\subsubsection{Sens direct}

On définit le sens direct également appelé le sens trigonométrique. C'est en pratique le sens inverse des aiguilles d'une montre. Il est noté positif.

\subsubsection{Sens indirect}

On définit le sens indirect également le sens horaire. Il est noté négatif.

\subsection{Angles dans le cercle}

% insert image

Il est possible de retrouver toutes ses grandeurs à partir des grandeurs données dans le premiers quart de cercle en jouant avec des opérations avec des fractions. On passe d'un quart de cercle à un autre en ajoutant $\frac{\pi}{2}$.

On peut également donner la valeur d'un angle en regardant dans le sens indirect. Il faut alors opposé la valeur symétrique par rapport à l'axe horizontal.

\section{Trigonométrie dans le cercle de rayon 1}

La première remarque fondamentale est que comme le rayon vaut 1, l'hypothénus est lui aussi égale à un. On peut alors récrécrire les relations de trignométrie.\\


\begin{figure}[h]
  \centering
  \includegraphics[width=1\textwidth]{sources/cours/trigocercle.pdf}

\end{figure}


\begin{Proposition}
  Soit un point $M$ appartenant au cercle et $x$ l'angle formé par IOM.\\
  Les coordonnées du points $M$ sont : $M(\cos(x) ; \sin(x) )$.
\end{Proposition}
On appelle l'axe horizontal, l'axe des cosinus, 


\section{Étude de fonctions}

\subsection{Cosinus}

\subsection{Sinus}

\section{Résolution d'équations}

\subsection{Cosinus}

\subsection{Sinus}

\end{document}

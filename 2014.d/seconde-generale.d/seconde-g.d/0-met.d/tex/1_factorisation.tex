%%%%%%%%%%%%%%%%%%%%%%%%%%%%%%%%%%%%%%%%%
% Short Sectioned Assignment
% LaTeX Template
% Version 1.0 (5/5/12)
%
% This template has been downloaded from:
% http://www.LaTeXTemplates.com
%
% Original author:
% Frits Wenneker (http://www.howtotex.com)
%
% License:
% CC BY-NC-SA 3.0 (http://creativecommons.org/licenses/by-nc-sa/3.0/)
%
%%%%%%%%%%%%%%%%%%%%%%%%%%%%%%%%%%%%%%%%%

%----------------------------------------------------------------------------------------
%	PACKAGES AND OTHER DOCUMENT CONFIGURATIONS
%----------------------------------------------------------------------------------------

\documentclass[paper=a4, fontsize=9pt]{scrartcl} % A4 paper and 11pt font size


\usepackage[T1]{fontenc} % Use 8-bit encoding that has 256 glyphs
\usepackage[english,francais]{babel} % Français et anglais
\usepackage[utf8]{inputenc}

\usepackage{amsmath,amsfonts,amsthm} % Math packages

\usepackage{enumitem}
\usepackage{lmodern}
\usepackage{url}
\usepackage{eurosym} % signe Euros
\usepackage{geometry} % Pour passer au format A4
\geometry{a4paper} %
\usepackage{graphicx} % Required for including pictures
\usepackage{float} % Allows putting an [H] in \begin{figure} to specify the exact location of the figure

\usepackage{multicol}

\usepackage{verbatim}

\usepackage{sectsty} % Allows customizing section commands
\allsectionsfont{\centering \normalfont\scshape} % Make all sections centered, the default font and small caps

%----------------------------------------------------------------------------------------
%	Pied de Page
%----------------------------------------------------------------------------------------

\setlength\parindent{0pt} % Removes all indentation from paragraphs - comment this line for an assignment with lots of text


%----------------------------------------------------------------------------------------
%	Titre
%----------------------------------------------------------------------------------------

\newcommand{\horrule}[1]{\rule{\linewidth}{#1}} % Create horizontal rule command with 1 argument of height


%----------------------------------------------------------------------------------------
%	Début du document
%----------------------------------------------------------------------------------------

\begin{document}

%----------------------------------------------------------------------------------------
% RE-DEFINITION
%----------------------------------------------------------------------------------------
% MATHS
%-----------

\newtheorem{Definition}{Définition}
\newtheorem{Theorem}{Théorème}
\newtheorem{Proposition}{Propriété}

% MATHS
%-----------
\renewcommand{\labelitemi}{$\bullet$}
\renewcommand{\labelitemii}{$\circ$}
%----------------------------------------------------------------------------------------
%	Titre
%----------------------------------------------------------------------------------------

%\maketitle % Print the title
\setlength{\columnseprule}{1pt}

\begin{multicols}{2}

  \subsection*{Factorisation - A}
  \textit{Factoriser au maximum le calcul suivant.}

  \begin{enumerate}
  \item[1.] $x^2 -2x + 1$
  \item[2.] $2(x + 1) - 7y(x + 1)$
  \item[3.] $-x^2 + 9y^2$
  \item[4.] $x - 4 - 7(x - 4) + y(x - 4)$
  \item[5.] $x^2 - x(2x - 1)$
  \end{enumerate}


  \subsection*{Factorisation - B}
  \textit{Factoriser au maximum le calcul suivant.}

  \begin{enumerate}
  \item[1.] $x^2 + 2x + 1$
  \item[2.] $3(x + 1) - 6y(x + 1)$
  \item[3.] $-x^2 + 25y^2$
  \item[4.] $(x + 2) - 8(x + 2) + 2y(x + 2)$
  \item[5.] $x^2 - (x + 2)x$
  \end{enumerate}

\end{multicols}

\vspace{1cm}

\begin{multicols}{2}

  \subsection*{Factorisation - C}
  \textit{Factoriser au maximum le calcul suivant.}

  \begin{enumerate}
  \item[1.] $x^2 -2x + 1$
  \item[2.] $-4(x+1) + 7y(x+1)$
  \item[3.] $-x^2 + 4y^2$
  \item[4.] $(x - 4) - 2(x - 4) + 2y(x - 4)$
  \item[5.] $x^2 - x(x - 3)$
  \end{enumerate}

  \subsection*{Factorisation - D}
  \textit{Factoriser au maximum le calcul suivant.}

  \begin{enumerate}
  \item[1.] $x^2 + 2x + 1$
  \item[2.] $2(3x + 1) - 7y(3x + 1)$
  \item[3.] $-x^2 + 36y^2$
  \item[4.] $(x - 1) - 5(x - 1) + 3y(x - 1)$
  \item[5.] $x^2 - (x + 5)x$
  \end{enumerate}

\end{multicols}

\vspace{1cm}

\begin{multicols}{2}

  \subsection*{Factorisation - E}
  \textit{Factoriser au maximum le calcul suivant.}

  \begin{enumerate}
  \item[1.] $x^2 -2x + 1$
  \item[2.] $5(z + 1) - 2y(z + 1)$
  \item[3.] $-x^2 + 16y^2$
  \item[4.] $(x - 4) - 2(x - 4) + 3y(x - 4)$
  \item[5.] $x^2 -  x(2\pi + 1)$
  \end{enumerate}

  \subsection*{Factorisation - F}
  \textit{Factoriser au maximum le calcul suivant.}

  \begin{enumerate}
  \item[1.] $x^2 +2x + 1$
  \item[2.] $3(x + 2) - 6y(x + 2)$
  \item[3.] $-x^2 + 9y^2$
  \item[4.] $(x - 8) - 6(x - 8) + z(x - 8)$
  \item[5.] $x^2 - x(\pi + 8)$
  \end{enumerate}

\end{multicols}

\vspace{1cm}

\begin{multicols}{2}

  \subsection*{Factorisation - H}
  \textit{Factoriser au maximum le calcul suivant.}

  \begin{enumerate}
  \item[1.] $x^2 -2x + 1$
  \item[2.] $2(y + 1) - x(y + 1)$
  \item[3.] $-x^2 + 9y^2$
  \item[4.] $(x - 3) - 4z(x - 3) + 3(x - 3)$
  \item[5.] $x^2 - x(2\pi + 7)$
  \end{enumerate}

  \subsection*{Factorisation - I}
  \textit{Factoriser au maximum le calcul suivant.}

  \begin{enumerate}
  \item[1.] $x^2 +2x + 1$
  \item[2.] $9(x + 2) - 6y(x + 2)$
  \item[3.] $-x^2 + 4y^2$
  \item[4.] $(x - 3) - 6y(x - 3) + 4(x - 3)$
  \item[5.] $x^2 - (x + 4)x$
  \end{enumerate}

\end{multicols}

\end{document}

%%%%%%%%%%%%%%%%%%%%%%%%%%%%%%%%%%%%%%%%%
% LaTeX Template
% http://www.LaTeXTemplates.com
%
% Original author:
% Linux and Unix Users Group at Virginia Tech Wiki 
% (https://vtluug.org/wiki/Example_LaTeX_chem_lab_report)
%
% License:
% CC BY-NC-SA 3.0 (http://creativecommons.org/licenses/by-nc-sa/3.0/)
%
%%%%%%%%%%%%%%%%%%%%%%%%%%%%%%%%%%%%%%%%%

%----------------------------------------------------------------------------------------
%	PACKAGES AND DOCUMENT CONFIGURATIONS
%----------------------------------------------------------------------------------------

\documentclass[10pt]{article}
\usepackage{geometry} % Pour passer au format A4
\geometry{hmargin=1cm, vmargin=1cm} % 

\usepackage{graphicx} % Required for including pictures
\usepackage{float} % 

%Français
\usepackage[T1]{fontenc} 
\usepackage[english,francais]{babel}
\usepackage[utf8]{inputenc}
\usepackage{eurosym}
\usepackage{lmodern}
\usepackage{url}
\usepackage{multicol}
\usepackage{fancybox} 

\usepackage{enumitem}
%Maths
\usepackage{amsmath,amsfonts,amssymb,amsthm}
%\usepackage[linesnumbered, ruled, vlined]{algorithm2e}
%\SetAlFnt{\small\sffamily}

%Autres
\linespread{1} % Line spacing
\setlength\parindent{0pt} % Removes all indentation from paragraphs

\renewcommand{\labelenumi}{\alph{enumi}.} % 
\pagestyle{empty}
%----------------------------------------------------------------------------------------
%	DOCUMENT INFORMATION
%----------------------------------------------------------------------------------------
\begin{document}
\subsection*{Exo 1 : Brevet des Collèges, Djibouti, Juin 2000}
\begin{multicols}{2}
	\textit{Voici un programme de calcul :}
	\begin{itemize}
		\item  on choisit un nombre,
		\item on lui ajoute 3,
		\item on élève le résultat au carré,
		\item on retranche 25 au nombre obtenu.
	\end{itemize}
	
	\begin{enumerate}
		\item Appliquer ce programme de calcul au nombre 2.
		Quel nombre obtient-on ?
		\item On appelle $n$ le nombre auquel on applique le programme de calcul précédent.
		Exprimer, en fonction de $n$, le résultat de ce programme de calcul.
		Tester l'expression obtenue en donnant à n la valeur 2.
	\end{enumerate}
\end{multicols}

\subsection*{Exo 2 : Brevet, France métropolitaine, Septembre 2009}

\begin{multicols}{2}
	\textit{On propose deux programmes de calcul :}
	\subsubsection*{Programme A}
		\begin{itemize}
			\item Choisir un nombre.
			\item Multiplier ce nombre par 3.
			\item Ajouter 7.
		\end{itemize}

	\subsubsection*{Programme B}
		\begin{itemize}
			\item Choisir un nombre.
			\item Multiplier ce nombre par 5.
			\item Retrancher 4.
			\item Multiplier par 2.
		\end{itemize}
\end{multicols}

\begin{enumerate}
	\item  On choisit 3 comme nombre de départ.
	Montrer que le résultat du programme B est 22.
	\item On choisit (-2) comme nombre de départ.
	Quel est le résultat avec le programme A ?
	\item 
	\begin{enumerate}
		\item Quel nombre de départ faut-il choisir pour que le résultat du programme A soit (-2) ?
		\item Quel nombre de départ faut-il choisir pour que le résultat du programme B soit 0 ?
	\end{enumerate}
	\item Quel nombre doit-on choisir pour obtenir le même résultat avec les deux programmes ?
	Faire apparaître sur la copie la démarche utilisée.
\end{enumerate}

\subsection*{Exo 3 : Brevet, France métropolitaine, Juin 2008}
\begin{multicols}{2}
	\textit{On donne le programme de calcul suivant :}

	\begin{itemize}
		\item Choisir un nombre.
		\item Multiplier ce nombre par 3.
		\item Ajouter le carré du nombre choisi.
		\item Multiplier par 2.
		\item Ecrire le résultat.
	\end{itemize}
	
	\begin{enumerate}
		\item Montrer que, si on choisit le nombre 10, le résultat obtenu est 260.
		\item Calculer la valeur exacte du résultat obtenu lorsque :
			\begin{enumerate}
			\item Le nombre choisi est $-5$.
			\item Le nombre choisi est $\dfrac{2}{3}$.
			\item Le nombre choisi est $\sqrt{5}$
			\end{enumerate}
		\item Quels nombres peut-on choisir pour que le résultat obtenu soit 0 ?
	\end{enumerate}
\end{multicols}

\subsection*{Exo 4 : Brevet, France métropolitaine, Juin 2010}
\begin{multicols}{2}
	\textit{On considère le programme de calcul suivant :}

	\begin{itemize}
		\item Choisir un nombre de départ.
		\item Multiplier ce nombre par $-2$
		\item Ajouter 5 au produit.
		\item Multiplier le résultat par 5.
		\item Ecrire le résultat obtenu.
	\end{itemize}

	\begin{enumerate}
		\item Vérifier que, lorsque le nombre de départ est 2, on obtient 5.
		\item Lorsque le nombre de départ est 3, quel résultat obtient on ?
		\item Quel nombre faut-il choisir au départ pour que le résultat obtenu soit 0 ?
		\item Arthur prétend que pour n’importe quel nombre de départ $x$, l’expression $(x-5)^{2} - x^{2}$ permet d’obtenir le résultat du programme de calcul. A-t-il raison ?
	\end{enumerate}
\end{multicols}

\subsection*{Exo 5}
\textit{Développer, et réduire si possible : }

\begin{multicols}{3}
\begin{itemize}[label=$\Diamond$]
\item $(3t + 1)^{2}$
\item $(2t - 9)(2t + 9)$
\item $(4a - 6)^{2}$
\item $(5a + 8)^{2}$
\item $(3a + 11)(3a - 11)$
\item $49 - (3x+ 2)^{2}$
\item $(x - 3)^{2} + (x - 3)(x + 3)$
\item $(3x + 2)^{2} - (5 - 2x)(3x+ 2)$
\item $(7a - 9)^{2} + (7a - 9)(3a + 1)$
\end{itemize}
\end{multicols}
\end{document}

%%%%%%%%%%%%%%%%%%%%%%%%%%%%%%%%%%%%%%%%%
% LaTeX Template
% http://www.LaTeXTemplates.com
%
% Original author:
% Linux and Unix Users Group at Virginia Tech Wiki 
% (https://vtluug.org/wiki/Example_LaTeX_chem_lab_report)
%
% License:
% CC BY-NC-SA 3.0 (http://creativecommons.org/licenses/by-nc-sa/3.0/)
%
%%%%%%%%%%%%%%%%%%%%%%%%%%%%%%%%%%%%%%%%%

%----------------------------------------------------------------------------------------
%	PACKAGES AND DOCUMENT CONFIGURATIONS
%----------------------------------------------------------------------------------------

\documentclass[11pt]{article}
\usepackage{geometry} % Pour passer au format A4
\geometry{hmargin=1cm, vmargin=1cm} % 

\usepackage{graphicx} % Required for including pictures
\usepackage{float} % 

%Français
\usepackage[T1]{fontenc} 
\usepackage[english,francais]{babel}
\usepackage[utf8]{inputenc}
\usepackage{eurosym}
\usepackage{lmodern}
\usepackage{url}
\usepackage{multicol}

%Maths
\usepackage{amsmath,amsfonts,amssymb,amsthm}
%\usepackage[linesnumbered, ruled, vlined]{algorithm2e}
%\SetAlFnt{\small\sffamily}

%Autres
\linespread{1} % Line spacing
\setlength\parindent{0pt} % Removes all indentation from paragraphs

\renewcommand{\labelenumi}{\alph{enumi}.} % 
\pagestyle{empty}
%----------------------------------------------------------------------------------------
%	DOCUMENT INFORMATION
%----------------------------------------------------------------------------------------
\begin{document}

%\maketitle % Insert the title, author and date

\begin{minipage}[t]{\textwidth}
  \raggedright
      {\bfseries Série A et B}\\[.35ex]
      {\bfseries $2^{nd}1$}\\[.35ex]
      \vspace*{-1cm}
      \raggedleft
          {\bfseries Fonctions 1}\\[.35ex]
          {\bfseries 29 Septembre 2014}\\[.35ex]
\end{minipage}\\[1em]

\begin{center}
  \textsf{La vie, c'est comme une bicyclette, il faut avancer pour ne pas perdre l'équilibre. - Albert Einstein}\\
\end{center}

\setlength{\columnseprule}{1pt}

% ------ Exercice 1 ------
\section*{1 - Programme de calcul}

\begin{verbatim}
- Choisir un nombre
- Soustraire 10
- Élever le résultat au carré
\end{verbatim}

\begin{enumerate}
% -- 1
\item[1]
  \begin{enumerate}
  \item On choisit le nombre $ 7$. Quel est le résultat du programme de calcul.\\
    \begin{eqnarray*}
	7 &=& 7\\
	7-10 &=& -3\\
	(-3)^2 &=& 9
  \end{eqnarray*}
  \item On choisit le nombre $-2$. Quel est le résultat du programme de calcul.
    \begin{eqnarray*}
	-2 &=& -2\\
	-2-10 &=& -12\\
	(-12)^2 &=& 144
  \end{eqnarray*}  
  \item On choisit le nombre $ 6$. Quel est le résultat du programme de calcul.
    \begin{eqnarray*}
	6 &=& 6\\
	6-10 &=& -4\\
	(-4)^2 &=& 16
  \end{eqnarray*}  
  \item On choisit le nombre $-3$. Quel est le résultat du programme de calcul.
    \begin{eqnarray*}
	-3 &=& -3\\
	-3-10 &=& -13\\
	(-13)^2 &=& 169
  \end{eqnarray*}  
  \end{enumerate}
% -- 2
\item[2] On décide de ne pas choisir de nombre précisément. On pose $x$ le nombre choisi.
  \begin{enumerate}
  \item Effectuer le programme de calcul avec $x$ le nombre choisi.\\
    \begin{eqnarray*}
	x\\
	x-10\\
	(x-10)^2
  \end{eqnarray*}    
  \item Charles trouve que le résultat est égale à $x^2 - 100$. A-t-il raison ? S'il a tort, proposer une solution juste.\\
  $(x-10)^2 = x^2 - 20x + 100 \neq x^2 - 100$\\
  Charles ne sait pas développer une identité remarquable.
  \end{enumerate}
\end{enumerate}

% ------ Exercice 2 ------
\noindent\hrulefill
\newpage
\section*{2 - Images et antécédents}

\begin{multicols}{2}

  \begin{figure}[H]
    \centering
    \includegraphics[width=\linewidth]{sources/ie/ie-courbe.pdf}
  \end{figure}

  \begin{center}
    \begin{tabular}{| l || c | c | c | c | c |}
      \hline			
      $x$    & -4 & -2 & -0.5 & 0  & 2\\
      \hline  
      $f(x)$ & -2 & -3 & -1   & -1  & 0.5\\
      \hline  
    \end{tabular}
  \end{center}


  \begin{enumerate}
    % -- 1
  \item[1] Remplir le tableau précédent :
    % -- 2
  \item[2] 
    \begin{enumerate}
    \item Quelle est l'image de $-2$ par la fonction $f$.\\
  L'image de $-2$ par la fonction $f$ est $-3$ : $f(-2) = -3$.  
    \item Quels sont les antécédents de $-2$ par la fonction $f$.
    Les antécédents de $-2$ par la fonction $f$ sont $-4$ et $-1.25$ : $f(-4) = -2$ et $f(--1.25) = -2$.
    \end{enumerate}
  \end{enumerate}

\end{multicols}

% ------ Exercice 3 ------
\noindent\hrulefill

\section*{3 - Un problème de longueur }

\begin{multicols}{2}

  Charles possède le terrain rectangulaire suivant. 

  \begin{figure}[H]
    \centering
    \includegraphics[width=0.8\linewidth]{sources/ie/ie-terrain.pdf}
  \end{figure}
 
  \begin{enumerate}
  \item[1] Exprimer en fonction de $x$ le périmètre du terrain.
  \textbf{Le périmitère d'un térrain est la somme de tous les côtés.} \\
  $P = 2 \left( (2x-1) + (x+5) \right) = 2 ( 3x + 4) = 6x + 8$
  Le périmètre du terrain mesure : $6x+8$
  \item[2A] Sachant que le périmètre du terrain mesure exactement $86m$.\\
  Quelles sont les dimensions (longueur ET largeur) du terrain.
     \begin{eqnarray*}
	6x + 8 &=& 86\\
	6x &=& 78\\
	x &=& \frac{78}{6}\\
	x &=& 13
  \end{eqnarray*}
  Le périmètre du terrain mesure $13$ mètres.\\
  \begin{itemize}
  \item La longueur mesure 25 mètres.\\
  $2x-1 = 2*13 -1 = 25$
   \item La largeur mesure 18 mètres\\
   $x+5 = 13 + 5 = 18$
   \end{itemize}
   
   \item[2B] Sachant que le périmètre du terrain mesure exactement $70m$.\\
  Quelles sont les dimensions (longueur ET largeur) du terrain.
     \begin{eqnarray*}
	6x + 8 &=& 70\\
	6x &=& 62\\
	x &=& \frac{62}{6}\\
	x &\simeq& 10.3
  \end{eqnarray*}
  Le périmètre du terrain mesure $10.3$ mètres.\\
  \begin{itemize}
  \item La longueur mesure 19.6 mètres.\\
  $2x-1 = 2*10.3 -1 = 19.6$
   \item La largeur mesure 15.3mètres\\
   $x+5 = 13 + 5 = 15.3$
   \end{itemize}  
  \end{enumerate}

\end{multicols}

\end{document}

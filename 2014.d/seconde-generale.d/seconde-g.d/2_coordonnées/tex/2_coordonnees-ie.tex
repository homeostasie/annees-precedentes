%%%%%%%%%%%%%%%%%%%%%%%%%%%%%%%%%%%%%%%%%
% LaTeX Template
% http://www.LaTeXTemplates.com
%
% Original author:
% Linux and Unix Users Group at Virginia Tech Wiki 
% (https://vtluug.org/wiki/Example_LaTeX_chem_lab_report)
%
% License:
% CC BY-NC-SA 3.0 (http://creativecommons.org/licenses/by-nc-sa/3.0/)
%
%%%%%%%%%%%%%%%%%%%%%%%%%%%%%%%%%%%%%%%%%

%----------------------------------------------------------------------------------------
%	PACKAGES AND DOCUMENT CONFIGURATIONS
%----------------------------------------------------------------------------------------

\documentclass[11pt]{article}
\usepackage{geometry} % Pour passer au format A4
\geometry{hmargin=1cm, vmargin=1cm} % 

\usepackage{graphicx} % Required for including pictures
\usepackage{float} % 

%Français
\usepackage[T1]{fontenc} 
\usepackage[english,francais]{babel}
\usepackage[utf8]{inputenc}
\usepackage{eurosym}
\usepackage{lmodern}
\usepackage{url}
\usepackage{multicol}

%Maths
\usepackage{amsmath,amsfonts,amssymb,amsthm}
%\usepackage[linesnumbered, ruled, vlined]{algorithm2e}
%\SetAlFnt{\small\sffamily}

%Autres
\linespread{1} % Line spacing
\setlength\parindent{0pt} % Removes all indentation from paragraphs

\renewcommand{\labelenumi}{\alph{enumi}.} % 
\pagestyle{empty}
%----------------------------------------------------------------------------------------
%	DOCUMENT INFORMATION
%----------------------------------------------------------------------------------------
\begin{document}

%\maketitle % Insert the title, author and date

\begin{minipage}[t]{\textwidth}
  \raggedright
      {\bfseries $2^{nd}1$}\\[.35ex]
      \vspace*{-1cm}
      \raggedleft
          {\bfseries Coordonnées}\\[.35ex]
          {\bfseries 17 Octobre 2014}\\[.35ex]
\end{minipage}\\[1em]

\begin{center}
  \textsf{Une méthode est un truc qui a été utilisé plusieurs fois. - George Polya}\\
\end{center}

\setlength{\columnseprule}{1pt}

\begin{multicols}{2}
% ------ Exercice 1 ------
\section*{1 - Dans un triangle}

Soit $A(-2;-2)$, $B(6.5;-2)$ et $C(2;2)$. On note également $O(0;0)$ l'origine du repère.

\begin{enumerate}

\item[1.] Placer les points $O$, $A$, $B$ et $C$ dans un repère orthonormal.
\item[2.] Montrer que $O$ est le milieu de $[AC]$.
\item[3.] 
	\begin{enumerate}
	\item[a.] Le triangle $ABC$ est-il isocèle. Justifier.
	\item[b.] Le triangle $ABC$ est-il rectangle. Justifier.
	\end{enumerate}

\end{enumerate}

% ------ Exercice 2 ------

\section*{2 - Dans un quadrilatère}

Soit $A(-5;-2)$, $B(2;0)$, $C(2;4)$ et $D(-5,5)$. 

\begin{enumerate}

\item[1.] Placer les points $A$, $B$, $C$ et $D$ dans un repère orthonormal.
\item[2.] Calculer $E$ le milieu de $[AC]$.
\item[3.] Calculer $E'$ le milieu de $[BD]$.
\item[4.] 
	\begin{enumerate}
	\item[a.] En déduire la nature du quadrilatère ABCD. Justifier.
	\item[b.] En déduire une relation entre [AB] et [CD]. Justifier.
	\end{enumerate}
\end{enumerate}
\end{multicols}

\end{document}

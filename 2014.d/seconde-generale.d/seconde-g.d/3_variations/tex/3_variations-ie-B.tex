%%%%%%%%%%%%%%%%%%%%%%%%%%%%%%%%%%%%%%%%%
% LaTeX Template
% http://www.LaTeXTemplates.com
%
% Original author:
% Linux and Unix Users Group at Virginia Tech Wiki
% (https://vtluug.org/wiki/Example_LaTeX_chem_lab_report)
%
% License:
% CC BY-NC-SA 3.0 (http://creativecommons.org/licenses/by-nc-sa/3.0/)
%
%%%%%%%%%%%%%%%%%%%%%%%%%%%%%%%%%%%%%%%%%

%----------------------------------------------------------------------------------------
%	PACKAGES AND DOCUMENT CONFIGURATIONS
%----------------------------------------------------------------------------------------

\documentclass[10pt]{article}
\usepackage{geometry} % Pour passer au format A4
\geometry{hmargin=1cm, vmargin=1cm} %

\usepackage{graphicx} % Required for including pictures
\usepackage{float} %

%Français
\usepackage[T1]{fontenc}
\usepackage[english,francais]{babel}
\usepackage[utf8]{inputenc}
\usepackage{eurosym}
\usepackage{lmodern}
\usepackage{url}
\usepackage{multicol}

%Maths
\usepackage{amsmath,amsfonts,amssymb,amsthm}
%\usepackage[linesnumbered, ruled, vlined]{algorithm2e}
%\SetAlFnt{\small\sffamily}

%Autres
\linespread{1} % Line spacing
\setlength\parindent{0pt} % Removes all indentation from paragraphs

\renewcommand{\labelenumi}{\alph{enumi}.} %
\pagestyle{empty}
%----------------------------------------------------------------------------------------
%	DOCUMENT INFORMATION
%----------------------------------------------------------------------------------------
\begin{document}

%\maketitle % Insert the title, author and date

\begin{minipage}[t]{\textwidth}
  \raggedright
      {\bfseries $2^{nd}1$}\\[.35ex]
      Série B\\
      \vspace*{-1cm}
      \raggedleft
          {\bfseries Variations}\\[.35ex]
          {\bfseries 10 Novembre 2014}\\[.35ex]
\end{minipage}\\[1em]

\begin{center}
  \textsf{Le principe de l'évolution est beaucoup plus rapide en informatique que chez le bipède. - Jean Dion}\\
\end{center}

\setlength{\columnseprule}{1pt}

\begin{multicols}{2}
  % ------ Exercice 1 ------
  \section*{1 - De la courbe au tableau}
  \begin{figure}[H]
    \centering
    \includegraphics[width=0.8\linewidth]{sources/ie/ie-1-B.pdf}
  \end{figure}

  Soit $f$ la fonction définie sur l'intervalle $I=[-0.5,2.5]$ à partir de sa représentation graphique.
  \begin{enumerate}
  \item[1.] À partir de la courbe, établir son tableau de variation complet.
  \item[2.]
    \begin{enumerate}
    \item[a.] Quel est le maximum de la fonction $f$. Pour quelle valeur est-il atteint ?
    \item[b.] Quel est le minimum de la fonction $f$. Pour quelle valeur est-il atteint ?
    \end{enumerate}
  \end{enumerate}

  % ------ Exercice 2 ------
  \section*{2 - Du tableau à la courbe}

  \begin{figure}[H]
    \centering
    \includegraphics[width=0.8\linewidth]{sources/ie/ie-2-B.pdf}
  \end{figure}

  Soit $f$ la fonction définie sur l'intervalle $I=[-6,4]$ à partir de son tableau de variation.
  \begin{enumerate}
  \item[1.] Proposer une représentation graphique (Échelle 1cm) pouvant correspondre à ce tableau de variation.
  \item[2.]
    \begin{enumerate}
    \item[a.] Quel est le maximum de la fonction $f$. Pour quelle valeur est-il atteint ?
    \item[b.] Quel est le minimum de la fonction $f$. Pour quelle valeur est-il atteint ?
    \end{enumerate}
  \end{enumerate}

  % ------ Exercice 2 ------
  \section*{3 - Comparaison d'images}

  Soit $f$ une fonction décroissante sur un intervalle $I=[-4,4]$. Comparer $f(-3)$ et $f(-1)$.
\end{multicols}
\end{document}

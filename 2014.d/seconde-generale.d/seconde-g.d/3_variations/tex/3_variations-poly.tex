%%%%%%%%%%%%%%%%%%%%%%%%%%%%%%%%%%%%%%%%%
% Short Sectioned Assignment
% LaTeX Template
% Version 1.0 (5/5/12)
%
% This template has been downloaded from:
% http://www.LaTeXTemplates.com
%
% Original author:
% Frits Wenneker (http://www.howtotex.com)
%
% License:
% CC BY-NC-SA 3.0 (http://creativecommons.org/licenses/by-nc-sa/3.0/)
%
%%%%%%%%%%%%%%%%%%%%%%%%%%%%%%%%%%%%%%%%%

%----------------------------------------------------------------------------------------
%	PACKAGES AND OTHER DOCUMENT CONFIGURATIONS
%----------------------------------------------------------------------------------------

\documentclass[paper=a4, fontsize=9pt]{scrartcl} % A4 paper and 11pt font size


\usepackage[T1]{fontenc} % Use 8-bit encoding that has 256 glyphs
\usepackage[english,francais]{babel} % Français et anglais
\usepackage[utf8]{inputenc} 

\usepackage{amsmath,amsfonts,amsthm} % Math packages

\usepackage{enumitem}
\usepackage{lmodern}
\usepackage{url}
\usepackage{eurosym} % signe Euros
\usepackage{geometry} % Pour passer au format A4
\geometry{a4paper} % 
\usepackage{graphicx} % Required for including pictures
\usepackage{float} % Allows putting an [H] in \begin{figure} to specify the exact location of the figure

\usepackage{multicol}

\usepackage{verbatim}

\usepackage{sectsty} % Allows customizing section commands
\allsectionsfont{\centering \normalfont\scshape} % Make all sections centered, the default font and small caps

%----------------------------------------------------------------------------------------
%	Pied de Page
%----------------------------------------------------------------------------------------


\usepackage{fancyhdr} % Custom headers and footers
\pagestyle{fancyplain} % Makes all pages in the document conform to the custom headers and footers
\fancyhead{} % No page header - if you want one, create it in the same way as the footers below
\fancyfoot[L]{$2^{nd}1$} % Empty left footer
\fancyfoot[C]{Chapitre 3 - Variations} % Empty center footer
\fancyfoot[R]{\thepage} % Page numbering for right footer

\renewcommand{\headrulewidth}{0pt} % Remove header underlines
\renewcommand{\footrulewidth}{0pt} % Remove footer underlines

\setlength{\headheight}{13.6pt} % Customize the height of the header


\setlength\parindent{0pt} % Removes all indentation from paragraphs - comment this line for an assignment with lots of text


%----------------------------------------------------------------------------------------
%	Titre
%----------------------------------------------------------------------------------------

\newcommand{\horrule}[1]{\rule{\linewidth}{#1}} % Create horizontal rule command with 1 argument of height


\title{	
  \vspace{-10ex}
  \horrule{0.5pt} \\[0.4cm] % Thin top horizontal rule
  \huge Chapitre 3 - Variations\\ % The assignment title
  \horrule{2pt} \\[0.5cm] % Thick bottom horizontal rule
}

\author{}
\date{\vspace{-10ex}} % Today's date or a custom date

%----------------------------------------------------------------------------------------
%	Début du document
%----------------------------------------------------------------------------------------

\begin{document}

%----------------------------------------------------------------------------------------
% RE-DEFINITION
%----------------------------------------------------------------------------------------
% MATHS
%-----------

\newtheorem{Definition}{Définition}
\newtheorem{Theorem}{Théorème}
\newtheorem{Proposition}{Propriété}

% MATHS
%-----------
\renewcommand{\labelitemi}{$\bullet$}
\renewcommand{\labelitemii}{$\circ$}
%----------------------------------------------------------------------------------------
%	Titre
%----------------------------------------------------------------------------------------

\maketitle % Print the title
\setlength{\columnseprule}{1pt}

\section{Variations d'une fonction sur un intervalle}

\begin{multicols}{2}

  \subsection{Fonction croissante sur un intervalle}

  \begin{Proposition}
    $f$ est une fonction croissante sur un intervalle $I$.
    Si les valeurs de $x$ augmente, alors les valeurs de $f(x)$ augmente. Pour $a,b \in I$ avec $a \leq b$ on a $f(a) \leq f(b)$
  \end{Proposition}

  \begin{figure}[H]
    \centering
    \includegraphics[width=\linewidth]{sources/cours/3_croissant.pdf}
  \end{figure}

  \subsection{Fonction décroissante sur un intervalle}

  \begin{Proposition} $f$ est une fonction constante sur un intervalle $I$.
    Pour toutes valeurs de $x$ sur I, alors les valeurs de $f(x)$ diminue. Pour $a,b \in I$ avec $a \leq b$ on a $f(a) \geq f(b)$
  \end{Proposition}

  \begin{figure}[H]
    \centering
    \includegraphics[width=\linewidth]{sources/cours/3_decroissant.pdf}
  \end{figure}

\end{multicols}

\subsection{Fonction constante sur un intervalle}

\begin{Proposition} $f$ est une fonction décroissante sur un intervalle $I$.
  Si les valeurs de $x$ augmente, alors les valeurs de f(x) ne change pas. Pour $a,b \in I$ on a $f(a) = f(b)$
\end{Proposition}

\paragraph{Remarques}~~\\
\begin{itemize}
\item Si $f$ est une fonction croissante et jamais constante, on dit qu'elle est strictement croissante.
\item Si $f$ est une fonction décroissante et jamais constante, on dit qu'elle est strictement décroissante.
\item Une fonction ni croissante, ni décroissante est une fonction quelconque.
\end{itemize}

\section{Extremum d'une fonction sur un intervalle}
\begin{multicols}{2}
  \subsection{Maximum}

  \begin{Definition}
    Soit $f$ une fonction définie sur un intervalle $I$. I admet un maximum $f(a)$ en $a \in I$ si pour tout $x \in I$ on a $f(x) \leq f(b)b$.  
  \end{Definition}

  \subsection{Minimum}

  \begin{Definition}
    Soit $f$ une fonction définie sur un intervalle $I$. I admet un minimum $f(a)$ en $a \in I$ si pour tout $x \in I$ on a $f(x) \geq f(a)$.  
  \end{Definition}
\end{multicols}

\newpage
\section{Méthodes}

\begin{multicols}{2}

  \begin{figure}[H]
    \centering
    \includegraphics[width=\linewidth]{sources/cours/3_courbe.pdf}
    \caption{$f$ est définie sur $[-1; 2]$}
  \end{figure}

  \begin{figure}[H]
    \centering
    \includegraphics[width=\linewidth]{sources/cours/3_tableau.pdf}
    \caption{$f$ est définie sur $[-1; 2]$}
  \end{figure}

  Il est très facile de remarquer les extremum à partir du tableau de variation.

\end{multicols}

\subsection{De la courbe au tableau de variation}

Soit $f$ une fonction dont la courbe représentative est donnée.

\begin{enumerate}
\item On repère le domaine de définition que seront les bornes de notre tableau.
\item On repère les variations par des flèches vers le haut pour croissante et vers le bas pour décroissante.
\item On remplit les différentes valeurs $x$ à chaque changement de variation.
\item On remplit alors les différentes valeurs de $f(x)$ manquantes.
\end{enumerate}

\subsection{Du tableau de variation à la courbe}

Soit une fonction $f$ dont on connaît le tableau de variation. Il est possible de tracer une (ou plusieurs) courbe correspondant au tableau.

\begin{enumerate}
\item On commence par placer les points $(x ; f(x) )$ présents sur le tableau.
\item On relie ces différents points en conservant les variations indiquées dans le tableau.
\end{enumerate}

\end{document}

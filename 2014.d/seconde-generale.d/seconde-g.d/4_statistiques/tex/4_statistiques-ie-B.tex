%%%%%%%%%%%%%%%%%%%%%%%%%%%%%%%%%%%%%%%%%
% LaTeX Template
% http://www.LaTeXTemplates.com
%
% Original author:
% Linux and Unix Users Group at Virginia Tech Wiki
% (https://vtluug.org/wiki/Example_LaTeX_chem_lab_report)
%
% License:
% CC BY-NC-SA 3.0 (http://creativecommons.org/licenses/by-nc-sa/3.0/)
%
%%%%%%%%%%%%%%%%%%%%%%%%%%%%%%%%%%%%%%%%%

%----------------------------------------------------------------------------------------
%	PACKAGES AND DOCUMENT CONFIGURATIONS
%----------------------------------------------------------------------------------------

\documentclass[11pt]{article}
\usepackage{geometry} % Pour passer au format A4
\geometry{hmargin=1cm, vmargin=1cm} %

\usepackage{graphicx} % Required for including pictures
\usepackage{float} %

%Français
\usepackage[T1]{fontenc}
\usepackage[english,francais]{babel}
\usepackage[utf8]{inputenc}
\usepackage{eurosym}
\usepackage{lmodern}
\usepackage{url}
\usepackage{multicol}

%Maths
\usepackage{amsmath,amsfonts,amssymb,amsthm}
%\usepackage[linesnumbered, ruled, vlined]{algorithm2e}
%\SetAlFnt{\small\sffamily}

%Autres
\linespread{1} % Line spacing
\setlength\parindent{0pt} % Removes all indentation from paragraphs

\renewcommand{\labelenumi}{\alph{enumi}.} %
\pagestyle{empty}
%----------------------------------------------------------------------------------------
%	DOCUMENT INFORMATION
%----------------------------------------------------------------------------------------
\begin{document}

%\maketitle % Insert the title, author and date

\begin{minipage}[t]{\textwidth}
  \raggedright
      {\bfseries $2^{nd}1$}\\[.35ex]
      Série B\\
      \vspace*{-1cm}
      \raggedleft
          {\bfseries Statistiques}\\[.35ex]
          {\bfseries 19 Décembre 2014}\\[.35ex]
\end{minipage}\\[1em]

\begin{center}
  \textsf{A une époque de supercherie universelle, dire la vérité est un acte révolutionnaire. - George Orwell}
\end{center}

\setlength{\columnseprule}{1pt}

\subsection*{I - Connecté}

\textit{Pour une classe de seconde, on relève par élève le temps en minutes passés par jour devant un écran.}

\begin{center}
  \begin{tabular}{| c | c | c | c | c | c | c | c | c | c | c | c | c | c | c | c | c | c | }
    \hline
    52 & 12 & 25 & 54 & 121 & 204 & 45 & 25 & 15 & 44 & 6 & 125 & 45 & 32 & 20 & 19 & 64 & 69 \\
    \hline
    32 & 22 & 54 & 44 & 104 & 14  & 25 & 34 & 31 & 22 & 0 & 129 & 23 &  4 & 32 & 16 & 42 & 21 \\
    \hline
  \end{tabular}
\end{center}

\begin{multicols}{2}
  \begin{enumerate}
  \item[1.] Quelle est la population statistique étudiée ? Quel est le caractère étudié ? Comment est-il ?
  \item[2.] En moyenne combien de temps les élèves de cette classe passent-ils sur l'ordinateur ?
  \item[3.]
    \begin{enumerate}
    \item[a)] Donner \textbf{votre} temps passé sur l'ordinateur hier.
    \item[b)] En indiquant vos calculs, re-calculer la moyenne en ajoutant votre temps dans les données. Êtes-vous au dessus ou en dessous de la moyenne de la classe ?
    \item[c)] Modifiez-vous grandement la moyenne précédente ? Pouvez-vous trouver une justification ?
    \end{enumerate}
  \end{enumerate}
\end{multicols}

\subsection*{II - SMS}

\textit{Pour une classe de seconde le jour des vacances, on relève le nombre de SMS envoyés par chaque élève pendant le cours de maths.}

\begin{center}
  \begin{tabular}{| c | c | c | c | c | c | c | c | c | c | c | c | c | c | c | c | c | c | }
    \hline
    4 & 2 & 5 & 6 & 1 & 1 & 0 & 0 & 5 & 7 & 4 & 0 & 2 & 2 & 1 & 1 & 3 & 2 \\
    \hline
  \end{tabular}
\end{center}

\begin{multicols}{2}
  \begin{enumerate}
  \item[1.] Proposer un tableau des effectifs. (Lire la question suivante)
  \item[2.] Sur la ligne d'après, proposer les fréquences cumulées croissantes
  \item[3.] En moyenne, combien de SMS ont été envoyés ? 
  \item[4.] À l'aide d'un histogramme représenté au mieux cette série d'effectifs. Utiliser le patron.
  \end{enumerate}
\end{multicols}

\subsection*{III - Notes - \textit{Les parties 1 et 2 sont indépendantes.} }

\subsubsection*{1 - Contrôle de stats}

\textit{Lors du dernier contrôle de mathématiques, on relève les notes suivantes sur 20}

\begin{center}
  \begin{tabular}{| c || c | c | c | c | c | c | c | c | c | c | c | c | c | c | c | c | c | }
    \hline
    Notes     & 3 & 4 & 7 & 8 & 9 & 11 & 12 & 14 & 16 & 18 & 20 \\
    \hline
    Effectifs & 1 & 3 & 2 & 4 & 7 & 6  & 5  & 2  & 3  & 1  & 1  \\
    \hline
    ECC       & \phantom{xxxxxx} & \phantom{xxxxxx} &\phantom{xxxxxx}  & \phantom{xxxxxx} & \phantom{xxxxxx} & \phantom{xxxxxx}  & \phantom{xxxxxx}  & \phantom{xxxxxx}  & \phantom{xxxxxx}  & \phantom{xxxxxx}  & \phantom{xxxxxx}  \\
    \hline
  \end{tabular}
\end{center}

\begin{multicols}{2}
  \begin{enumerate}
  \item[1.] Calculer la moyenne de cette série.
  \item[2.] En rajoutant une ligne au tableau, proposer les effectifs cumulés croissants.
  \item[3.] Calculer la médiane de cette série. Interpréter ce résultat à l'aide d'une phrase.
  \item[4.] Calculer les quartiles Q1 et Q3.
  \item[5.] Quelle est l'étendue de la série.
  \end{enumerate}
\end{multicols}

\begin{multicols}{2}

\subsubsection*{2 - Cas particulier de Jean-Heude}

  \textit{Pour toutes notes (sur 20)  en mathématiques, Jean-Heude a obtenu :}
  \begin{itemize}
  \item 13 et 18 coefficient 1.
  \item 14 coefficient 2.
  \item 15 et 18 coefficient 3.
  \end{itemize}

  \begin{enumerate}
  \item[1.] Calculer la moyenne Jean-Heude pour ce trimestre.
  \item[2.] Comme souvent, le professeur de mathématiques s'est trompé en comptant les points de Jean-Heude. Sa note coefficient $2$ n'est pas de $14/20$ mais de $15/20$. Calculer sa nouvelle moyenne.
  \item[3.] En partant de la nouvelle moyenne. Jean-Heude se rend compte qu'il lui manque encore une note coefficient $3$. Combien doit-il obtenir \textbf{au minimum} afin d'obtenir une moyenne de 18 ? La notation est de 0.5 en 0.5.
  \end{enumerate}
\end{multicols}

\newpage



\begin{center}
  \begin{tabular}{| c || c | c | c | c | c | c | c |}
    \hline
    NBR  SMS     & \phantom{xxxxxxxxx} & \phantom{xxxxxxxxx} &\phantom{xxxxxxxxx} & \phantom{xxxxxxxxx} & \phantom{xxxxxxxxx} & \phantom{xxxxxxxxx} & \phantom{xxxxxxxx}\\
                 & \phantom{xxxxxxxxx} & \phantom{xxxxxxxxx} &\phantom{xxxxxxxxx} & \phantom{xxxxxxxxx} & \phantom{xxxxxxxxx} & \phantom{xxxxxxxxx} & \phantom{xxxxxxxx}\\
 
    \hline
    Effectifs    & \phantom{xxxxxxxxx} & \phantom{xxxxxxxxx} &\phantom{xxxxxxxxx} & \phantom{xxxxxxxxx} & \phantom{xxxxxxxxx} & \phantom{xxxxxxxxx} & \phantom{xxxxxxxx}\\
                 & \phantom{xxxxxxxxx} & \phantom{xxxxxxxxx} &\phantom{xxxxxxxxx} & \phantom{xxxxxxxxx} & \phantom{xxxxxxxxx} & \phantom{xxxxxxxxx} & \phantom{xxxxxxxx}\\ 

    \hline
    FCC          & \phantom{xxxxxxxxx} & \phantom{xxxxxxxxx} &\phantom{xxxxxxxxx} & \phantom{xxxxxxxxx} & \phantom{xxxxxxxxx} & \phantom{xxxxxxxxx} & \phantom{xxxxxxxx}\\
                 & \phantom{xxxxxxxxx} & \phantom{xxxxxxxxx} &\phantom{xxxxxxxxx} & \phantom{xxxxxxxxx} & \phantom{xxxxxxxxx} & \phantom{xxxxxxxxx} & \phantom{xxxxxxxx}\\ 
    \hline
  \end{tabular}
\end{center}

\vspace{0.2cm }


\begin{figure}[H]
\centering
\includegraphics[width=0.55\linewidth]{sources/ie/histogramme.pdf}
\end{figure}

\begin{center}
  \begin{tabular}{| c || c | c | c | c | c | c | c |}
    \hline
    NBR  SMS     & \phantom{xxxxxxxxx} & \phantom{xxxxxxxxx} &\phantom{xxxxxxxxx} & \phantom{xxxxxxxxx} & \phantom{xxxxxxxxx} & \phantom{xxxxxxxxx} & \phantom{xxxxxxxx}\\
                 & \phantom{xxxxxxxxx} & \phantom{xxxxxxxxx} &\phantom{xxxxxxxxx} & \phantom{xxxxxxxxx} & \phantom{xxxxxxxxx} & \phantom{xxxxxxxxx} & \phantom{xxxxxxxx}\\
 
    \hline
    Effectifs    & \phantom{xxxxxxxxx} & \phantom{xxxxxxxxx} &\phantom{xxxxxxxxx} & \phantom{xxxxxxxxx} & \phantom{xxxxxxxxx} & \phantom{xxxxxxxxx} & \phantom{xxxxxxxx}\\
                 & \phantom{xxxxxxxxx} & \phantom{xxxxxxxxx} &\phantom{xxxxxxxxx} & \phantom{xxxxxxxxx} & \phantom{xxxxxxxxx} & \phantom{xxxxxxxxx} & \phantom{xxxxxxxx}\\ 

    \hline
    FCC          & \phantom{xxxxxxxxx} & \phantom{xxxxxxxxx} &\phantom{xxxxxxxxx} & \phantom{xxxxxxxxx} & \phantom{xxxxxxxxx} & \phantom{xxxxxxxxx} & \phantom{xxxxxxxx}\\
                 & \phantom{xxxxxxxxx} & \phantom{xxxxxxxxx} &\phantom{xxxxxxxxx} & \phantom{xxxxxxxxx} & \phantom{xxxxxxxxx} & \phantom{xxxxxxxxx} & \phantom{xxxxxxxx}\\ 
    \hline
  \end{tabular}
\end{center}

\begin{figure}[H]
\centering
\includegraphics[width=0.55\linewidth]{sources/ie/histogramme.pdf}
\end{figure}



\newpage



\begin{center}
  \begin{tabular}{| c || c | c | c | c | c | c | c |}
    \hline
    NBR  SMS     & \phantom{xxxxxxxxx} & \phantom{xxxxxxxxx} &\phantom{xxxxxxxxx} & \phantom{xxxxxxxxx} & \phantom{xxxxxxxxx} & \phantom{xxxxxxxxx} & \phantom{xxxxxxxx}\\
                 & \phantom{xxxxxxxxx} & \phantom{xxxxxxxxx} &\phantom{xxxxxxxxx} & \phantom{xxxxxxxxx} & \phantom{xxxxxxxxx} & \phantom{xxxxxxxxx} & \phantom{xxxxxxxx}\\
 
    \hline
    Effectifs    & \phantom{xxxxxxxxx} & \phantom{xxxxxxxxx} &\phantom{xxxxxxxxx} & \phantom{xxxxxxxxx} & \phantom{xxxxxxxxx} & \phantom{xxxxxxxxx} & \phantom{xxxxxxxx}\\
                 & \phantom{xxxxxxxxx} & \phantom{xxxxxxxxx} &\phantom{xxxxxxxxx} & \phantom{xxxxxxxxx} & \phantom{xxxxxxxxx} & \phantom{xxxxxxxxx} & \phantom{xxxxxxxx}\\ 

    \hline
    FCC          & \phantom{xxxxxxxxx} & \phantom{xxxxxxxxx} &\phantom{xxxxxxxxx} & \phantom{xxxxxxxxx} & \phantom{xxxxxxxxx} & \phantom{xxxxxxxxx} & \phantom{xxxxxxxx}\\
                 & \phantom{xxxxxxxxx} & \phantom{xxxxxxxxx} &\phantom{xxxxxxxxx} & \phantom{xxxxxxxxx} & \phantom{xxxxxxxxx} & \phantom{xxxxxxxxx} & \phantom{xxxxxxxx}\\ 
    \hline
  \end{tabular}
\end{center}

\vspace{0.2cm }


\begin{figure}[H]
\centering
\includegraphics[width=0.55\linewidth]{sources/ie/histogramme.pdf}
\end{figure}

\begin{center}
  \begin{tabular}{| c || c | c | c | c | c | c | c |}
    \hline
    NBR  SMS     & \phantom{xxxxxxxxx} & \phantom{xxxxxxxxx} &\phantom{xxxxxxxxx} & \phantom{xxxxxxxxx} & \phantom{xxxxxxxxx} & \phantom{xxxxxxxxx} & \phantom{xxxxxxxx}\\
                 & \phantom{xxxxxxxxx} & \phantom{xxxxxxxxx} &\phantom{xxxxxxxxx} & \phantom{xxxxxxxxx} & \phantom{xxxxxxxxx} & \phantom{xxxxxxxxx} & \phantom{xxxxxxxx}\\
 
    \hline
    Effectifs    & \phantom{xxxxxxxxx} & \phantom{xxxxxxxxx} &\phantom{xxxxxxxxx} & \phantom{xxxxxxxxx} & \phantom{xxxxxxxxx} & \phantom{xxxxxxxxx} & \phantom{xxxxxxxx}\\
                 & \phantom{xxxxxxxxx} & \phantom{xxxxxxxxx} &\phantom{xxxxxxxxx} & \phantom{xxxxxxxxx} & \phantom{xxxxxxxxx} & \phantom{xxxxxxxxx} & \phantom{xxxxxxxx}\\ 

    \hline
    FCC          & \phantom{xxxxxxxxx} & \phantom{xxxxxxxxx} &\phantom{xxxxxxxxx} & \phantom{xxxxxxxxx} & \phantom{xxxxxxxxx} & \phantom{xxxxxxxxx} & \phantom{xxxxxxxx}\\
                 & \phantom{xxxxxxxxx} & \phantom{xxxxxxxxx} &\phantom{xxxxxxxxx} & \phantom{xxxxxxxxx} & \phantom{xxxxxxxxx} & \phantom{xxxxxxxxx} & \phantom{xxxxxxxx}\\ 
    \hline
  \end{tabular}
\end{center}

\begin{figure}[H]
\centering
\includegraphics[width=0.55\linewidth]{sources/ie/histogramme.pdf}
\end{figure}

\end{document}

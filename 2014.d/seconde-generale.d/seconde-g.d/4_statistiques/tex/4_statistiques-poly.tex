%%%%%%%%%%%%%%%%%%%%%%%%%%%%%%%%%%%%%%%%%
% Short Sectioned Assignment
% LaTeX Template
% Version 1.0 (5/5/12)
%
% This template has been downloaded from:
% http://www.LaTeXTemplates.com
%
% Original author:
% Frits Wenneker (http://www.howtotex.com)
%
% License:
% CC BY-NC-SA 3.0 (http://creativecommons.org/licenses/by-nc-sa/3.0/)
%
%%%%%%%%%%%%%%%%%%%%%%%%%%%%%%%%%%%%%%%%%

%----------------------------------------------------------------------------------------
%	PACKAGES AND OTHER DOCUMENT CONFIGURATIONS
%----------------------------------------------------------------------------------------

\documentclass[paper=a4, fontsize=9pt]{scrartcl} % A4 paper and 11pt font size


\usepackage[T1]{fontenc} % Use 8-bit encoding that has 256 glyphs
\usepackage[english,francais]{babel} % Français et anglais
\usepackage[utf8]{inputenc}

\usepackage{amsmath,amsfonts,amsthm} % Math packages

\usepackage{enumitem}
\usepackage{lmodern}
\usepackage{url}
\usepackage{eurosym} % signe Euros
\usepackage{geometry} % Pour passer au format A4
\geometry{a4paper} %
\usepackage{graphicx} % Required for including pictures
\usepackage{float} % Allows putting an [H] in \begin{figure} to specify the exact location of the figure

\usepackage{multicol}

\usepackage{verbatim}

\usepackage{sectsty} % Allows customizing section commands
\allsectionsfont{\centering \normalfont\scshape} % Make all sections centered, the default font and small caps

%----------------------------------------------------------------------------------------
%	Pied de Page
%----------------------------------------------------------------------------------------


\usepackage{fancyhdr} % Custom headers and footers
\pagestyle{fancyplain} % Makes all pages in the document conform to the custom headers and footers
\fancyhead{} % No page header - if you want one, create it in the same way as the footers below
\fancyfoot[L]{$2^{nd}1$} % Empty left footer
\fancyfoot[C]{Chapitre 4 - Statistiques} % Empty center footer
\fancyfoot[R]{\thepage} % Page numbering for right footer

\renewcommand{\headrulewidth}{0pt} % Remove header underlines
\renewcommand{\footrulewidth}{0pt} % Remove footer underlines

\setlength{\headheight}{13.6pt} % Customize the height of the header


\setlength\parindent{0pt} % Removes all indentation from paragraphs - comment this line for an assignment with lots of text


%----------------------------------------------------------------------------------------
%	Titre
%----------------------------------------------------------------------------------------

\newcommand{\horrule}[1]{\rule{\linewidth}{#1}} % Create horizontal rule command with 1 argument of height


\title{
  \vspace{-10ex}
  \horrule{0.5pt} \\[0.4cm] % Thin top horizontal rule
  \huge Chapitre 4 - Statistiques\\ % The assignment title
  \horrule{2pt} \\[0.5cm] % Thick bottom horizontal rule
}

\author{}
\date{\vspace{-10ex}} % Today's date or a custom date

%----------------------------------------------------------------------------------------
%	Début du document
%----------------------------------------------------------------------------------------

\begin{document}

%----------------------------------------------------------------------------------------
% RE-DEFINITION
%----------------------------------------------------------------------------------------
% MATHS
%-----------

\newtheorem{Definition}{Définition}
\newtheorem{Theorem}{Théorème}
\newtheorem{Proposition}{Propriété}

% MATHS
%-----------
\renewcommand{\labelitemi}{$\bullet$}
\renewcommand{\labelitemii}{$\circ$}
%----------------------------------------------------------------------------------------
%	Titre
%----------------------------------------------------------------------------------------

\maketitle % Print the title
\setlength{\columnseprule}{1pt}

%-----------------------------------111111111111111111111111111111111111
\section{Généralités sur la Statistique}
%-----------------------------------------------------------------------
\begin{multicols}{2}

  \subsection{Introduction}

  \begin{Definition}
    On appelle Statistique l'ensemble des méthodes et des techniques permettant d'analyser et de traiter des ensembles d'observations appelées données.
  \end{Definition}

  \begin{Definition}
    La statistique descriptive regroupe des méthodes dont l'objet principal est la description des données étudiées.
    Cette description des données se fait à travers l'étude de leur caractéristique et leur représentation graphique
  \end{Definition}

  \paragraph{Remarques}~~\\
  \begin{itemize}
  \item Un autre groupe de méthodes provient de la Statistique inférentielle. On les utilise après celle de Statistique descriptive.
  \item De manière générale, les méthodes statistiques relève du domaine des mathématiques et font largement appel à l'outil informatique.
  \end{itemize}

  \begin{Definition}
    Une population (statistique) est composé d'un ensemble d'individu. Une étude statistique se fait à l'aide de données recueillies
    sur un caractère d'étude. Ses caractères peuvent être numérique (ou  qualitative).
  \end{Definition}

  \subsubsection{Activité de présentation}~~\\
  En s'intéressant à la population de la classe entière de seconde 1, un individu est un élève. On s'intéresse au caractère d'étude du nombre de frère et sœur.
  L'organisation de la récupération de données est une part importante du travail.

  \begin{center}
    \begin{tabular}{| l || c | c | c | c | c |}
      \hline
      Personnes    &  a. & b. & c. & d. & ...\\
      \hline
      Nombre de frères et sœurs & 1 & 0 & 3 & 1 & ...\\
      \hline
    \end{tabular}
  \end{center}

  Une meilleur organisation consiste à s'intéresser aux effectifs et aux fréquences.
\end{multicols}
%-----------------------------------111111111111111111111111111111111111
\section{Structure de données}
%----------------------------------------------------------------------
\begin{multicols}{2}

  \subsection{Effectifs et fréquences}
  %----------------------------------------------------------------------

  \begin{Definition}~~\\
    \begin{enumerate}
    \item L'\textbf{effectif} est le nombre d'individus dans une population possédant une certaine \textbf{valeur} d'un caractère.
    \item L'\textbf{effectif total} noté n est le nombre total d'individu. On parle également de la taille de la population.
    \item La \textbf{fréquence} d'une valeur est le quotient de l'effectif par l'effectif total : $\text{fréquence} = \dfrac{\text{effectif}}{\text{effectif total}}$
    \end{enumerate}
  \end{Definition}

  \subsection{Effectifs et fréquences cumulé croissant}
  %----------------------------------------------------------------------

  \begin{Definition}~~\\
    \begin{enumerate}
    \item L'\textbf{effectif cumulé croissant} d'une valeur est la somme des effectifs de toutes les valeurs plus petites ou égales.
    \item La \textbf{fréquence cumulée croissante} d'une valeur est la somme des fréquences de toutes les valeurs plus petites ou égales.
    \end{enumerate}
  \end{Definition}

\end{multicols}

\subsection{Activité de présentation}

L'effectif total de la classe est n=35.

\begin{multicols}{2}

  \begin{center}
    \begin{tabular}{| l || c | c | c | c | c |}
      \hline
      Nbr de frères & 0 & 1 & 2 & 3 & ...\\
      \hline
      Effectif      & 4 & 2 & 6 & 1 & ...\\
      \hline
      Fréquence     & 0.11 & 0.057 & 0.17 & 0.029 & ...\\
      \hline
    \end{tabular}
  \end{center}

  \begin{center}
    \begin{tabular}{| l || c | c | c | c | c |}
      \hline
      Nbr de frères & 0 & 1 & 2 & 3 & ...\\
      \hline
      Effectif cumulé croissant & 4 & 6 & 12 & 13 & ...\\
      \hline
      Fréquence cumulée croissante & 0.11 & 0.17 & 0.34 & 0.35 & ...\\
      \hline
    \end{tabular}
  \end{center}

\end{multicols}
\begin{multicols}{2}
  %-----------------------------------111111111111111111111111111111111111
  \section{Représentation graphique des données}
  %----------------------------------------------------------------------

  La représentation graphique privilégié en statistique est l'histogramme. On le préfère au nuage de point ainsi qu'au diagramme circulaire.

  %-----------------------------------111111111111111111111111111111111111
  \section{Caractéristiques des données}
  %----------------------------------------------------------------------

  Soit la série statistique donnée par le tableau suivant. On note $N = n_1 + n_2 + ... + n_3$.

  \begin{center}
    \begin{tabular}{| l || c | c | c | c | c |}
      \hline
      Valeur    & $x_1$ & $x_2$ & $x_3$ & ... & $x_p$\\
      \hline
      Effectif  & $n_1$ & $n_2$ & $n_3$ & ... & $n_p$\\
      \hline
      Fréquence & $f_1$ & $f_2$ & $f_3$ & ... & $f_p$\\
      \hline
    \end{tabular}
  \end{center}
\end{multicols}
\subsection{Caractéristiques de position}
%----------------------------------------------------------------------

\begin{multicols}{2}
  \subsubsection{Moyenne}~~\\

  \begin{Definition} La moyenne\\
    La moyenne est le nombre noté $\bar{x} = \dfrac{n_1 x_1 + n_2 x_2 + n_3 x_3 + ... + n_p x_p}{N}$\\
  \end{Definition}

  \begin{Proposition}
    On peut également calculer la moyenne à l'aide des fréquences.\\
    $\bar{x} = f_1 x_1 + f_2 x_2 + f_3 x_3 + ... + f_p x_p$\\
  \end{Proposition}

  \subsubsection{Médiane}~~\\

  \begin{Definition}La médiane\\
    On considère une série statistique dont les valeurs sont ordonnées. Selon la valeur de N l'effectif total la médiane n'est pas la même.
    \begin{itemize}
    \item Si N est impair, la médiane est le terme du milieu : $x_{\frac{N + 1}{2}}$.
    \item Si N est pair, la médiane est la somme des deux termes du milieu : $\dfrac{x_{\frac{N}{2}} + x_{\frac{N+1}{2}}}{2}$.
    \end{itemize}
  \end{Definition}
\end{multicols}
\subsection{Caractéristiques de dispersion}

\begin{Definition}Quartiles\\
  On considère une série statistique dont les valeurs sont ordonnées.

  \begin{enumerate}
  \item Le premier quartile, noté $Q_1$ est la plus petite valeur de la série telle que 25\% des valeurs lui soient inférieurs : $x_\frac{N}{4}$.
  \item Le troisième quartile, noté $Q_3$ est la plus petite valeur de la série telle que 75\% des valeurs lui soient inférieurs : $x_\frac{3N}{4}$.
  \item Le deuxième quartile est la médiane (avec 50\% des valeurs inférieurs)
  \end{enumerate}
\end{Definition}

\begin{Definition}Étendue\\
  L'étendue d'une série statistique est la différence entre la plus grande et la plus petite des valeurs.
\end{Definition}

\subsection{Exemple de cours - Ex 49 p159}
\begin{multicols}{2}
  Soit la série : 12, 9, 6, 13, 10, 9, 8, 16, 11, 17, 9, 9, 16, 13, 17, 9, 14
  \begin{enumerate}
  \item On calcul l'effectif total : $N = 17$. Puis la moyenne : $\bar{x}  = \frac{1}{17} \times (12+9+6+13+10+9+8+16+11+17+9+9+16+13+17+9+14) = \frac{198}{17} = 11.65$ 
  \item Il faut proposer une série statistique ordonnée.
    \begin{itemize}
    \item N est impair, on choisit la valeur du milieu, la médiane est 11.
    \item $\frac{17}{4} = 4.25$, on s'intéresse à la cinquième valeur : $Q_1 = 9$
    \item $\frac{3*17}{4} = 12.75$, on s'intéresse à la treizième valeur : $Q2 = 14$
    \end{itemize}
  \end{enumerate}
\end{multicols}
\begin{verbatim}
# Sous R
data=c(12, 9, 6, 13, 10, 9, 8, 16, 11, 17, 9, 9, 16, 13, 17, 9, 14) # Données
hist(data,breaks=17,col=''red'') # Histogramme
summary(data) # Caractéristiques
\end{verbatim}
\end{document}

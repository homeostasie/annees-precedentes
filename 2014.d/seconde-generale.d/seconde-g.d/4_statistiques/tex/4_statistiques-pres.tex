\documentclass{beamer}

\usepackage{geometry} % Pour passer au format A4
\usepackage{graphicx} % Required for including pictures
\usepackage{float} %

\usepackage{amsmath,amsfonts,amssymb,amsthm}
\usepackage[T1]{fontenc}
\usepackage[english,francais]{babel}
\usepackage[utf8]{inputenc}
\usepackage{lmodern}
\usepackage{eurosym} % signe Euros
\usepackage{verbatim}

\usetheme{Warsaw}

\title{Statistique}
\author{$2^{nd}1$}

\begin{document}

\frame{\titlepage}

\section{Généralités sur la Statistique}
\subsection{Introduction}

\begin{frame}
  \frametitle{I - Généralités sur la Statistique}

  \begin{block}{1 - Introduction}
    \begin{alertblock}{Définition 1}
      On appelle Statistique l'ensemble des méthodes et des techniques permettant d'analyser et de traiter des ensembles d'observations appelées données.
    \end{alertblock}

    \begin{alertblock}{Définition 2}
      La statistique descriptive regroupe des méthodes dont l'objet principal est la description des données étudiées.
      Cette description des données se fait à travers l'étude de leur caractéristique et leur représentation graphique
    \end{alertblock}
  \end{block}
\end{frame}

\begin{frame}
  \frametitle{I - Généralités sur la Statistique}

  \begin{exampleblock}{Remarques}
    \begin{itemize}
    \item Un autre groupe de méthodes provient de la Statistique inférentielle. On les utilise après celle de Statistique descriptive.
    \item De manière générale, les méthodes statistiques relève du domaine des mathématiques et font largement appel à l'outil informatique.
    \end{itemize}
  \end{exampleblock}

  \begin{alertblock}{Définition 3}
    Une population (statistique) est composé d'un ensemble d'individu. Une étude statistique se fait à l'aide de données recueillies
    sur un caractère d'étude. Ses caractères peuvent être numérique (ou  qualitative).
  \end{alertblock}
\end{frame}

\begin{frame}
  \frametitle{I - Généralités sur la Statistique}

  \begin{exampleblock}{Activité de chapitre}
    En s'intéressant à la population de la classe entière de seconde 1, un individu est un élève. On s'intéresse au caractère d'étude du nombre de frère et sœur.\\
    L'organisation de la récupération de données est une part importante du travail.
  \end{exampleblock}
\end{frame}

\section{Structure de données}
\subsection{Effectifs et fréquences}

\begin{frame}
  \frametitle{II - Structure de données}

  \begin{block}{1 - Effectifs et fréquences}
    \begin{alertblock}{Définition 4}
      \begin{enumerate}
      \item L'\textbf{effectif} est le nombre d'individus dans une population possédant une certaine \textbf{valeur} d'un caractère.
      \item L'\textbf{effectif total} noté $n$ est le nombre total d'individus. On parle également de la taille de la population.
      \item La \textbf{fréquence} d'une valeur est le quotient de l'effectif par l'effectif total : $\text{fréquence} = \dfrac{\text{effectif}}{\text{effectif total}}$
      \end{enumerate}
    \end{alertblock}
  \end{block}
\end{frame}

\subsection{Effectifs et fréquences cumulés croissants}

\begin{frame}
  \frametitle{II - Structure de données}

  \begin{block}{2 - Effectifs et fréquences}
    \begin{alertblock}{Définition 5}
      \begin{enumerate}
      \item L'\textbf{effectif cumulé croissant} d'une valeur est la somme des effectifs de toutes les valeurs plus petites ou égales.
      \item La \textbf{fréquence cumulée croissante} d'une valeur est la somme des fréquences de toutes les valeurs plus petites ou égales.
      \end{enumerate}
    \end{alertblock}
  \end{block}
\end{frame}


%-----------------------------------111111111111111111111111111111111111
\section{Représentation graphique des données}
%----------------------------------------------------------------------

\begin{frame}
  \frametitle{III - Représentation graphique des données}

  \begin{exampleblock}{}
    La représentation graphique privilégiée en statistique est l'\textbf{histogramme}. On le préfère au nuages de point ainsi qu'au diagramme circulaire.
  \end{exampleblock}
\end{frame}

%-----------------------------------111111111111111111111111111111111111
\section{Caractéristiques des données}
%----------------------------------------------------------------------
\subsection{Caractèristiques de position}
%----------------------------------------------------------------------

\begin{frame}
  \frametitle{IV - Caractéristiques des données}

  \begin{block}{Données}
    Soit la série statistique donnée par le tableau suivant. On note $N = n_1 + n_2 + ... + n_3$.

    \begin{center}
      \begin{tabular}{| l || c | c | c | c | c |}
        \hline
        Valeur    & $x_1$ & $x_2$ & $x_3$ & ... & $x_p$\\
        \hline
        Effectif  & $n_1$ & $n_2$ & $n_3$ & ... & $n_p$\\
        \hline
        Fréquence & $f_1$ & $f_2$ & $f_3$ & ... & $f_p$\\
        \hline
      \end{tabular}
    \end{center}
  \end{block}
\end{frame}

\begin{frame}
  \frametitle{1 - Caractéristiques de position}
  
  \begin{block}{a) La Moyenne}
  \end{block}
  
  \begin{alertblock}{Définition 6 - Moyenne}
    La moyenne est le nombre noté $\bar{x} = \dfrac{n_1 x_1 + n_2 x_2 + n_3 x_3 + ... + n_p x_p}{N}$\\
  \end{alertblock}
  
  \begin{block}{Propriété 1}
    On peut également calculer la moyenne à l'aide des fréquences.\\
    $\bar{x} = f_1 x_1 + f_2 x_2 + f_3 x_3 + ... + f_p x_p$\\
  \end{block}
\end{frame}


\begin{frame}
  \frametitle{1 - Caractéristiques de position}
  
  \begin{block}{b) La Médiane}
  \end{block}
  
  \begin{alertblock}{Définition 7 - Médiane}
    On considère une série statistique dont les valeurs sont ordonnées. Selon la valeur de N l'effectif total la médiane n'est pas la même.
    \begin{itemize}
    \item Si N est impair, la médiane est le terme du milieu : $x_{\frac{N + 1}{2}}$.
    \item Si N est pair, la médiane est la somme des deux termes du milieu : $\dfrac{x_{\frac{N}{2}} + x_{\frac{N+1}{2}}}{2}$.
    \end{itemize}
  \end{alertblock}
\end{frame}

\subsection{Caractéristiques de dispersion}
%----------------------------------------------------------------------

\begin{frame}
  \frametitle{2 - Caractéristiques de dispersion }

  \begin{alertblock}{Définition 8 - Quartiles}
    On considère une série statistique dont les valeurs sont ordonnées.

    \begin{enumerate}
    \item Le premier quartile, noté $Q_1$ est la plus petite valeur de la série telle que 25\% des valeurs lui soient inférieurs : $x_\frac{N}{4}$.
    \item Le troisième quartile, noté $Q_3$ est la plus petite valeur de la série telle que 75\% des valeurs lui soient inférieurs : $x_\frac{3N}{4}$.
    \item Le deuxième quartile est la médiane (avec 50\% des valeurs inférieurs)
    \end{enumerate}
  \end{alertblock}

  \begin{alertblock}{Définition 9 - Étendue}
    L'étendue d'une série statistique est la différence entre la plus grande et la plus petite des valeurs.
  \end{alertblock}
\end{frame}

\end{document}

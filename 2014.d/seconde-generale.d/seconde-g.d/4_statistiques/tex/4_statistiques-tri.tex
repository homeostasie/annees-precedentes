%%%%%%%%%%%%%%%%%%%%%%%%%%%%%%%%%%%%%%%%%
% Short Sectioned Assignment
% LaTeX Template
% Version 1.0 (5/5/12)
%
% This template has been downloaded from:
% http://www.LaTeXTemplates.com
%
% Original author:
% Frits Wenneker (http://www.howtotex.com)
%
% License:
% CC BY-NC-SA 3.0 (http://creativecommons.org/licenses/by-nc-sa/3.0/)
%
%%%%%%%%%%%%%%%%%%%%%%%%%%%%%%%%%%%%%%%%%

%----------------------------------------------------------------------------------------
%	PACKAGES AND OTHER DOCUMENT CONFIGURATIONS
%----------------------------------------------------------------------------------------

\documentclass[paper=a4, fontsize=9pt]{scrartcl} % A4 paper and 11pt font size


\usepackage[T1]{fontenc} % Use 8-bit encoding that has 256 glyphs
\usepackage[english,francais]{babel} % Français et anglais
\usepackage[utf8]{inputenc}

\usepackage{amsmath,amsfonts,amsthm} % Math packages

\usepackage{enumitem}
\usepackage{lmodern}
\usepackage{url}
\usepackage{eurosym} % signe Euros
\usepackage{geometry} % Pour passer au format A4
\geometry{a4paper} %
\usepackage{graphicx} % Required for including pictures
\usepackage{float} % Allows putting an [H] in \begin{figure} to specify the exact location of the figure

\usepackage{multicol}

\usepackage{verbatim}

\usepackage{sectsty} % Allows customizing section commands
\allsectionsfont{\centering \normalfont\scshape} % Make all sections centered, the default font and small caps

%----------------------------------------------------------------------------------------
%	Pied de Page
%----------------------------------------------------------------------------------------


\usepackage{fancyhdr} % Custom headers and footers
\pagestyle{fancyplain} % Makes all pages in the document conform to the custom headers and footers
\fancyhead{} % No page header - if you want one, create it in the same way as the footers below
\fancyfoot[L]{$2^{nd}1$} % Empty left footer
\fancyfoot[C]{Chapitre 4 - Statistiques} % Empty center footer
\fancyfoot[R]{\thepage} % Page numbering for right footer

\renewcommand{\headrulewidth}{0pt} % Remove header underlines
\renewcommand{\footrulewidth}{0pt} % Remove footer underlines

\setlength{\headheight}{13.6pt} % Customize the height of the header


\setlength\parindent{0pt} % Removes all indentation from paragraphs - comment this line for an assignment with lots of text


%----------------------------------------------------------------------------------------
%	Titre
%----------------------------------------------------------------------------------------

\newcommand{\horrule}[1]{\rule{\linewidth}{#1}} % Create horizontal rule command with 1 argument of height


\title{
  \vspace{-10ex}
  \horrule{0.5pt} \\[0.4cm] % Thin top horizontal rule
  \huge Chapitre 4 - Statistiques\\ % The assignment title
  Tris
  \horrule{2pt} \\[0.5cm] % Thick bottom horizontal rule
}

\author{}
\date{\vspace{-10ex}} % Today's date or a custom date

%----------------------------------------------------------------------------------------
%	Début du document
%----------------------------------------------------------------------------------------

\begin{document}

%----------------------------------------------------------------------------------------
% RE-DEFINITION
%----------------------------------------------------------------------------------------
% MATHS
%-----------

\newtheorem{Definition}{Définition}
\newtheorem{Theorem}{Théorème}
\newtheorem{Proposition}{Propriété}

% MATHS
%-----------
\renewcommand{\labelitemi}{$\bullet$}
\renewcommand{\labelitemii}{$\circ$}
%----------------------------------------------------------------------------------------
%	Titre
%----------------------------------------------------------------------------------------

\maketitle % Print the title
\setlength{\columnseprule}{1pt}

Pour une liste d'éléments de taille $n$ (numérotés de 1 à n)
%-----------------------------------111111111111111111111111111111111111
\section{Tri par sélection}

\textit{On recherche le plus petit élément qu'on place en bas d'un tas et ainsi de suite: 6; 2; 8; 1; 5; 3; 7; 9; 4; 0.}\\
Fonction : échanger.

\begin{enumerate}
\item Rechercher le plus petit élément du tableau, et l'échanger avec l'élément d'indice 1.
\item Rechercher le second plus petit élément du tableau, et l'échanger avec l'élément d'indice 2.
\item Continuer de cette façon jusqu'à ce que la liste soit entièrement trié.
\end{enumerate}

\section{Tri par insertion}

\textit{Le tri typique des jeux de carte : 6; 2; 8; 1; 5; 3; 7; 9; 4; 0.}

\begin{enumerate}
\item On prend les deux premières cartes et on crée un tas dans la main gauche.
\item On ordonne les deux cartes de la main gauche.
\item On prend la prochaine carte dans le tas restant.
\item On l'insert correctement dans le tas de gauche.
\item Et on repart de 3.
\end{enumerate}

\section{Tri par bulle}

\textit{On fait remonter une bulle :  5 1 4 2 8.}\\
Fonction : échanger
\begin{enumerate}
\item On choisit le premièr nombre.
\item On le compare au nombre du dessus.
  \begin{itemize}
  \item S'il est plus grand, on les permutes deux à deux et on repart en 2.
  \item Sinon on choisit le deuxième nombre et on repart en 2.
  \end{itemize}
\end{enumerate}


\section{Tri par fusion}

\textit{On sépare en tout petite tas qu'on fusionne petit à petit facilement :  38 ; 27 ; 43 ; 3 ; 9 ; 82 ; 10}\\

\begin{enumerate}
\item On prend un tas, on le sépare en deux tas égaux.
\item On répète cette étape jusqu'à obtenir des tas de taille 1.
\item On ordonne chaque tas de deux.
\item On fusionne les tas deux par deux.
\item ... jusqu'à retrouver le premier tas trier.
\end{enumerate}


\newpage

\begin{multicols}{2}
  \section{Tri par sélection}

\begin{verbatim}
pour i de 1 à n - 1
          min = i
          pour j de i + 1 à n
              si t[j] < t[min]
                      alors min = j
          fin pour
          si min != i, 
                 alors échanger t[i] et t[min]
      fin pour
\end{verbatim}

\begin{enumerate}
\item 6281537\textbf{9}40 $\to$ 6281537\textbf{0}4\textbf{9} 
\item 62\textbf{8}1537049 $\to$ 62\textbf{4}15370\textbf{8}9
\item 624153\textbf{7}089 $\to$ 624153\textbf{07}89
\item \textbf{6}241530789 $\to$ \textbf{0}24153\textbf{6}789
\item 0241\textbf{5}36789 $\to$ 0241\textbf{35}6789
\item 02\textbf{4}1356789 $\to$ 023\textbf{14}56789
\item 02\textbf{3}1456789 $\to$ 02\textbf{13}456789
\item 0\textbf{2}13456789 $\to$ 0\textbf{12}3456789
\item 0123456789
\end{enumerate}

\section{Tri par insertion}

\begin{verbatim}
pour i de 1 à n-1
     x = t[i]
     j = i
     tant que j > 0 et t[j - 1] > x
          t[j] = t[j - 1]
          j = j - 1
     fin tant que
     t[j] = x
fin pour
\end{verbatim}

\begin{enumerate}
\item 6281537940
\item 62 ; 81537940  
\item 26 ; 81537940
\item 268 ; 1537940
\item 1268 ; 537940
\item 12568 ; 37940
\item 123568 ; 7940
\item 1235678 ; 940
\item 12356789 ; 40
\item 123456789 ; 0
\item 0123456789
\end{enumerate}

\section{Tri par bulle}

\begin{verbatim}
pour i de n - 1 à 1
    aucun_échange = vrai
    pour j de 1 à i
         si t[j] > t[j + 1], alors
             échanger t[j] et t[j + 1]
             aucun_échange = faux
         fin si
    fin pour 
    si aucun_échange : fin procédure
fin pour
\end{verbatim}

\begin{enumerate}
\item Première étape:
  \begin{enumerate}
  \item ( 5 1 4 2 8 ) $\to$ ( 1 5 4 2 8 ) Les éléments 5 et 1 sont comparés, et comme 5 > 1, l'algorithme les intervertit.
  \item ( 1 5 4 2 8 ) $\to$ ( 1 4 5 2 8 ) Interversion car 5 > 4.
  \item ( 1 4 5 2 8 ) $\to$ ( 1 4 2 5 8 ) Interversion car 5 > 2.
  \item ( 1 4 2 5 8 ) $\to$ ( 1 4 2 5 8 ) Comme 5 < 8, les éléments ne sont pas échangés.
  \end{enumerate}
\item Deuxième étape:
  \begin{enumerate}
  \item ( 1 4 2 5 8 ) $\to$ ( 1 4 2 5 8 ) Même principe qu'à l'étape 1.
  \item ( 1 4 2 5 8 ) $\to$ ( 1 2 4 5 8 )
  \item ( 1 2 4 5 8 ) $\to$ ( 1 2 4 5 8 )
    À ce stade, la liste est triée, mais pour le détecter, l'algorithme doit effectuer un dernier parcours.
  \end{enumerate}
\item Troisième étape:
  \begin{enumerate}
  \item ( 1 2 4 5 8 ) $\to$ ( 1 2 4 5 8 )
  \item ( 1 2 4 5 8 ) $\to$ ( 1 2 4 5 8 )
    Comme la liste est triée, aucune interversion n'a lieu à cette étape, ce qui provoque l'arrêt de l'algorithme.
  \end{enumerate}
\end{enumerate}
\section{Tri par fusion}

\begin{enumerate}
\item 38 ; 27 ; 43 ; 3 ; 9 ; 82 ; 10
\item 38 ; 27 ; 43 ; 3 || 9 ;82 ; 10
\item 38 ; 27 || 43 ; 3 || 9 ; 82 || 10
\item 38 || 27 || 43 || 3 || 9 || 82 || 10
\item 27 ; 38 || 3 ; 43 || 9 ; 82 || 10
\item 3 ; 27 ; 38 ; 43 || 9 ; 10 ; 82
\item 3 ; 9 ; 10 ; 27 ; 38 ; 43 ; 82
\end{enumerate}
\end{multicols}
\end{document}

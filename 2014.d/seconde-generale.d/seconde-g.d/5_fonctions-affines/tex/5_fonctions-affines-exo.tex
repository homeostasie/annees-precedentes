%%%%%%%%%%%%%%%%%%%%%%%%%%%%%%%%%%%%%%%%%
% Short Sectioned Assignment
% LaTeX Template
% Version 1.0 (5/5/12)
%
% This template has been downloaded from:
% http://www.LaTeXTemplates.com
%
% Original author:
% Frits Wenneker (http://www.howtotex.com)
%
% License:
% CC BY-NC-SA 3.0 (http://creativecommons.org/licenses/by-nc-sa/3.0/)
%
%%%%%%%%%%%%%%%%%%%%%%%%%%%%%%%%%%%%%%%%%

%----------------------------------------------------------------------------------------
%	PACKAGES AND OTHER DOCUMENT CONFIGURATIONS
%----------------------------------------------------------------------------------------

\documentclass[paper=a4, fontsize=9pt]{scrartcl} % A4 paper and 11pt font size


\usepackage[T1]{fontenc} % Use 8-bit encoding that has 256 glyphs
\usepackage[english,francais]{babel} % Français et anglais
\usepackage[utf8]{inputenc}

\usepackage{amsmath,amsfonts,amsthm} % Math packages

\usepackage{enumitem}
\usepackage{lmodern}
\usepackage{url}
\usepackage{eurosym} % signe Euros
\usepackage{geometry} % Pour passer au format A4
\geometry{a4paper} %
\usepackage{graphicx} % Required for including pictures
\usepackage{float} % Allows putting an [H] in \begin{figure} to specify the exact location of the figure

\usepackage{multicol}

\usepackage{verbatim}

\usepackage{sectsty} % Allows customizing section commands
\allsectionsfont{\centering \normalfont\scshape} % Make all sections centered, the default font and small caps

%----------------------------------------------------------------------------------------
%	Pied de Page
%----------------------------------------------------------------------------------------


\usepackage{fancyhdr} % Custom headers and footers
\pagestyle{fancyplain} % Makes all pages in the document conform to the custom headers and footers
\fancyhead{} % No page header - if you want one, create it in the same way as the footers below
\fancyfoot[L]{$2^{nd}1$} % Empty left footer
\fancyfoot[C]{Chapitre 5 - Fonctions affines} % Empty center footer
\fancyfoot[R]{\thepage} % Page numbering for right footer

\renewcommand{\headrulewidth}{0pt} % Remove header underlines
\renewcommand{\footrulewidth}{0pt} % Remove footer underlines

\setlength{\headheight}{13.6pt} % Customize the height of the header


\setlength\parindent{0pt} % Removes all indentation from paragraphs - comment this line for an assignment with lots of text


%----------------------------------------------------------------------------------------
%	Titre
%----------------------------------------------------------------------------------------

\newcommand{\horrule}[1]{\rule{\linewidth}{#1}} % Create horizontal rule command with 1 argument of height


\title{
  \vspace{-10ex}
  \horrule{0.5pt} \\[0.4cm] % Thin top horizontal rule
  \huge Chapitre 5 - Fonctions affines\\ % The assignment title
  \horrule{2pt} \\[0.5cm] % Thick bottom horizontal rule
}

\author{}
\date{\vspace{-10ex}} % Today's date or a custom date

%----------------------------------------------------------------------------------------
%	Début du document
%----------------------------------------------------------------------------------------

\begin{document}

%----------------------------------------------------------------------------------------
% RE-DEFINITION
%----------------------------------------------------------------------------------------
% MATHS
%-----------

\newtheorem{Definition}{Définition}
\newtheorem{Theorem}{Théorème}
\newtheorem{Proposition}{Propriété}

% MATHS
%-----------
\renewcommand{\labelitemi}{$\bullet$}
\renewcommand{\labelitemii}{$\circ$}
%----------------------------------------------------------------------------------------
%	Titre
%----------------------------------------------------------------------------------------

%\maketitle % Print the title
\setlength{\columnseprule}{1pt}


\subsection*{Cartouche d'encre}

\begin{itemize}
\item Dans un magasin, une cartouche d'encre coûte 15\euro. On appelle $f_m$ le prix à payer par cartouche en magasin.
\item Sur Internet, on peut acheter des cartouches pour seulement 10\euro \text{ }l'unité. Il faut également prendre en compte les frais de ports qui sont fixes et s'élèvent à 40\euro \text{ } quel que soit le nombre de cartouche achetées. On appelle $f_i$ le prix à payer par cartouche sur Internet.
\end{itemize}


\begin{enumerate}
\item Reproduire et compléter le tableau suivant : 
  \begin{center}
    \begin{tabular}{| l || c | c | c | c | c |}
      \hline
      Nombre de cartouches achetées     & 2 & 5 & 11 & 14 \\
      \hline
      Prix à payer en magasin : $f_m$   & \phantom{azerty}  & \phantom{azerty}  & \phantom{azerty} & \phantom{azerty}\\
      \hline
      Prix à payer sur Internet : $f_i$ & \phantom{azerty}  & \phantom{azerty}  & \phantom{azerty} & \phantom{azerty}\\     
      \hline
    \end{tabular}
  \end{center}

\item  On note $x$ le nombre de cartouche achetées.
  \begin{enumerate}
  \item Donner l'expression de $f_m$ en fonction de $x$.
  \item Donner l'expression de $f_i$ en fonction de $x$.
  \end{enumerate} 

\item Graphiquement
  \begin{enumerate}
  \item Tracer les courbes représentatives $\mathcal{C}_m$ et $\mathcal{C}_i$ des fonctions $f_m$ et $f_i$.
  \item Quand est-il plus intéréssant d'acheter ses cartouches sur Internet ? Écrire également la réponse à l'aide d'intervalle.
  \end{enumerate} 

\item On se propose de retrouver ces résultats par le calcul à l'aide de la fonction :  $g(x) = f_m(x) - f_i(x)$
  \begin{enumerate}
  \item Donner l'expression littérale de $g(x)$.
  \item Étudier le signe de $g(x)$ sur $[0 ; \infty[$.
  \end{enumerate} 

\end{enumerate}

\end{document}

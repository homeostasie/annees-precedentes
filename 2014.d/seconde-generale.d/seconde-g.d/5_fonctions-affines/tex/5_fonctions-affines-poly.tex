%%%%%%%%%%%%%%%%%%%%%%%%%%%%%%%%%%%%%%%%%
% Short Sectioned Assignment
% LaTeX Template
% Version 1.0 (5/5/12)
%
% This template has been downloaded from:
% http://www.LaTeXTemplates.com
%
% Original author:
% Frits Wenneker (http://www.howtotex.com)
%
% License:
% CC BY-NC-SA 3.0 (http://creativecommons.org/licenses/by-nc-sa/3.0/)
%
%%%%%%%%%%%%%%%%%%%%%%%%%%%%%%%%%%%%%%%%%

%----------------------------------------------------------------------------------------
%	PACKAGES AND OTHER DOCUMENT CONFIGURATIONS
%----------------------------------------------------------------------------------------

\documentclass[paper=a4, fontsize=9pt]{scrartcl} % A4 paper and 11pt font size


\usepackage[T1]{fontenc} % Use 8-bit encoding that has 256 glyphs
\usepackage[english,francais]{babel} % Français et anglais
\usepackage[utf8]{inputenc}

\usepackage{amsmath,amsfonts,amsthm} % Math packages

\usepackage{enumitem}
\usepackage{lmodern}
\usepackage{url}
\usepackage{eurosym} % signe Euros
\usepackage{geometry} % Pour passer au format A4
\geometry{a4paper} %
\usepackage{graphicx} % Required for including pictures
\usepackage{float} % Allows putting an [H] in \begin{figure} to specify the exact location of the figure

\usepackage{multicol}

\usepackage{verbatim}

\usepackage{sectsty} % Allows customizing section commands
\allsectionsfont{\centering \normalfont\scshape} % Make all sections centered, the default font and small caps

%----------------------------------------------------------------------------------------
%	Pied de Page
%----------------------------------------------------------------------------------------


\usepackage{fancyhdr} % Custom headers and footers
\pagestyle{fancyplain} % Makes all pages in the document conform to the custom headers and footers
\fancyhead{} % No page header - if you want one, create it in the same way as the footers below
\fancyfoot[L]{$2^{nd}1$} % Empty left footer
\fancyfoot[C]{Chapitre 5 - Fonctions affines} % Empty center footer
\fancyfoot[R]{\thepage} % Page numbering for right footer

\renewcommand{\headrulewidth}{0pt} % Remove header underlines
\renewcommand{\footrulewidth}{0pt} % Remove footer underlines

\setlength{\headheight}{13.6pt} % Customize the height of the header


\setlength\parindent{0pt} % Removes all indentation from paragraphs - comment this line for an assignment with lots of text


%----------------------------------------------------------------------------------------
%	Titre
%----------------------------------------------------------------------------------------

\newcommand{\horrule}[1]{\rule{\linewidth}{#1}} % Create horizontal rule command with 1 argument of height


\title{
  \vspace{-10ex}
  \horrule{0.5pt} \\[0.4cm] % Thin top horizontal rule
  \huge Chapitre 5 - Fonctions affines\\ % The assignment title
  \horrule{2pt} \\[0.5cm] % Thick bottom horizontal rule
}

\author{}
\date{\vspace{-10ex}} % Today's date or a custom date

%----------------------------------------------------------------------------------------
%	Début du document
%----------------------------------------------------------------------------------------

\begin{document}

%----------------------------------------------------------------------------------------
% RE-DEFINITION
%----------------------------------------------------------------------------------------
% MATHS
%-----------

\newtheorem{Definition}{Définition}
\newtheorem{Theorem}{Théorème}
\newtheorem{Proposition}{Propriété}

% MATHS
%-----------
\renewcommand{\labelitemi}{$\bullet$}
\renewcommand{\labelitemii}{$\circ$}
%----------------------------------------------------------------------------------------
%	Titre
%----------------------------------------------------------------------------------------

\maketitle % Print the title
\setlength{\columnseprule}{1pt}

%-----------------------------------111111111111111111111111111111111111
\section{Formule}
%-----------------------------------------------------------------------

Soit $f$ la fonction définie sur $\mathbb{R}$ par $x : \to f(x) = ax + b$.\\

\begin{itemize}
\item $f$ est une fonction affine.
\item La représentation graphique de $f$ est une droite qu'on note souvent $d$.
\end{itemize}

\subsection{Ordonnée à l'origine}

Le paramètre $b$ est l'ordonnée à l'origine. Il se trouve algébriquement en calculant $f(0)$ et graphiquement en cherchant l'ordonnée du point d'intersection entre la droite $d$ et la droite d'équation x=0.

$$f(0) = a \times 0 + b = 0 + b = b$$


\subsection{Coefficient directeur}

Le paramètre $a$ est le coefficient directeur de la droite $d$. Il correspond à la pente de la droite.\\
Soient $A(x_a, y_a)$ et $B(x_b, y_b)$ deux points appartenant à la droite d. On a les relations $y_a = ax_a + b$ et $y_b = ax_b + b$.
$$\dfrac{y_b - y_a}{x_b - x_a} = \dfrac{ax_b + b - ax_a + b}{x_b - x_a} = \dfrac{a(x_b - x_a)}{x_b - x_a} = a = \dfrac{\Delta y}{\Delta x}$$

\begin{multicols}{2}
  \begin{figure}[H]
    \centering
    \includegraphics[width=0.8\linewidth]{sources/cours/droite-1.pdf}
  \end{figure}

  \begin{figure}[H]
    \centering
    \includegraphics[width=0.8\linewidth]{sources/cours/droite-2.pdf}
  \end{figure}
\end{multicols}

\newpage

%-----------------------------------111111111111111111111111111111111111
\section{Étude des variations d'une fonction affine}
%-----------------------------------------------------------------------

\subsection{Cas général}

\begin{Proposition}
  Soit $f$ la fonction affine définie par $f : x \to ax + b$.
  \begin{itemize}
  \item Si $a > 0$, la fonction $f$ est croissante.
  \item Si $a < 0$, la fonction $f$ est décroissante.
  \item Si $a = 0$, la fonction $f$ est constante et égale à $b$ pour tout $x$. 
  \end{itemize}
\end{Proposition}

\paragraph{Preuve}
Soit $f$ la fonction affine définie par $f : x \to ax + b$.
Soient $x_1$ et $x_2$ deux réels distincts tel que $x_1 < x_2$.

\begin{multicols}{2}
  \begin{enumerate}
  \item Si $a > 0$.\\
    \begin{eqnarray*}
      x_1 &<& x_2\\
      a x_1 &<& a x_2\\
      a x_1 + b &<& a x_2 + b\\
      f(x_1) &<& f(x_2)
    \end{eqnarray*}
    Si $a>0$, alors la fonction $f$ est croissante.
  \item Si $a < 0$.\\
    \begin{eqnarray*}
      x_1 &<& x_2\\
      a x_1 &>& a x_2\\
      a x_1 + b &>& a x_2 + b\\
      f(x_1) &>& f(x_2)
    \end{eqnarray*}
    Si $a>0$, alors la fonction $f$ est décroissante.
  \end{enumerate}
\end{multicols}

\subsection{Exemples}

\begin{itemize}
\item $f_1(x) = 7 - 3x$ est décroissante.
\item $f_1(x) = \frac{1}{-2} x$ est décroissante.
\item $f_1(x) = 3 + x$ est croissante.
\item $f_1(x) = \frac{1}{x} + 3$ n'est pas une fonction affine.
\end{itemize}

\newpage

%-----------------------------------111111111111111111111111111111111111
\section{Étude du signe d'une fonction affine}
%-----------------------------------------------------------------------

Il est possible de déterminer graphiquement et algébriquement le signe du fonction affine. Cela revient à résoudre des équations et des inéquations avec des fonctions affines. Soit $f$ la fonction affine définie par $f : x \to ax + b$ et $d$ sa représentation graphique.

\subsection{Résolution de l'équation $f(x) = 0$}

\begin{multicols}{2}

  \begin{eqnarray*}
    a x + b &=& 0\\
    a x &=& -b\\
    x &=& -\dfrac{b}{a}
  \end{eqnarray*}

  $x = -\dfrac{b}{a}$ est solution de l'équation $f(x) = 0$. \\
  Le point $(-\dfrac{b}{a} ; 0)$ est situé à l'intersection de la droite $d$ et de l'axe des abscisses.

\end{multicols}

\subsection{Résolution de $f(x) \leq 0$}

\begin{multicols}{2}
  \begin{Proposition}
    Soit $f_1(x) = \frac{1}{3} x - 1$.\\
    La solution de l'équation $f(x) = 0$ est $x = -\dfrac{b}{a} = 3$.\\
    Comme $a = \frac{1}{3} > 0$, la fonction est strictement croissante sur $\mathbb{R}$.\\ 
    $f$ est négative jusqu'à 3 où elle s'annule puis devient positive.
  \end{Proposition}

  \paragraph{Preuve}

  \begin{eqnarray*}
    f_1(x) & \leq & 0\\
    \frac{1}{3} x - 1 & \leq & 0 \\
    \frac{1}{3} x & \leq & 1 \\
    x & \leq & 3
  \end{eqnarray*}



  \begin{figure}[H]
    \centering
    \includegraphics[width=\linewidth]{sources/cours/signe-1.pdf}
  \end{figure}

\end{multicols}

\begin{multicols}{2}
  \begin{Proposition}
    Soit $f_2(x) = -x + 4$.\\
    La solution de l'équation $f(x) = 0$ est $x = -\dfrac{b}{a} = 4$.\\
    Comme $a = -1 < 0$, la fonction est strictement décroissante sur $\mathbb{R}$.\\ 
    $f$ est positive jusq'à 4 où elle s'annule puis devient négative.
  \end{Proposition}
  \paragraph{Preuve}

  \begin{eqnarray*}
    f_2(x) & \leq & 0\\
    -x + 4 & \leq & 0 \\
    -x & \leq & -4 \\
    x & \geq & 4
  \end{eqnarray*}

  \begin{figure}[H]
    \centering
    \includegraphics[width=\linewidth]{sources/cours/signe-2.pdf}
  \end{figure}

\end{multicols}

\newpage 

%-----------------------------------111111111111111111111111111111111111
\section{Équations et inéquations}
%-----------------------------------------------------------------------

\subsection{Les problèmes du type : $f(x) = k$, $f(x) \leq k$ et $f(x) \geq k$}

\begin{multicols}{2}

  Dans ces problèmes de modélisation, on étudie quand notre fonction est au dessous ou au dessus de la droite horizontale d'équation $y = k$.

  Algébriquement, il est possible de résoudre ces (in)équations de manière similaire.\\
  \begin{eqnarray*}
    f(x) & = & k\\
    f(x) - k & = & 0\\
    ax + b - k & = & 0\\
    ax &=& k-b\\
    x &=& \dfrac{k-b}{a}
  \end{eqnarray*}

  \begin{figure}[H]
    \centering
    \includegraphics[width=\linewidth]{sources/cours/ineq-1.pdf}
  \end{figure}

\end{multicols}

\subsection{Les problèmes du type : $f(x) = g(x)$, $f(x) \leq g(x)$ et $f(x) \geq g(x)$}

\begin{multicols}{2}

  Dans ces problèmes de modélisation, on étudie la position relative de notre fonction $f$ avec une autre fonction $g$. On cherche ainsi quand $f$ est au dessous ou au dessus de $g$.

  Si $g$ est une fonction affine définie par $ g : x \to cx + d$, il es possible de les résoudre algébriquement.


  \begin{eqnarray*}
    f(x) & = & g(x)\\
    f(x) - g(x) & = & 0\\
    ax + b - (cx + d) & = & 0\\
    (a - c) x + (b - d) &=& 0\\
    x &=& \dfrac{b-d}{a-c}
  \end{eqnarray*}

  \begin{figure}[H]
    \centering
    \includegraphics[width=\linewidth]{sources/cours/ineq-2.pdf}
  \end{figure}

\end{multicols}

\subsection{(In)Équations produit}

\begin{Proposition}
  \begin{itemize}
  \item Un produit de facteur est nul si l'un de ses facteurs est nulle.
  \item Un produit de facteurs est positif s'il possède un nombre pair de facteurs négatifs. 
  \item Un produit de facteurs est négatif s'il possède un nombre impair de facteurs négatifs.
  \end{itemize} 
\end{Proposition}


\subsection{(In)Équations quotient}

\begin{Proposition}
  \begin{itemize}
  \item Un quotient de facteur est nul si son nominateur est nul et que so dénominateur est non nul.
  \item Un quotient de facteurs est positif si son dénominateur et son numérateur sont du même signe. 
  \item Un quotient de facteurs est négatif si son dénominateur et son numérateur sont de signe contraire.
  \end{itemize} 
\end{Proposition}

\end{document}

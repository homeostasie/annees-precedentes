%%%%%%%%%%%%%%%%%%%%%%%%%%%%%%%%%%%%%%%%%
% Short Sectioned Assignment
% LaTeX Template
% Version 1.0 (5/5/12)
%
% This template has been downloaded from:
% http://www.LaTeXTemplates.com
%
% Original author:
% Frits Wenneker (http://www.howtotex.com)
%
% License:
% CC BY-NC-SA 3.0 (http://creativecommons.org/licenses/by-nc-sa/3.0/)
%
%%%%%%%%%%%%%%%%%%%%%%%%%%%%%%%%%%%%%%%%%

%----------------------------------------------------------------------------------------
%	PACKAGES AND OTHER DOCUMENT CONFIGURATIONS
%----------------------------------------------------------------------------------------

\documentclass[paper=a4, fontsize=9pt]{scrartcl} % A4 paper and 11pt font size


\usepackage[T1]{fontenc} % Use 8-bit encoding that has 256 glyphs
\usepackage[english,francais]{babel} % Français et anglais
\usepackage[utf8]{inputenc}

\usepackage{amsmath,amsfonts,amsthm} % Math packages

\usepackage{enumitem}
\usepackage{lmodern}
\usepackage{url}
\usepackage{eurosym} % signe Euros
\usepackage{geometry} % Pour passer au format A4
\geometry{a4paper} %
\usepackage{graphicx} % Required for including pictures
\usepackage{float} % Allows putting an [H] in \begin{figure} to specify the exact location of the figure

\usepackage{multicol}

\usepackage{verbatim}

\usepackage{sectsty} % Allows customizing section commands
\allsectionsfont{\centering \normalfont\scshape} % Make all sections centered, the default font and small caps

%----------------------------------------------------------------------------------------
%	Pied de Page
%----------------------------------------------------------------------------------------


\usepackage{fancyhdr} % Custom headers and footers
\pagestyle{fancyplain} % Makes all pages in the document conform to the custom headers and footers
\fancyhead{} % No page header - if you want one, create it in the same way as the footers below
\fancyfoot[L]{$2^{nd}1$} % Empty left footer
\fancyfoot[C]{Chapitre 8 - Fonctions carrées} % Empty center footer
\fancyfoot[R]{\thepage} % Page numbering for right footer

\renewcommand{\headrulewidth}{0pt} % Remove header underlines
\renewcommand{\footrulewidth}{0pt} % Remove footer underlines

\setlength{\headheight}{13.6pt} % Customize the height of the header


\setlength\parindent{0pt} % Removes all indentation from paragraphs - comment this line for an assignment with lots of text


%----------------------------------------------------------------------------------------
%	Titre
%----------------------------------------------------------------------------------------

\newcommand{\horrule}[1]{\rule{\linewidth}{#1}} % Create horizontal rule command with 1 argument of height


\title{
  \vspace{-10ex}
  \horrule{0.5pt} \\[0.4cm] % Thin top horizontal rule
  \huge Chapitre 8 - Fonctions carrées\\ % The assignment title
  \horrule{2pt} \\[0.5cm] % Thick bottom horizontal rule
}

\author{}
\date{\vspace{-10ex}} % Today's date or a custom date

%----------------------------------------------------------------------------------------
%	Début du document
%----------------------------------------------------------------------------------------

\begin{document}

%----------------------------------------------------------------------------------------
% RE-DEFINITION
%----------------------------------------------------------------------------------------
% MATHS
%-----------

\newtheorem{Definition}{Définition}
\newtheorem{Theorem}{Théorème}
\newtheorem{Proposition}{Propriété}

% MATHS
%-----------
\renewcommand{\labelitemi}{$\bullet$}
\renewcommand{\labelitemii}{$\circ$}
%----------------------------------------------------------------------------------------
%	Titre
%----------------------------------------------------------------------------------------

\maketitle % Print the title
\setlength{\columnseprule}{1pt}

%-----------------------------------111111111111111111111111111111111111
\section{La fonction $x^2$}
%-----------------------------------------------------------------------

Aspect : La courbe est un parabole.

Le domaine de défintion est $D_ff = \mathbb{R}$

%-----------------------------------111111111111111111111111111111111111
\section{Vocabulaires associés}
%-----------------------------------------------------------------------

Polynôme du second degrée

$ax^2 + bx + c$

Trinôme du second degrée

Domaine de definition

%-----------------------------------111111111111111111111111111111111111
\section{Variations}
%-----------------------------------------------------------------------

La fonction est décroissante puis croissante si $a>0$.
La fonction est croissante puis décroissante si $a<0$.

Et on trace la fonction
 

%-----------------------------------111111111111111111111111111111111111
\section{Variations}
%-----------------------------------------------------------------------




\end{document}

%%%%%%%%%%%%%%%%%%%%%%%%%%%%%%%%%%%%%%%%%
% Short Sectioned Assignment
% LaTeX Template
% Version 1.0 (5/5/12)
%
% This template has been downloaded from:
% http://www.LaTeXTemplates.com
%
% Original author:
% Frits Wenneker (http://www.howtotex.com)
%
% License:
% CC BY-NC-SA 3.0 (http://creativecommons.org/licenses/by-nc-sa/3.0/)
%
%%%%%%%%%%%%%%%%%%%%%%%%%%%%%%%%%%%%%%%%%

%----------------------------------------------------------------------------------------
%	PACKAGES AND OTHER DOCUMENT CONFIGURATIONS
%----------------------------------------------------------------------------------------

\documentclass[paper=a4, fontsize=9pt]{scrartcl} % A4 paper and 11pt font size


\usepackage[T1]{fontenc} % Use 8-bit encoding that has 256 glyphs
\usepackage[english,francais]{babel} % Français et anglais
\usepackage[utf8]{inputenc}

\usepackage{amsmath,amsfonts,amsthm} % Math packages

\usepackage{enumitem}
\usepackage{lmodern}
\usepackage{url}
\usepackage{eurosym} % signe Euros
\usepackage{geometry} % Pour passer au format A4
\geometry{a4paper} %
\usepackage{graphicx} % Required for including pictures
\usepackage{float} % Allows putting an [H] in \begin{figure} to specify the exact location of the figure

\usepackage{multicol}

\usepackage{verbatim}

\usepackage{sectsty} % Allows customizing section commands
\allsectionsfont{\centering \normalfont\scshape} % Make all sections centered, the default font and small caps

%----------------------------------------------------------------------------------------
%	Pied de Page
%----------------------------------------------------------------------------------------


\begin{document}

%----------------------------------------------------------------------------------------
% RE-DEFINITION
%----------------------------------------------------------------------------------------
% MATHS
%-----------

\newtheorem{Definition}{Définition}
\newtheorem{Theorem}{Théorème}
\newtheorem{Proposition}{Propriété}

% MATHS
%-----------
\renewcommand{\labelitemi}{$\bullet$}
\renewcommand{\labelitemii}{$\circ$}
%----------------------------------------------------------------------------------------
%	Titre
%----------------------------------------------------------------------------------------

%\maketitle % Print the title
\setlength{\columnseprule}{1pt}

\textbf{Nom, Prénom :}

\subsection*{fonction carrée - Étude graphique.}

Soit $f$ la fonction du second degré définie sur tout $\mathbb{R}$ par $f(x) = x^3 - \dfrac{5}{2}x^2 - 3x + \dfrac{11}{2}$.

\subsubsection*{Obtenir un graphique exploitable}

On se propose dans cette partie d'étudier la fonction à partir de sa représentation graphique sur l'intervalle [-2 ; 4].

\begin{enumerate}
\item Reproduire et compléter le tableau de données. (utiliser la fonction table de votre calculette)
  \begin{center}
    \begin{tabular}{| l || c | c | c | c | c | c | c | c| c | c | c | c | c |}
      \hline
      $x$ & -2.0 & -1.5 & -1.0 & -0.5 & 0.0 & 0.5 & 1.0 & 1.5 & 2.0 & 2.5 & 3.0 & 3.5 & 4.0\\
      \hline
      $f(x)$ &  &  &  &  &  &  &  &  &  &  &  &  & \\
             &  &  &  &  &  &  &  &  &  &  &  &  & \\

      \hline
    \end{tabular}
  \end{center}

\item Après avoir reporter les points sur le graphique, tracer la courbe. (Vérifier que la fonction tracée corresponde à celle obtenue à l'aide de votre calculette sur ce même intervalle.
\end{enumerate}

\subsubsection*{Obtenir un graphique exploitable}

On propose maintenant d'exploiter au mieux la trace obtenue.

\begin{enumerate}
\item Placer en rouge les solutions de l'équation $f(x) = 0$. On note ces solutions $x_1$, $x_2$ et $x_3$. Donner les coordonnées de ces points. $A_1 = \text{\phantom{azertyuipazertyuiop}} A_2 = \text{\phantom{azertyuipazertyuiop}} A_3 =$
\item Placer en bleu les solutions de l'équation $f(x) = 4$. On note ces solutions $z_1$, $z_2$ et $z_3$. Donner les coordonnées de ces points. $B_1 = \text{\phantom{azertyuipazertyuiop}} B_2 = \text{\phantom{azertyuipazertyuiop}} B_3$
\item Placer en vert la solution de l'équation $f(x) = 7$. On note cette solution $u_1$. Donner les coordonnées de ce point. $D =$ 
\item Placer des points au niveau des extremums locaux. Donner leurs coordonnées. $E_1 = \text{\phantom{azertyuipazertyuiop}} E_2 = $
 
\end{enumerate}


\subsection*{Annexe}

\begin{figure}[H]
  \centering
  \includegraphics[width=\linewidth]{grille.png}
\end{figure}


\end{document}

%%%%%%%%%%%%%%%%%%%%%%%%%%%%%%%%%%%%%%%%%
% LaTeX Template
% http://www.LaTeXTemplates.com
%
% Original author:
% Linux and Unix Users Group at Virginia Tech Wiki 
% (https://vtluug.org/wiki/Example_LaTeX_chem_lab_report)
%
% License:
% CC BY-NC-SA 3.0 (http://creativecommons.org/licenses/by-nc-sa/3.0/)
%
%%%%%%%%%%%%%%%%%%%%%%%%%%%%%%%%%%%%%%%%%

%----------------------------------------------------------------------------------------
%	PACKAGES AND DOCUMENT CONFIGURATIONS
%----------------------------------------------------------------------------------------

\documentclass[12pt]{article}
\usepackage{geometry} % Pour passer au format A4
\geometry{hmargin=1cm, vmargin=1cm} % 

\usepackage{graphicx} % Required for including pictures
\usepackage{float} % 

%Français
\usepackage[T1]{fontenc} 
\usepackage[english,francais]{babel}
\usepackage[utf8]{inputenc}
\usepackage{eurosym}
\usepackage{lmodern}
\usepackage{url}
\usepackage{multicol}

%Maths
\usepackage{amsmath,amsfonts,amssymb,amsthm}
%\usepackage[linesnumbered, ruled, vlined]{algorithm2e}
%\SetAlFnt{\small\sffamily}

%Autres
\linespread{1} % Line spacing
\setlength\parindent{0pt} % Removes all indentation from paragraphs

\renewcommand{\labelenumi}{\alph{enumi}.} % 
\pagestyle{empty}
%----------------------------------------------------------------------------------------
%	DOCUMENT INFORMATION
%----------------------------------------------------------------------------------------
\begin{document}

%\maketitle % Insert the title, author and date

\begin{center}
\textit{Les mathématiques ne sont une moindre immensité que la mer.} - \textbf{Victor Hugo}
\end{center}

\subsection*{Exercice 1 - Oliveraie}

À partir d'un hectare de champ d'olivier, on produit 50 litres d'huile d'olive. 

\begin{center}
  \begin{tabular}{| c ||       c | c | c |  c |  c |  c|}
    \hline
    Nombres d'hectare (ha) &  1 &   5&    7 &  10 &   20 & 90 \\
    \hline
    Huile produite (L)     & 50 & 250 & 350 & 500 & 1000 & 4500 \\ 
    \hline
  \end{tabular}
\end{center}


\begin{enumerate}
\item[1.] Remplir le tableau de proportionnalité. 
\item[2.] Quelle est la valeur du cœfficient de proportionnalité ?\\
Le coœfficient de proportionnalité est \textbf{50}.\\
\end{enumerate}


\textit{Remarque : 1 hectare est une surface de 100m par 100m.}

\subsection*{Exercice 2 - Copie de fichiers}

Sur un ordinateur, le temps de copie est proportionnel au nombre de fichiers à copier. En lançant la copie de 20 fichiers, l'ordinateur met 600 secondes. 

\begin{enumerate}
\item[1.] Remplir le tableau de proportionnalité.

  \begin{center}
    \begin{tabular}{| c || c | c | c |}
      \hline
      Nombre de fichiers & 20  &  120 & 60  \\
      \hline
      Temps de copie (s) & 600 & 3600 & 1800\\ 
      \hline
    \end{tabular}
  \end{center}

\item[2.] Quel est le cœfficient de proportionnalité ?\\
Le cœfficient de proportionnalité est \textbf{30}.

\item[3.] Résoudre le problème avec la collection de 120 fichiers. \\
 $20 \times 6 = 120$. donc $600 \times 6 = 3600$.\\
Le temps de copie de 120 fichier est de 3600s.

\item[4.] Combien peut-on copier de fichiers en 30 \textbf{minutes} ? \\
  $30min = 30 \times 60 = 1800s$. De plus, $600 \times 3 = 1800$. Donc $20 \times 3 = 60$.\\
On peut copier 60 fichiers en 30min.

\end{enumerate}
\textit{Remarque : une minute dure 60 secondes.}

\subsection*{Exercice 3 - Master Tacos}

Le prix d'un tacos est de 4 \euro{}. On se pose la question du nombre maximum de tacos qu'il est possible de s'acheter avec 24.50 \euro{}?

\begin{enumerate}
\item[1.] Remplir le tableau de proportionnalité en mettant un ? ou un $x$ dans la case avec la valeur manquante.

  \begin{center}
    \begin{tabular}{| c || c | c |}
      \hline
      Nombre de tacos & 1 &  ?   \\
      \hline
      Prix (\euro{})  & 4 & 24.50\\
      \hline
    \end{tabular}
  \end{center}

\item[2.] Répondre à la question précédemment posée. \textit{Laisser une trace des calculs et répondre par une phrase.}\\

$4 \times 6 = 24$ : On peut acheter \textbf{6 tacos}. \\
$4 \times 7 = 28$ : On ne peut pas acheter 7 tacos.

\end{enumerate}


\end{document}

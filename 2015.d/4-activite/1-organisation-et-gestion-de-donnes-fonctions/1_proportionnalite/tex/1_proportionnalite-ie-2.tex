%%%%%%%%%%%%%%%%%%%%%%%%%%%%%%%%%%%%%%%%%
% LaTeX Template
% http://www.LaTeXTemplates.com
%
% Original author:
% Linux and Unix Users Group at Virginia Tech Wiki 
% (https://vtluug.org/wiki/Example_LaTeX_chem_lab_report)
%
% License:
% CC BY-NC-SA 3.0 (http://creativecommons.org/licenses/by-nc-sa/3.0/)
%
%%%%%%%%%%%%%%%%%%%%%%%%%%%%%%%%%%%%%%%%%

%----------------------------------------------------------------------------------------
%	PACKAGES AND DOCUMENT CONFIGURATIONS
%----------------------------------------------------------------------------------------

\documentclass[12pt]{article}
\usepackage{geometry} % Pour passer au format A4
\geometry{hmargin=1cm, vmargin=1cm} % 

\usepackage{graphicx} % Required for including pictures
\usepackage{float} % 

%Français
\usepackage[T1]{fontenc} 
\usepackage[english,francais]{babel}
\usepackage[utf8]{inputenc}
\usepackage{eurosym}
\usepackage{lmodern}
\usepackage{url}
\usepackage{multicol}

%Maths
\usepackage{amsmath,amsfonts,amssymb,amsthm}
%\usepackage[linesnumbered, ruled, vlined]{algorithm2e}
%\SetAlFnt{\small\sffamily}

%Autres
\linespread{1} % Line spacing
\setlength\parindent{0pt} % Removes all indentation from paragraphs

\renewcommand{\labelenumi}{\alph{enumi}.} % 
\pagestyle{empty}
%----------------------------------------------------------------------------------------
%	DOCUMENT INFORMATION
%----------------------------------------------------------------------------------------
\begin{document}

%\maketitle % Insert the title, author and date

\textbf{Nom(s), Prénom(s) :}

\begin{center}
\textit{Les mathématiques ne sont une moindre immensité que la mer.} - \textbf{Victor Hugo}
\end{center}

\subsubsection*{Exercice 1 - Oliveraie}

À partir d'un hectare de champ d'olivier, on produit 60 litres d'huile d'olive. 

\begin{center}
  \begin{tabular}{| c ||           c | c | c |  c |  c |  c|}
    \hline
    Nombres d'hectare (ha)      &  1 & 5 & 7 & 10 & 20 & 90 \\
    \hline
    Huile produite (L) & \phantom{$\frac{aze}{aazertyui}$} & \phantom{$\frac{aze}{aazertyui}$} & \phantom{$\frac{aze}{aazertyui}$} & \phantom{$\frac{aze}{aazertyui}$} &  \phantom{$\frac{aze}{aazertyui}$} & \phantom{$\frac{aze}{aazertyui}$} \\ 
    \hline
  \end{tabular}
\end{center}


\begin{enumerate}
\item[1.] Remplir le tableau de proportionnalité. 
\item[2.] Quelle est la valeur du cœfficient de proportionnalité ? \textit{( Répondre par une phrase.)}
\vspace{1cm}
\end{enumerate}

\textit{Remarque : 1 hectare est une surface de 100m par 100m.}

\subsubsection*{Exercice 2 - Copie de fichiers}

Sur un ordinateur, le temps de copie est proportionnel au nombre de fichiers à copier. En lançant la copie de 20 fichiers, l'ordinateur met 50 secondes. 

\begin{enumerate}
\item[1.] Remplir le tableau de proportionnalité.
  \begin{center}
    \begin{tabular}{| c || c | c | c |}
      \hline
      \phantom{$\frac{aze}{aazertyuiop}$ \phantom{azertyuiop}} & \phantom{$\frac{aze}{aazertyuiop}$} & \phantom{$\frac{aze}{aazertyuiop}$}  & \phantom{$\frac{aze}{aazertyuiop}$}\\
      \hline
       & & & \\ 
      \hline
    \end{tabular}
  \end{center}
\item[2.] Quel est le cœfficient de proportionnalité ? \textit{( Répondre par une phrase.)}
\vspace{1cm}

\item[3.] Résoudre le problème avec une collection de 120 fichiers. \textit{( Répondre par une phrase.)}
  \vspace{1cm}
\item[4.] Combien peut-on copier de fichiers en 30 \textbf{minutes} ? \textit{( Laisser une trace des calculs et répondre par une phrase.)} 
  \vspace{2cm}
\end{enumerate}
\textit{Remarque : une minute dure 60 secondes.}

\subsubsection*{Exercice 3 - Master Tacos}

Le prix d'un tacos est de $4,50$ \euro{}. On se pose la question du nombre maximum de tacos qu'il est possible de s'acheter avec 24.50 \euro{}?

\begin{enumerate}
\item[1.] Remplir le tableau de proportionnalité en mettant un ? ou un $x$ dans la case avec la valeur manquante.

  \begin{center}
    \begin{tabular}{| c || c | c |}
      \hline
      \phantom{$\frac{aze}{aazertyuiop azertyuiop}$} & \phantom{$\frac{aze}{aazertyuiop}$} & \phantom{$\frac{aze}{aazertyuiop}$} \\
      \hline
      \phantom{$\frac{aze}{aazertyuiop azertyuiop}$} & \phantom{$\frac{aze}{aazertyuiop}$} & \phantom{$\frac{aze}{aazertyuiop}$} \\ 
      \hline
    \end{tabular}
  \end{center}

\item[2.] Répondre à la question posée dans l'énoncé. \textit{( Laisser une trace des calculs et répondre par une phrase.)}
  \vspace{2cm}

\end{enumerate}


\end{document}

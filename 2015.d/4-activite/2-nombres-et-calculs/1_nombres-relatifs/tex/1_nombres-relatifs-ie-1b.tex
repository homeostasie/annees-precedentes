%%%%%%%%%%%%%%%%%%%%%%%%%%%%%%%%%%%%%%%%%
% LaTeX Template
% http://www.LaTeXTemplates.com
%
% Original author:
% Linux and Unix Users Group at Virginia Tech Wiki 
% (https://vtluug.org/wiki/Example_LaTeX_chem_lab_report)
%
% License:
% CC BY-NC-SA 3.0 (http://creativecommons.org/licenses/by-nc-sa/3.0/)
%
%%%%%%%%%%%%%%%%%%%%%%%%%%%%%%%%%%%%%%%%%

%----------------------------------------------------------------------------------------
%	PACKAGES AND DOCUMENT CONFIGURATIONS
%----------------------------------------------------------------------------------------

\documentclass[12pt]{article}
\usepackage{geometry} % Pour passer au format A4
\geometry{hmargin=1cm, vmargin=1cm} % 

\usepackage{graphicx} % Required for including pictures
\usepackage{float} % 

%Français
\usepackage[T1]{fontenc} 
\usepackage[english,francais]{babel}
\usepackage[utf8]{inputenc}
\usepackage{eurosym}
\usepackage{lmodern}
\usepackage{url}
\usepackage{multicol}

%Maths
\usepackage{amsmath,amsfonts,amssymb,amsthm}
%\usepackage[linesnumbered, ruled, vlined]{algorithm2e}
%\SetAlFnt{\small\sffamily}

%Autres
\linespread{1} % Line spacing
\setlength\parindent{0pt} % Removes all indentation from paragraphs

\renewcommand{\labelenumi}{\alph{enumi}.} % 
\pagestyle{empty}
%----------------------------------------------------------------------------------------
%	DOCUMENT INFORMATION
%----------------------------------------------------------------------------------------
\begin{document}

%\maketitle % Insert the title, author and date

\textbf{Nom, Prénom :} \hspace{8cm} \textbf{Classe :} \hspace{3cm} \textbf{Date :}

\begin{center}
  \textit{Il vient une heure où protester ne suffit plus : après la philosophie, il faut l’action.}  - \textbf{Victor Hugo}

\end{center}

\textbf{La calculatrice n'est pas autorisée.}

\subsection*{Ex1 - Entourer le + si le résultat est positif, le - sinon}

\begin{multicols}{2}
  
  % Exercice 1
  \begin{enumerate}
  \item[] \textbf{( + ) | ( - )} : $ -62 \times -23 \times 10$\\
  \item[] \textbf{( + ) | ( - )} : $ -45 \times -36 \times 100 \times -7.3$\\
  \item[] \textbf{( + ) | ( - )} : $ -100 \times -123 \times -234 \times -2$\\
  \item[] \textbf{( + ) | ( - )} : $ 2.5 \times -\pi \times -12.4 \times \sqrt{10}$\\
  \end{enumerate}

  \begin{enumerate}
  \item[] \textbf{( + ) | ( - )} :  $ \dfrac{2}{-3} \times \dfrac{-1}{9} $ \\
  \item[] \textbf{( + ) | ( - )} :  $ - \dfrac{-4}{5} \times \dfrac{-3}{-2} \times \dfrac{-5}{-6}$ \\
  \item[] \textbf{( + ) | ( - )} :  $ \dfrac{-5}{-3} \times \dfrac{-34}{-56} \times \dfrac{-12}{-98} \times  \dfrac{-34}{-68} $ \\
  \item[] \textbf{( + ) | ( - )} :  $ -23.2 \times \dfrac{-4}{-2} \times \dfrac{-7}{4} \div -2.3$ \\
  \end{enumerate}

\end{multicols}

% Exercice 2

\subsection*{Ex2 - Remplir le tableau}

\begin{center}
  \begin{tabular}{| l || c | c | c | c | c | c | c | c | }
    \hline
    Nombre & 2                 & 5.4               & -5               & $\dfrac{2}{7}$   & $\dfrac{-2}{8}$  & \phantom{azerty} & \phantom{azerty} & \phantom{azerty}  \\
    \hline
    Inverse & \phantom{azerty}  & \phantom{azerty}  & \phantom{azerty} & \phantom{azerty} & \phantom{azerty} & $\dfrac{1}{4}$   & $\dfrac{-5}{2}$  & $ \sqrt{2}$ \\
    \hline
  \end{tabular}
\end{center}

% Exercice 3

\subsection*{Ex3}
\textit{Calculer. Laisser une trace des calculs. Mettre le nombre mininal de signe moins.}

\begin{multicols}{2}
  
\begin{enumerate}
\item[1a] $2 \times  -5 \times 9 = $\\
  \rule{\linewidth}{0.5pt}
\item[1b] $4 \times  -10 \times -5 \times 2 = $\\
  \rule{\linewidth}{0.5pt}
\item[1c] $-2 \times 17 \times 0 \times -32 = $\\
  \rule{\linewidth}{0.5pt}
\item[1d] $-1 \times -1 \times -1 \times -1 \times -1  \times -1 = $\\
  \rule{\linewidth}{0.5pt}
\item[1e] $2.2 \times -5 \times -2 \times 10 = $\\
  \rule{\linewidth}{0.5pt}   
\end{enumerate}


\begin{enumerate}
\item[2a] $ \dfrac{2}{9} \times \dfrac{4}{9} = $\\
  \rule{\linewidth}{0.5pt}
\item[2b] $ \dfrac{-7}{4} \times \dfrac{1}{-8} = $\\
  \rule{\linewidth}{0.5pt}
\item[2c] $ 5 \times \dfrac{-5}{3} \times \dfrac{7}{-4} = $\\
    \rule{\linewidth}{0.5pt}
\item[2d] $ \dfrac{2.2}{-10} \times \dfrac{5}{3.4} = $\\
  \rule{\linewidth}{0.5pt}
\end{enumerate}

\end{multicols}

\begin{enumerate}
\item[3a] $ \dfrac{2}{9} \div \dfrac{4}{9} = $\\
  \rule{\linewidth}{0.5pt}
\item[3b] $ \dfrac{-7}{4} \div \dfrac{1}{-8} = $\\
  \rule{\linewidth}{0.5pt}
\item[3c] $ 4 \div \dfrac{-5}{3} = $\\
    \rule{\linewidth}{0.5pt}
\item[3d] $ \dfrac{2}{_10} \div -3 \div \dfrac{5}{3} = $\\
  \rule{\linewidth}{0.5pt}
\end{enumerate}


\end{document}

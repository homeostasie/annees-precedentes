%%%%%%%%%%%%%%%%%%%%%%%%%%%%%%%%%%%%%%%%%
% LaTeX Template
% http://www.LaTeXTemplates.com
%
% Original author:
% Linux and Unix Users Group at Virginia Tech Wiki 
% (https://vtluug.org/wiki/Example_LaTeX_chem_lab_report)
%
% License:
% CC BY-NC-SA 3.0 (http://creativecommons.org/licenses/by-nc-sa/3.0/)
%
%%%%%%%%%%%%%%%%%%%%%%%%%%%%%%%%%%%%%%%%%

%----------------------------------------------------------------------------------------
%	PACKAGES AND DOCUMENT CONFIGURATIONS
%----------------------------------------------------------------------------------------

\documentclass[12pt]{article}
\usepackage{geometry} % Pour passer au format A4
\geometry{hmargin=1cm, vmargin=1cm} % 

\usepackage{graphicx} % Required for including pictures
\usepackage{float} % 

%Français
\usepackage[T1]{fontenc} 
\usepackage[english,francais]{babel}
\usepackage[utf8]{inputenc}
\usepackage{eurosym}
\usepackage{lmodern}
\usepackage{url}
\usepackage{multicol}

%Maths
\usepackage{amsmath,amsfonts,amssymb,amsthm}
%\usepackage[linesnumbered, ruled, vlined]{algorithm2e}
%\SetAlFnt{\small\sffamily}

%Autres
\linespread{1} % Line spacing
\setlength\parindent{0pt} % Removes all indentation from paragraphs

\renewcommand{\labelenumi}{\alph{enumi}.} % 
\pagestyle{empty}
%----------------------------------------------------------------------------------------
%	DOCUMENT INFORMATION
%----------------------------------------------------------------------------------------
\begin{document}

%\maketitle % Insert the title, author and date

\subsubsection*{Exercice 1 - Champions League}
Lors d'un match de la \textit{Champions League} \textbf{trois} amis commandent une pizza. Ils la coupent en huit. 

\begin{enumerate}
\item[1.] M. Audibert prend trois parts. Quelle fraction de la pizza cela représente-t-il ?
\vspace{2cm}
\item[2.] M. Bachelier prend seulement deux parts. Quelle fraction de la pizza cela représente-t-il ?
\vspace{2cm}
\item[3.] Combien de parts reste-il de disponibles pour Mme. Abellaneda ?
\vspace{2cm}
\begin{multicols}{2}
\item[4.a] Compléter :
$1 = \dfrac{}{8}$
\item[4.b] Faire le calcul suivant :\\
\vspace{1cm}
$1 - \dfrac{3}{8} + \dfrac{2}{8} = $
\end{multicols}
\end{enumerate}

\vspace{1cm}
\noindent\hrulefill
\vspace{1cm}

\subsubsection*{Exercice 1 - Champions League}
Lors d'un match de la \textit{Champions League} \textbf{trois} amis commandent une pizza. Ils la coupent en huit. 

\begin{enumerate}
\item[1.] M. Audibert prend trois parts. Quelle fraction de la pizza cela représente-t-il ?
\vspace{2cm}
\item[2.] M. Bachelier prend seulement deux parts. Quelle fraction de la pizza cela représente-t-il ?
\vspace{2cm}
\item[3.] Combien de parts reste-il de disponibles pour Mme. Abellaneda ?
\vspace{2cm}
\begin{multicols}{2}
\item[4.a] Compléter :
$1 = \dfrac{}{8}$
\item[4.b] Faire le calcul suivant :\\
\vspace{0.2cm}
$1 - \dfrac{3}{8} + \dfrac{2}{8} = $
\end{multicols}
\end{enumerate}

\end{document}

\documentclass[10t]{article}
\usepackage{geometry} % Pour passer au format A4
\geometry{hmargin=0.7cm, vmargin=0.7cm} % 

\usepackage{graphicx} % Required for including pictures
\usepackage{float} % 

%Français
\usepackage[T1]{fontenc} 
\usepackage[english,francais]{babel}
\usepackage[utf8]{inputenc}
\usepackage{eurosym}
\usepackage{lmodern}
\usepackage{url}
\usepackage{multicol}
\usepackage{multido}
%Maths
\usepackage{amsmath,amsfonts,amssymb,amsthm}
%\usepackage[linesnumbered, ruled, vlined]{algorithm2e}
%\SetAlFnt{\small\sffamily}

%Autres
\linespread{1} % Line spacing
\setlength\parindent{0pt} % Removes all indentation from paragraphs

\renewcommand{\labelenumi}{\alph{enumi}.} %
\newcommand{\horrule}[1]{\rule{\linewidth}{#1}} % Create horizontal rule command with 1 argument of height
\newcommand{\Pointille}[1][3]{\multido{}{#1}{ \makebox[\linewidth]{\dotfill}\\[\parskip]}}

\pagestyle{empty}
%----------------------------------------------------------------------------------------
%	DOCUMENT INFORMATION
%----------------------------------------------------------------------------------------
\begin{document}


\subsection*{Papyrus Rhind}
Le Papyrus Rhind aurait été écrit par le scribe Ahmès, qui vécu vers $1700$ av. J.-C. Son nom vient d'un Écossais qui l'acheta en $1858$ à Louxor. Il aurait été découvert sur le site de la ville de Thèbes. Actuellement conservé au British Museum de Londres, il contient $87$ problèmes résolus d'arithmétique, d'algèbre, de géométrie et d'arpentage, sur plus de $5m$ de longueur et $32cm$ de large. Voici un des problèmes que l’on trouve dans ce papyrus.\\

`` Dans chacune des 7 cabanes, il y a 7 chats. Chaque chat surveille 7 souris. Chaque souris a 7 épis de blé. Chaque épi est composé de 7 grains. \textbf{Combien de grains de blé y a-t-il en tout ?}''

\subsection*{Distance Terre-Lune en papier}
Une feuille de papier mesure $0,1 mm$ d’épaisseur. La distance entre la Terre et la Lune est d’environ $384 400 km$. En pliant une feuille de papier en deux, on double son épaisseur. En la repliant en quatre, l’épaisseur quadruple et ainsi de suite.\\
\textbf{Combien de fois faut-il plier la feuille de papier pour obtenir la distance Terre-Lune ?}\\

\textit{Il est judicieux commencer par travailler sur la conversion des unités. On rappel : $1km = 1 000m = 10^3 m$ et $1m = 1 000 mm = 10^3 mm$.}

\subsection*{Rumeur}

Mademoiselle Jeanne habite à Feyzin. Cette nuit, elle a rêvé qu’elle prenait son petit-déjeuner avec son chanteur préféré de PNL. En arrivant au collège à 9 h, elle raconte le fait à ses trois amies, mais elle oublie de leur dire qu'il s'agissait simplement d'un rêve.\\

Naturellement, les trois amies se hâtent de faire les intéressantes et chacune d’entre elles annonce ce qu’elle vient d’apprendre à trois nouvelles personnes. Évidemment, chacune de ces nouvelles personnes raconte cette histoire à trois autres personnes et ainsi de suite.\\

Sachant qu'il y a environ 400 élèves au collège et que l'information est répétée à de nouveaux groupes de trois personnes toutes les 10 minutes, \textbf{à quelle heure tout le collège pense que Jeanne a pris son petit déjeuner avec le chanteur de PNL ?}

\horrule{1px}

\subsection*{Papyrus Rhind}
Le Papyrus Rhind aurait été écrit par le scribe Ahmès, qui vécu vers $1700$ av. J.-C. Son nom vient d'un Écossais qui l'acheta en $1858$ à Louxor. Il aurait été découvert sur le site de la ville de Thèbes. Actuellement conservé au British Museum de Londres, il contient $87$ problèmes résolus d'arithmétique, d'algèbre, de géométrie et d'arpentage, sur plus de $5m$ de longueur et $32cm$ de large. Voici un des problèmes que l’on trouve dans ce papyrus.\\

`` Dans chacune des 7 cabanes, il y a 7 chats. Chaque chat surveille 7 souris. Chaque souris a 7 épis de blé. Chaque épi est composé de 7 grains. \textbf{Combien de grains de blé y a-t-il en tout ?}''

\subsection*{Distance Terre-Lune en papier}
Une feuille de papier mesure $0,1 mm$ d’épaisseur. La distance entre la Terre et la Lune est d’environ $384 400 km$. En pliant une feuille de papier en deux, on double son épaisseur. En la repliant en quatre, l’épaisseur quadruple et ainsi de suite.\\
\textbf{Combien de fois faut-il plier la feuille de papier pour obtenir la distance Terre-Lune ?}\\

\textit{Il est judicieux commencer par travailler sur la conversion des unités. On rappel : $1km = 1 000m = 10^3 m$ et $1m = 1 000 mm = 10^3 mm$.}

\subsection*{Rumeur}

Mademoiselle Jeanne habite à Feyzin. Cette nuit, elle a rêvé qu’elle prenait son petit-déjeuner avec son chanteur préféré de PNL. En arrivant au collège à 9 h, elle raconte le fait à ses trois amies, mais elle oublie de leur dire qu'il s'agissait simplement d'un rêve.\\

Naturellement, les trois amies se hâtent de faire les intéressantes et chacune d’entre elles annonce ce qu’elle vient d’apprendre à trois nouvelles personnes. Évidemment, chacune de ces nouvelles personnes raconte cette histoire à trois autres personnes et ainsi de suite.\\

Sachant qu'il y a environ 400 élèves au collège et que l'information est répétée à de nouveaux groupes de trois personnes toutes les 10 minutes, \textbf{à quelle heure tout le collège pense que Jeanne a pris son petit déjeuner avec le chanteur de PNL ?}

\end{document}

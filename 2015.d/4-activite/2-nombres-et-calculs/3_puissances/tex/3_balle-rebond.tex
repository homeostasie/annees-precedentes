\documentclass[11pt]{article}
\usepackage{geometry} % Pour passer au format A4
\geometry{hmargin=0.7cm, vmargin=0.7cm} % 

\usepackage{graphicx} % Required for including pictures
\usepackage{float} % 

%Français
\usepackage[T1]{fontenc} 
\usepackage[english,francais]{babel}
\usepackage[utf8]{inputenc}
\usepackage{eurosym}
\usepackage{lmodern}
\usepackage{url}
\usepackage{multicol}
\usepackage{multido}
%Maths
\usepackage{amsmath,amsfonts,amssymb,amsthm}
%\usepackage[linesnumbered, ruled, vlined]{algorithm2e}
%\SetAlFnt{\small\sffamily}

%Autres
\linespread{1} % Line spacing
\setlength\parindent{0pt} % Removes all indentation from paragraphs

\renewcommand{\labelenumi}{\alph{enumi}.} %
\newcommand{\horrule}[1]{\rule{\linewidth}{#1}} % Create horizontal rule command with 1 argument of height
\newcommand{\Pointille}[1][3]{\multido{}{#1}{ \makebox[\linewidth]{\dotfill}\\[\parskip]}}

\pagestyle{empty}
%----------------------------------------------------------------------------------------
%	DOCUMENT INFORMATION
%----------------------------------------------------------------------------------------
\begin{document}


On laisse tomber une balle d'une hauteur de 1 mètre.\\
À chaque rebond, elle rebondit des $\dfrac{3}{4}$ de la hauteur d'où elle est tombée.\\
\begin{enumerate}
\item Faire un schéma en prenant pour échelle 1m <=> 10cm.
\item Quelle hauteur atteint la balle au cinquième rebonds ?
  \begin{itemize}
  \item Donner le résultat sous la forme d'une fraction.
  \item Donner le résultat sous la forme décimale. \textit{(Arrondir au cm)}
  \item Donner le résultat sous la forme scientifique \textit{(Arrondir au cm)}
  \end{itemize}
\end{enumerate}

\horrule{1px}
\vspace{0.3cm}

On laisse tomber une balle d'une hauteur de 1 mètre.\\
À chaque rebond, elle rebondit des $\dfrac{3}{4}$ de la hauteur d'où elle est tombée.\\
\begin{enumerate}
\item Faire un schéma en prenant pour échelle 1m <=> 10cm.
\item Quelle hauteur atteint la balle au cinquième rebonds ?
  \begin{itemize}
  \item Donner le résultat sous la forme d'une fraction.
  \item Donner le résultat sous la forme décimale. \textit{(Arrondir au cm)}
  \item Donner le résultat sous la forme scientifique \textit{(Arrondir au cm)}
  \end{itemize}
\end{enumerate}

\horrule{1px}
\vspace{0.3cm}

On laisse tomber une balle d'une hauteur de 1 mètre.\\
À chaque rebond, elle rebondit des $\dfrac{3}{4}$ de la hauteur d'où elle est tombée.\\
\begin{enumerate}
\item Faire un schéma en prenant pour échelle 1m <=> 10cm.
\item Quelle hauteur atteint la balle au cinquième rebonds ?
  \begin{itemize}
  \item Donner le résultat sous la forme d'une fraction.
  \item Donner le résultat sous la forme décimale. \textit{(Arrondir au cm)}
  \item Donner le résultat sous la forme scientifique \textit{(Arrondir au cm)}
  \end{itemize}
\end{enumerate}

\horrule{1px}
\vspace{0.3cm}

On laisse tomber une balle d'une hauteur de 1 mètre.\\
À chaque rebond, elle rebondit des $\dfrac{3}{4}$ de la hauteur d'où elle est tombée.\\
\begin{enumerate}
\item Faire un schéma en prenant pour échelle 1m <=> 10cm.
\item Quelle hauteur atteint la balle au cinquième rebonds ?
  \begin{itemize}
  \item Donner le résultat sous la forme d'une fraction.
  \item Donner le résultat sous la forme décimale. \textit{(Arrondir au cm)}
  \item Donner le résultat sous la forme scientifique \textit{(Arrondir au cm)}
  \end{itemize}
\end{enumerate}

\horrule{1px}
\vspace{0.3cm}

On laisse tomber une balle d'une hauteur de 1 mètre.\\
À chaque rebond, elle rebondit des $\dfrac{3}{4}$ de la hauteur d'où elle est tombée.\\
\begin{enumerate}
\item Faire un schéma en prenant pour échelle 1m <=> 10cm.
\item Quelle hauteur atteint la balle au cinquième rebonds ?
  \begin{itemize}
  \item Donner le résultat sous la forme d'une fraction.
  \item Donner le résultat sous la forme décimale. \textit{(Arrondir au cm)}
  \item Donner le résultat sous la forme scientifique \textit{(Arrondir au cm)}
  \end{itemize}
\end{enumerate}

\end{document}

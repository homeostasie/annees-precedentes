\documentclass[12pt]{article}
\usepackage{geometry} % Pour passer au format A4
\geometry{hmargin=1.2cm, vmargin=1.2cm} % 

\usepackage{graphicx} % Required for including pictures
\usepackage{float} % 

%Français
\usepackage[T1]{fontenc} 
\usepackage[english,francais]{babel}
\usepackage[utf8]{inputenc}
\usepackage{eurosym}
\usepackage{lmodern}
\usepackage{url}
\usepackage{multicol}
\usepackage{multido}
%Maths
\usepackage{amsmath,amsfonts,amssymb,amsthm}
\usepackage{numprint}


%Autres
\linespread{1} % Line spacing
\setlength\parindent{0pt} % Removes all indentation from paragraphs

\renewcommand{\labelenumi}{\alph{enumi}.} %
\newcommand{\horrule}[1]{\rule{\linewidth}{#1}} % Create horizontal rule command with 1 argument of height
\newcommand{\Pointille}[1][3]{\multido{}{#1}{ \makebox[\linewidth]{\dotfill}\\[\parskip]}}

\pagestyle{empty}
%----------------------------------------------------------------------------------------
%	DOCUMENT INFORMATION
%----------------------------------------------------------------------------------------
\begin{document}

%\maketitle % Insert the title, author and date

\textbf{Nom, Prénom :} \hspace{8cm} \textbf{Classe :} \hspace{3cm} \textbf{Date :}

\begin{center}
  \textit{N'oublions pas que les petites émotions sont les grands capitaines de nos vies et qu'à celles-là nous y obéissons sans le savoir.} - \textbf{Vincent Van Gogh}
\end{center}

\subsection*{Exercice 1 - Donner l'écriture scientifique :}

\begin{enumerate}
\item[1a.] $ 128     $ 
\item[1b.] $ 0,0041  $ 
\item[1c.] $ 134,04  $ 
\item[1d.] $ -40,056 $ 
\item[1e.] $ 148 000 $
\item[1f.] $ 1028.04 $
\end{enumerate}

\subsection*{Exercice 2 - Faire les calculs suivants :}
\textbf{Donner l'écriture décimal ET scientifique.}

\begin{enumerate}
\item[2a.] $ 2000 \times 10^{-1} $  
\item[2b.] $ 2^{12}              $  
\item[2c.] $ (-5)^{9}            $  
\item[2d.] $ 2,1^{-4}            $ 
\item[2e.] $ 12^0                $ 
\item[2f.] $ 3^5                 $ 
\end{enumerate}

\textit{Pour les deux exercices suivants, la rédaction est prise en compte dans la notation. En particulier, il est demandé de répondre par une phrase.}

\subsection*{Exercice 3 - Rebond}

On laisse tomber une balle d'une hauteur de $10$ mètres. À chaque rebond, elle rebondit des $\dfrac{3}{4}$ de la hauteur d'où elle est tombée.\\

\textbf{Quelle hauteur atteint la balle au cinquième rebonds ?}


\subsection*{Exercice 4 - Année Lumière}

La  lumière  parcourt  \numprint{300  000  000}  mètres  par  seconde  (m/s)  environ. Une année est constituée d’environ \numprint{32 000 000} de secondes (s).

\begin{enumerate}

\item[4a.] Exprimer ces deux quantités en écriture scientifique. 
  \begin{itemize}
  \item \numprint{300 000 000} = 
  \item \numprint{32 000 000} = 
  \end{itemize}

\item[4b.] Calculer une année lumière, c’est à dire la distance que parcourt la lumière en une année.

\item[4c.] Calculer le temps que met la lumière pour nous parvenir du Soleil qui est situé en 
  moyenne à 150 millions de km de la Terre.

\item[4d.] L'Étoile polaire est à environ 350  a.l. de la Terre.\\
Exprimer cette distance en km.
\end{enumerate}

\end{document}

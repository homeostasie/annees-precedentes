\documentclass[11pt]{article}
\usepackage{geometry} % Pour passer au format A4
\geometry{hmargin=0.7cm, vmargin=0.7cm} % 

\usepackage{graphicx} % Required for including pictures
\usepackage{float} % 

%Français
\usepackage[T1]{fontenc} 
\usepackage[english,francais]{babel}
\usepackage[utf8]{inputenc}
\usepackage{eurosym}
\usepackage{lmodern}
\usepackage{url}
\usepackage{multicol}
\usepackage{multido}
%Maths
\usepackage{amsmath,amsfonts,amssymb,amsthm}
%\usepackage[linesnumbered, ruled, vlined]{algorithm2e}
%\SetAlFnt{\small\sffamily}

%Autres
\linespread{1} % Line spacing
\setlength\parindent{0pt} % Removes all indentation from paragraphs

\renewcommand{\labelenumi}{\alph{enumi}.} %
\newcommand{\horrule}[1]{\rule{\linewidth}{#1}} % Create horizontal rule command with 1 argument of height
\newcommand{\Pointille}[1][3]{\multido{}{#1}{ \makebox[\linewidth]{\dotfill}\\[\parskip]}}

\pagestyle{empty}
%----------------------------------------------------------------------------------------
%	DOCUMENT INFORMATION
%----------------------------------------------------------------------------------------
\begin{document}

%\maketitle % Insert the title, author and date

\textbf{Nom, Prénom :} \hspace{8cm} \textbf{Classe :} \hspace{3cm} \textbf{Date :}

\begin{center}
  \textit{Personne par la guerre, ne devient grand.} - The Empire Strikes Back - \textbf{Yoda}
\end{center}

\textbf{Exercice 1 - Donner l'écriture scientifique :}

  \begin{enumerate}
  \item[1a.] $128      =$ \Pointille[1]
  \item[1b.] $0,0041   =$ \Pointille[1]
  \item[1c.] $134,04   =$ \Pointille[1]
  \item[1d.] $-40,056  =$ \Pointille[1]
  \end{enumerate}

\textbf{Exercice 2}
\begin{enumerate}
\item[A] \textbf{Donner l'écriture décimale :}
  \begin{enumerate}
  \item[2a.] $2000 \times 10^{-1} =$  \Pointille[1]
  \item[2b.] $2175 \times 10^{-5} =$  \Pointille[1]
  \item[2c.] $281  \times 10^{-2} =$  \Pointille[1]
  \item[2d.] $0,9  \times 10^{5}  =$  \Pointille[1]
  \item[2e.] $5,1  \times 10^{0}  =$  \Pointille[1]
  \item[2f.] $3    \times 10^{-3} =$  \Pointille[1]
  \end{enumerate}
  
\item[B] \textbf{Donner l'écriture scientifique :}
  \begin{enumerate}
  \item[2a.] $2000 \times 10^{-1} =$  \Pointille[1]
  \item[2b.] $2175 \times 10^{-5} =$  \Pointille[1]
  \item[2c.] $281  \times 10^{-2} =$  \Pointille[1]
  \item[2d.] $0,9  \times 10^{5}  =$  \Pointille[1]
  \item[2e.] $5,1  \times 10^{0}  =$  \Pointille[1]
  \item[2f.] $3    \times 10^{-3} =$  \Pointille[1]
  \end{enumerate}

\end{enumerate}

\textbf{Cours - Donner la définition de la notation scientifique :}\\

\Pointille[6]
\end{document}

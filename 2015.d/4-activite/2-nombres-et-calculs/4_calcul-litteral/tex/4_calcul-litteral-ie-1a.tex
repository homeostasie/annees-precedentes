%%%%%%%%%%%%%%%%%%%%%%%%%%%%%%%%%%%%%%%%%
% LaTeX Template
% http://www.LaTeXTemplates.com
%
% Original author:
% Linux and Unix Users Group at Virginia Tech Wiki 
% (https://vtluug.org/wiki/Example_LaTeX_chem_lab_report)
%
% License:
% CC BY-NC-SA 3.0 (http://creativecommons.org/licenses/by-nc-sa/3.0/)
%
%%%%%%%%%%%%%%%%%%%%%%%%%%%%%%%%%%%%%%%%%

%----------------------------------------------------------------------------------------
%	PACKAGES AND DOCUMENT CONFIGURATIONS
%----------------------------------------------------------------------------------------

\documentclass[12pt]{article}
\usepackage{geometry} % Pour passer au format A4
\geometry{hmargin=1cm, vmargin=1cm} % 

\usepackage{graphicx} % Required for including pictures
\usepackage{float} % 

%Français
\usepackage[T1]{fontenc} 
\usepackage[english,francais]{babel}
\usepackage[utf8]{inputenc}
\usepackage{eurosym}
\usepackage{lmodern}
\usepackage{url}
\usepackage{multicol}

%Maths
\usepackage{amsmath,amsfonts,amssymb,amsthm}
%\usepackage[linesnumbered, ruled, vlined]{algorithm2e}
%\SetAlFnt{\small\sffamily}

%Autres
\linespread{1} % Line spacing
\setlength\parindent{0pt} % Removes all indentation from paragraphs

\newcommand{\horrule}[1]{\rule{\linewidth}{#1}} % Create horizontal rule 
\renewcommand{\labelenumi}{\alph{enumi}.} % 
\pagestyle{empty}
%----------------------------------------------------------------------------------------
%	DOCUMENT INFORMATION
%----------------------------------------------------------------------------------------
\begin{document}

%\maketitle % Insert the title, author and date

\begin{center}
  \textit{Un tableau ne vit que par celui qui le regarde.} - \textbf{Pablo Picasso}
\end{center}

\setlength{\columnseprule}{1pt}

\begin{multicols}{2}

  \subsection*{Exercice 1 - Programme de calcul}

  \begin{itemize}
  \item Choisir un nombre
  \item Ajouter 2.
  \item Multiplier par 3.
  \item Soustraire 6.
  \item Afficher le résultat
  \end{itemize}

  \begin{enumerate}
  \item[1a)] Vérifier qu'avec $5$ comme nombre de départ on obtient $15$.
  \item[1b)] Effectuer le programme avec $10$ ; $-4$ et $2.2$.
  \item[2a)] Faire une conjecture.
  \item[2b)] Essayer de prouver cette hypothèse en prenant $x$ comme nombre de départ.
  \end{enumerate}

\end{multicols}

\horrule{1px}

\begin{multicols}{2}

  \subsection*{Exercice 2 - Réduction}
  \textbf{Réduire les expressions suivantes afin de les écrire de la manière la plus simple et compacte possible.}

  \begin{enumerate}
  \item $A = 10x + 8 - 2x - 4 + 4x$
  \item $B = 7y -2x + 10 + y + 22$
  \item $C = 8x^2 + 12 - 5x + 2 - 3x$ 
  \item $D = 9y - (2y -3) + 8y^2$
  \end{enumerate}

  \subsection*{Exercice 3 - Distribution}

  \textbf{Développer les expressions suivantes.}
  \begin{enumerate}
  \item $A = 4(x + 2)$
  \item $B = 2(3y - 5)$
  \item $C = x(2x +12)$ 
  \item $D = -2(-3y - 4)$
  \end{enumerate}

\end{multicols}

\horrule{1px}

\begin{multicols}{2}
  \subsection*{Exercice 4 - Évaluer}

  \begin{enumerate}
  \item Évaluer l'expression : $2x +5$ pour $x = 4$.
  \item Évaluer l'expression : $5x^2 - 2x + 5$ pour $x = 2$.
  \item Évaluer l'expression : $x + 2(x +3)$ pour $x = -10$.
  \end{enumerate}

  \subsection*{Exercice 5 - Périmètre}

  % include figure
  Donner les expressions des \textbf{périmètres} des trois figures ci-dessous.

\end{multicols}

\horrule{0.5px}

\begin{multicols}{3}

  \begin{figure}[H]
    \centering
    \includegraphics[width=0.5\linewidth]{sources/1/peri-1a.pdf}
  \end{figure}

  \begin{figure}[H]
    \centering
    \includegraphics[width=0.8\linewidth]{sources/1/peri-2a.pdf}
  \end{figure}

  \begin{figure}[H]
    \centering
    \includegraphics[width=0.8\linewidth]{sources/1/peri-3a.pdf}
  \end{figure}

\end{multicols}


\horrule{1px}

\begin{multicols}{2}
  \subsection*{Exercice 6 - Rectangle}

  \textit{Laisser une trace des calculs et justifier la méthode utilisée.}\\

  Données : 
  \begin{itemize}
  \item $AC = 8$, $CB = 2$ et $CD = 5$
  \item Les points $A$, $B$ et $C$ sont alignés.\\
  \end{itemize}

  \textbf{Calculer la longueur $AD$}.

  \begin{figure}[H]
    \centering
    \includegraphics[width=0.7\linewidth]{sources/1/pytha-a.pdf}
  \end{figure}

\end{multicols}

\end{document}

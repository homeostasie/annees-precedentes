\documentclass[11pt]{article}
\usepackage{geometry} % Pour passer au format A4
\geometry{hmargin=0.7cm, vmargin=0.7cm} % 

\usepackage{graphicx} % Required for including pictures
\usepackage{float} % 

%Français
\usepackage[T1]{fontenc} 
\usepackage[english,francais]{babel}
\usepackage[utf8]{inputenc}
\usepackage{eurosym}
\usepackage{lmodern}
\usepackage{url}
\usepackage{multicol}
\usepackage{multido}
%Maths
\usepackage{amsmath,amsfonts,amssymb,amsthm}
%\usepackage[linesnumbered, ruled, vlined]{algorithm2e}
%\SetAlFnt{\small\sffamily}

%Autres
\linespread{1} % Line spacing
\setlength\parindent{0pt} % Removes all indentation from paragraphs

\renewcommand{\labelenumi}{\alph{enumi}.} %
\newcommand{\horrule}[1]{\rule{\linewidth}{#1}} % Create horizontal rule command with 1 argument of height
\newcommand{\Pointille}[1][3]{\multido{}{#1}{ \makebox[\linewidth]{\dotfill}\\[\parskip]}}

\pagestyle{empty}
%----------------------------------------------------------------------------------------
%	DOCUMENT INFORMATION
%----------------------------------------------------------------------------------------
\begin{document}

\textbf{Nom, Prénom :}\\

\textbf{MET2 : Réduire les expressions littéral.}



\horrule{1pt}

\begin{enumerate}
  
\item $2x + 2        = $ \Pointille[1]
\item $x \times 32   = $ \Pointille[1]
\item $3(x + 4)      = $ \Pointille[1]
\item $2x + 1 + x    = $ \Pointille[1]
\item $3x + 2y + 7x  = $ \Pointille[1]
  
\end{enumerate}

\vspace{-0.3cm}
\horrule{1pt}

\begin{enumerate}
  
\item \Pointille[1]
\item \Pointille[1]
\item \Pointille[1]
\item \Pointille[1]
\item \Pointille[1]
  
\end{enumerate}

\vspace{-0.3cm}
\horrule{1pt}

\begin{enumerate}
  
\item \Pointille[1]
\item \Pointille[1]
\item \Pointille[1]
\item \Pointille[1]
\item \Pointille[1]
  
\end{enumerate}

\vspace{-0.3cm}
\horrule{1pt}

\begin{enumerate}
  
\item \Pointille[1]
\item \Pointille[1]
\item \Pointille[1]
\item \Pointille[1]
\item \Pointille[1]
  
\end{enumerate}
  
\end{document}

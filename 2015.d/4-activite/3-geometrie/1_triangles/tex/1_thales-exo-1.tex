%%%%%%%%%%%%%%%%%%%%%%%%%%%%%%%%%%%%%%%%%
% LaTeX Template
% http://www.LaTeXTemplates.com
%
% Original author:
% Linux and Unix Users Group at Virginia Tech Wiki 
% (https://vtluug.org/wiki/Example_LaTeX_chem_lab_report)
%
% License:
% CC BY-NC-SA 3.0 (http://creativecommons.org/licenses/by-nc-sa/3.0/)
%
%%%%%%%%%%%%%%%%%%%%%%%%%%%%%%%%%%%%%%%%%

%----------------------------------------------------------------------------------------
%	PACKAGES AND DOCUMENT CONFIGURATIONS
%----------------------------------------------------------------------------------------

\documentclass[11pt]{article}
\usepackage{geometry} % Pour passer au format A4
\geometry{hmargin=1cm, vmargin=1cm} % 

\usepackage{graphicx} % Required for including pictures
\usepackage{float} % 

%Français
\usepackage[T1]{fontenc} 
\usepackage[english,francais]{babel}
\usepackage[utf8]{inputenc}
\usepackage{eurosym}
\usepackage{lmodern}
\usepackage{url}
\usepackage{multicol}

%Maths
\usepackage{amsmath,amsfonts,amssymb,amsthm}
%\usepackage[linesnumbered, ruled, vlined]{algorithm2e}
%\SetAlFnt{\small\sffamily}

%Autres
\linespread{1} % Line spacing
\setlength\parindent{0pt} % Removes all indentation from paragraphs

\renewcommand{\labelenumi}{\alph{enumi}.} % 
\pagestyle{empty}
%----------------------------------------------------------------------------------------
%	DOCUMENT INFORMATION
%----------------------------------------------------------------------------------------
\begin{document}

%\maketitle % Insert the title, author and date

\setlength{\columnseprule}{1pt}

\subsection*{1 - Tracer du triangle}

\begin{enumerate}
\item Tracer un triangle $ABC$ tel que $BC = 12cm$, $AB = 8cm$ et $AC = 10cm$.
\item Placer le point $D$ sur $[AB]$ tel que $AD = 2cm$.
\item Tracer la droite parallèle à $(BC)$ passant par $D$. On nomme $E$ le point d'intersection entre la droite et $(AC)$.
\end{enumerate}

\subsection*{2 - Des petits calculs}

\begin{enumerate}

\item Remplir le tableau.
  \begin{center}
    \begin{tabular}{| c | c | c | c | c | c |}
      \hline
      $AB$    & $AC$ & $BC$ & $AD$ & $AE$  & DE \\
      \hline
      \phantom{1234567890} & \phantom{1234567890} & \phantom{1234567890} & \phantom{1234567890} & \phantom{1234567890} &  \phantom{1234567890} \\
      \hline
    \end{tabular}
  \end{center}

\item En déduire.
  \begin{multicols}{3}
    \begin{itemize}
    \item[*] $\dfrac{AE}{AC} = $
    \item[*] $\dfrac{AD}{AB} = $
    \item[*] $\dfrac{ED}{BC} = $
    \end{itemize}
  \end{multicols}
\end{enumerate}

\vspace{1cm}
\subsection*{1 - Tracer du triangle}

\begin{enumerate}
\item Tracer un triangle $ABC$ tel que $BC = 12cm$, $AB = 8cm$ et $AC = 10cm$.
\item Placer le point $D$ sur $[AB]$ tel que $AD = 2cm$.
\item Tracer la droite parallèle à $(BC)$ passant par $D$. On nomme $E$ le point d'intersection entre la droite et $(AC)$.
\end{enumerate}

\subsection*{2 - Des petits calculs}

\begin{enumerate}

\item Remplir le tableau.
  \begin{center}
    \begin{tabular}{| c | c | c | c | c | c |}
      \hline
      $AB$    & $AC$ & $BC$ & $AD$ & $AE$  & DE \\
      \hline
      \phantom{1234567890} & \phantom{1234567890} & \phantom{1234567890} & \phantom{1234567890} & \phantom{1234567890} &  \phantom{1234567890} \\
      \hline
    \end{tabular}
  \end{center}

\item En déduire.
  \begin{multicols}{3}
    \begin{itemize}
    \item[*] $\dfrac{AE}{AC} = $
    \item[*] $\dfrac{AD}{AB} = $
    \item[*] $\dfrac{ED}{BC} = $
    \end{itemize}
  \end{multicols}
\end{enumerate}

\vspace{1cm}
\subsection*{1 - Tracer du triangle}

\begin{enumerate}
\item Tracer un triangle $ABC$ tel que $BC = 12cm$, $AB = 8cm$ et $AC = 10cm$.
\item Placer le point $D$ sur $[AB]$ tel que $AD = 2cm$.
\item Tracer la droite parallèle à $(BC)$ passant par $D$. On nomme $E$ le point d'intersection entre la droite et $(AC)$.
\end{enumerate}

\subsection*{2 - Des petits calculs}

\begin{enumerate}

\item Remplir le tableau.
  \begin{center}
    \begin{tabular}{| c | c | c | c | c | c |}
      \hline
      $AB$    & $AC$ & $BC$ & $AD$ & $AE$  & DE \\
      \hline
      \phantom{1234567890} & \phantom{1234567890} & \phantom{1234567890} & \phantom{1234567890} & \phantom{1234567890} &  \phantom{1234567890} \\
      \hline
    \end{tabular}
  \end{center}

\item En déduire.
  \begin{multicols}{3}
    \begin{itemize}
    \item[*] $\dfrac{AE}{AC} = $
    \item[*] $\dfrac{AD}{AB} = $
    \item[*] $\dfrac{ED}{BC} = $
    \end{itemize}
  \end{multicols}
\end{enumerate}

\newpage

\subsection*{1 - Tracer du triangle}

\begin{enumerate}
\item Tracer un triangle $ABC$ rectangle en $B$ et tel que $BC = 16cm$ et $AB = 10cm$.
\item Placer le point $D$ sur $[AB]$ tel que $BD = 8cm$.
\item Tracer la droite parallèle à $(BC)$ passant par $D$. On nomme $E$ le point d'intersection entre la droite et $(AC)$.
\end{enumerate}

\subsection*{2 - Des petits calculs}

\begin{enumerate}

\item Remplir le tableau.
  \begin{center}
    \begin{tabular}{| c | c | c | c | c | c |}
      \hline
      $AB$    & $AC$ & $BC$ & $AD$ & $AE$  & DE \\
      \hline
      \phantom{1234567890} & \phantom{1234567890} & \phantom{1234567890} & \phantom{1234567890} & \phantom{1234567890} &  \phantom{1234567890} \\
      \hline
    \end{tabular}
  \end{center}

\item En déduire.
  \begin{multicols}{3}
    \begin{itemize}
    \item[*] $\dfrac{AE}{AC} = $
    \item[*] $\dfrac{AD}{AB} = $
    \item[*] $\dfrac{ED}{BC} = $
    \end{itemize}
  \end{multicols}
\end{enumerate}

\subsection*{1 - Tracer du triangle}

\begin{enumerate}
\item Tracer un triangle $ABC$ rectangle en $B$ et tel que $BC = 16cm$ et $AB = 10cm$.
\item Placer le point $D$ sur $[AB]$ tel que $BD = 8cm$.
\item Tracer la droite parallèle à $(BC)$ passant par $D$. On nomme $E$ le point d'intersection entre la droite et $(AC)$.
\end{enumerate}

\subsection*{2 - Des petits calculs}

\begin{enumerate}

\item Remplir le tableau.
  \begin{center}
    \begin{tabular}{| c | c | c | c | c | c |}
      \hline
      $AB$    & $AC$ & $BC$ & $AD$ & $AE$  & DE \\
      \hline
      \phantom{1234567890} & \phantom{1234567890} & \phantom{1234567890} & \phantom{1234567890} & \phantom{1234567890} &  \phantom{1234567890} \\
      \hline
    \end{tabular}
  \end{center}

\item En déduire.
  \begin{multicols}{3}
    \begin{itemize}
    \item[*] $\dfrac{AE}{AC} = $
    \item[*] $\dfrac{AD}{AB} = $
    \item[*] $\dfrac{ED}{BC} = $
    \end{itemize}
  \end{multicols}
\end{enumerate}

\subsection*{1 - Tracer du triangle}

\begin{enumerate}
\item Tracer un triangle $ABC$ rectangle en $B$ et tel que $BC = 16cm$ et $AB = 10cm$.
\item Placer le point $D$ sur $[AB]$ tel que $BD = 8cm$.
\item Tracer la droite parallèle à $(BC)$ passant par $D$. On nomme $E$ le point d'intersection entre la droite et $(AC)$.
\end{enumerate}

\subsection*{2 - Des petits calculs}

\begin{enumerate}

\item Remplir le tableau.
  \begin{center}
    \begin{tabular}{| c | c | c | c | c | c |}
      \hline
      $AB$    & $AC$ & $BC$ & $AD$ & $AE$  & DE \\
      \hline
      \phantom{1234567890} & \phantom{1234567890} & \phantom{1234567890} & \phantom{1234567890} & \phantom{1234567890} &  \phantom{1234567890} \\
      \hline
    \end{tabular}
  \end{center}

\item En déduire.
  \begin{multicols}{3}
    \begin{itemize}
    \item[*] $\dfrac{AE}{AC} = $
    \item[*] $\dfrac{AD}{AB} = $
    \item[*] $\dfrac{ED}{BC} = $
    \end{itemize}
  \end{multicols}
\end{enumerate}

\end{document}

%%%%%%%%%%%%%%%%%%%%%%%%%%%%%%%%%%%%%%%%%
% LaTeX Template
% http://www.LaTeXTemplates.com
%
% Original author:
% Linux and Unix Users Group at Virginia Tech Wiki 
% (https://vtluug.org/wiki/Example_LaTeX_chem_lab_report)
%
% License:
% CC BY-NC-SA 3.0 (http://creativecommons.org/licenses/by-nc-sa/3.0/)
%
%%%%%%%%%%%%%%%%%%%%%%%%%%%%%%%%%%%%%%%%%

%----------------------------------------------------------------------------------------
%	PACKAGES AND DOCUMENT CONFIGURATIONS
%----------------------------------------------------------------------------------------

\documentclass[11pt]{article}
\usepackage{geometry} % Pour passer au format A4
\geometry{hmargin=1cm, vmargin=1cm} % 

\usepackage{graphicx} % Required for including pictures
\usepackage{float} % 

%Français
\usepackage[T1]{fontenc} 
\usepackage[english,francais]{babel}
\usepackage[utf8]{inputenc}
\usepackage{eurosym}
\usepackage{lmodern}
\usepackage{url}
\usepackage{multicol}

%Maths
\usepackage{amsmath,amsfonts,amssymb,amsthm}
%\usepackage[linesnumbered, ruled, vlined]{algorithm2e}
%\SetAlFnt{\small\sffamily}

%Autres
\linespread{1} % Line spacing
\setlength\parindent{0pt} % Removes all indentation from paragraphs

\renewcommand{\labelenumi}{\alph{enumi}.} % 
\pagestyle{empty}
%----------------------------------------------------------------------------------------
%	DOCUMENT INFORMATION
%----------------------------------------------------------------------------------------
\begin{document}

%\maketitle % Insert the title, author and date

\setlength{\columnseprule}{1pt}

\subsection*{3 - En déduire des longueurs}

\begin{multicols}{2}

\begin{figure}[H]
  \centering
  \includegraphics[width=0.7\linewidth]{sources/exo/1_longueurs-thales.pdf}
\end{figure}

Dans le triangle $ABC$, 

\begin{itemize}
\item $(EF)$ est parallèle à $(BC)$.
\item $AF = 8$, $AC = 12$, $AB = 9$, $BC = 14$.
\end{itemize}

\begin{enumerate}
\item Écrire les égalités découlant du théorème de Thalès.
\item Calculer AE.
\item Calculer EF.
\end{enumerate}

\end{multicols}

\vspace{1cm}

\subsection*{3 - En déduire des longueurs}

\begin{multicols}{2}

\begin{figure}[H]
  \centering
  \includegraphics[width=0.7\linewidth]{sources/exo/1_longueurs-thales-b.pdf}
\end{figure}

Dans le triangle $ZUT$, 

\begin{itemize}
\item $(AB)$ est parallèle à $(UT)$.
\item $ZA = 2$, $ZU = 3$, $ZB = 3$, $AB = 3$.
\end{itemize}

\begin{enumerate}
\item Écrire les égalités découlant du théorème de Thalès.
\item Calculer ZT.
\item Calculer UT.
\end{enumerate}

\end{multicols}

\vspace{1cm}

\subsection*{3 - En déduire des longueurs}

\begin{multicols}{2}

\begin{figure}[H]
  \centering
  \includegraphics[width=0.7\linewidth]{sources/exo/1_longueurs-thales.pdf}
\end{figure}

Dans le triangle $ABC$, 

\begin{itemize}
\item $(EF)$ est parallèle à $(BC)$.
\item $AF = 8$, $AC = 12$, $AB = 9$, $BC = 14$.
\end{itemize}

\begin{enumerate}
\item Écrire les égalités découlant du théorème de Thalès.
\item Calculer AE.
\item Calculer EF.
\end{enumerate}

\end{multicols}

\vspace{1cm}

\subsection*{3 - En déduire des longueurs}

\begin{multicols}{2}

\begin{figure}[H]
  \centering
  \includegraphics[width=0.7\linewidth]{sources/exo/1_longueurs-thales-b.pdf}
\end{figure}

Dans le triangle $ZUT$, 

\begin{itemize}
\item $(AB)$ est parallèle à $(UT)$.
\item $ZA = 2$, $ZU = 3$, $ZB = 3$, $AB = 3$.
\end{itemize}

\begin{enumerate}
\item Écrire les égalités découlant du théorème de Thalès.
\item Calculer ZT.
\item Calculer UT.
\end{enumerate}

\end{multicols}

\end{document}

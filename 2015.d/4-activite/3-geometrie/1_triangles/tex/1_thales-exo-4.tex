%%%%%%%%%%%%%%%%%%%%%%%%%%%%%%%%%%%%%%%%%
% LaTeX Template
% http://www.LaTeXTemplates.com
%
% Original author:
% Linux and Unix Users Group at Virginia Tech Wiki 
% (https://vtluug.org/wiki/Example_LaTeX_chem_lab_report)
%
% License:
% CC BY-NC-SA 3.0 (http://creativecommons.org/licenses/by-nc-sa/3.0/)
%
%%%%%%%%%%%%%%%%%%%%%%%%%%%%%%%%%%%%%%%%%

%----------------------------------------------------------------------------------------
%	PACKAGES AND DOCUMENT CONFIGURATIONS
%----------------------------------------------------------------------------------------

\documentclass[12pt]{article}
\usepackage{geometry} % Pour passer au format A4
\geometry{hmargin=1cm, vmargin=1cm} % 

\usepackage{graphicx} % Required for including pictures
\usepackage{float} % 

%Français
\usepackage[T1]{fontenc} 
\usepackage[english,francais]{babel}
\usepackage[utf8]{inputenc}
\usepackage{eurosym}
\usepackage{lmodern}
\usepackage{url}
\usepackage{multicol}

%Maths
\usepackage{amsmath,amsfonts,amssymb,amsthm}
%\usepackage[linesnumbered, ruled, vlined]{algorithm2e}
%\SetAlFnt{\small\sffamily}

%Autres
\linespread{1} % Line spacing
\setlength\parindent{0pt} % Removes all indentation from paragraphs

\renewcommand{\labelenumi}{\alph{enumi}.} % 
\pagestyle{empty}
%----------------------------------------------------------------------------------------
%	DOCUMENT INFORMATION
%----------------------------------------------------------------------------------------
\begin{document}

%\maketitle % Insert the title, author and date

\setlength{\columnseprule}{1pt}

\textbf{Nom(s), Prénom(s) :}

\begin{multicols}{2}
\subsection*{Exercice 1 - Calculer des proportions}

  \begin{figure}[H]
    \centering
    \includegraphics[width=\linewidth]{sources/exo/1_calcul-thales-1.pdf}
  \end{figure}


  \begin{enumerate}
  \item[1.] Dans le triangle $ABC$.\\ 
    \phantom{abc}\\
    \phantom{abc}\\
    D'après le théorème de Thalès, on en déduit :\\
    \phantom{abc}\\
    \phantom{abc}\\
    
  \item[2.] Avec les longueurs : $AM = 3$ et $AC = 12$.
    \begin{enumerate}
    \item[a)] Calculer : $\dfrac{AM}{AC} = $\\
    \item[b)] En déduire : $\dfrac{AN}{AB} = $\\
    \end{enumerate}

  \item[3.] On a aussi : $AB = 25$.\\
    En déduire : $AN$.\\
  \end{enumerate}
\end{multicols}

\vspace{0.8cm}
\noindent\hrulefill
\vspace{0.8cm}

\begin{multicols}{2}
\subsection*{Exercice 2 - Calculer des proportions}

  \begin{figure}[H]
    \centering
    \includegraphics[width=\linewidth]{sources/exo/1_calcul-thales-2.pdf}
  \end{figure}


  \begin{enumerate}
  \item[1.] Dans le triangle $XYZ$.\\ 
    \phantom{abc}\\
    \phantom{abc}\\
    D'après le théorème de Thalès, on en déduit :\\
    \phantom{abc}\\
    \phantom{abc}\\
    
  \item[2.] Avec les longueurs : $XA = 5$ et $XY = 25$.
    \begin{enumerate}
    \item[a)] Calculer : $\dfrac{XA}{XY} = $\\
    \item[b)] En déduire : $\dfrac{XB}{XZ} = $\\
    \end{enumerate}

  \item[3.] On a aussi : $AB = 2$.\\
    En déduire : $YZ$.\\
  \end{enumerate}
\end{multicols}

\end{document}

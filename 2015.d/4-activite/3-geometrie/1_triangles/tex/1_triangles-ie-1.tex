%%%%%%%%%%%%%%%%%%%%%%%%%%%%%%%%%%%%%%%%%
% LaTeX Template
% http://www.LaTeXTemplates.com
%
% Original author:
% Linux and Unix Users Group at Virginia Tech Wiki 
% (https://vtluug.org/wiki/Example_LaTeX_chem_lab_report)
%
% License:
% CC BY-NC-SA 3.0 (http://creativecommons.org/licenses/by-nc-sa/3.0/)
%
%%%%%%%%%%%%%%%%%%%%%%%%%%%%%%%%%%%%%%%%%

%----------------------------------------------------------------------------------------
%	PACKAGES AND DOCUMENT CONFIGURATIONS
%----------------------------------------------------------------------------------------

\documentclass[13pt]{article}
\usepackage{geometry} % Pour passer au format A4
\geometry{hmargin=0.5cm, vmargin=0.5cm} % 

\usepackage{graphicx} % Required for including pictures
\usepackage{float} % 

%Français
\usepackage[T1]{fontenc} 
\usepackage[english,francais]{babel}
\usepackage[utf8]{inputenc}
\usepackage{eurosym}
\usepackage{lmodern}
\usepackage{url}
\usepackage{multicol}

%Maths
\usepackage{amsmath,amsfonts,amssymb,amsthm}
%\usepackage[linesnumbered, ruled, vlined]{algorithm2e}
%\SetAlFnt{\small\sffamily}

%Autres
\linespread{1} % Line spacing
\setlength\parindent{0pt} % Removes all indentation from paragraphs

\renewcommand{\labelenumi}{\alph{enumi}.} % 
\pagestyle{empty}
%----------------------------------------------------------------------------------------
%	DOCUMENT INFORMATION
%----------------------------------------------------------------------------------------
\begin{document}

%\maketitle % Insert the title, author and date

\setlength{\columnseprule}{1pt}

\textbf{Nom(s), Prénom(s) :}

\begin{multicols}{2}

  \subsection*{Exercice 1}

  \begin{figure}[H]
    \centering
    \includegraphics[width=.8\linewidth]{sources/ie/1_ie-intro-1.pdf}
  \end{figure}


  \textit{Mesurer en centimètre et à l'aide d'une règle ou d'une equerre graduée les longueurs suivantes :}\\

  \begin{itemize}
  \item $AB = $\\
  \item $BC = $\\
  \item $AC = $\\
  \item $AE = $\\
  \item $ED = $\\
  \item $EC = $\\
  \end{itemize}

\end{multicols}

\subsection*{Exercice 2}

\textit{Faire les calculs suivants à partir des données suivantes :}  $AB = 12$, $AC = 6$, $BC = 2$.

\begin{enumerate}
\item $\dfrac{AB}{BC} = $\\
\item $AB - 2*(AC + BC) = $\\
\item $5 + \dfrac{AB * BC}{AC} = $\\
\item $2 * \dfrac{AB + BC}{BC} - 4= $\\
\end{enumerate}

\noindent\hrulefill

\textbf{Nom(s), Prénom(s) :}

\begin{multicols}{2}

  \subsection*{Exercice 1}

  \begin{figure}[H]
    \centering
    \includegraphics[width=.8\linewidth]{sources/ie/1_ie-intro-2.pdf}
  \end{figure}

  \textit{Mesurer en centimètre et à l'aide d'une règle ou d'une equerre graduée les longueurs suivantes :}\\

  \begin{itemize}
  \item $AB = $\\
  \item $BC = $\\
  \item $AC = $\\
  \item $AE = $\\
  \item $ED = $\\
  \item $EC = $\\
  \end{itemize}

\end{multicols}

\subsection*{Exercice 2}

\textit{Faire les calculs suivants à partir des données suivantes :}  $AB = 18$, $AC = 6$, $BC = 2$.

\begin{enumerate}
\item $\dfrac{AB}{BC} = $\\
\item $AB - 2*(AC + BC) = $\\
\item $5 + \dfrac{AB * BC}{AC} = $\\
\item $2 * \dfrac{AB + BC}{BC} - 4= $\\
\end{enumerate}

\end{document}

%%%%%%%%%%%%%%%%%%%%%%%%%%%%%%%%%%%%%%%%%
% LaTeX Template
% http://www.LaTeXTemplates.com
%
% Original author:
% Linux and Unix Users Group at Virginia Tech Wiki 
% (https://vtluug.org/wiki/Example_LaTeX_chem_lab_report)
%
% License:
% CC BY-NC-SA 3.0 (http://creativecommons.org/licenses/by-nc-sa/3.0/)
%
%%%%%%%%%%%%%%%%%%%%%%%%%%%%%%%%%%%%%%%%%

%----------------------------------------------------------------------------------------
%	PACKAGES AND DOCUMENT CONFIGURATIONS
%----------------------------------------------------------------------------------------

\documentclass[12pt]{article}
\usepackage{geometry} % Pour passer au format A4
\geometry{hmargin=1cm, vmargin=1cm} % 

\usepackage{graphicx} % Required for including pictures
\usepackage{float} % 

%Français
\usepackage[T1]{fontenc} 
\usepackage[english,francais]{babel}
\usepackage[utf8]{inputenc}
\usepackage{eurosym}
\usepackage{lmodern}
\usepackage{url}
\usepackage{multicol}

%Maths
\usepackage{amsmath,amsfonts,amssymb,amsthm}
%\usepackage[linesnumbered, ruled, vlined]{algorithm2e}
%\SetAlFnt{\small\sffamily}

%Autres
\linespread{1} % Line spacing
\setlength\parindent{0pt} % Removes all indentation from paragraphs

\renewcommand{\labelenumi}{\alph{enumi}.} % 
\pagestyle{empty}
%----------------------------------------------------------------------------------------
%	DOCUMENT INFORMATION
%----------------------------------------------------------------------------------------
\begin{document}

%\maketitle % Insert the title, author and date

\setlength{\columnseprule}{1pt}

\textbf{Nom(s), Prénom(s) :}

\subsection*{Exercice 1 - Écrire les égalités de Thalès}

\begin{figure}[H]
  \centering
  \includegraphics[width=0.8\linewidth]{sources/ie/1_config-thales-3.pdf}
\end{figure}

\begin{multicols}{3}
  \begin{enumerate}
  \item Dans le triangle $ABC$.\\ 
  \phantom{abc}\\
  \phantom{abc}\\
  D'après le théorème de Thalès, on en déduit :\\
   \phantom{abc}\\
  \phantom{abc}\\
  \item Dans le triangle $MNP$.\\ 
  \phantom{abc}\\
  \phantom{abc}\\
  D'après le théorème de Thalès, on en déduit :\\
  \phantom{abc}\\
  \phantom{abc}\\
  \item Dans le triangle $RST$\\
  \phantom{abc}\\
  \phantom{abc}\\
  D'après le théorème de Thalès, on en déduit :\\
  \phantom{abc}\\
  \phantom{abc}\\
  \end{enumerate}
\end{multicols}

\vspace{0.3cm}
\noindent\hrulefill
\vspace{0.3cm}

\subsection*{Exercice 2 - Écrire les égalités de Thalès}

\begin{figure}[H]
  \centering
  \includegraphics[width=0.8\linewidth]{sources/ie/1_config-thales-4.pdf}
\end{figure}

\begin{multicols}{3}
  \begin{enumerate}
  \item Dans le triangle $ABC$.\\ 
  \phantom{abc}\\
  \phantom{abc}\\
  D'après le théorème de Thalès, on en déduit :\\
   \phantom{abc}\\
  \phantom{abc}\\
  \item Dans le triangle $IJK$.\\ 
  \phantom{abc}\\
  \phantom{abc}\\
  D'après le théorème de Thalès, on en déduit :\\
  \phantom{abc}\\
  \phantom{abc}\\
  \item Dans le triangle $XYZ$\\
  \phantom{abc}\\
  \phantom{abc}\\
  D'après le théorème de Thalès, on en déduit :\\
  \phantom{abc}\\
  \phantom{abc}\\
  \end{enumerate}
\end{multicols}

\newpage

\textbf{Nom(s), Prénom(s) :}

\subsection*{Exercice 1 - Écrire les égalités de Thalès}

\begin{figure}[H]
  \centering
  \includegraphics[width=0.8\linewidth]{sources/ie/1_config-thales-3b.pdf}
\end{figure}

\begin{multicols}{3}
  \begin{enumerate}
  \item Dans le triangle $ABC$.\\ 
  \phantom{abc}\\
  \phantom{abc}\\
  D'après le théorème de Thalès, on en déduit :\\
   \phantom{abc}\\
  \phantom{abc}\\
  \item Dans le triangle $MNP$.\\ 
  \phantom{abc}\\
  \phantom{abc}\\
  D'après le théorème de Thalès, on en déduit :\\
  \phantom{abc}\\
  \phantom{abc}\\
  \item Dans le triangle $RST$\\
  \phantom{abc}\\
  \phantom{abc}\\
  D'après le théorème de Thalès, on en déduit :\\
  \phantom{abc}\\
  \phantom{abc}\\
  \end{enumerate}
\end{multicols}

\vspace{0.3cm}
\noindent\hrulefill
\vspace{0.3cm}

\subsection*{Exercice 2 - Écrire les égalités de Thalès}

\begin{figure}[H]
  \centering
  \includegraphics[width=0.8\linewidth]{sources/ie/1_config-thales-4b.pdf}
\end{figure}

\begin{multicols}{3}
  \begin{enumerate}
  \item Dans le triangle $ABC$.\\ 
  \phantom{abc}\\
  \phantom{abc}\\
  D'après le théorème de Thalès, on en déduit :\\
   \phantom{abc}\\
  \phantom{abc}\\
  \item Dans le triangle $IJK$.\\ 
  \phantom{abc}\\
  \phantom{abc}\\
  D'après le théorème de Thalès, on en déduit :\\
  \phantom{abc}\\
  \phantom{abc}\\
  \item Dans le triangle $XYZ$\\
  \phantom{abc}\\
  \phantom{abc}\\
  D'après le théorème de Thalès, on en déduit :\\
  \phantom{abc}\\
  \phantom{abc}\\
  \end{enumerate}
\end{multicols}

\end{document}

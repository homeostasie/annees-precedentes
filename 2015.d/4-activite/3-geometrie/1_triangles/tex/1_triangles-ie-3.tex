%%%%%%%%%%%%%%%%%%%%%%%%%%%%%%%%%%%%%%%%%
% LaTeX Template
% http://www.LaTeXTemplates.com
%
% Original author:
% Linux and Unix Users Group at Virginia Tech Wiki 
% (https://vtluug.org/wiki/Example_LaTeX_chem_lab_report)
%
% License:
% CC BY-NC-SA 3.0 (http://creativecommons.org/licenses/by-nc-sa/3.0/)
%
%%%%%%%%%%%%%%%%%%%%%%%%%%%%%%%%%%%%%%%%%

%----------------------------------------------------------------------------------------
%	PACKAGES AND DOCUMENT CONFIGURATIONS
%----------------------------------------------------------------------------------------

\documentclass[12pt]{article}
\usepackage{geometry} % Pour passer au format A4
\geometry{hmargin=1cm, vmargin=1cm} % 

\usepackage{graphicx} % Required for including pictures
\usepackage{float} % 

%Français
\usepackage[T1]{fontenc} 
\usepackage[english,francais]{babel}
\usepackage[utf8]{inputenc}
\usepackage{eurosym}
\usepackage{lmodern}
\usepackage{url}
\usepackage{multicol}

%Maths
\usepackage{amsmath,amsfonts,amssymb,amsthm}
%\usepackage[linesnumbered, ruled, vlined]{algorithm2e}
%\SetAlFnt{\small\sffamily}

%Autres
\linespread{1} % Line spacing
\setlength\parindent{0pt} % Removes all indentation from paragraphs

\renewcommand{\labelenumi}{\alph{enumi}.} % 
\pagestyle{empty}
%----------------------------------------------------------------------------------------
%	DOCUMENT INFORMATION
%----------------------------------------------------------------------------------------
\begin{document}

%\maketitle % Insert the title, author and date

\setlength{\columnseprule}{1pt}

\textbf{Nom(s), Prénom(s) :}

\subsection*{Exercice 1}

\begin{multicols}{2}

  \begin{figure}[H]
    \centering
    \includegraphics[width=0.7\linewidth]{sources/ie/1_longueur-thales-1.pdf}
  \end{figure}

  Dans le triangle $ABC$, 

  \begin{itemize}
  \item $(EF)$ est parallèle à $(BC)$.
  \item $AF = 18$, $AC = 24$, $AB = 18$, $BC = 16$.
  \end{itemize}

  \begin{enumerate}
  \item Écrire les égalités découlant du théorème de Thalès.
  \item Calculer AE.
  \item Calculer EF.
  \end{enumerate}

\end{multicols}
\subsection*{Exercice 2}

\begin{multicols}{2}

  \begin{figure}[H]
    \centering
    \includegraphics[width=0.6\linewidth]{sources/ie/1_longueur-thales-2.pdf}
  \end{figure}

  Dans le triangle $MNP$, 

  \begin{itemize}
  \item $(XY)$ est parallèle à $(NP)$.
  \item $MX = 15$, $MY = 25$, $MN = 20$, $XY = 15$.
  \end{itemize}

  \begin{enumerate}
  \item Écrire les égalités découlant du théorème de Thalès.
  \item Calculer MP.
  \item Calculer NP.
  \end{enumerate}

\end{multicols}

\noindent\hrulefill

\textbf{Nom(s), Prénom(s) :}

\subsection*{Exercice 1}

\begin{multicols}{2}

  \begin{figure}[H]
    \centering
    \includegraphics[width=0.7\linewidth]{sources/ie/1_longueur-thales-1.pdf}
  \end{figure}

  Dans le triangle $ABC$, 

  \begin{itemize}
  \item $(EF)$ est parallèle à $(BC)$.
  \item $AF = 15$, $AC = 25$, $AB = 20$, $BC = 15$.
  \end{itemize}

  \begin{enumerate}
  \item Écrire les égalités découlant du théorème de Thalès.
  \item Calculer AE.
  \item Calculer EF.
  \end{enumerate}

\end{multicols}
\subsection*{Exercice 2}

\begin{multicols}{2}

  \begin{figure}[H]
    \centering
    \includegraphics[width=0.6\linewidth]{sources/ie/1_longueur-thales-2.pdf}
  \end{figure}

  Dans le triangle $MNP$, 

  \begin{itemize}
  \item $(XY)$ est parallèle à $(NP)$.
  \item $MX = 18$, $MY = 24$, $MN = 32$, $XY = 16$.
  \end{itemize}

  \begin{enumerate}
  \item Écrire les égalités découlant du théorème de Thalès.
  \item Calculer MP.
  \item Calculer NP.
  \end{enumerate}

\end{multicols}


\end{document}

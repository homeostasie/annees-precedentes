%%%%%%%%%%%%%%%%%%%%%%%%%%%%%%%%%%%%%%%%%
% LaTeX Template
% http://www.LaTeXTemplates.com
%
% Original author:
% Linux and Unix Users Group at Virginia Tech Wiki 
% (https://vtluug.org/wiki/Example_LaTeX_chem_lab_report)
%
% License:
% CC BY-NC-SA 3.0 (http://creativecommons.org/licenses/by-nc-sa/3.0/)
%
%%%%%%%%%%%%%%%%%%%%%%%%%%%%%%%%%%%%%%%%%

%----------------------------------------------------------------------------------------
%	PACKAGES AND DOCUMENT CONFIGURATIONS
%----------------------------------------------------------------------------------------

\documentclass[11pt]{article}
\usepackage{geometry} % Pour passer au format A4
\geometry{hmargin=0.7cm, vmargin=0.7cm} % 

\usepackage{graphicx} % Required for including pictures
\usepackage{float} % 

%Français
\usepackage[T1]{fontenc} 
\usepackage[english,francais]{babel}
\usepackage[utf8]{inputenc}
\usepackage{eurosym}
\usepackage{lmodern}
\usepackage{url}
\usepackage{multicol}
\usepackage{multido}
%Maths
\usepackage{amsmath,amsfonts,amssymb,amsthm}
%\usepackage[linesnumbered, ruled, vlined]{algorithm2e}
%\SetAlFnt{\small\sffamily}

%Autres
\linespread{1} % Line spacing
\setlength\parindent{0pt} % Removes all indentation from paragraphs

\renewcommand{\labelenumi}{\alph{enumi}.} %
\newcommand{\horrule}[1]{\rule{\linewidth}{#1}} % Create horizontal rule command with 1 argument of height
\newcommand{\Pointille}[1][3]{\multido{}{#1}{ \makebox[\linewidth]{\dotfill}\\[\parskip]}}

\pagestyle{empty}
%----------------------------------------------------------------------------------------
%	DOCUMENT INFORMATION
%----------------------------------------------------------------------------------------
\begin{document}

%\maketitle % Insert the title, author and date

\textbf{Nom, Prénom :} \hspace{8cm} \textbf{Classe :} \hspace{3cm} \textbf{Date :}

\begin{center}
  \textit{On ne craint que ce que l'on ne connaît pas.}  - \textbf{Marie Curie}
\end{center}


% Exercice 1

\begin{multicols}{2}

  \subsection*{Ex1}

  \begin{itemize}
  \item[a.] Dans le rond central d'un terrain de foot, on a oublié de signaler le centre par un point. Retrouver le centre du cercle uniquement à l'aide de votre équerre.
  \item[b.] Écrire le théorème utilisé.\newline
    \Pointille[6]
  \end{itemize}

  \begin{figure}[H]
    \centering
    \includegraphics[width=0.6\linewidth]{sources/2/exo1.pdf}
  \end{figure}
  
\end{multicols}

\horrule{1px}

\begin{multicols}{2}

\textbf{Ex2 :}  On sait que (MN) est parallèle à (YZ).\newline
  Montrer que N est le milieu de [XY].
  
  \begin{figure}[H]
    \centering
    \includegraphics[width=0.5\linewidth]{sources/2/exo2b.pdf}
  \end{figure}

  \Pointille[7]

\end{multicols}

\horrule{1px}
\setlength{\columnseprule}{1pt}
\begin{multicols}{2}


\textbf{Ex3 :} MNOP est un parallélogramme tel que NO = 4 cm et MN = 1,7 cm.
  \begin{figure}[H]
    \centering
    \includegraphics[width=0.4\linewidth]{sources/2/exo4b.pdf}
  \end{figure}
  \begin{itemize}
  \item[a.] Que peux-tu dire des droites (IM) et (NO) ? Justifie. \newline
    \Pointille[3]
  \item[b.] Montre que I est le milieu du segment [OA]. \newline
    \Pointille[5]
  \item[c.] Calcule MI. \newline
    \Pointille[2] 
  \end{itemize}



\textbf{Ex4 :} ABCD est un trapèze dont les côtés [AB] et [CD] sont parallèles.
  \begin{figure}[H]
    \centering
    \includegraphics[width=0.5\linewidth]{sources/2/exo5b.pdf}
  \end{figure}
  \begin{itemize}
  \item[a.] Montre que (AB) et (PN) sont parallèles \newline
    \Pointille[7]
  \item[b.] Montre que (PN) et (DC) sont parallèles. \newline
    \Pointille[7]
  \end{itemize}

\end{multicols}

\end{document}

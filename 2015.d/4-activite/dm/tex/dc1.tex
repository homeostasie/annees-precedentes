%%%%%%%%%%%%%%%%%%%%%%%%%%%%%%%%%%%%%%%%%
% LaTeX Template
% http://www.LaTeXTemplates.com
%
% Original author:
% Linux and Unix Users Group at Virginia Tech Wiki 
% (https://vtluug.org/wiki/Example_LaTeX_chem_lab_report)
%
% License:
% CC BY-NC-SA 3.0 (http://creativecommons.org/licenses/by-nc-sa/3.0/)
%
%%%%%%%%%%%%%%%%%%%%%%%%%%%%%%%%%%%%%%%%%

%----------------------------------------------------------------------------------------
%	PACKAGES AND DOCUMENT CONFIGURATIONS
%----------------------------------------------------------------------------------------

\documentclass[12pt]{article}
\usepackage{geometry} % Pour passer au format A4
\geometry{hmargin=1cm, vmargin=1cm} % 

\usepackage{graphicx} % Required for including pictures
\usepackage{float} % 

%Français
\usepackage[T1]{fontenc} 
\usepackage[english,francais]{babel}
\usepackage[utf8]{inputenc}
\usepackage{eurosym}
\usepackage{lmodern}
\usepackage{url}
\usepackage{multicol}

%Maths
\usepackage{amsmath,amsfonts,amssymb,amsthm}
%\usepackage[linesnumbered, ruled, vlined]{algorithm2e}
%\SetAlFnt{\small\sffamily}

%Autres
\linespread{1} % Line spacing
\setlength\parindent{0pt} % Removes all indentation from paragraphs

\newcommand{\horrule}[1]{\rule{\linewidth}{#1}} % Create horizontal rule 
\renewcommand{\labelenumi}{\alph{enumi}.} % 
\pagestyle{empty}
%----------------------------------------------------------------------------------------
%	DOCUMENT INFORMATION
%----------------------------------------------------------------------------------------
\begin{document}

\subsection*{Exercice 1 (7 points)}

Une commune souhaite aménager des parcours de santé sur son territoire. On fait deux propositions au conseil municipal, schématisées ci-dessous :

\begin{itemize}
\item le parcours ACDA
\item le parcours AEFA
\end{itemize}

Ils souhaitent faire un parcours dont la longueur s’approche le plus possible de 4 km.

Peux-tu les aider à choisir le parcours ? Justifie.

\begin{multicols}{2}

  \begin{figure}[H]
    \centering
    \includegraphics[width=0.8\linewidth]{sources/exo1.pdf}
  \end{figure}

  \begin{itemize}
  \item L’angle $\widehat{EAF}$ dans le triangle AEF vaut $30^{\circ}$.
  \item $(E' F') $\slash{}\slash{}$ (EF)$
  \item $AC = 1,4 km$
  \item $CD = 1,05 km$
  \item $AE' = 0,5 km$
  \item $AE = 1,3 km$
  \item $AF = 1,6 km$
  \item $E'F' = 0,4 km$  \\
  \end{itemize}

  \textit{Attention : la figure proposée au conseil municipal n’est pas à l’échelle, mais les codages et les dimensions données sont correctes.}
\end{multicols}

\subsection*{Exercice 2 (3 points)}

On laisse tomber une balle d’une hauteur de 1 mètre. A chaque rebond elle rebondit des $\dfrac{3}{4}$ de la hauteur d’où elle est tombée. \newline
Quelle hauteur atteint la balle au cinquième rebond ? \textit{Arrondir au cm près.}

\subsection*{Exercice 3 (3 points)}

\begin{multicols}{2}
  
Les alvéoles des nids d’abeilles présentent une ouverture ayant la forme d’un hexagone régulier de côté 3 mm environ.
Construire un agrandissement de cet hexagone de rapport 10. \textit{(Aucune justification de la construction n’est attendue.)}

\begin{figure}[H]
  \centering
  \includegraphics[width=0.3\linewidth]{sources/exo3.pdf}
\end{figure}

\end{multicols}

\newpage

\subsection*{Exercice 4 (6 points)}

\begin{multicols}{2}

\textit{Attention les figures tracées ne respectent ni les mesures de longueur, ni les mesures d’angle.} \newline

Répondre par ``vrai'' ou ``faux'' ou ``on ne peut pas savoir'' à chacune des affirmations suivantes et expliquer votre choix.
  
\begin{enumerate}
\item Tout triangle inscrit dans un cercle est rectangle.
\item Si un point M appartient à la médiatrice d’un segment [AB] alors le triangle AMB est isocèle.
\item Dans le triangle ABC suivant, AB = 4 cm.
\item Le quadrilatère ABCD ci-contre est un carré.
\end{enumerate}

\begin{figure}[H]
  \centering
  \includegraphics[width=0.5\linewidth]{sources/exo4a.pdf}
\end{figure}

\begin{figure}[H]
  \centering
  \includegraphics[width=0.5\linewidth]{sources/exo4b.pdf}
\end{figure}

\end{multicols}

\subsection*{Exercice 5 (4 points)}

Cédric s’entraîne pour l’épreuve de vélo d’un triathlon. La courbe ci-dessous représente la distance en kilomètres en fonction du temps écoulé en minutes.

\begin{figure}[H]
  \centering
  \includegraphics[width=0.6\linewidth]{sources/exo5.pdf}
\end{figure}

\textit{Pour les trois premières questions, les réponses seront données grâce à des lectures graphiques. Aucune justification n’est attendue sur la copie.}

\begin{enumerate}
\item Quelle distance Cédric a-t-il parcourue au bout de 20 minutes ?
\item Combien de temps a mis Cédric pour faire les 30 premiers kilomètres ?
\item Le circuit de Cédric comprend une montée, une descente et deux portions plates. Reconstituer dans l’ordre le trajet parcouru par Cédric.
\item Calculer la vitesse moyenne de Cédric (exprimée en km/h) sur l'ensemble du trajet.
\end{enumerate}

\end{document}


\documentclass{beamer}

\usepackage{geometry} % Pour passer au format A4
\usepackage{graphicx} % Required for including pictures
\usepackage{float} %

\usepackage{amsmath,amsfonts,amssymb,amsthm}
\usepackage[T1]{fontenc}
\usepackage[english,francais]{babel}
\usepackage[utf8]{inputenc}
\usepackage{lmodern}
\usepackage{eurosym} % signe Euros
\usepackage{verbatim}
\usepackage{multicol}
\usefonttheme[onlymath]{serif}
\usetheme{m} 

\title{\rmfamily{\textsc{V}}itesse moyenne}

\begin{document}

\frame{\titlepage}

\section{ \rmfamily{\textsc{I - P}}roportionnalité et vitesse moyenne}
\begin{frame}
  \frametitle{\rmfamily{\textsc{I - P}}roportionnalité et vitesse moyenne}

  \begin{alertblock}{Définition}
    La vitesse moyenne sur un trajet est le  quotient de la distance parcourue par la durée du trajet.
    $$v = \frac{d}{t}$$
  \end{alertblock}

  \begin{block}{Propriété}
    À partir de cette relation et de deux grandeurs, on peut toujours en déduire la troisième.
    \begin{multicols}{2}
      $$d = v \times t$$
      $$t = \frac{d}{v}$$
    \end{multicols}
  \end{block}
\end{frame}

\begin{frame}
  \frametitle{\rmfamily{\textsc{I - P}}roportionnalité et vitesse moyenne}
  \begin{block}{Remarque}
    La distance parcourue est \textbf{proportionnelle} au temps de parcours. Le coefficient de proportionnalité est la vitesse moyenne. On parle alors de \textbf{mouvement uniforme}.
  \end{block}
\end{frame}



\section{\rmfamily{\textsc{II - C}}onversions d'unités}
\begin{frame}
  \frametitle{\rmfamily{\textsc{II - C}}onversions d'unités}

  \begin{block}{Unités de vitesse}
    Il existe plusieurs unités de vitesse. 

    \begin{itemize}
    \item Pour les voitures, scooters et coureurs. On parle de kilomètre par heure.\\
      1 kilomètre en 1 heure : $1km/h$ ou $1km.h^{-1}$
    \item Pour les calculs scientifiques. On utilise les unités internationnales. On parle de mètre par seconde.\\
      1 mètre en 1 seconde   : $1m/s$  ou $m.s^{-1}$ 
    \end{itemize}
  \end{block}
  
\end{frame}

\begin{frame}
  \frametitle{\rmfamily{\textsc{II - C}}onversions d'unités}

  \begin{exampleblock}{Exemples}
    Pour comparer des vitesses, il faut qu'elles aient la même unité. \\On sait que : $1km = 1000m$, $1h = 60min = 3600s$ et $1min = 60s$

    \begin{multicols}{2}
      \begin{enumerate}
      \item[1.] Convertir 54 km /h en m/s
        \begin{eqnarray*}
          54 km /h &=& 54 000 m/h\\
          &=& 900 m/min\\
          &=& 15 m/s
        \end{eqnarray*}

      \item[2.] Convertir 2m/s en km/h.
        \begin{eqnarray*}
          2 m /s &=& 120  m /min\\
          &=& 7200 m /h\\
          &=& 7.2 km /h
        \end{eqnarray*}
      \end{enumerate}
    \end{multicols}
  \end{exampleblock}
  
\end{frame}

\end{document}

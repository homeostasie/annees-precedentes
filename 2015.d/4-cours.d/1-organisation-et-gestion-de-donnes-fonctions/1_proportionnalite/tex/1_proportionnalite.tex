%%%%%%%%%%%%%%%%%%%%%%%%%%%%%%%%%%%%%%%%%
% Short Sectioned Assignment
% LaTeX Template
% Version 1.0 (5/5/12)
%
% This template has been downloaded from:
% http://www.LaTeXTemplates.com
%
% Original author:
% Frits Wenneker (http://www.howtotex.com)
%
% License:
% CC BY-NC-SA 3.0 (http://creativecommons.org/licenses/by-nc-sa/3.0/)
%
%%%%%%%%%%%%%%%%%%%%%%%%%%%%%%%%%%%%%%%%%

%----------------------------------------------------------------------------------------
%	PACKAGES AND OTHER DOCUMENT CONFIGURATIONS
%----------------------------------------------------------------------------------------

\documentclass[paper=a4, fontsize=9pt]{scrartcl} % A4 paper and 11pt font size


\usepackage[T1]{fontenc} % Use 8-bit encoding that has 256 glyphs
\usepackage[english,francais]{babel} % Français et anglais
\usepackage[utf8]{inputenc} 

\usepackage{amsmath,amsfonts,amsthm} % Math packages

\usepackage{enumitem}
\usepackage{lmodern}
\usepackage{url}
\usepackage{eurosym} % signe Euros
\usepackage{geometry} % Pour passer au format A4
\geometry{a4paper} % 
\usepackage{graphicx} % Required for including pictures
\usepackage{float} % Allows putting an [H] in \begin{figure} to specify the exact location of the figure

\usepackage{multicol}

\usepackage{verbatim}

\usepackage{sectsty} % Allows customizing section commands
\allsectionsfont{\centering \normalfont\scshape} % Make all sections centered, the default font and small caps

%----------------------------------------------------------------------------------------
%	Pied de Page
%----------------------------------------------------------------------------------------


\usepackage{fancyhdr} % Custom headers and footers
\pagestyle{fancyplain} % Makes all pages in the document conform to the custom headers and footers
\fancyhead{} % No page header - if you want one, create it in the same way as the footers below
\fancyfoot[C]{Proportionnalité} % Empty center footer
\fancyfoot[R]{\thepage} % Page numbering for right footer

\renewcommand{\headrulewidth}{0pt} % Remove header underlines
\renewcommand{\footrulewidth}{0pt} % Remove footer underlines

%\usepackage{titling}
%\setlength{\droptitle}{-1cm}
%\setlength{\headheight}{13.6pt} % Customize the height of the header

\setlength\parindent{0pt} % Removes all indentation from paragraphs - comment this line for an assignment with lots of text


%----------------------------------------------------------------------------------------
%	Titre
%----------------------------------------------------------------------------------------

\newcommand{\horrule}[1]{\rule{\linewidth}{#1}} % Create horizontal rule command with 1 argument of height

\title{
  \vspace{-10ex}
  \horrule{0.5pt} % Thin top horizontal rule
  \huge Proportionnalité\\
  \horrule{2pt}
}

\author{}
\date{\vspace{-10ex}} % Today's date or a custom date

%----------------------------------------------------------------------------------------
%	Début du document
%----------------------------------------------------------------------------------------
\begin{document}

%----------------------------------------------------------------------------------------
% RE-DEFINITION
%----------------------------------------------------------------------------------------
% MATHS
%-----------

\newtheorem{Definition}{Définition}
\newtheorem{Theorem}{Théorème}
\newtheorem{Proposition}{Propriété}

% MATHS
%-----------
\renewcommand{\labelitemi}{$\bullet$}
\renewcommand{\labelitemii}{$\circ$}
%----------------------------------------------------------------------------------------
%	Titre
%----------------------------------------------------------------------------------------

\maketitle % Print the title
\setlength{\columnseprule}{1pt}

\section{Reconnaître une situation de proportionnalité et produit en croix}

\begin{multicols}{2}

  \subsection{Tableau de proportionnelle}


  % ajouter le résultat à droite.
  \begin{center}
    \begin{tabular}{| c || c | c | c|}
      \hline
      Quantité & 1 & 5  & 12\\
      \hline
      Prix     & 4 & 20 & 48\\ 
      \hline
    \end{tabular}
  \end{center}


  Un tableau est de proportionnalité si on peut passer d'une ligne à l'autre en multipliant par un même nombre. \\
  On appelle ce nombre le \textbf{cœfficient de proportionnalité}. \textit{C'est également la valeur d'une unité.}\\

  On calcule le coœfficient en divisant le nombre du bas par le nombre du haut : $\dfrac{20}{5}$.\\
  Pour avoir la réponse souhaitée, il ne nous reste plus qu'à multiplier la quantité par ce cœfficient.

\end{multicols}

\begin{multicols}{2}

  \subsection*{L'égalité des produits en croix}

  \begin{center}
    \begin{tabular}{| c || c | c |}
      \hline
      Quantité & 5 & 32  \\
      \hline
      Prix     & 20 & $x$ \\ 
      \hline
    \end{tabular}
  \end{center}

  On peut voir un tableau de proportionnel comme des fractions dont les rapports sont égaux. 

  $$\dfrac{5}{20} = \dfrac{32}{x}$$

  On parle alors d'égalité des produits en croix.
  $$ 5 \times x = 20 \times 30 $$

  On peut donc résoudre tout problème de proportionnalité et trouver la quatrième proportionnelle en multipliant les nombres sur la diagonales qu'on divise par la troisième valeur du tabelau.

  $$x = \dfrac{4 \times 20}{5}$$

\end{multicols}



\section{Pourcentages}

Un pourcentage représente une situation de proportionnalité. 

\subsection{Appliquer un pourcentage}

\subsection{Appliquer une réduction ou une augmentation}


\subsection{Calculer un pourcentage}


\section{Représentation graphique}

\begin{Proposition}
  Une situation de proportionnalité se représente graphiquement par des points alignés avec l'origine du repère. On parle également de \textbf{droite passant par l'origine}.
\end{Proposition}

\begin{figure}[H]
  \centering
  \includegraphics[width=0.8\linewidth]{sources/3/prop-repre.pdf}
\end{figure}

\section{Mouvements uniformes - Vitesse}
\subsection{Proportionnalité et vitesse moyenne}

\begin{Definition}
  La vitesse moyenne sur un trajet est le  quotient de la distance parcourue par la durée du trajet.
  $$v = \frac{d}{t}$$
\end{Definition}

\begin{Proposition}
  À partir de cette relation et de deux grandeurs, on peut toujours en déduire la troisième.
  \begin{multicols}{2}
    $$d = v \times t$$
    $$t = \frac{d}{v}$$
  \end{multicols}
\end{Proposition}

\paragraph{Remarque}~~\\
La distance parcourue est \textbf{proportionnelle} au temps de parcours. Le coefficient de proportionnalité est la vitesse moyenne. On parle alors de \textbf{mouvement uniforme}.

\subsection{Conversions d'unité}

Il existe plusieurs unités de vitesse. 

\begin{itemize}
\item Pour les voitures, scooters et coureurs. On parle de kilomètre par heure.\\
  1 kilomètre en 1 heure : $1km/h$ ou $1km.h^{-1}$
\item Pour les calculs scientifiques. On utilise les unités internationnales. On parle de mètre par seconde.\\
  1 mètre en 1 seconde   : $1m/s$  ou $m.s^{-1}$ 
\end{itemize}

Pour comparer des vitesses, il faut qu'elles aient la même unité. \\On sait que : $1km = 1000m$, $1h = 60min = 3600s$ et $1min = 60s$

\begin{multicols}{2}
  \begin{enumerate}
  \item[1.] Convertir 54 km /h en m/s
    \begin{eqnarray*}
      54 km /h &=& 54 000 m/h\\
      &=& 900 m/min\\
      &=& 15 m/s
    \end{eqnarray*}

  \item[2.] Convertir 2m/s en km/h.
    \begin{eqnarray*}
      2 m /s &=& 120  m /min\\
      &=& 7200 m /h\\
      &=& 7.2 km /h
    \end{eqnarray*}
  \end{enumerate}
\end{multicols}

\section{Agrandissements - Réductions}
\section{Conversions d'unités}

\end{document}

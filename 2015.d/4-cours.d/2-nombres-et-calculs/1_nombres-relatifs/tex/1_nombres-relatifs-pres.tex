\documentclass{beamer}

\usepackage{geometry} % Pour passer au format A4
\usepackage{graphicx} % Required for including pictures
\usepackage{float} %

\usepackage{amsmath,amsfonts,amssymb,amsthm}
\usepackage[T1]{fontenc}
\usepackage[english,francais]{babel}
\usepackage[utf8]{inputenc}
\usepackage{lmodern}
\usepackage{eurosym} % signe Euros
\usepackage{verbatim}
\usepackage{multicol}
\usepackage{array}
\usefonttheme[onlymath]{serif}
\usetheme{m} 

\title{\rmfamily{\textsc{O}}pération avec des relatifs}

\begin{document}

\frame{\titlepage}

\section{\rmfamily{\textsc{I - P}}roduit de relatifs}

\begin{frame}
  \frametitle{\rmfamily{\textsc{1 - R}}ègle de signes}

  \alert{1 - Règle de signes}

  \begin{block}{\textcolor{blue}{Propriété}}
    \begin{itemize}
    \item Le produit de deux nombres de même signe est positif.
    \item Le produit de deux nombres de signe contraire est négatif.
    \end{itemize}

    \begin{multicols}{2}
      $$(+) \times (+) = (+)$$
      $$(-) \times (-) = (+)$$

      $$(+) \times (-) = (-)$$
      $$(-) \times (+) = (-)$$
    \end{multicols}
  \end{block}
\end{frame}

\begin{frame}
  \frametitle{\rmfamily{\textsc{1 - R}}ègle de signes}

  \begin{block}{\textcolor{blue}{Propriété}}
    \begin{itemize}
    \item Un produit est positif si le nombre de facteurs négatifs (nombre de signe -) est paire. 
    \item Un produit est négatif si le nombre de facteurs négatifs (nombre de signe -) est impaire. 
    \end{itemize}
  \end{block}

  \begin{block}{\textcolor{blue}{Propriété}}
    Un produit est nul si l'un des facteurs est nul.
  \end{block}



\end{frame}

\begin{frame}
  \frametitle{\rmfamily{\textsc{2 - O}}pposé d'un nombre}

  \alert{2 - Opposé d'un nombre}

  \begin{alertblock}{Définition}
    Deux nombres sont \alert{opposés} l'un de l'autre si leur somme est égale à zéro.
  \end{alertblock}

  \begin{block}{\textcolor{blue}{Propriété}}
    Soit $x$ un nombre. L'opposé de $x$ est $-x$.
  \end{block}

  \begin{exampleblock}{Exemples}

    \begin{multicols}{2}
      \begin{itemize}
      \item L'opposé de $2$ est $-2$.
      \item L'opposé de $12$ est $-12$.
      \item L'opposé de $-5$ est $5$.
      \item L'opposé de $7.2$ est $-7.2$.
      \item L'opposé de $\Pi$ est $-\Pi$.
      \item L'opposé de $0$ est $0$.
      \end{itemize}
    \end{multicols}
  \end{exampleblock}
\end{frame}

\begin{frame}
  \frametitle{\rmfamily{\textsc{3 - P}}roduit de fractions}
  
  \alert{3 - Produit de fractions}
  
  \begin{block}{\textcolor{blue}{Propriété}}
    Pour multiplier deux fractions, on multiplie les numérateurs entre eux et les dénominateurs entre eux.
  \end{block}
  
  \begin{exampleblock}{Exemples}
    
    \begin{itemize}
    \item $\dfrac{7}{3} \times \dfrac{5}{2} = \dfrac{7 \times 5}{3 \times 2} = \dfrac{35}{6}$
    \item $2 \times \dfrac{9}{7} = \dfrac{2}{1} \times \dfrac{9}{7} = \dfrac{2 \times 9}{1 \times 7} = \dfrac{18}{7}$
    \end{itemize}
    
  \end{exampleblock}
\end{frame}


\section{\rmfamily{\textsc{II - Q}}uotient de relatifs}

\begin{frame}
  \frametitle{\rmfamily{\textsc{1 - R}}ègle de signes}

  \alert{1 - Règle de signes}

  \begin{block}{\textcolor{blue}{Propriété}}
    \begin{itemize}
    \item Le quotient de deux nombres de même signe est positif.
    \item Le quotient de deux nombres de signe contraire est négatif.
    \end{itemize}

    \begin{multicols}{2}
      $$ \dfrac{ (+) }{ (+) }= (+)$$
      $$ \dfrac{ (-) }{ (-) }= (+)$$

      $$ \dfrac{ (+) }{ (-) }= (-)$$
      $$ \dfrac{ (-) }{ (+) }= (-)$$
    \end{multicols}
  \end{block}

  \begin{exampleblock}{Exemples}
    \begin{multicols}{2}
      \begin{itemize}
      \item $ \dfrac{-2}{3} = \dfrac{2}{-3} = -\dfrac{2}{3} = -\dfrac{-2}{-3}$ 
      \item $ \dfrac{-2}{-3} = \dfrac{2}{3}$ 
      \end{itemize}
    \end{multicols}
  \end{exampleblock}

\end{frame}


\begin{frame}


  \begin{center}
    \begin{tabular}{| m{1.2cm} || m{1.2cm} | m{1.2cm} | m{1.2cm} | m{1.2cm} | m{1.2cm} | m{1.2cm} |}
      \hline
      $\times$ & $\dfrac{1}{2}$ & $\dfrac{1}{3}$ & $\dfrac{1}{4}$ & $\dfrac{1}{5}$ & $\dfrac{3}{7}$ & $\dfrac{-5}{7}$ \\
      \hline
      \hline
      2               & \phantom{$\dfrac{azerty}{a}$} & & & & & \\ 
      \hline
      3               & \phantom{$\dfrac{azerty}{a}$} & & & & & \\ 
      \hline
      4               & \phantom{$\dfrac{azerty}{a}$} & & & & & \\ 
      \hline
      5               & \phantom{$\dfrac{azerty}{a}$} & & & & & \\ 
      \hline
      $\dfrac{7}{3}$  & \phantom{$\dfrac{azerty}{a}$} & & & & & \\ 
      \hline
      $\dfrac{-5}{3}$ & \phantom{$\dfrac{azerty}{a}$} & & & & & \\ 
      \hline
    \end{tabular}
  \end{center}


\end{frame}



\begin{frame}
  \frametitle{\rmfamily{\textsc{2 - I}}nverse d'un nombre}

  \alert{2 - Inverse d'un nombre}

  \begin{alertblock}{Définition}
    Deux nombres sont inverses l'un de l'autre si leur produit est égal à un.
  \end{alertblock}

  \begin{block}{\textcolor{blue}{Propriété}}
    \begin{itemize}
    \item Soit $x \neq 0$ un nombre. L'inverse de $x$ est $\dfrac{1}{x}$. 
    \item Soient $a$ et $b \neq 0$ deux nombres. L'inverse de $\dfrac{a}{b}$ est $\dfrac{b}{a}$. 
    \end{itemize}
  \end{block}

  \begin{block}{\textcolor{blue}{Preuve}}
    \begin{multicols}{2}
      \begin{itemize}
      \item $x \times \dfrac{1}{x} = \dfrac{x}{x} = 1$ 
      \item $\dfrac{a}{b} \times \dfrac{b}{a} = \dfrac{ab}{ab} = 1$ 
      \end{itemize}
    \end{multicols}
  \end{block}
\end{frame}

\begin{frame}
  \frametitle{\rmfamily{\textsc{3 - Q}}uotient de relatifs}

  \alert{3 - Quotient de relatifs}

  \begin{block}{\textcolor{blue}{Propriété}}
    Divisier par un nombre non nul revient par multiplier par son inverse.
  \end{block}

  \begin{exampleblock}{Exemples}

    \begin{itemize}
    \item $2 \div 3 = \dfrac{2}{3} = 2 \times \dfrac{1}{3}$ 
    \item $3 \div \dfrac{4}{5} = 3 \times \dfrac{5}{4} = \dfrac{15}{4}$
    \item $\dfrac{6}{7} \div \dfrac{8}{9} = \dfrac{6}{7} \times \dfrac{9}{8} = \dfrac{54}{56}$
    \item $\dfrac{\dfrac{1}{2}}{\dfrac{3}{4}} = \dfrac{1}{2} \times \dfrac{4}{3} = \dfrac{4}{6}$
    \end{itemize}

  \end{exampleblock}
\end{frame}

\end{document}

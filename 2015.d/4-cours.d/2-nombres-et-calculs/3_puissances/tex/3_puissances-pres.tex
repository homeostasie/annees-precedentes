\documentclass{beamer}

\usepackage{geometry} % Pour passer au format A4
\usepackage{graphicx} % Required for including pictures
\usepackage{float} %

\usepackage{amsmath,amsfonts,amssymb,amsthm}
\usepackage[T1]{fontenc}
\usepackage[english,francais]{babel}
\usepackage[utf8]{inputenc}
\usepackage{lmodern}
\usepackage{eurosym} % signe Euros
\usepackage{verbatim}
\usepackage{multicol}
\usepackage{numprint}

\usefonttheme[onlymath]{serif}
\usetheme{m} 

\title{\rmfamily{\textsc{P}}uissances}

\begin{document}

\frame{\titlepage}

\begin{frame}
  \begin{alertblock}{Définition 1. Lecture}
        $6^5$ se lit 6 à la puissance 5 ou 6 exposant 5.
  \end{alertblock}
\end{frame}

\section{\rmfamily{\textsc{I - N}}otations et calculs}

\begin{frame}
  \frametitle{\rmfamily{\textsc{I - N}}otations et calculs}

  \begin{alertblock}{Définition 2a. Puissances positives}

    \begin{eqnarray*}
      6^2 &=& 6 \times 6 = 36\\
      6^3 &=& 6 \times 6 \times 6 = 216\\
      6^4 &=& 6 \times 6 \times 6 \times 6 = 1296\\
      &...& \\
      6^{12} &=& 6 \times ... \times 6 = \numprint{2 176 782 336} \\
      6^n &=& 6 \times ... \times 6
    \end{eqnarray*}
   
  \end{alertblock}
\end{frame}


\begin{frame}
  \frametitle{\rmfamily{\textsc{I - N}}otations et calculs}

  \begin{alertblock}{Définition 2b. Puissances négatives}

    \begin{eqnarray*}
      3^{-2} &=& \dfrac{1}{3^{2}} = \dfrac{1}{9} = 0.111\\
      3^{-3} &=& \dfrac{1}{3^{3}} = \dfrac{1}{27} = 0.037\\
      &...& \\
      3^{-7} &=& \dfrac{1}{3^{7}} = \dfrac{1}{2187} = \numprint{0.00046}
    \end{eqnarray*}

  \end{alertblock}

  \begin{exampleblock}{Remarque}

On obtient des nombres très grands ou très proche de 0 qui \textit{dépassent les capacités des calculatrices} comme $5^{200}$ ou $7^{-150}$.

  \end{exampleblock}

\end{frame}

\begin{frame}
  \frametitle{\rmfamily{\textsc{I - N}}otations et calculs}

  \begin{alertblock}{Définition 3. Quelques conventions}

    \begin{itemize}
    \item  Un nombre à la puissance 0 est égal à 1. \newline
      $1^0 = 5^0 = 12^0 = 0^0 = ... = 15.7^0 = 1$ \\
    \item Un nombre à la puissance 1 est égal à lui-même. \newline
      $5^1 = 5$
    \item Un nombre à la puissance -1 est l'inverse de ce nombre. \newline
      $7^{-1} = \dfrac{1}{7}$
    \end{itemize}

  \end{alertblock}

\end{frame}

\section{\rmfamily{\textsc{II - Q}}uelques simplifications possibles}

\begin{frame}
  \frametitle{\rmfamily{\textsc{II - Q}}uelques simplifications possibles}

  \begin{block}{\textcolor{blue}{Propriété 1.}}

	Peu de simplifications sont possibles avec les puissances. \textbf{Il faut toujours le même nombre sous les puissances.}
	
    \begin{itemize}
    \item Lors d'une multiplication on additionne les exposants. \\
      $5^3 \times 5^7 = 5^{3+7} = 5^{10}$
    \item Lors d'une division on soustrait l'exposant du dessous à celui de haut. \\
      $\dfrac{12^{10}}{12^{6}} = 12^{10 - 6} = 12^4$
    \item On distribue l'ensemble des facteurs dans la parenthèse. \\
      $(3 x)^4 = 3^4 \times x^4$
    \end{itemize}
  
  \end{block}
\end{frame}


\begin{frame}
  \frametitle{\rmfamily{\textsc{II - Q}}uelques simplifications possibles}

  \begin{block}{\textcolor{blue}{Preuve - Propriété 1.}}
  \textit{par un retour au produit.}
  
  \begin{itemize}
  \item $5^3 \times 5^7 = 5 \times ... \times 5 \times  5 \times ... \times 5 =  5 \times ... \times 5 = 5^{10}$
  \item $\dfrac{12^{10}}{12^{6}} = \dfrac{12 \times ... \times 12}{12 \times ... \times 12} =12 \times ... \times 12 = 12^4$
  \item $(3 x)^4 = (3x) \times ... \times (3x) = 3 \times ... \times 3 \times x \times ... \times x = 3^4 \times x^4.$
  \end{itemize}

  \begin{exampleblock}{Attention}
on ne connaît pas directement le résultat d'une addition ou d'une soustraction de nombre avec des puissances : $8^4 + 8^2 \neq 8^6$, $(3 + x)^2 \neq 3^2 + x^2 $.

  \end{exampleblock}

  \end{block}
\end{frame}


\section{\rmfamily{\textsc{III - L}}'écriture scientifique}


\begin{frame}
  \frametitle{\rmfamily{\textsc{III - L}}'écriture scientifique}

Traiter avec des nombres très grands (distance entre les galaxies en mètre) et des nombres très petits (masse d'un atome en kg) est courant en Sciences. \\
Pour faciliter les calculs, on introduit la \textbf{notation scientifique}.

 \begin{alertblock}{Définition 4.}
	Tout nombre peut s'écrire en notation scientifique sous la forme : $ \boxed{ a \times 10^n} $\\
  où $a$ est un nombre décimal compris entre 1 et 10 appelé mantisse. ($a$ peut être négatif)

  \end{alertblock}

\end{frame}

\begin{frame}
  \frametitle{\rmfamily{\textsc{III - L}}'écriture scientifique}


 \begin{block}{Exemples}
	  \begin{itemize}
  \item $\numprint{1 234} = 1,234 \times 10^3$
  \item $\numprint{0,00659} = 6,59 \times 10^{-3}$
  \item Distance de la Terre au soleil :  $\numprint{149 600 000 000}m = 1,49 \times 10^{11}$
  \item Masse d'un électron : $9,1 \times 10^{-31}kg$
  \item Vitesse de la lumière : $\numprint{300 000 000}m/s = 3 \times 10 ^8 m/s$
  \end{itemize}

  \end{block}

\end{frame}

\begin{frame}
  \frametitle{\rmfamily{\textsc{III - L}}'écriture scientifique}

Cette écriture simplifie les calculs.

 \begin{block}{Pour les produits}

  \begin{eqnarray*}
    & & \numprint{65 000 000 000} \times \numprint{420 000 000} \\
    &=& 6,, \times 10 ^{10} \times 4,2 \times 10 ^{8} \\
    &=& 6,5 \times 4,2 \times 10^{10} \times 10^8 \\
    &=& 27,3 \times 10^{10 + 8}\\
    &=& 27,3 \times 10^{18} \\
    &=& 2,73 \times 10^{19}
  \end{eqnarray*}


  \end{block}

\end{frame}


\begin{frame}
  \frametitle{\rmfamily{\textsc{III - L}}'écriture scientifique}

Cette écriture simplifie les calculs.

 \begin{block}{Pour les quotients}

  \begin{eqnarray*}
    \dfrac{\numprint{0, 000 74}}{\numprint{530 000}} &=& \dfrac{7,4 \times 10^{-4}}{5,3 \times 10^{5}} \\
    &=& \dfrac{7,4}{5,3} \times \dfrac{10^{-4}}{10^{5}} \\
    &=& 1,4 \times 10^{( -4 - 5 )} \\
    &=& 1,4 \times 10^{-9}
  \end{eqnarray*}

  \end{block}

\end{frame}
\end{document}

%%%%%%%%%%%%%%%%%%%%%%%%%%%%%%%%%%%%%%%%%
% Short Sectioned Assignment
% LaTeX Template
% Version 1.0 (5/5/12)
%
% This template has been downloaded from:
% http://www.LaTeXTemplates.com
%
% Original author:
% Frits Wenneker (http://www.howtotex.com)
%
% License:
% CC BY-NC-SA 3.0 (http://creativecommons.org/licenses/by-nc-sa/3.0/)
%
%%%%%%%%%%%%%%%%%%%%%%%%%%%%%%%%%%%%%%%%%

%----------------------------------------------------------------------------------------
%	PACKAGES AND OTHER DOCUMENT CONFIGURATIONS
%----------------------------------------------------------------------------------------

\documentclass[10pt]{article}
\usepackage{geometry} % Pour passer au format A4
\geometry{hmargin=0.7cm, vmargin=0.7cm} % 

\usepackage[T1]{fontenc} % Use 8-bit encoding that has 256 glyphs
\usepackage[english,francais]{babel} % Français et anglais
\usepackage[utf8]{inputenc} 

\usepackage{amsmath,amsfonts,amsthm} % Math packages

\usepackage{enumitem}
\usepackage{lmodern}
\usepackage{url}
\usepackage{eurosym} % signe Euros
\usepackage{geometry} % Pour passer au format A4
\geometry{a4paper} % 
\usepackage{graphicx} % Required for including pictures
\usepackage{float} % Allows putting an [H] in \begin{figure} to specify the exact location of the figure

\usepackage{multicol}
\usepackage{numprint}

\usepackage{verbatim}

\usepackage{sectsty} % Allows customizing section commands
\allsectionsfont{\centering \normalfont\scshape} % Make all sections centered, the default font and small caps

%----------------------------------------------------------------------------------------
%	Pied de Page
%----------------------------------------------------------------------------------------


\usepackage{fancyhdr} % Custom headers and footers
\pagestyle{fancyplain} % Makes all pages in the document conform to the custom headers and footers
\fancyhead{} % No page header - if you want one, create it in the same way as the footers below
\fancyfoot[C]{Puissances} % Empty center footer
\fancyfoot[R]{\thepage} % Page numbering for right footer

\renewcommand{\headrulewidth}{0pt} % Remove header underlines
\renewcommand{\footrulewidth}{0pt} % Remove footer underlines

%\usepackage{titling}
%\setlength{\droptitle}{-2.5cm}
%\setlength{\headheight}{13.6pt} % Customize the height of the header

\setlength\parindent{0pt} % Removes all indentation from paragraphs - comment this line for an assignment with lots of text


%----------------------------------------------------------------------------------------
%	Titre
%----------------------------------------------------------------------------------------

\newcommand{\horrule}[1]{\rule{\linewidth}{#1}} % Create horizontal rule command with 1 argument of height

\title{	
  \vspace{-6ex}
  \horrule{0.5pt} \\[0.4cm] % Thin top horizontal rule
  \huge Puissances \\ % The assignment title
  \horrule{2pt} \\[0.5cm] % Thick bottom horizontal rule
}

\author{}
\date{\vspace{-10ex}} % Today's date or a custom date

%----------------------------------------------------------------------------------------
%	Début du document
%----------------------------------------------------------------------------------------

\begin{document}

%----------------------------------------------------------------------------------------
% RE-DEFINITION
%----------------------------------------------------------------------------------------
% MATHS
%-----------

\newtheorem{Definition}{Définition}
\newtheorem{Theorem}{Théorème}
\newtheorem{Proposition}{Propriété}

% MATHS
%-----------
\renewcommand{\labelitemi}{$\bullet$}
\renewcommand{\labelitemii}{$\circ$}
%----------------------------------------------------------------------------------------
%	Titre
%----------------------------------------------------------------------------------------
\setlength{\columnseprule}{1pt}

\maketitle % Print the title

\vspace{-4ex}

\begin{multicols}{2}

  \begin{Definition}{Lecture}\\
    $6^5$ se lit 6 à la puissance 5 ou 6 exposant 5.
  \end{Definition}
  
  \subsection*{I - Notations et calculs}
  
\end{multicols}

\horrule{0.5px}

\begin{multicols}{2}
  
  \begin{Definition}{Puissances positives}
    
    \vspace{-4ex}

    \begin{eqnarray*}
      6^2 &=& 6 \times 6 = 36\\
      6^3 &=& 6 \times 6 \times 6 = 216\\
      6^4 &=& 6 \times 6 \times 6 \times 6 = 1296\\
      &...& \\
      6^{12} &=& 6 \times ... \times 6 = \numprint{2 176 782 336} \\
      6^n &=& 6 \times ... \times 6
    \end{eqnarray*}
    Puissances négatives
    \begin{eqnarray*}
      3^{-2} &=& \dfrac{1}{3^{2}}  = \dfrac{1}{9} = 0.111\\
      3^{-3} &=& \dfrac{1}{3^{3}} = \dfrac{1}{27} = 0.037\\
      &...& \\
      3^{-7} &=& \dfrac{1}{3^{7}} = \dfrac{1}{2187} = \numprint{0.00046}
    \end{eqnarray*}
    
  \end{Definition}
\end{multicols}

On obtient des nombres très grands ou très proche de 0 qui \textit{dépassent les capacités des calculatrices} comme $5^{200}$ ou $7^{-150}$.

\horrule{0.5px}

\begin{multicols}{2}

  \begin{Definition}{Quelques conventions}
    \begin{itemize}
    \item  Un nombre à la puissance 0 est égal à 1. \newline
      $1^0 = 5^0 = 12^0 = 0^0 = ... = 15.7^0 = 1$ \\
    \item Un nombre à la puissance 1 est égal à lui-même. \newline
      $5^1 = 5$
    \item Un nombre à la puissance -1 est l'inverse de ce nombre. \newline
      $7^{-1} = \dfrac{1}{7}$
    \end{itemize}
  \end{Definition}

\end{multicols}

\horrule{0.5px}

\subsection*{II - Quelques simplifications possibles}

\begin{multicols}{2}

  \begin{Proposition} {Peu de simplifications sont possibles avec les puissances. \textbf{Il faut toujours le même nombre sous les puissances.}}
    \begin{itemize}
    \item Lors d'une multiplication on additionne les exposants. \\
      $5^3 \times 5^7 = 5^{3+7} = 5^{10}$
    \item Lors d'une division on soustrait l'exposant du dessous à celui de haut. \\
      $\dfrac{12^{10}}{12^{6}} = 12^{10 - 6} = 12^4$
    \item On distribue l'ensemble des facteurs dans la parenthèse. \\
      $(3 x)^4 = 3^4 \times x^4$
    \end{itemize}
  \end{Proposition}

  \textbf{Preuve} par un retour au produit.
  
  \begin{itemize}
  \item $5^3 \times 5^7 = 5 \times ... \times 5 \times  5 \times ... \times 5 =  5 \times ... \times 5 = 5^{10}$
  \item $\dfrac{12^{10}}{12^{6}} = \dfrac{12 \times ... \times 12}{12 \times ... \times 12} =12 \times ... \times 12 = 12^4$
  \item $(3 x)^4 = (3x) \times ... \times (3x) = 3 \times ... \times 3 \times x \times ... \times x = 3^4 \times x^4.$
  \end{itemize}

\end{multicols}

\textbf{Attention}, on ne connaît pas directement le résultat d'une addition ou d'une soustraction de nombre avec des puissances : $8^4 + 8^2 \neq 8^6$, $(3 + x)^2 \neq 3^2 + x^2 $.

\horrule{0.5px}

\subsection*{ III - L'écriture scientifique}

Traiter avec des nombres très grands (distance entre les galaxies en mètre) et des nombres très petits (masse d'un atome) est courant en Sciences. \\
Pour facilité les calculs, on introduit la \textbf{notation scientifique}.

\begin{Definition}
  Tout nombre peut s'écrire en notation scientifique sous la forme : $ \boxed{ a \times 10^n} $\\
  où a est un nombre décimal compris entre 1 et 10 appelé mantisse. (a peut être négatif)
\end{Definition}

\begin{multicols}{3}

  Exemples :
  \begin{itemize}
  \item $\numprint{1 234} = 1,234 \times 10^3$
  \item $\numprint{0,006 59} = 6,59 \times 10^{-3}$
  \item Distance de la Terre au soleil :  $\numprint{149 600 000 000}m = 1,49 \times 10^{11}$
  \item Masse d'un électron : $9,1 \times 10^{-31}kg$
  \item Vitesse de la lumière : $\numprint{300 000 000}m/s = 3 \times 10 ^8 m/s$
  \end{itemize}

  Cette écriture simplifie les calculs.

  \begin{eqnarray*}
    & & \numprint{65 000 000 000} \times \numprint{420 000 000} \\
    &=& 6,5 \times 10 ^{10} \times 4,2 \times 10 ^{8} \\
    &=& 6,5 \times 4,2 \times 10^{10} \times 10^8 \\
    &=& 27,3 \times 10^{10 + 8}\\
    &=& 27,3 \times 10^{18} \\
    &=& 2,73 \times 10^{19}
  \end{eqnarray*}

  \begin{eqnarray*}
    \dfrac{\numprint{0. 000 74}}{\numprint{530 000}} &=& \dfrac{7,4 \times 10^{-4}}{5,3 \times 10^{5}} \\
    &=& \dfrac{7,4}{5,3} \times \dfrac{10^{-4}}{10^{5}} \\
    &=& 1,4 \times 10^{( -4 - 5 )} \\
    &=& 1,4 \times 10^{-9}
  \end{eqnarray*}

\end{multicols}


\end{document}

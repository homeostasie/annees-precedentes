\documentclass{beamer}

\usepackage{geometry} % Pour passer au format A4
\usepackage{graphicx} % Required for including pictures
\usepackage{float} %

\usepackage{amsmath,amsfonts,amssymb,amsthm}
\usepackage[T1]{fontenc}
\usepackage[english,francais]{babel}
\usepackage[utf8]{inputenc}
\usepackage{lmodern}
\usepackage{eurosym} % signe Euros
\usepackage{verbatim}
\usepackage{multicol}
\usefonttheme[onlymath]{serif}
\usetheme{m} 

\title{\rmfamily{\textsc{C}}alcul littéral}

\begin{document}

\frame{\titlepage}

\begin{frame}
  \begin{alertblock}{Définition}
    Un calcul littéral est un calcul où l'ensemble des nombres ne sont pas connus. On représente ces \textbf{inconnus} par des \textbf{lettres}.
  \end{alertblock}
\end{frame}

\section{\rmfamily{\textsc{I - P}}rogramme de calcul}

\begin{frame}
  \frametitle{\rmfamily{\textsc{I - P}}rogramme de calcul}

  \begin{block}{\textcolor{blue}{Propriété}}

    \begin{multicols}{2}
      
      Un programme de calcul est un ensemble d'instructions et d'opérations à faire les unes à la suite des autres.\vspace{2cm}
      
      \fbox{\parbox{\textwidth}{
          \begin{itemize}
          \item Choisir un nombre.
          \item Ajouter $2$.
          \item Multiplier par $2$.
          \item Soustraire $4$.
          \item Ajouter le nombre de départ.
          \item Afficher le résultat.
          \end{itemize}
      }}
    \end{multicols}
  \end{block}
\end{frame}

\begin{frame}
  \frametitle{\rmfamily{\textsc{I - P}}rogramme de calcul}
  \begin{exampleblock}{Exercice}
    
    \begin{multicols}{2}
      
      \begin{enumerate}
      \item[a)] Montrer qu'en choississant $5$ comme nombre de départ, on obtient $15$.
      \item[b)] Essayer avec $-2$, $4$ et $10$.
      \item[c)] Démontrer que le résultat est toujours le triple du nombre de départ.
      \end{enumerate}
      
      \fbox{\parbox{\textwidth}{
          \begin{itemize}
          \item Choisir un nombre.
          \item Ajouter $2$.
          \item Multiplier par $2$.
          \item Soustraire $4$.
          \item Ajouter le nombre de départ.
          \item Afficher le résultat.
          \end{itemize}
      }}
    \end{multicols}
    

  \end{exampleblock}
\end{frame}

\begin{frame}
  \frametitle{\rmfamily{\textsc{I - P}}rogramme de calcul}
  \begin{block}{\textcolor{blue}{Propriétés}}
    \begin{enumerate}
    \item[1.] Pour \textbf{démontrer} qu'un résultat est vrai, il faut démontrer qu'il est vrai tout le temps et pour toutes valeurs. On doit utiliser le calcul littéral.
    \item[2.] Simplifier l'écriture d'une expréssion littéral s'appelle : \textbf{réduire}. Il faut faire tous les calculs possibles.
    \end{enumerate}
  \end{block}
\end{frame}

\section{ \rmfamily{\textsc{II - R}}ègles d'écriture}

\begin{frame}
  \frametitle{ \rmfamily{\textsc{II - R}}ègles d'écriture}
  \begin{block}{\textcolor{blue}{Propriétés}}
    Le signe $\times$ de la multiplication n'est pas nécessairement écrit.
    \begin{itemize}
    \item $2 \times x          = 2x$
    \item $3 \times (2 + x)    = 3(2 + x)$
    \item $v \times t          = vt$
    \item $2 \times a \times 3 = 2 \times 3 \times a = 6 \times a = 6a$
    \end{itemize}
    Le 1 devant une lettre ne doit pas être écrit. Zéro fois une lettre est égale à 0. 
    \begin{itemize}
    \item $1 \times x          = 1x = x$
    \item $0 \times x          = 0x = 0$
    \end{itemize}
  \end{block}
\end{frame}

\begin{frame}
  \frametitle{ \rmfamily{\textsc{II - R}}ègles d'écriture}
  \begin{block}{\textcolor{blue}{Propriétés}}
    On peut additionner (ou soustraire) des mêmes lettres.
    \begin{itemize}
    \item $2x + 3x - 10x = 5x - 10x = -5x$
    \end{itemize}
    On ne peut pas additionner ou soustraire une lettre et un nombre.
    \begin{itemize}
    \item $1 + 2x + 3 - 4x = 4 - 2x$
    \end{itemize}
    On ne peut pas additionner ou soustraire deux lettres différentes.
    \begin{itemize} 
    \item $1 + a + 2b + 3a +4b = 1 + 4a + 6b$
    \item $1 + x + x^2 + 2 + 3x + 4x^2 = 3 + 4x + 5x^2$
    \end{itemize}
  \end{block}  
\end{frame}

\section{ \rmfamily{\textsc{III - D}}istributivité}

\begin{frame}
  \frametitle{ \rmfamily{\textsc{III - D}}istributivité}

  \begin{multicols}{2}
    \textbf{Programme A}
    \begin{itemize}
    \item Choisir un nombre.
    \item Ajouter $3$.
    \item Multiplier par $2$.
    \item Afficher le résultat.
    \end{itemize}

    \textbf{Programme B}  
    \begin{itemize}
    \item Choisir un nombre.
    \item Multiplier par $2$.
    \item Ajouter $6$.
    \item Afficher le résultat.
    \end{itemize}
  \end{multicols}

  \begin{enumerate}
  \item[a)] Essayer les programmes A et B avec les nombres $1$; $2$ et $5$.
  \item[b)] Faire une conjecture
  \item[c)] Produire les expressions littérales.
  \end{enumerate}
  
\end{frame}


\begin{frame}
  \frametitle{ \rmfamily{\textsc{III - D}}istributivité}

 \begin{block}{Preuve}
   \begin{enumerate}
   \item[a)] Calculer l'aire du grand rectangle.
   \item[b)] Calculer la somme des aires des petits rectangles.
   \item[c)] En dédurie une égalité.
   \end{enumerate}
 \end{block}

  \begin{figure}[H]
    \centering
    \includegraphics[width=0.6\linewidth]{sources/1/simple-distri.pdf}
  \end{figure}
\end{frame}

\begin{frame}
  \frametitle{ \rmfamily{\textsc{III - D}}istributivité}

  \begin{figure}[H]
    \centering
    \includegraphics[width=0.6\linewidth]{sources/1/simple-distri.pdf}
  \end{figure}

  $$2 \times (x + 3) = 2 \times x + 2 \times 3 $$

\end{frame}


\begin{frame}
  \frametitle{ \rmfamily{\textsc{III - D}}istributivité}
  \begin{alertblock}{Distributivité  simple}
    $$k \times (a + b) = k \times a + k \times b $$
    $$k \times (a - b) = k \times a - k \times b $$ 
  \end{alertblock}
\end{frame}

\section{ \rmfamily{\textsc{E}}t si on démontrait le premier programme de calcul ?}

\begin{frame}
  \frametitle{\rmfamily{\textsc{I - P}}rogramme de calcul}
  \begin{exampleblock}{Exercice}
    
    \begin{multicols}{2}
      
      \begin{enumerate}
      \item[a)] Montrer qu'en choississant $5$ comme nombre de départ, on obtient $15$.
      \item[b)] Essayer avec $-2$, $4$ et $10$.
      \item[c)] Démontrer que le résultat est toujours le triple du nombre de départ.
      \end{enumerate}
      
      \fbox{\parbox{\textwidth}{
          \begin{itemize}
          \item Choisir un nombre.
          \item Ajouter $2$.
          \item Multiplier par $2$.
          \item Soustraire $4$.
          \item Ajouter le nombre de départ.
          \item Afficher le résultat.
          \end{itemize}
      }}
    \end{multicols}
    

  \end{exampleblock}
\end{frame}

\end{document}

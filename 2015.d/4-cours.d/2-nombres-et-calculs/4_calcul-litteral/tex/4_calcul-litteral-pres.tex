\documentclass{beamer}

\usepackage{geometry} % Pour passer au format A4
\usepackage{graphicx} % Required for including pictures
\usepackage{float} %

\usepackage{amsmath,amsfonts,amssymb,amsthm}
\usepackage[T1]{fontenc}
\usepackage[english,francais]{babel}
\usepackage[utf8]{inputenc}
\usepackage{lmodern}
\usepackage{eurosym} % signe Euros
\usepackage{verbatim}
\usepackage{multicol}
\usefonttheme[onlymath]{serif}
\usetheme{m} 

\title{\rmfamily{\textsc{C}}alcul littéral}

\begin{document}

\frame{\titlepage}

\begin{frame}
  \begin{alertblock}{Définition}
    Un calcul littéral est un calcul où l'ensemble des nombres ne sont pas connus. On représente ces \textbf{inconnus} par des \textbf{lettres}.
  \end{alertblock}
\end{frame}

\section{\rmfamily{\textsc{I - P}}rogramme de calcul}

\begin{frame}
  \frametitle{\rmfamily{\textsc{I - P}}rogramme de calcul}

  \begin{block}{\textcolor{blue}{Propriété}}

    \begin{multicols}{2}
      Un programme de calcul est un ensemble d'instructions et d'opérations à faire les uns à la suite des autres.
      \begin{itemize}
      \item Choisir un nombre.
      \item Multiplier par 2.
      \item Soustraire 3 au résultat.
      \item Afficher le résultat.
      \end{itemize}
    \end{multicols}
  \end{block}
  
  \begin{exampleblock}{Exercice}
    \begin{enumerate}
    \item[a)] Essayer avec les nombres suivants : $3$, $10$, $12$ et $-2$.
    \item[b)] On prend $x$ comme nombre de départ. Exécuter le programme de calcul. 
    \item[c)] Évaluer le programme avec $x=0$, $x=1$ , $x=2$ et $x=3$.
    \end{enumerate}

  \end{exampleblock}
\end{frame}

\begin{frame}
  \frametitle{\rmfamily{\textsc{I - P}}rogramme de calcul}
  \begin{block}{\textcolor{blue}{Propriété}}
    Pour trouver le résultat pour une certaine valeur de $x$, on remplace tous les $x$ présents par la valeurs souhaitée. On a \textbf{évalué} pour une certaine valeur de $x$.
  \end{block}
\end{frame}

\section{ \rmfamily{\textsc{II - R}}ègles d'écriture}

\begin{frame}
  \frametitle{ \rmfamily{\textsc{II - R}}ègles d'écriture}
  \begin{block}{\textcolor{blue}{Propriétés}}
    Le signe $\times$ de la multiplication n'est pas nécessairement écrit.
    \begin{itemize}
    \item $2 \times x          = 2x$
    \item $3 \times (2 + x)    = 3(2 + x)$
    \item $v \times t          = vt$
    \item $2 \times a \times 3 = 2 \times 3 \times a = 6 \times a = 6a$
    \end{itemize}
    Le 1 devant une lettre ne doit pas être écrit. Zéro fois une lettre est égale à 0. 
    \begin{itemize}
    \item $1 \times x          = 1x = x$
    \item $0 \times x          = 0x = 0$
    \end{itemize}
  \end{block}
  \end{frame}

\begin{frame}
  \frametitle{ \rmfamily{\textsc{II - R}}ègles d'écriture}
  \begin{block}{\textcolor{blue}{Propriétés}}
    On peut additionner (ou soustraire) des mêmes lettres.
    \begin{itemize}
    \item $2x + 3x - 10x = 5x - 10x = -5x$
    \end{itemize}
    On ne peut pas additionner ou soustraire une lettre et un nombre.
    \begin{itemize}
    \item $1 + 2x + 3 - 4x = 4 - 2x$
    \end{itemize}
    On ne peut pas additionner ou soustraire deux lettres différentes.
    \begin{itemize} 
    \item $1 + a + 2b + 3a +4b = 1 + 4a + 6b$
    \item $1 + x + x^2 + 2 + 3x + 4x^2 = 3 + 4x + 5x^2$
    \end{itemize}
  \end{block}  
\end{frame}

\section{ \rmfamily{\textsc{III - D}}istributivité}

\begin{frame}
  \frametitle{ \rmfamily{\textsc{III - D}}istributivité}
  \begin{alertblock}{Comment simplifier les parenthèses dans un calcul ?}
    $$k (a + b) = ka + kb$$
    $$k (a - b) = ka - kb$$ 
  \end{alertblock}
\end{frame}

\begin{frame}
  \frametitle{ \rmfamily{\textsc{III - D}}istributivité}

  \begin{exampleblock}{Exemple}
    \begin{enumerate}
    \item[a)] \textbf{Réduire} : $2(x + 3)$\\
      $ 2(x + 3) = 2 \times x + 2 \times 3$ \\
      $ 2(x + 3) = 2x + 6$\\

    \item[b)] \textbf{Vérifier} pour $x= 4$.\\
      $ 2(4 + 3) = 2 \times 7 = 14$\\
      $ 2(4 + 3) = 2 \times 4 + 2 \times 3 = 8 + 6 = 14$\\
    \end{enumerate}
  \end{exampleblock}

  \begin{figure}[H]
    \centering
    \includegraphics[width=0.6\linewidth]{sources/1/simple-distri.pdf}
  \end{figure}
\end{frame}

\end{document}

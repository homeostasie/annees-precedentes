\documentclass[11pt,a4paper]{article}
\usepackage[textheight=27cm,textwidth=19cm,headheight=10mm,footskip=10mm]{geometry}
\usepackage[UTF8]{inputenc}
\usepackage[T1]{fontenc}
\usepackage[french]{babel}
\usepackage{fourier}
\usepackage{amsmath,amsfonts,amssymb}
\usepackage{paralist}
\usepackage{pgf,tikz}
\usetikzlibrary{shapes,snakes,arrows}
\usepackage{graphicx,multicol}
\usepackage{varwidth}
\usepackage{array}
\usepackage{color,colortbl}
\usepackage{pstricks-add}

\pagestyle{empty}

\newcounter {exercice}
\newcommand{\exdeb}[1]{\par\addvspace{24pt}\noindent\stepcounter{exercice}\textbf{\underline{{Exercice }\theexercice\, :}\; (#1 points)} }
%fin de la commande \exdeb (entête d'exercice)

\newcommand{\Exdeb}[1]{\par\noindent\stepcounter{exercice}\textbf{\underline{{Exercice }\theexercice\,: }\; (#1 points)}}
%fin de la commande \exdeb (entête d'exercice 1)

\newcommand{\exfin}{\par\vspace{24 pt}}

\newcommand{\ent}[3]{
\fbox{\hspace*{2mm}\begin{minipage}{0.98\linewidth}
$\ $\vspace{-3pt}\\
\begin{Large}
Collège Frédéric MISTRAL \hfill #1 $\ $\\
\end{Large}
\begin{center}
\psset{xunit=1.0cm,yunit=1.0cm,algebraic=true,dotstyle=o,dotsize=3pt 0,linewidth=2pt,arrowsize=3pt 2,arrowinset=0.25}
\begin{pspicture*}(-8,-0.1)(8,6.1)
\rput(0,5){\scalebox{4}{\textbf{Classe de 4$^{\text{ème}}$ }}}
\rput(0,3){\scalebox{4}{\textbf{BREVET BLANC DE}}}
\rput(0,1){\scalebox{4}{\textbf{MATHEMATIQUES}}}
\end{pspicture*}
\end{center}
\end{minipage}} \vspace{12pt}
\begin{center}
\begin{Large}
L'usage de la calculatrice est autorisé ainsi que les instruments usuels de dessin. \vspace{12pt} \\
Le sujet comporte #3 pages numérotées 1/#3 à #3/#3
\end{Large}
\end{center} }
%fin de la commande \ent (entête du contrôle)

\begin{document}
% entête
\noindent \ent{ mai 2016}{1}{4} \vspace{24pt}
\begin{center}
\begin{Large}
\renewcommand{\arraystretch}{1.25}
\begin{tabular}{|l|c|}
\hline
Exercice n°1&5 points\\
\hline
Exercice n°2&4 points\\
\hline
Exercice n°3&7 points\\
\hline
Exercice n°4&3 points\\
\hline
Exercice n°5&4 points\\
\hline
Exercice n°6&4 points\\
\hline
Exercice n°7&3 points\\
\hline
Exercice n°8&6 points\\
\hline
Maîtrise de la langue&4 points\\
\hline
\end{tabular} 
\vspace{48pt} \\
Le sujet comporte 8 exercices indépendants qui peuvent être traités dans le désordre. \vspace{12pt} \\
\textbf{Toutes les réponses doivent être justifiées}, sauf si une indication contraire est donnée. \vspace{12pt} \\
Pour chaque question, si le travail n'est pas terminé, laisser tout de même une trace de la recherche, elle sera prise en compte au moment de la notation.
\end{Large}
\end{center}
$\ $\vspace{1.5cm}\\
\begin{flushright}
\textbf{1/4}
\end{flushright}
\exdeb{5} \vspace{6pt} \\% Exercice 1
\begin{minipage}{0.5\linewidth}
Supposons que la hauteur OB du volcan (de la base jusqu'au sommet) soit de 1800 m et que la nuée ardente dévale la pente AB à une vitesse de 4,5 km/min.
\begin{enumerate}[1{)}]
\item
Quelle est la longueur de la pente du volcan ?
\item
Transformer la vitesse en m/s.
\item
Combien de temps la nuée ardente va-t-elle mettre pour dévaler la pente ?\\
Exprimer le résultat en seconde.
\end{enumerate}
\end{minipage}
\begin{minipage}{0.05\linewidth}
$\ $
\end{minipage}
\begin{minipage}{0.45\linewidth}
\includegraphics[scale=0.5]{Brevet_4e_Volcan}
\end{minipage}
\exdeb{4} \vspace{3pt} \\% Exercice 2
\textit{Cet exercice est un questionnaire à choix multiples (QCM). Aucune justification n'est demandée.\\
Pour chacune des questions, une seule réponse est exacte. L'absence de réponse ou une réponse fausse ne retire aucun point.\\
Indiquer sur votre copie le numéro de la question et recopier la réponse exacte.\vspace{9 pt}\\}
\renewcommand{\arraystretch}{2}
\begin{tabular}{|c|p{10.5cm}|>{\centering\arraybackslash}p{20mm}|>{\centering\arraybackslash}p{20mm}|>{\centering\arraybackslash}p{20mm}|}
\cline{3-5}
\multicolumn{2}{c|}{$\ $}&\textbf{Réponse A}&\textbf{Réponse B}&\textbf{Réponse C}\\
\hline
1&Lorsqu'on regarde un angle de 18° à la loupe de grossissement 2, on voit un angle de :&9°&18°&36° \\
\hline
\rule[-2ex]{0pt}{6ex} 2&A quelle autre expression le nombre $\dfrac{7}{3}\ - \ \dfrac{4}{3}\ \div \ \dfrac{5}{2}$  est-il égal ?&$\dfrac{3}{3}\ \div \ \dfrac{5}{2}$&$\dfrac{7}{3}\ - \ \dfrac{3}{4}\ \times \ \dfrac{2}{5}$&$\dfrac{7}{3}\ - \ \dfrac{8}{15}$\\
\hline
3&On donne : 1To (téraoctet) = 10$^{12}$ octets et 1 Go (gigaoctet) = 10$^9$ octets. On partage un disque dur de 1,5 To en dossiers de 60 Go chacun. Le nombre de dossiers obtenus est égal à :&25&2,5 $\times$ 10$^{19}$&4 $\times$ 10$^{22}$\\
\hline
4&Quel nombre est en écriture scientifique ?&$17,3 \times 10^{-3}$&$0,97 \times 10^7$&$1,52 \times 10^3$\\
\hline
%5&&&&\\
%\hline
\end{tabular}
\exdeb{7} \vspace{3pt} \\% Exercice 3
Une commune souhaite aménager des parcours de santé sur son territoire.\\
On fait deux propositions au conseil municipal, schématisées ci-dessous : \vspace{3pt} \\
•	le parcours ACDA \vspace{3pt} \\
•	le parcours AEFA \vspace{3pt} \\
Ils souhaitent faire un parcours dont la longueur s'approche le plus possible de 4 km. \vspace{3pt} \\
Peux-tu les aider à choisir le parcours? Justifie. \vspace{3pt} \\
\textit{Attention : la figure proposée au conseil municipal n'est pas à l'échelle, mais les codages et les dimensions données sont correctes.}\\
\psset{xunit=1.0cm,yunit=1.0cm,algebraic=true,dotstyle=o,dotsize=3pt 0,linewidth=0.8pt,arrowsize=3pt 2,arrowinset=0.25}
\begin{pspicture*}(-3,-1.5)(13,4.5)
\psline(0,0)(0,3.5)(2,3.5)(0,0)(4,0.5)(4.5,-1)(0,0)
\psline(0,3.06)(0.44,3.06)(0.44,3.5)
\psline(2.5,0.31)(2.82,-0.63)
\rput[tl](0,-0.1){A} \rput[b](0,3.6){C} \rput[b](2,3.6){D} \rput[b](4,0.65){E} \rput[t](4.5,-1.15){F} \rput[b](2.5,0.46){E'} \rput[t](2.82,-0.78){F'}
\rput[l](7,1.5){\parbox{7 cm}{AC = 1,4 km \vspace{3pt} \\  CD = 1,05 km \vspace{3pt} \\ AE' = 0,5 km \vspace{3pt} \\ AE = 1,3 km \vspace{3pt} \\ AF = 1,6 km \vspace{3pt} \\ E'F'=0,4 km \vspace{3pt} \\ (EF) // (E'F') \vspace{3pt} \\ L'angle $\widehat{\text{A}}$ dans le triangle AEF vaut 30°.}}
\psline{->}(-1,-1)(-0.1,-0.1)
\rput(-1,-1.25){Départ et arrivée}
\end{pspicture*}
$\ $\vspace{-.5cm}\\
\begin{flushright}
\textbf{2/4}
\end{flushright}
\exdeb{3} \vspace{3pt} \\% Exercice 4
On laisse tomber une balle d'une hauteur de 1 mètre.\\
A chaque rebond elle rebondit des $\dfrac{3}{4}$ de la hauteur d'où elle est tombée.\\
Quelle hauteur atteint la balle au cinquième rebonds ? Arrondir au cm près.
\exdeb{4} \vspace{3pt} \\% Exercice 5
Cédric s'entraîne pour l'épreuve de vélo d'un triathlon.\\
La courbe ci-dessous représente la distance en kilomètres en fonction du temps écoulé en minutes.\\
\psset{xunit=1.0mm,yunit=2.0mm}
\begin{pspicture*}(-20,-5)(150,49)
% axe abscices
\multido{\i=10+10}{10}{\psline[linestyle=dashed,dash=3pt 3pt,linecolor=lightgray](\i ,0)(\i ,45)}
\psline[arrowscale=2.5]{->}(-1.5,0)(100,0)
\multido{\i=10+10}{9}{\psline(\i ,-0.75)(\i ,0.75) \rput[t](\i ,-1.5){\i}}
\rput[l](102,0){\textbf{Durée (min)}}
% axe ordonnées
\multido{\i=5+5}{9}{\psline[linestyle=dashed,dash=3pt 3pt,linecolor=lightgray](0,\i)(100,\i)}
\psline[arrowscale=2.5]{->}(0,-0.75)(0,45)
\multido{\i=10+10}{4}{\psline(-1.5,\i)(1.5,\i) \rput[r](-3,\i){\i}}
\rput[b](0,46){\textbf{Distance (km)}}
% Courbe
\psline[linewidth=1.2pt](0,0)(20,10)(30,20)(70,40)(90,45)
\end{pspicture*}\\
\textit{Aucune justification n'est attendue pour les trois premières questions, les réponses seront données grâce à des lectures \\graphiques.}
\begin{enumerate}[1{)}]
\item
Quelle distance Cédric a-t-il parcourue au bout de 20 minutes?
\item
Combien de temps a mis Cédric pour faire les 30 premiers kilomètres?
\item
Le circuit de Cédric comprend une montée, une descente et deux portions plates.\\
Reconstituer dans l'ordre le trajet parcouru par Cédric.
\item
Calculer la vitesse moyenne de Cédric (exprimée en km/h) sur la première des quatre parties du trajet.
\end{enumerate}
\exdeb{4} \vspace{3pt} \\% Exercice 6
Voici un programme de calcul sur lequel travaillent trois élèves. \vspace*{-3pt}
\begin{center}
\fbox{\hspace*{2mm}\begin{minipage}{5cm}
\begin{itemize}
\item[\textbullet]
Prendre un nombre.
\item[\textbullet]
Lui ajouter 8.
\item[\textbullet]
Multiplier le résultat par 3.
\item[\textbullet]
Enlever 24.
\item[\textbullet]
Enlever le nombre de départ.
\end{itemize}
\end{minipage}}
\end{center}
Voici ce qu'ils affirment : \\
Sophie : \og Quand je prends 4 comme nombre de départ, j'obtiens 8.\fg{} \vspace{3pt} \\
%Martin : \og En appliquant le programme à 0, je trouve 0. \fg{} \vspace{3pt} \\
Gabriel : \og Moi, j'ai pris -3 au départ et j'ai obtenu -9 \fg{} \vspace{3pt} \\
Faïza : \og Le résultat final est égal à $\dfrac{4}{3}$ si je choisis $\dfrac{2}{3}$ comme nombre de départ. \fg{} \vspace{9pt} \\
Pour chacun de ces quatre élèves expliquer s'il a raison ou s'il a tord.
$\ $\vspace{-1.25cm}\\
\begin{flushright}
\textbf{3/4}
\end{flushright}
\exdeb{3} \vspace{3pt} \\% Exercice 7
\begin{minipage}{0.75\linewidth}
Les alvéoles des nids d'abeilles présentent une ouverture ayant la forme d'un hexagone régulier de côté 3 mm environ.\\ 
Construire un agrandissement de cet hexagone de rapport 10. (aucune justification de la construction n'est attendue)\vspace{12pt}\\
\end{minipage}
\begin{minipage}{0.25\linewidth}
$\ $ \vspace{-36pt}\\
\psset{xunit=1.0cm,yunit=1.0cm,algebraic=true,dotstyle=o,dotsize=3pt 0,linewidth=0.8pt,arrowsize=3pt 2,arrowinset=0.25}
\begin{pspicture*}(-3,-1.6)(1.6,1.6)
\pspolygon(-1.3,0.75)(0,1.5)(1.3,0.75)(1.3,-0.75)(0,-1.5)(-1.3,-0.75)
\parametricplot{-2.6179938779914944}{-0.5235987755982987}{0.5*cos(t)+0|0.5*sin(t)+1.5}
\rput[bl](-0.2,0.65){120$\textrm{\degre}$}
\end{pspicture*}
\end{minipage}
\exdeb{6} \vspace{3pt} \\% Exercice 8
\textbf{Attention les figures tracées ne respectent ni les mesures de longueur, ni les mesures d'angle.} \vspace{6pt} \\
Répondre par \og vrai \fg{} ou \og faux \fg{} ou \og on ne peut pas savoir \fg{} à chacune des affirmations suivantes et \underline{expliquer votre choix}. \vspace{-9pt}
\begin{enumerate}[1{)}]
\item
Tout triangle inscrit dans un cercle est rectangle. \vspace{6pt} 
\item
Si un point M appartient à la médiatrice d'un segment [AB] alors le triangle AMB est isocèle. \vspace{6pt} 
\item
Dans le triangle ABC suivant, AB = 4 cm.\\
\psset{xunit=1.0cm,yunit=1.0cm,algebraic=true,dotstyle=o,dotsize=3pt 0,linewidth=0.8pt,arrowsize=3pt 2,arrowinset=0.25}
\begin{pspicture*}(-2,0)(4.5,3.5)
\pspolygon(1,1)(4,1)(1,3)
\psline(1.39,1)(1.39,1.39)(1,1.39)
\parametricplot{2.5535900500422257}{3.141592653589793}{0.55*cos(t)+4|0.55*sin(t)+1}
\rput[tr](0.9,0.9){A} \rput[tl](4.1,0.9){B} \rput[b](1,3.1){C}
\rput[r](3.4,1.2){60°} \rput[l](2.5,2.3){8 cm}
\end{pspicture*}
\item
Le quadrilatère ABCD ci-dessous est un carré.\\
\psset{xunit=1.0cm,yunit=1.0cm,algebraic=true,dotstyle=o,dotsize=3pt 0,linewidth=0.8pt,arrowsize=3pt 2,arrowinset=0.25}
\begin{pspicture*}(0,0)(5,5)
\pspolygon (1,1)(1,4)(3.98,4.28)(4.15,0.72)
\psline (1,3.6)(1.4,3.6)(1.4,4.05)
\psline(0.85,2.6)(1.15,2.4)
\psline(3.92,2.6)(4.22,2.4)
\psline(2.59,3.99)(2.39,4.29)
\psline(2.47,1.01)(2.67,0.71)
\rput[tr](0.95,0.95){A} \rput[br](0.95,4.05){B} \rput[bl](4.02,4.33){C} \rput[tl](4.2,0.67){D}
\end{pspicture*}
\end{enumerate}
$\ $\vspace{8.1cm}\\
\begin{flushright}
\textbf{4/4}
\end{flushright}

\end{document}


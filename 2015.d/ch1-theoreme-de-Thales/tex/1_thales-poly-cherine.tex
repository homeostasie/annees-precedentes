%%%%%%%%%%%%%%%%%%%%%%%%%%%%%%%%%%%%%%%%%
% Short Sectioned Assignment
% LaTeX Template
% Version 1.0 (5/5/12)
%
% This template has been downloaded from:
% http://www.LaTeXTemplates.com
%
% Original author:
% Frits Wenneker (http://www.howtotex.com)
%
% License:
% CC BY-NC-SA 3.0 (http://creativecommons.org/licenses/by-nc-sa/3.0/)
%
%%%%%%%%%%%%%%%%%%%%%%%%%%%%%%%%%%%%%%%%%

%----------------------------------------------------------------------------------------
%	PACKAGES AND OTHER DOCUMENT CONFIGURATIONS
%----------------------------------------------------------------------------------------

\documentclass[paper=a4, fontsize=12pt]{scrartcl} % A4 paper and 11pt font size


\usepackage[T1]{fontenc} % Use 8-bit encoding that has 256 glyphs
\usepackage[english,francais]{babel} % Français et anglais
\usepackage[utf8]{inputenc} 

\usepackage{amsmath,amsfonts,amsthm} % Math packages

\usepackage{enumitem}
\usepackage{lmodern}
\usepackage{url}
\usepackage{eurosym} % signe Euros
\usepackage{geometry} % Pour passer au format A4
\geometry{a4paper} % 
\usepackage{graphicx} % Required for including pictures
\usepackage{float} % Allows putting an [H] in \begin{figure} to specify the exact location of the figure

\usepackage{multicol}

\usepackage{verbatim}

\usepackage{sectsty} % Allows customizing section commands
\allsectionsfont{\centering \normalfont\scshape} % Make all sections centered, the default font and small caps

%----------------------------------------------------------------------------------------
%	Pied de Page
%----------------------------------------------------------------------------------------


\usepackage{fancyhdr} % Custom headers and footers
\pagestyle{fancyplain} % Makes all pages in the document conform to the custom headers and footers
\fancyhead{} % No page header - if you want one, create it in the same way as the footers below

\fancyfoot[L]{Cherine Afri} % Empty center footer
\fancyfoot[C]{Chapitre 1 - Théorème de Thalès} % Empty center footer
\fancyfoot[R]{\thepage} % Page numbering for right footer

\renewcommand{\headrulewidth}{0pt} % Remove header underlines
\renewcommand{\footrulewidth}{0pt} % Remove footer underlines

\usepackage{titling}
\setlength{\droptitle}{-2.5cm}

\setlength{\headheight}{13.6pt} % Customize the height of the header

\setlength\parindent{0pt} % Removes all indentation from paragraphs - comment this line for an assignment with lots of text


%----------------------------------------------------------------------------------------
%	Titre
%----------------------------------------------------------------------------------------

\newcommand{\horrule}[1]{\rule{\linewidth}{#1}} % Create horizontal rule command with 1 argument of height


\title{	
  \vspace{-10ex}
  \horrule{0.5pt} \\[0.4cm] % Thin top horizontal rule
  \huge Chapitre 1 - Théorème de Thalès \\ % The assignment title
  \horrule{2pt} \\[0.5cm] % Thick bottom horizontal rule
}

\author{}
\date{\vspace{-10ex}} % Today's date or a custom date

%----------------------------------------------------------------------------------------
%	Début du document
%----------------------------------------------------------------------------------------

\begin{document}

%----------------------------------------------------------------------------------------
% RE-DEFINITION
%----------------------------------------------------------------------------------------
% MATHS
%-----------

\newtheorem{Definition}{Définition}
\newtheorem{Theorem}{Théorème}
\newtheorem{Proposition}{Propriété}

% MATHS
%-----------
\renewcommand{\labelitemi}{$\bullet$}
\renewcommand{\labelitemii}{$\circ$}
%----------------------------------------------------------------------------------------
%	Titre
%----------------------------------------------------------------------------------------

\maketitle % Print the title
\setlength{\columnseprule}{1pt}

\section{Formulation}

\begin{multicols}{2}

\begin{Theorem}{Théorème de Thalès}

Soit $ABC$ un triangle avec : 
\begin{itemize}
\item $M$ un point appartenant au segment $[AB]$
\item $N \in [AC]$
\end{itemize}
\textbf{Si} les droites $(MN)$ et $(BC)$ sont \textbf{parallèles}, alors :
$$\dfrac{AM}{AB} = \dfrac{AN}{AC} = \dfrac{MN}{BC}$$

\end{Theorem}

\begin{figure}[H]
  \centering
  \includegraphics[width=0.8\linewidth]{sources/poly/1_triangle-thales.pdf}
\end{figure}

\end{multicols}

\section{Exploiter le théorème de Thalès}

\begin{Proposition}{
Le théorème de Thalès se révèle particulièrement efficace pour chercher des longueurs manquantes sur une figure.}

$$AM \times AC = AN \times AB$$

Toutes longueurs manquantes $AM$, $AC$, $AN$ où $AB$ peuvent être déduites à partir des trois autres.
\end{Proposition}


\begin{Proposition}{Le théorème de Thalès représente une situation proportionnelle.}
\end{Proposition}


\section{Appartées géométriques}

\subsection{Reconnaître des droites parallèles}

\begin{multicols}{3}

 \begin{Proposition}{$d_1$ et $d_2$ sont parallèles, alors :}\\
\begin{itemize}
    \item Les angles correspondants sont égaux;
    \item Les angles alternes-internes sont égaux.
\end{itemize}
  \end{Proposition}

\begin{figure}[H]
  \centering
  \includegraphics[width=0.8\linewidth]{sources/poly/3_angle_ai.pdf}
\end{figure}


\begin{figure}[H]
  \centering
  \includegraphics[width=0.8\linewidth]{sources/poly/4_angle_cor.pdf}
\end{figure}

\end{multicols}

\end{document}

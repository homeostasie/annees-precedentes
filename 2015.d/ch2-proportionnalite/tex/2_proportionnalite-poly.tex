%%%%%%%%%%%%%%%%%%%%%%%%%%%%%%%%%%%%%%%%%
% Short Sectioned Assignment
% LaTeX Template
% Version 1.0 (5/5/12)
%
% This template has been downloaded from:
% http://www.LaTeXTemplates.com
%
% Original author:
% Frits Wenneker (http://www.howtotex.com)
%
% License:
% CC BY-NC-SA 3.0 (http://creativecommons.org/licenses/by-nc-sa/3.0/)
%
%%%%%%%%%%%%%%%%%%%%%%%%%%%%%%%%%%%%%%%%%

%----------------------------------------------------------------------------------------
%	PACKAGES AND OTHER DOCUMENT CONFIGURATIONS
%----------------------------------------------------------------------------------------

\documentclass[paper=a4, fontsize=9pt]{scrartcl} % A4 paper and 11pt font size


\usepackage[T1]{fontenc} % Use 8-bit encoding that has 256 glyphs
\usepackage[english,francais]{babel} % Français et anglais
\usepackage[utf8]{inputenc} 

\usepackage{amsmath,amsfonts,amsthm} % Math packages

\usepackage{enumitem}
\usepackage{lmodern}
\usepackage{url}
\usepackage{eurosym} % signe Euros
\usepackage{geometry} % Pour passer au format A4
\geometry{a4paper} % 
\usepackage{graphicx} % Required for including pictures
\usepackage{float} % Allows putting an [H] in \begin{figure} to specify the exact location of the figure

\usepackage{multicol}

\usepackage{verbatim}

\usepackage{sectsty} % Allows customizing section commands
\allsectionsfont{\centering \normalfont\scshape} % Make all sections centered, the default font and small caps

%----------------------------------------------------------------------------------------
%	Pied de Page
%----------------------------------------------------------------------------------------


\usepackage{fancyhdr} % Custom headers and footers
\pagestyle{fancyplain} % Makes all pages in the document conform to the custom headers and footers
\fancyhead{} % No page header - if you want one, create it in the same way as the footers below
\fancyfoot[C]{Chapitre 2 - Proportionnalite} % Empty center footer
\fancyfoot[R]{\thepage} % Page numbering for right footer

\renewcommand{\headrulewidth}{0pt} % Remove header underlines
\renewcommand{\footrulewidth}{0pt} % Remove footer underlines

%\usepackage{titling}
%\setlength{\droptitle}{-1cm}
%\setlength{\headheight}{13.6pt} % Customize the height of the header

\setlength\parindent{0pt} % Removes all indentation from paragraphs - comment this line for an assignment with lots of text


%----------------------------------------------------------------------------------------
%	Titre
%----------------------------------------------------------------------------------------

\newcommand{\horrule}[1]{\rule{\linewidth}{#1}} % Create horizontal rule command with 1 argument of height

\title{
  \vspace{-10ex}
  \horrule{0.5pt} % Thin top horizontal rule
  \huge Chapitre 2 - Proportionnalité\\
  \horrule{2pt}
}

\author{}
\date{\vspace{-10ex}} % Today's date or a custom date

%----------------------------------------------------------------------------------------
%	Début du document
%----------------------------------------------------------------------------------------
\begin{document}

%----------------------------------------------------------------------------------------
% RE-DEFINITION
%----------------------------------------------------------------------------------------
% MATHS
%-----------

\newtheorem{Definition}{Définition}
\newtheorem{Theorem}{Théorème}
\newtheorem{Proposition}{Propriété}

% MATHS
%-----------
\renewcommand{\labelitemi}{$\bullet$}
\renewcommand{\labelitemii}{$\circ$}
%----------------------------------------------------------------------------------------
%	Titre
%----------------------------------------------------------------------------------------

\maketitle % Print the title
\setlength{\columnseprule}{1pt}

\section{Quatrième Proportionnelle}

\begin{Proposition}{Tableau de proportionnalité}

  \begin{multicols}{2}

    % ajouter le résultat à droite.
    \begin{center}
      \begin{tabular}{| c || c | c |}
        \hline
        Volume (en L) & 5 & 15  \\
        \hline
        Masse (en kg) & 4 & $x$ \\ 
        \hline
      \end{tabular}
    \end{center}

    Pour calculer la quatrième proportionnelle $x$ de ce tableau de proportionnalité, on peut utiliser :

    \begin{itemize}
    \item L'égalité des produits en croix : 
      $$ 5 \times x = 4 \times 15 $$

    \item Un coefficient de proportionnalité : \textit{Comment passer de la première ligne à la deuxième ligne.}

      % ajouter le coef à droite.
      \begin{center}
        \begin{tabular}{| c || c | c |}
          \hline
          Volume (en L) & 5 & 15  \\
          \hline
          Masse (en kg) & 4 & $x$ \\ 
          \hline
        \end{tabular}
      \end{center}

    \item Un rapport de linéarité : \textit{Comment passer de la première colonne à la deuxième colonne.}

      % ajouter le coeff en haut
      \begin{center}
        \begin{tabular}{| c || c | c |}
          \hline
          Volume (en L) & 5 & 15  \\
          \hline
          Masse (en kg) & 4 & $x$ \\ 
          \hline
        \end{tabular}
      \end{center}
    \end{itemize}
  \end{multicols}
\end{Proposition}

\section{Représentation graphique}

\begin{Proposition}
  Dans un repère une situation de proportionnalité est représentée graphiquement par des points alignés avec l'origine du repère. On peut également parler de \textbf{droite passant par l'origine}.
\end{Proposition}

\begin{figure}[H]
  \centering
  \includegraphics[width=0.8\linewidth]{sources/cours/2_prop-repre.pdf}
\end{figure}


\section{Pourcentage}

\subsection{Appliquer un pourcentage}

\subsection{Calculer un pourcentage}

\section{Échelle}

\section{Vitesse Moyenne}
\end{document}

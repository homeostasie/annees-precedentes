%%%%%%%%%%%%%%%%%%%%%%%%%%%%%%%%%%%%%%%%%
% LaTeX Template
% http://www.LaTeXTemplates.com
%
% Original author:
% Linux and Unix Users Group at Virginia Tech Wiki 
% (https://vtluug.org/wiki/Example_LaTeX_chem_lab_report)
%
% License:
% CC BY-NC-SA 3.0 (http://creativecommons.org/licenses/by-nc-sa/3.0/)
%
%%%%%%%%%%%%%%%%%%%%%%%%%%%%%%%%%%%%%%%%%

%----------------------------------------------------------------------------------------
%	PACKAGES AND DOCUMENT CONFIGURATIONS
%----------------------------------------------------------------------------------------

\documentclass[13pt]{article}
\usepackage{geometry} % Pour passer au format A4
\geometry{hmargin=0.5cm, vmargin=0.5cm} % 

\usepackage{graphicx} % Required for including pictures
\usepackage{float} % 

%Français
\usepackage[T1]{fontenc} 
\usepackage[english,francais]{babel}
\usepackage[utf8]{inputenc}
\usepackage{eurosym}
\usepackage{lmodern}
\usepackage{url}
\usepackage{multicol}

%Maths
\usepackage{amsmath,amsfonts,amssymb,amsthm}
%\usepackage[linesnumbered, ruled, vlined]{algorithm2e}
%\SetAlFnt{\small\sffamily}

%Autres
\linespread{1} % Line spacing
\setlength\parindent{0pt} % Removes all indentation from paragraphs

\renewcommand{\labelenumi}{\alph{enumi}.} % 
\pagestyle{empty}
%----------------------------------------------------------------------------------------
%	DOCUMENT INFORMATION
%----------------------------------------------------------------------------------------
\begin{document}

%\maketitle % Insert the title, author and date

\setlength{\columnseprule}{1pt}

\begin{multicols}{2}

  \textit{Éffectuer les calculs suivants :}

  \begin{enumerate}
  \item $2 + 9 \times 3 - 1 = \phantom{\dfrac{1}{1}}$ 
  \item $2 \times (9 - 2) = \phantom{\dfrac{1}{1}} $
  \item $\dfrac{6 + 8}{2} = $
  \item $3 \times \dfrac{8}{2} = $
  \end{enumerate}

  \textit{Éffectuer les calculs suivants :}

  \begin{enumerate}
  \item $9 \times \dfrac{2}{18} = $ 
  \item $5 \times (-9 + 12) = \phantom{\dfrac{1}{1}}$
  \item $\dfrac{-6 - 8}{2} = $
  \item $10 + \dfrac{10}{2} = $
  \end{enumerate}

\end{multicols}

\vspace{0.3cm}
\noindent\hrulefill
\vspace{0.3cm}

\begin{multicols}{2}

  \textit{Éffectuer les calculs suivants :}

  \begin{enumerate}
  \item $\dfrac{1}{2} times \dfrac{6}{4} = \phantom{\dfrac{1}{1}}$ 
  \item $\dfrac{3}{4} times \dfrac{10}{4} = \phantom{\dfrac{1}{1}}$ 
  \item $2 \dfrac{12}{5} = \phantom{\dfrac{1}{1}}$ 
  \end{enumerate}

  \textit{Éffectuer les calculs suivants :}

  \begin{enumerate}
  \item $2 + 9 \times 3 - 1 = \phantom{\dfrac{1}{1}}$ 
  \item $2 \times (9 - 2) = \phantom{\dfrac{1}{1}}$
  \item $\dfrac{6 + 8}{2} = $
  \item $3 \times \dfrac{8}{2} = $
  \end{enumerate}

\end{multicols}

\vspace{0.3cm}
\noindent\hrulefill
\vspace{0.3cm}

\begin{multicols}{2}

  \textit{Éffectuer les calculs suivants :}

  \begin{enumerate}
  \item $2 + 9 \times 3 - 1 = \phantom{\dfrac{1}{1}}$ 
  \item $2 \times (9 - 2) = \phantom{\dfrac{1}{1}} $
  \item $\dfrac{6 + 8}{2} = $
  \item $3 \times \dfrac{8}{2} = $
  \end{enumerate}

  \textit{Éffectuer les calculs suivants :}

  \begin{enumerate}
  \item $2 + 9 \times 3 - 1 = \phantom{\dfrac{1}{1}}$ 
  \item $2 \times (9 - 2) = \phantom{\dfrac{1}{1}}$
  \item $\dfrac{6 + 8}{2} = $
  \item $3 \times \dfrac{8}{2} = $
  \end{enumerate}

\end{multicols}

\vspace{0.3cm}
\noindent\hrulefill
\vspace{0.3cm}

\begin{multicols}{2}

  \textit{Éffectuer les calculs suivants :}

  \begin{enumerate}
  \item $2 + 9 \times 3 - 1 = \phantom{\dfrac{1}{1}}$ 
  \item $2 \times (9 - 2) = \phantom{\dfrac{1}{1}} $
  \item $\dfrac{6 + 8}{2} = $
  \item $3 \times \dfrac{8}{2} = $
  \end{enumerate}

  \textit{Éffectuer les calculs suivants :}

  \begin{enumerate}
  \item $2 + 9 \times 3 - 1 = \phantom{\dfrac{1}{1}}$ 
  \item $2 \times (9 - 2) = \phantom{\dfrac{1}{1}}$
  \item $\dfrac{6 + 8}{2} = $
  \item $3 \times \dfrac{8}{2} = $
  \end{enumerate}

\end{multicols}

\vspace{0.3cm}
\noindent\hrulefill
\vspace{0.3cm}


\begin{multicols}{2}

  \textit{Éffectuer les calculs suivants :}

  \begin{enumerate}
  \item $2 + 9 \times 3 - 1 = \phantom{\dfrac{1}{1}}$ 
  \item $2 \times (9 - 2) = \phantom{\dfrac{1}{1}} $
  \item $\dfrac{6 + 8}{2} = $
  \item $3 \times \dfrac{8}{2} = $
  \end{enumerate}

  \textit{Éffectuer les calculs suivants :}

  \begin{enumerate}
  \item $2 + 9 \times 3 - 1 = \phantom{\dfrac{1}{1}}$ 
  \item $2 \times (9 - 2) = \phantom{\dfrac{1}{1}}$
  \item $\dfrac{6 + 8}{2} = $
  \item $3 \times \dfrac{8}{2} = $
  \end{enumerate}

\end{multicols}


\end{document}

%----------------------------------------------------------------------------------------
%	PACKAGES AND DOCUMENT CONFIGURATIONS
%----------------------------------------------------------------------------------------

\documentclass[10pt]{article}
\usepackage{geometry} % Pour passer au format A4
\geometry{hmargin=1cm, vmargin=0.8cm} %

\usepackage{graphicx} % Required for including pictures
\usepackage{float} %

%Français
\usepackage[T1]{fontenc}
\usepackage[english,francais]{babel}
\usepackage[utf8]{inputenc}
\usepackage{eurosym}
\usepackage{lmodern}
\usepackage{url}
\usepackage{multicol,multido}

%Maths
\usepackage{amsmath,amsfonts,amssymb,amsthm,gensymb}
\usepackage{siunitx}

%Autres
\linespread{1} % Line spacing
\setlength\parindent{0pt} % Removes all indentation from paragraphs

\renewcommand{\labelenumi}{\alph{enumi}.} %
\pagestyle{empty}
\newcommand{\horrule}[1]{\rule{\linewidth}{#1}} % Create horizontal rule command with 1 argument of height
\newcommand{\Pointille}[1][3]{\multido{}{#1}{ \makebox[\linewidth]{\dotfill}\\[\parskip]}}

%----------------------------------------------------------------------------------------
%	DOCUMENT INFORMATION
%----------------------------------------------------------------------------------------
\begin{document}

%\maketitle % Insert the title, author and date
\setlength{\columnseprule}{1pt}

\begin{multicols}{3}
  $2x + 3 = 0$ \\
  \vspace{2cm} \\

  $5x + 15 = 0$ \\
  \vspace{2cm}\\

  $7x -  4= 0$ \\
  \vspace{2cm}\\
\end{multicols}

\horrule{1px}

\begin{multicols}{3}
  $12x + 3 = 13$ \\
  \vspace{2cm} \\

  $6x + 15 = 5$ \\
  \vspace{2cm}\\

  $12x -  4 = 4$ \\
  \vspace{2cm}\\
\end{multicols}

\horrule{1px}

\begin{multicols}{3}
  $12x + 3 = -13$ \\
  \vspace{2cm} \\

  $6x + 15 = -5$ \\
  \vspace{2cm}\\

  $12x -  4 = -4$ \\
  \vspace{2cm}\\
\end{multicols}


\horrule{1px}

\begin{multicols}{3}
  $12x + 3 = 2x$ \\
  \vspace{2cm} \\

  $6x - 5 = 5x$ \\
  \vspace{2cm}\\

  $2x +  4 = 8x$ \\
  \vspace{2cm}\\
\end{multicols}

\horrule{1px}

\begin{multicols}{3}
  $12x + 3 = 2x$ \\
  \vspace{2cm} \\

  $6x - 5 = 5x$ \\
  \vspace{2cm}\\

  $22x + 3 = -2x + 20$ \\
  \vspace{2cm}\\
\end{multicols}

\horrule{1px}

\begin{multicols}{3}
  $6x - 5 = 15x - 11$ \\
  \vspace{1cm}\\
  $2x +  4 = 2x + 8$ \\
  \vspace{1cm}\\
  $x +  1 = x + 1$\\



\end{multicols}

\end{document}

%%%%%%%%%%%%%%%%%%%%%%%%%%%%%%%%%%%%%%%%%
% LaTeX Template
% http://www.LaTeXTemplates.com
%
% Original author:
% Linux and Unix Users Group at Virginia Tech Wiki 
% (https://vtluug.org/wiki/Example_LaTeX_chem_lab_report)
%
% License:
% CC BY-NC-SA 3.0 (http://creativecommons.org/licenses/by-nc-sa/3.0/)
%
%%%%%%%%%%%%%%%%%%%%%%%%%%%%%%%%%%%%%%%%%

%----------------------------------------------------------------------------------------
%	PACKAGES AND DOCUMENT CONFIGURATIONS
%----------------------------------------------------------------------------------------

\documentclass[12pt]{article}
\usepackage{geometry} % Pour passer au format A4
\geometry{hmargin=1cm, vmargin=1cm} % 

\usepackage{graphicx} % Required for including pictures
\usepackage{float} % 

%Français
\usepackage[T1]{fontenc} 
\usepackage[english,francais]{babel}
\usepackage[utf8]{inputenc}
\usepackage{eurosym}
\usepackage{lmodern}
\usepackage{url}
\usepackage{multicol}

%Maths
\usepackage{amsmath,amsfonts,amssymb,amsthm}
%\usepackage[linesnumbered, ruled, vlined]{algorithm2e}
%\SetAlFnt{\small\sffamily}

%Autres
\linespread{1} % Line spacing
\setlength\parindent{0pt} % Removes all indentation from paragraphs

\renewcommand{\labelenumi}{\alph{enumi}.} % 
\pagestyle{empty}
%----------------------------------------------------------------------------------------
%	DOCUMENT INFORMATION
%----------------------------------------------------------------------------------------
\begin{document}

%\maketitle % Insert the title, author and date

\textbf{Nom, Prénom :} \hspace{8cm} \textbf{Classe :} \hspace{3cm} \textbf{Date :}\\
\textbf{Calculatrice :}

\begin{center}
  \textit{On ne craint que ce que l'on ne connaît pas.}  - \textbf{Marie Curie}
\end{center}


\textbf{L'exercice 1 est à faire sur l'énoncé, les exercices 2 et 3 sont à faire sur une copie correctement présentée.}

% Exercice 1 
\subsection*{Exercice 1} 

\textbf{Corriger la copie de Léa }. Il faut corriger sur le côté, chaque réponse fausse. Chaque réponse juste vaut un point. \textbf{Quelle est sa note ?}

\begin{multicols}{3}

  \begin{eqnarray*}
    A &=& 6 - 4 \times 3 \\
    A &=& 2 \times 3     \\
    A &=& 6 
  \end{eqnarray*}

\vspace{-0.5cm}

  \begin{eqnarray*}
    B &=& 17 - 16 + 4  \\
    B &=& 17 - 20      \\
    B &=& -3
  \end{eqnarray*}

\vspace{-0.5cm}

  \begin{eqnarray*}
    C &=& -8 \times -2 + 4 \\
    C &=& 16 + 4           \\
    C &=& 20
  \end{eqnarray*}

\vspace{-0.5cm}

  \begin{eqnarray*}
    D &=& 6 \times (-5 - 3) \\
    D &=& 6 \times -8       \\
    D &=& -48
  \end{eqnarray*}

\vspace{-0.5cm}

  \begin{eqnarray*}
    E &=& -9 - (-3) + 9 \\
    E &=& -12 + 9       \\
    E &=& -5
  \end{eqnarray*}

\vspace{-0.5cm}

  \begin{eqnarray*}
    F &=& 6 - 3 \times -2 \\
    F &=& 6 - 6           \\
    F &=& 0
  \end{eqnarray*}

\vspace{-0.5cm}

  \begin{eqnarray*}
    G &=& -4 \times 2 + -5 \times -4 \\
    G &=& -8 + 20                    \\
    G &=& 12 
  \end{eqnarray*}

\vspace{-0.5cm}

  \begin{eqnarray*}
    H &=& 5 \times (3 + (-7))        \\
    H &=& 5 \times 3 + 5 \times (-7) \\
    H &=& 15 - 35  \\
    H &=& -20  
  \end{eqnarray*}

\vspace{-0.5cm}

  \begin{eqnarray*}
    I &=& 18 - (-2 - 4) \\
    I &=& 18 - 6       \\
    I &=& 12
  \end{eqnarray*}

\end{multicols}

\subsection*{Exercice 2}


\begin{enumerate}
\item[1.] Auguste est adopté par Jules César en $-45$ alors qu'il a 18 ans. Plus tard Auguste succèdera à Jules César. \textbf{En quelle année est né Auguste ?}

\item[2.] Julia est né en -39 et morte en 14. Elle est la fille d'Auguste. Elle se marie en -25 avec Marcellus (né en -41), puis en -22 avec Agrippa (né en -63), puis en -11 avec Tibère (né en -42).

  \begin{enumerate}
  \item[2a.] Quel est l'âge de Julia au moment de ses différents mariages ?
  \item[2b.] Quel est l'âge de chacun de ses maris au moment du mariage ?
  \item[2c.] À quel âge Julia est-elle décédée ?
  \end{enumerate}
\end{enumerate}



\subsection*{Exercice 3}

Un jeu-concours de mathématiques comporte 20 questions à choix multiples. Le classement des concurrents est déterminé selon le barème suivant : \\
\textbf{Bonne réponse : $5$ points, Réponse fausse : $-3$ points, Absence de réponse : $-2$ points}


\begin{itemize}
\item Elvis n’a répondu qu’à 12 questions et 10 de ses réponses sont correctes.
\item Léa a répondu à toutes les questions ; 7 de ses réponses sont fausses.
\item Il n’y a que 5 questions auxquelles Mehdi n’a pas répondu, et il a fait 4 erreurs.
\end{itemize}

\textbf{Quel est le classement de Léa, Elvis et Mehdi ?}\\

\textit{bonus.} Dominique a obtenu un score nul. Peut-on trouver combien de réponses correctes
il a obtenues ?



\end{document}

%%%%%%%%%%%%%%%%%%%%%%%%%%%%%%%%%%%%%%%%%
% LaTeX Template
% http://www.LaTeXTemplates.com
%
% Original author:
% Linux and Unix Users Group at Virginia Tech Wiki 
% (https://vtluug.org/wiki/Example_LaTeX_chem_lab_report)
%
% License:
% CC BY-NC-SA 3.0 (http://creativecommons.org/licenses/by-nc-sa/3.0/)
%
%%%%%%%%%%%%%%%%%%%%%%%%%%%%%%%%%%%%%%%%%

%----------------------------------------------------------------------------------------
%	PACKAGES AND DOCUMENT CONFIGURATIONS
%----------------------------------------------------------------------------------------

\documentclass[12pt]{article}
\usepackage{geometry} % Pour passer au format A4
\geometry{hmargin=1cm, vmargin=1cm} % 

\usepackage{graphicx} % Required for including pictures
\usepackage{float} % 

%Français
\usepackage[T1]{fontenc} 
\usepackage[english,francais]{babel}
\usepackage[utf8]{inputenc}
\usepackage{eurosym}
\usepackage{lmodern}
\usepackage{url}
\usepackage{multicol}

%Maths
\usepackage{amsmath,amsfonts,amssymb,amsthm}
%\usepackage[linesnumbered, ruled, vlined]{algorithm2e}
%\SetAlFnt{\small\sffamily}

%Autres
\linespread{1} % Line spacing
\setlength\parindent{0pt} % Removes all indentation from paragraphs
\newcommand{\horrule}[1]{\rule{\linewidth}{#1}} 

\renewcommand{\labelenumi}{\alph{enumi}.} % 
\pagestyle{empty}
%----------------------------------------------------------------------------------------
%	DOCUMENT INFORMATION
%----------------------------------------------------------------------------------------
\begin{document}

%\maketitle % Insert the title, author and date

\textbf{Nom, Prénom :} \hspace{8cm} \textbf{Classe :} \hspace{3cm} \textbf{Date :}
\begin{center}
  \textit{La réalité, c'est ce qui refuse de disparaître quand on cesse d'y croire.}  - \textbf{ - Philip K. Dick}
\end{center}
\textit{La calculatrice n'est pas autorisée. Les étapes de calcul sont à écrire. Le soin apporté à la rédaction est évalué.}

\begin{multicols}{3}

% Exercice 1 
\subsection*{Exercice 1} 

\begin{enumerate}
\item[1.] $-9 - 9 + 17 $
\item[2.] $-2 + 8 +4   $
\item[3.] $4 + 6 -(-12)$
\item[4.] $-2 + 11 - 13$
\item[4.] $-7 - 3 - 10 $
\item[6.] $4 - 12 - 4  $
\end{enumerate}

\subsection*{Exercice 2}


\begin{enumerate}
\item[1.] $\dfrac{1}{2} + \dfrac{5}{2}$
\item[2.] 2 + $\dfrac{5}{7}$
\item[3.] $\dfrac{-3}{4} - \dfrac{2}{4}$
\item[4.] $\dfrac{3}{15} + \dfrac{10}{5}$
\end{enumerate}



\subsection*{Exercice 3}

\begin{enumerate}
\item[1.] $\dfrac{1}{2} + \dfrac{2}{3}$
\item[2.] $\dfrac{-2}{3} + \dfrac{-4}{5} + \dfrac{10}{15}$
\item[3.] $\dfrac{3}{4} - \dfrac{-6}{7} + \dfrac{16}{7}$
\item[4.] $\dfrac{4}{-5} + \dfrac{8}{9} + 0$
\end{enumerate}

\end{multicols}

\vspace{0.4cm}
\horrule{1px}
\vspace{0.4cm}

\textbf{Nom, Prénom :} \hspace{8cm} \textbf{Classe :} \hspace{3cm} \textbf{Date :}
\begin{center}
  \textit{La réalité, c'est ce qui refuse de disparaître quand on cesse d'y croire.}  - \textbf{ - Philip K. Dick}
\end{center}
\textit{La calculatrice n'est pas autorisée. Les étapes de calcul sont à écrire. Le soin apporté à la rédaction est évalué.}

\begin{multicols}{3}

% Exercice 1 
\subsection*{Exercice 1} 

\begin{enumerate}
\item[1.] $-9 - 9 + 17 $
\item[2.] $-2 + 8 + 6  $
\item[3.] $4 + 8 -(-12)$
\item[4.] $-2 + 12 - 13$
\item[4.] $-7 - 3 - 20 $
\item[6.] $4 - 14 - 4  $
\end{enumerate}

\subsection*{Exercice 2}


\begin{enumerate}
\item[1.] $\dfrac{1}{3} + \dfrac{5}{3}$
\item[2.] 2 + $\dfrac{5}{9}$
\item[3.] $\dfrac{-5}{4} - \dfrac{2}{4}$
\item[4.] $\dfrac{3}{5} + \dfrac{10}{15}$
\end{enumerate}



\subsection*{Exercice 3}

\begin{enumerate}
\item[1.] $\dfrac{1}{2} + \dfrac{2}{3}$
\item[2.] $\dfrac{-2}{3} + \dfrac{-4}{5} + \dfrac{12}{15}$
\item[3.] $\dfrac{3}{4} - \dfrac{-6}{9} + \dfrac{16}{9}$
\item[4.] $\dfrac{4}{-7} + \dfrac{8}{9} + 0$
\end{enumerate}

\end{multicols}

\vspace{0.4cm}
\horrule{1px}
\vspace{0.4cm}

\textbf{Nom, Prénom :} \hspace{8cm} \textbf{Classe :} \hspace{3cm} \textbf{Date :}
\begin{center}
  \textit{La réalité, c'est ce qui refuse de disparaître quand on cesse d'y croire.}  - \textbf{ - Philip K. Dick}
\end{center}
\textit{La calculatrice n'est pas autorisée. Les étapes de calcul sont à écrire. Le soin apporté à la rédaction est évalué.}

\begin{multicols}{3}

% Exercice 1 
\subsection*{Exercice 1} 

\begin{enumerate}
\item[1.] $-9 - 9 + 17 $
\item[2.] $-2 + 6 + 8  $
\item[3.] $4 + 9 -(-12)$
\item[4.] $-2 + 12 - 11$
\item[4.] $-6 - 4 - 10 $
\item[6.] $4 - 16 - 4  $
\end{enumerate}

\subsection*{Exercice 2}


\begin{enumerate}
\item[1.] $\dfrac{1}{6} + \dfrac{5}{6}$
\item[2.] 2 + $\dfrac{3}{5}$
\item[3.] $\dfrac{-7}{4} - \dfrac{2}{4}$
\item[4.] $\dfrac{3}{4} + \dfrac{10}{12}$
\end{enumerate}



\subsection*{Exercice 3}

\begin{enumerate}
\item[1.] $\dfrac{1}{2} + \dfrac{2}{3}$
\item[2.] $\dfrac{-2}{3} + \dfrac{-4}{5} + \dfrac{8}{15}$
\item[3.] $\dfrac{3}{4} - \dfrac{-6}{5} + \dfrac{16}{5}$
\item[4.] $\dfrac{4}{-3} + \dfrac{8}{9} + 0$
\end{enumerate}

\end{multicols}
\end{document}

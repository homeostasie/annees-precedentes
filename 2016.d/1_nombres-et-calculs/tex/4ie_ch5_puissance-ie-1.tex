%----------------------------------------------------------------------------------------
%	PACKAGES AND DOCUMENT CONFIGURATIONS
%----------------------------------------------------------------------------------------

\documentclass[10pt]{article}
\usepackage{geometry} % Pour passer au format A4
\geometry{hmargin=1cm, vmargin=1cm} % 

\usepackage{graphicx} % Required for including pictures
\usepackage{float} % 

%Français
\usepackage[T1]{fontenc} 
\usepackage[english,francais]{babel}
\usepackage[utf8]{inputenc}
\usepackage{eurosym}
\usepackage{lmodern}
\usepackage{url}
\usepackage{multicol,multido}

%Maths
\usepackage{amsmath,amsfonts,amssymb,amsthm,gensymb}
\usepackage{siunitx}

%Autres
\linespread{1} % Line spacing
\setlength\parindent{0pt} % Removes all indentation from paragraphs

\renewcommand{\labelenumi}{\alph{enumi}.} % 
\pagestyle{empty}
\newcommand{\horrule}[1]{\rule{\linewidth}{#1}} % Create horizontal rule command with 1 argument of height
\newcommand{\Pointille}[1][3]{\multido{}{#1}{ \makebox[\linewidth]{\dotfill}\\[\parskip]}}

%----------------------------------------------------------------------------------------
%	DOCUMENT INFORMATION
%----------------------------------------------------------------------------------------
\begin{document}

%\maketitle % Insert the title, author and date

\textbf{Nom, Prénom :} \hspace{8cm} \textbf{Classe :} \hspace{3cm} \textbf{Date :}\\
\textbf{La calculatrice.}

\begin{center}
  \textit{La plus coûteuse des dépenses, c’est la perte de temps.}  - \textbf{Théophraste}
\end{center}

% Exercice 1 
\subsection*{Calcul 1 - Calculer et donner sous forme décimale} 
\begin{eqnarray*}
	4^4                  &=& \\
	5.1 \times 10^4      &=& \\
	2.1 \times 10^{-5}   &=& \\
	(-5)^6               &=& \\
	-2^10                &=& \\
	124^3 \times 10^{-3} &=& \\
	12^0                 &=& 
\end{eqnarray*}
% Exercice 1 
\subsection*{Calcul 2 - Mettre sous forme scientifique} 
\begin{eqnarray*}
	20 000 000  &=& \\
	0.0150      &=& \\
	0.63        &=& \\
	0.004       &=& \\
	130         &=& \\
	158 000     &=& \\
	0.1         &=& \\
	0.000 178   &=& \\
	107 000 000 &=& \\
	40 001.5    &=& \\
\end{eqnarray*}

% Exercice 1 
\subsection*{Restituer}
Donner la définition d'un nombre sous forme scientifique. 

% Exercice 1 
\subsection*{Exercice 1 - Terre Soleil} 

La lumière se propage à une vitesse de $3 \times 10^8 m/s$. Un rayon partant du Soleil arrive sur Terre au bout de 8 min 30s. \\
Quelle est la distance Terre - Soleil ? \\
On donnera l'écriture scientifique en m puis l'écriture décimale en km.


% Exercice 1 
\subsection*{Exercice 2 - Porte-avions} 

Un porte-avions coûte environ 3 miliards d'euros. \\
Quelle hauteur atteindra une pile de billets de 50 \euro \: représentant cette somme.

\begin{itemize}
\item Un billet de 50 \euro \: a une épaisseurs de $\SI{80}{\micro\meter}$. 
\item $\SI{1}{\micro \meter} = 10^{-6}m$.
\end{itemize}

% Exercice 1 
\subsection*{Exercice 3 - Rebonds}

On laisse tomber une balle du haut d'un immeuble de 12 mètres.\\
À chaque rebond, elle rebondit des $\dfrac{2}{3}$ de la hauteur d'où elle est tombée.\\
\begin{enumerate}
\item[1.] Faire un schéma en prenant pour échelle 1m <=> 1cm.

\item[2.] Quelle hauteur atteint la balle au cinquième rebonds ?

\item[3.] Quelle hauteur atteint la balle au centième rebonds ? On donnera le résultat sous forme exacte puis sous forme scientifique.
\end{enumerate}

\end{document}

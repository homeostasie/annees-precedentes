%----------------------------------------------------------------------------------------
%	PACKAGES AND DOCUMENT CONFIGURATIONS
%----------------------------------------------------------------------------------------

\documentclass[10pt]{article}
\usepackage{geometry} % Pour passer au format A4
\geometry{hmargin=1cm, vmargin=1cm} % 

\usepackage{graphicx} % Required for including pictures
\usepackage{float} % 

%Français
\usepackage[T1]{fontenc} 
\usepackage[english,francais]{babel}
\usepackage[utf8]{inputenc}
\usepackage{eurosym}
\usepackage{lmodern}
\usepackage{url}
\usepackage{multicol,multido}

%Maths
\usepackage{amsmath,amsfonts,amssymb,amsthm,gensymb}
\usepackage{siunitx}

%Autres
\linespread{1} % Line spacing
\setlength\parindent{0pt} % Removes all indentation from paragraphs

\renewcommand{\labelenumi}{\alph{enumi}.} % 
\pagestyle{empty}
\newcommand{\horrule}[1]{\rule{\linewidth}{#1}} % Create horizontal rule command with 1 argument of height
\newcommand{\Pointille}[1][3]{\multido{}{#1}{ \makebox[\linewidth]{\dotfill}\\[\parskip]}}

%----------------------------------------------------------------------------------------
%	DOCUMENT INFORMATION
%----------------------------------------------------------------------------------------
\begin{document}

%\maketitle % Insert the title, author and date

\textbf{Nom, Prénom :} \hspace{8cm} \textbf{Classe :} \hspace{3cm} \textbf{Date :}\\
\textbf{La calculatrice.}

\begin{center}
  \textit{La plus coûteuse des dépenses, c’est la perte de temps.}  - \textbf{Théophraste}
\end{center}

% Exercice 1 
\subsection*{Calcul 1 - Calculer et donner sous forme décimale} 
\begin{eqnarray*}
	4^4                  &=& \\
	5.1 \times 10^4      &=& \\
	2.1 \times 10^{-5}   &=& \\
	(-5)^6               &=& \\
	-2^{10}                &=& \\
	124^3 \times 10^{-3} &=& \\
	12^0                 &=& 
\end{eqnarray*}
% Exercice 1 
\subsection*{Calcul 2 - Mettre sous forme scientifique} 
\begin{eqnarray*}
	20 000 000  &=& \\
	0.0150      &=& \\
	0.63        &=& \\
	0.004       &=& \\
	130         &=& \\
	158 000     &=& \\
	0.1         &=& \\
	0.000 178   &=& \\
	107 000 000 &=& \\
	40 001.5    &=& \\
\end{eqnarray*}

% Exercice 1 
\subsection*{Restituer}
Donner la définition d'un nombre sous forme scientifique. 

% Exercice 1 
\subsection*{Exercice 1 - Sondages} 

Pour un sondage on utilise un questionnaire comportant dix questions. À chaque question, on peut répondre par "oui", "non" ou "sans opinion".

\begin{enumerate}
\item[a.] Combien y a-t-il de façons différentes de répondre à ce questionnaire ? 
\item[b.] Même question pour un sondage avec 200 questions. 
\end{enumerate}

% Exercice 1 
\subsection*{Exercice 2 - Porte-avions} 

Un porte-avions coûte environ 3 miliards d'euros. \\
Quelle hauteur atteindra une pile de billets de 50 \euro \: représentant cette somme.

\begin{itemize}
\item Un billet de 50 \euro \: a une épaisseurs de $\SI{80}{\micro\meter}$. 
\item $\SI{1}{\micro \meter} = 10^{-6}m$.
\end{itemize}

% Exercice 1 
\subsection*{Exercice 3 - Fourmis}

La reine des fourmis lève une armée. Elle nomme un général qui choisit 5 colonels, qui prennent chacun 5 capitaines, qui prennent chacun 5 sergents, qui choisissent chacun 25 soldats!
\begin{enumerate}
\item[a.] Calculer le nombre total de soldats.
\item[b.] Montrer que le nombre total de soldats est une puissance de 5.
\item[c.] La reine des termites,elle, lève une armée dont l'éffectif total est une puissance de 12. 
Quel est l'exposant minimum de cette puissance pour que les termites soient plus nombreux que les fourmis ?
\end{enumerate}

\end{document}

%%%%%%%%%%%%%%%%%%%%%%%%%%%%%%%%%%%%%%%%%
% LaTeX Template
% http://www.LaTeXTemplates.com
%
% Original author:
% Linux and Unix Users Group at Virginia Tech Wiki 
% (https://vtluug.org/wiki/Example_LaTeX_chem_lab_report)
%
% License:
% CC BY-NC-SA 3.0 (http://creativecommons.org/licenses/by-nc-sa/3.0/)
%
%%%%%%%%%%%%%%%%%%%%%%%%%%%%%%%%%%%%%%%%%

%----------------------------------------------------------------------------------------
%	PACKAGES AND DOCUMENT CONFIGURATIONS
%----------------------------------------------------------------------------------------

\documentclass[12pt]{article}
\usepackage{geometry} % Pour passer au format A4
\geometry{hmargin=1cm, vmargin=1cm} % 

\usepackage{graphicx} % Required for including pictures
\usepackage{float} % 

%Français
\usepackage[T1]{fontenc} 
\usepackage[english,francais]{babel}
\usepackage[utf8]{inputenc}
\usepackage{eurosym}
\usepackage{lmodern}
\usepackage{url}
\usepackage{multicol}

%Maths
\usepackage{amsmath,amsfonts,amssymb,amsthm}
%\usepackage[linesnumbered, ruled, vlined]{algorithm2e}
%\SetAlFnt{\small\sffamily}

%Autres
\linespread{1} % Line spacing
\setlength\parindent{0pt} % Removes all indentation from paragraphs

\renewcommand{\labelenumi}{\alph{enumi}.} % 
\pagestyle{empty}
%----------------------------------------------------------------------------------------
%	DOCUMENT INFORMATION
%----------------------------------------------------------------------------------------
\begin{document}

%\maketitle % Insert the title, author and date

\textbf{Nom, Prénom :} \hspace{8cm} \textbf{Classe :} \hspace{3cm} \textbf{Date :}\\
\textbf{Calculatrice :}

\begin{center}
  \textit{On ne craint que ce que l'on ne connaît pas.}  - \textbf{Marie Curie}
\end{center}


\textbf{L'exercice 1 est à faire sur l'énoncé, les exercices 2 et 3 sont à faire sur une copie correctement présentée.}

% Exercice 1 
\subsection*{Exercice 1} 

Corriger la copie de Léa. Chaque réponse juste vaut un point. Quelle est sa note ?

\begin{multicols}{3}

  \begin{enumerate}
  \item \begin{eqnarray*}
    A &=& 6 + 4 \times 3 \\
    A &=& 10 \times 3 \\
    A &=& 30 
  \end{eqnarray*}

  \item \begin{eqnarray*}
    B &=& 17 - 6 + 4  \\
    B &=& 11 + 4      \\
    B &=& 15
  \end{eqnarray*}

  \item \begin{eqnarray*}
    C &=& 18 \div 2 + 4 \\
    C &=& 9 + 4         \\
    C &=& 13
  \end{eqnarray*}
  \end{enumerate}
  
\end{multicols}

\begin{multicols}{3}

  \begin{enumerate}
\item[d.] \begin{eqnarray*}
  D &=& 6 \times (5 + 3) \\
  D &=& 6 \times 8       \\
  D &=& 48
\end{eqnarray*}

\item[e.] \begin{eqnarray*}
  E &=& 9 + \dfrac{18}{3}\\
  E &=& 27 \div 3        \\
  E &=& 3
\end{eqnarray*}

\item[f.] \begin{eqnarray*}
  F &=& 6 - 3 \times 2 \\
  F &=& 16 \times 6    \\
  F &=& 10
\end{eqnarray*}
\end{enumerate}
\end{multicols}
\begin{multicols}{3}
  \begin{enumerate}
\item[g.] \begin{eqnarray*}
  G &=& 20 - 12 + 5 \\
  G &=& 20 - 17     \\
  G &=& 3 
\end{eqnarray*}

\item[h.] \begin{eqnarray*}
  H &=& 5 \times 3 + 7 \\
  H &=& 15 + 7         \\
  H &=& 3
\end{eqnarray*}

\item[i.] \begin{eqnarray*}
  I &=& 18 \div (2 + 4) \\
  I &=& 18 \div 4       \\
  I &=& 3
\end{eqnarray*}


\end{enumerate}

\end{multicols}

\subsection*{Exercice 2}
Chasser l'intrus.


\begin{multicols}{2}

\begin{enumerate}
\item $A = 5,8 + 10 \times 0.02$
\item $B = 6,3 \div 2 - 2$
\item $C = 1,7 \times 2 + 0,6$
\item $D = 15 \times 3 - 12 \times 3 - 2 \times 3$
\end{enumerate}

\end{multicols}

\subsection*{Exercice 3}

\textit{Il est demandé d'écrire UNE expression numérique menant au résultat, de faire les calculs avec les étapes et de répondre par une phrase réponse.}

\begin{enumerate}

\item Jérémy offre à sa mère un parfum qui coûte 36 \euro et à son père un CD qui coûte 22 \euro. Jérémy avait 250 \euro d'économie.\\ \textbf{Que restera-t-il à Jérémy après ses achats ?}

\item Sylvain a encore assez d'essence dans son scooter pour faire 28 km. Il doit aller voir son frère à 9,8 km, puis aller à la librairie à 4,6 km et enfin rentrer chez lui à 11 km.\\ \textbf{Sylvain doit-il remettre de l'essence dans son scooter ?}

\item Dans un club sportif, on a acheté pour chacun des 15 membres de l'équipe une paire de chaussures à 55 \euro la paire et un maillot à 45 \euro.\\ \textbf{Calculer la dépense du club.}

\end{enumerate}


\end{document}

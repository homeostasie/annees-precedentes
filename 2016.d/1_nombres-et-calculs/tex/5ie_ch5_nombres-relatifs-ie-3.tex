%----------------------------------------------------------------------------------------
%	PACKAGES AND DOCUMENT CONFIGURATIONS
%----------------------------------------------------------------------------------------

\documentclass[10pt]{article}
\usepackage{geometry} % Pour passer au format A4
\geometry{hmargin=1cm, vmargin=1cm} % 

\usepackage{graphicx} % Required for including pictures
\usepackage{float} % 

%Français
\usepackage[T1]{fontenc} 
\usepackage[english,francais]{babel}
\usepackage[utf8]{inputenc}
\usepackage{eurosym}
\usepackage{lmodern}
\usepackage{url}
\usepackage{multicol,multido}

%Maths
\usepackage{amsmath,amsfonts,amssymb,amsthm,gensymb}


%Autres
\linespread{1} % Line spacing
\setlength\parindent{0pt} % Removes all indentation from paragraphs

\renewcommand{\labelenumi}{\alph{enumi}.} % 
\pagestyle{empty}
\newcommand{\horrule}[1]{\rule{\linewidth}{#1}} % Create horizontal rule command with 1 argument of height
\newcommand{\Pointille}[1][3]{\multido{}{#1}{ \makebox[\linewidth]{\dotfill}\\[\parskip]}}

%----------------------------------------------------------------------------------------
%	DOCUMENT INFORMATION
%----------------------------------------------------------------------------------------
\begin{document}

%\maketitle % Insert the title, author and date

\textbf{Nom, Prénom :} \hspace{8cm} \textbf{Classe :} \hspace{3cm} \textbf{Date :}\\
\textbf{La calculatrice n'est pas autorisée.}

\begin{center}
  \textit{Si nous faisions tout ce dont nous sommes capables, nous nous surprendrions vraiment.}  - \textbf{Thomas Edison}
\end{center}


\begin{itemize}
\item Restituer
\item Calculer
\item Representer
\end{itemize}

% Exercice 1 
\subsection*{Exercice 1} 

\textbf{1. Calculer}
\begin{multicols}{2}

  \begin{eqnarray*}
    (-8) + 3     &=& \\ 
    4 + (-6 )    &=& \\ 
    (-8) + (-4 ) &=& \\ 
    (-9) + 8     &=& \\ 
    (-4) + 4     &=& \\ 
    7 + (-5 )    &=& \\ 
    2 + 7        &=& \\ 
    (-3) + 6     &=& \\ 
    5 + (-2 )    &=& \\ 
   (-7) + (-7 )  &=& 
  \end{eqnarray*}
\end{multicols}
\textbf{Donner deux nombres opposés ainsi que la définition.}\\
\Pointille[3]


\begin{multicols}{2}
  \begin{figure}[H]
    \centering
    \includegraphics[width=0.6\linewidth]{sources/1/5ie_ch5_bat-3.pdf}
  \end{figure}
  \subsection*{Exercice 2}

  \textbf{1. Donner les coordonnées de tous les tirs afin d'abattre tous les bateaux ci-contre.}


\end{multicols}


\begin{multicols}{2}
  \begin{figure}[H]
    \centering
    \includegraphics[width=0.6\linewidth]{sources/1/5ie_ch5_quad.pdf}
  \end{figure}    

  \textbf{2. Placer les points aux coordonées suivantes}

  \begin{itemize}
  \item $A(2  ;  4)$
  \item $B(-3 ;  2)$
  \item $C(2  ; -1)$
  \item $D(-4 ; -2)$
  \item $E(0  ;  2)$
  \end{itemize}

\end{multicols}


\end{document}

%%%%%%%%%%%%%%%%%%%%%%%%%%%%%%%%%%%%%%%%%
% LaTeX Template
% http://www.LaTeXTemplates.com
%
% Original author:
% Linux and Unix Users Group at Virginia Tech Wiki 
% (https://vtluug.org/wiki/Example_LaTeX_chem_lab_report)
%
% License:
% CC BY-NC-SA 3.0 (http://creativecommons.org/licenses/by-nc-sa/3.0/)
%
%%%%%%%%%%%%%%%%%%%%%%%%%%%%%%%%%%%%%%%%%

%----------------------------------------------------------------------------------------
%	PACKAGES AND DOCUMENT CONFIGURATIONS
%----------------------------------------------------------------------------------------

\documentclass[12pt]{article}
\usepackage{geometry} % Pour passer au format A4
\geometry{hmargin=1cm, vmargin=1cm} % 

\usepackage{graphicx} % Required for including pictures
\usepackage{float} % 

%Français
\usepackage[T1]{fontenc} 
\usepackage[english,francais]{babel}
\usepackage[utf8]{inputenc}
\usepackage{eurosym}
\usepackage{lmodern}
\usepackage{url}
\usepackage{multicol}

%Maths
\usepackage{amsmath,amsfonts,amssymb,amsthm}
\usepackage{gensymb} % \degree


%Autres
\linespread{1} % Line spacing
\newcommand{\horrule}[1]{\rule{\linewidth}{#1}} % Create horizontal rule 
\setlength\parindent{0pt} % Removes all indentation from paragraphs

\renewcommand{\labelenumi}{\alph{enumi}.} % 
\pagestyle{empty}
%----------------------------------------------------------------------------------------
%	DOCUMENT INFORMATION
%----------------------------------------------------------------------------------------
\begin{document}

%\maketitle % Insert the title, author and date

\textbf{Nom, Prénom :} \hspace{8cm} \textbf{Classe :} \hspace{3cm} \textbf{Date :}\\
\textbf{Calculatrice :}

\begin{center}
  \textit{Un peuple prêt à sacrifier un peu de liberté pour un peu de sécurité ne mérite ni l'une ni l'autre, et finit par perdre les deux.} - \textbf{Benjamin Franklin}
\end{center}



% Exercice 1 
\setlength{\columnseprule}{1pt}


\textit{Un professeur de SVT a demandé à tous les élèves d'une classe de faire germer des graines de blé chez eux. Le tableau suivant donne la taille des plantules dix jours après la germination.}

\begin{center}
  \begin{tabular}{| l || l | c | c | c  | c  | c  | c  | c  | c  | c  | c  | c |}
    \hline
    &        A & B & C &  D &  E &  F &  G &  H &  I &  J &  K & L \\ 
    \hline
    1 &   Taille & 2 & 8 & 12 & 14 & 16 & 17 & 18 & 19 & 20 & 21 & 22 \\
    \hline 
    2 & Effectif & 1 & 2 &  3 &  5 &  3 &  4 &  7 &  8 &  4 &  3 &  2 \\
    \hline
  \end{tabular}
\end{center}

\begin{enumerate}
\item Quel est le caractère étudié dans cette étude statistique ?
\item Quel est l'effectif total de la série étudiée ?
\item Calculer l'étendue de la série puis la moyenne arrondie au dixième.
\item Déterminer la médiane de cette série et \textbf{interpréter} le résultat.
\item Un élève a bien respecté le protocole si la taille de la plantule à 10 jours est supérieur à 14cm. Quel pourcentage d'élève à bien respecté le protocole ?

\item[bonus.] Proposer une formule de tableur (commençant par = et en utilisant des cases) permettant de calculer l'étendue.
\end{enumerate}


\vspace{0.5cm}
\horrule{1px}
\vspace{0.5cm}

\textbf{Nom, Prénom :} \hspace{8cm} \textbf{Classe :} \hspace{3cm} \textbf{Date :}\\
\textbf{Calculatrice :}

\begin{center}
  \textit{Un peuple prêt à sacrifier un peu de liberté pour un peu de sécurité ne mérite ni l'une ni l'autre, et finit par perdre les deux.} - \textbf{Benjamin Franklin}
\end{center}



% Exercice 1 
\setlength{\columnseprule}{1pt}


\textit{Un professeur de SVT a demandé à tous les élèves d'une classe de faire germer des graines de blé chez eux. Le tableau suivant donne la taille des plantules dix jours après la germination.}

\begin{center}
  \begin{tabular}{| l || l | c | c | c  | c  | c  | c  | c  | c  | c  | c  | c |}
    \hline
    &        A & B & C &  D &  E &  F &  G &  H &  I &  J &  K & L \\ 
    \hline
    1 &   Taille & 2 & 8 & 12 & 14 & 16 & 17 & 18 & 19 & 20 & 21 & 22 \\
    \hline 
    2 & Effectif & 1 & 2 &  3 &  5 &  3 &  4 &  7 &  8 &  4 &  3 &  2 \\
    \hline
  \end{tabular}
\end{center}

\begin{enumerate}
\item Quel est le caractère étudié dans cette étude statistique ?
\item Quel est l'effectif total de la série étudiée ?
\item Calculer l'étendue de la série puis la moyenne arrondie au dixième.
\item Déterminer la médiane de cette série et \textbf{interpréter} le résultat.
\item Un élève a bien respecté le protocole si la taille de la plantule à 10 jours est supérieur à 14cm. Quel pourcentage d'élève à bien respecté le protocole ?

\item[bonus.] Proposer une formule de tableur (commençant par = et en utilisant des cases) permettant de calculer l'étendue.
\end{enumerate}

\end{document}

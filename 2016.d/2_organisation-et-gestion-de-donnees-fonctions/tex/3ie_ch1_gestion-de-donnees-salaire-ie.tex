%%%%%%%%%%%%%%%%%%%%%%%%%%%%%%%%%%%%%%%%%
% LaTeX Template
% http://www.LaTeXTemplates.com
%
% Original author:
% Linux and Unix Users Group at Virginia Tech Wiki 
% (https://vtluug.org/wiki/Example_LaTeX_chem_lab_report)
%
% License:
% CC BY-NC-SA 3.0 (http://creativecommons.org/licenses/by-nc-sa/3.0/)
%
%%%%%%%%%%%%%%%%%%%%%%%%%%%%%%%%%%%%%%%%%

%----------------------------------------------------------------------------------------
%	PACKAGES AND DOCUMENT CONFIGURATIONS
%----------------------------------------------------------------------------------------

\documentclass[12pt]{article}
\usepackage{geometry} % Pour passer au format A4
\geometry{hmargin=1cm, vmargin=1cm} % 

\usepackage{graphicx} % Required for including pictures
\usepackage{float} % 

%Français
\usepackage[T1]{fontenc} 
\usepackage[english,francais]{babel}
\usepackage[utf8]{inputenc}
\usepackage{eurosym}
\usepackage{lmodern}
\usepackage{url}
\usepackage{multicol}

%Maths
\usepackage{amsmath,amsfonts,amssymb,amsthm}
\usepackage{gensymb} % \degree


%Autres
\linespread{1} % Line spacing
\newcommand{\horrule}[1]{\rule{\linewidth}{#1}} % Create horizontal rule 
\setlength\parindent{0pt} % Removes all indentation from paragraphs

\renewcommand{\labelenumi}{\alph{enumi}.} % 
\pagestyle{empty}
%----------------------------------------------------------------------------------------
%	DOCUMENT INFORMATION
%----------------------------------------------------------------------------------------
\begin{document}

%\maketitle % Insert the title, author and date

\textbf{Nom, Prénom :} \hspace{8cm} \textbf{Classe :} \hspace{3cm} \textbf{Date :}\\
\textbf{Calculatrice :}

\begin{center}
  \textit{Le courage est le juste milieu entre la peur et l'audace} - \textbf{Aristote}
\end{center}

\textit{Voici les salaires pour l'année 2015 de votre professeur de mathématiques.}

\begin{center}
  \begin{tabular}{| l || c | c | c  | c  | c  | c  | c  | c  | c  | c  | c |}
    \hline
    Mois & Janvier & Février & Mars & Avril &  Mai & Juin & Juillet & Aout & septembre & Octobre & Novembre \\
    \hline 
    Salaire (\euro) &   1886  &    1888 & 2048 &  1887 & 1887 & 2273    & 1761 & 1761 &      1775 &    2029 &     3179 \\
    \hline
  \end{tabular}
\end{center}

\begin{enumerate}
\item[1.] Représentation graphique.
  \begin{enumerate}
  \item[1a.] Représenter graphiquement cette série de données par un diagramme en bâton.
  \item[1b.] Le diagramme circulaire est-il une représentation graphique adaptée pour cette série de données ? Justifier à l'aide d'une phrase.
  \end{enumerate}

\item[2.] Étude des caractéristiques de la série.

  \begin{enumerate}
  \item[2a.] Quel est le caractère étudié de cette étude statistique ?
  \item[2b.] Calculer l'étendu de la série.
  \item[2c.] Déterminer la médiane de la série et \textbf{interpréter} le résultat.
  \item[2d.] Calculer le salaire moyen perçu par M. Lafond pour ces 11 mois.
  \item[2e.] Combien doit toucher M. Lafond en décembre pour que son salaire moyen pour l'année 2015 soit de 2200 \euro.		
  \end{enumerate}

\end{enumerate}

\begin{enumerate}
\item[bonus.] Dans quelle grande ville se situe l'arrondissement de Manhattan ? Connaissez-vous le nom d'autres arrondissement de cette ville ?
\end{enumerate}

\horrule{1px}


\textbf{Nom, Prénom :} \hspace{8cm} \textbf{Classe :} \hspace{3cm} \textbf{Date :}\\
\textbf{Calculatrice :}

\begin{center}
  \textit{Le courage est le juste milieu entre la peur et l'audace} - \textbf{Aristote}
\end{center}

\textit{Voici les salaires pour l'année 2015 de votre professeur de mathématiques.}

\begin{center}
  \begin{tabular}{| l || c | c | c  | c  | c  | c  | c  | c  | c  | c  | c |}
    \hline
    Mois & Janvier & Février & Mars & Avril &  Mai & Juin & Juillet & Aout & septembre & Octobre & Novembre \\
    \hline 
    Salaire (\euro) &   1886  &    1888 & 2048 &  1887 & 1887 & 2273    & 1761 & 1761 &      1775 &    2029 &     3179 \\
    \hline
  \end{tabular}
\end{center}

\begin{enumerate}
\item[1.] Représentation graphique.
  \begin{enumerate}
  \item[1a.] Représenter graphiquement cette série de données par un diagramme en bâton.
  \item[1b.] Le diagramme circulaire est-il une représentation graphique adaptée pour cette série de données ? Justifier à l'aide d'une phrase.
  \end{enumerate}

\item[2.] Étude des caractéristiques de la série.

  \begin{enumerate}
  \item[2a.] Quel est le caractère étudié de cette étude statistique ?
  \item[2b.] Calculer l'étendu de la série.
  \item[2c.] Déterminer la médiane de la série et \textbf{interpréter} le résultat.
  \item[2d.] Calculer le salaire moyen perçu par M. Lafond pour ces 11 mois.
  \item[2e.] Combien doit toucher M. Lafond en décembre pour que son salaire moyen pour l'année 2015 soit de 2200 \euro.		
  \end{enumerate}

\end{enumerate}

\begin{enumerate}
\item[bonus.] Dans quelle grande ville se situe l'arrondissement de Manhattan ? Connaissez-vous le nom d'autres arrondissement de cette ville ?
\end{enumerate}

\end{document}

%%%%%%%%%%%%%%%%%%%%%%%%%%%%%%%%%%%%%%%%%
% LaTeX Template
% http://www.LaTeXTemplates.com
%
% Original author:
% Linux and Unix Users Group at Virginia Tech Wiki 
% (https://vtluug.org/wiki/Example_LaTeX_chem_lab_report)
%
% License:
% CC BY-NC-SA 3.0 (http://creativecommons.org/licenses/by-nc-sa/3.0/)
%
%%%%%%%%%%%%%%%%%%%%%%%%%%%%%%%%%%%%%%%%%

%----------------------------------------------------------------------------------------
%	PACKAGES AND DOCUMENT CONFIGURATIONS
%----------------------------------------------------------------------------------------

\documentclass[12pt]{article}
\usepackage{geometry} % Pour passer au format A4
\geometry{hmargin=1cm, vmargin=1cm} % 

\usepackage{graphicx} % Required for including pictures
\usepackage{float} % 

%Français
\usepackage[T1]{fontenc} 
\usepackage[english,francais]{babel}
\usepackage[utf8]{inputenc}
\usepackage{eurosym}
\usepackage{lmodern}
\usepackage{url}
\usepackage{multicol}

%Maths
\usepackage{amsmath,amsfonts,amssymb,amsthm}
\usepackage{gensymb} % \degree


%Autres
\linespread{1} % Line spacing
\newcommand{\horrule}[1]{\rule{\linewidth}{#1}} % Create horizontal rule 
\setlength\parindent{0pt} % Removes all indentation from paragraphs

\renewcommand{\labelenumi}{\alph{enumi}.} % 
\pagestyle{empty}
%----------------------------------------------------------------------------------------
%	DOCUMENT INFORMATION
%----------------------------------------------------------------------------------------
\begin{document}

%\maketitle % Insert the title, author and date

\textbf{Nom, Prénom :} \hspace{8cm} \textbf{Classe :} \hspace{3cm} \textbf{Date :}\\
\textbf{Calculatrice :}

\begin{center}
  \textit{On ne craint que ce que l'on ne connaît pas.}  - \textbf{Marie Curie}
\end{center}


% Exercice 1 
\setlength{\columnseprule}{1pt}

\begin{multicols}{2}

  \subsection*{Exercice 1} 

  \textit{Dans un journal, on a compté le nombre de ligne pour chacune des 120 petites annonces parues ce jour-là. Les résultats sont présentés dans le tableau suivant.}

  \begin{center}
    \begin{tabular}{| l || c | c | c | c | c | }
      \hline
      Nombre de lignes  & 2 &  3 &  4 &  5 &  6 \\
      \hline
      Effectif          & 5 & 28 & 42 & 10 & 35 \\ 
      \hline
      Fréquence (\%)    &   &    &    &    &    \\
      \hline
    \end{tabular}
  \end{center}

  \begin{enumerate}
  \item Reproduire et compléter le tableau des fréquences en pourcentage.
  \item Constuire un diagramme en bâtons représentant ces données. \\
    \textit{Rappel : La qualité, la précision et la beauté graphique sont prises en compte dans la notation.}

  \end{enumerate}

\end{multicols}

\subsection*{Exercice 2} 

\textit{Le tableau suivant donne le nombre de membres par catégories dans un club de football.}

\begin{center}
  \begin{tabular}{| l || c | c | c | c | c | }
    \hline
    Catégorie            & Poussins & Benjamins & Minimes & Cadets & Total         \\
    \hline
    Effectif             &       25 &        45 &      39 &    11  &               \\
    \hline
    Angle (en $\degree$) &          &           &         &        & $360 \degree$ \\
    \hline
  \end{tabular}
\end{center}

\begin{enumerate}
\item Reproduire et compléter le tableau.
\item Constuire un diagramme circulaire.
\item Donner un exemple de série de données statistiques pour laquelle le diagramme circulaire n'est pas une bonne représentation graphique et justifier avec une argumentation de 2/3 lignes.
\end{enumerate}

\vspace{0.2cm}
\horrule{1px}
\vspace{0.2cm}

\textbf{Nom, Prénom :} \hspace{8cm} \textbf{Classe :} \hspace{3cm} \textbf{Date :}\\
\textbf{Calculatrice :}

\begin{center}
  \textit{On ne craint que ce que l'on ne connaît pas.}  - \textbf{Marie Curie}
\end{center}


% Exercice 1 
\setlength{\columnseprule}{1pt}

\begin{multicols}{2}

  \subsection*{Exercice 1} 

  \textit{Dans un journal, on a compté le nombre de ligne pour chacune des 120 petites annonces parues ce jour-là. Les résultats sont présentés dans le tableau suivant.}

  \begin{center}
    \begin{tabular}{| l || c | c | c | c | c | }
      \hline
      Nombre de lignes  & 2 &  3 &  4 &  5 &  6 \\
      \hline
      Effectif          & 15 & 18 & 32 & 20 & 35 \\ 
      \hline
      Fréquence (\%)    &   &    &    &    &    \\
      \hline
    \end{tabular}
  \end{center}

  \begin{enumerate}
  \item Reproduire et compléter le tableau des fréquences en pourcentage.
  \item Constuire un diagramme en bâtons représentant ces données. \\
    \textit{Rappel : La qualité, la précision et la beauté graphique sont prises en compte dans la notation.}

  \end{enumerate}

\end{multicols}

\subsection*{Exercice 2} 

\textit{Le tableau suivant donne le nombre de membres par catégories dans un club de football.}

\begin{center}
  \begin{tabular}{| l || c | c | c | c | c | }
    \hline
    Catégorie            & Poussins & Benjamins & Minimes & Cadets & Total         \\
    \hline
    Effectif             &       14 &        56 &      35 &    25  &               \\
    \hline
    Angle (en $\degree$) &          &           &         &        & $360 \degree$ \\
    \hline
  \end{tabular}
\end{center}

\begin{enumerate}
\item Reproduire et compléter le tableau.
\item Constuire un diagramme circulaire.
\item Donner un exemple de série de données statistiques pour laquelle le diagramme circulaire n'est pas une bonne représentation graphique et justifier avec une argumentation de 2/3 lignes.
\end{enumerate}

\end{document}

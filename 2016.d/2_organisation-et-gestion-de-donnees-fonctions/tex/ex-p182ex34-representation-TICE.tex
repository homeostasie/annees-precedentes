\documentclass[12pt]{article}
\usepackage{geometry} % Pour passer au format A4
\geometry{hmargin=0.7cm, vmargin=0.7cm} % 

\usepackage{graphicx} % Required for including pictures
\usepackage{float} % 

%Français
\usepackage[T1]{fontenc} 
\usepackage[english,francais]{babel}
\usepackage[utf8]{inputenc}
\usepackage{eurosym}
\usepackage{lmodern}
\usepackage{url}
\usepackage{multicol}
\usepackage{multido}
%Maths
\usepackage{amsmath,amsfonts,amssymb,amsthm}
%\usepackage[linesnumbered, ruled, vlined]{algorithm2e}
%\SetAlFnt{\small\sffamily}

%Autres
\linespread{1} % Line spacing
\setlength\parindent{0pt} % Removes all indentation from paragraphs

\renewcommand{\labelenumi}{\alph{enumi}.} %
\newcommand{\horrule}[1]{\rule{\linewidth}{#1}} % Create horizontal rule command with 1 argument of height
\newcommand{\Pointille}[1][3]{\multido{}{#1}{ \makebox[\linewidth]{\dotfill}\\[\parskip]}}

\pagestyle{empty}
%----------------------------------------------------------------------------------------
%	DOCUMENT INFORMATION
%----------------------------------------------------------------------------------------
\begin{document}

\textbf{Nom, Prénom :} \hspace{8cm} \textbf{Classe :} \hspace{3cm} \textbf{Date :}

\subsection*{Exercice 34 - p182}

Le tableau suivant donne l'évolution de la population d'un village depuis les années 1990.

\begin{center}
  \begin{tabular}{| l || c | c | c | c | c | c | }
    \hline
    Année     & 1990 & 1995 & 2000 & 2005 & 2010 & 2015 \\
    \hline
    Effectif &  672 &  815 &  910 &  985 & 1058 & 1354 \\ 
    \hline
  \end{tabular}
\end{center}



\begin{enumerate}
\item[a.] Saisir ces données dans une feuille de calcul, puis les représenter par différents types de graphiques : diagramme en bâtons, diagramme circulaire et courbe.
\item[b.] Laquelle de ces représentaions semble le mieux adaptée pour montrer l'évolution de la population de ce village ? Justifier. \\
\end{enumerate}

\Pointille[5] \\

\horrule{1px}
\vspace{0.3cm}

\textbf{Nom, Prénom :} \hspace{8cm} \textbf{Classe :} \hspace{3cm} \textbf{Date :}

\subsection*{Exercice 34 - p182}

Le tableau suivant donne l'évolution de la population d'un village depuis les années 1990.

\begin{center}
  \begin{tabular}{| l || c | c | c | c | c | c | }
    \hline
    Année     & 1990 & 1995 & 2000 & 2005 & 2010 & 2015 \\
    \hline
    Effectif &  672 &  815 &  910 &  985 & 1058 & 1354 \\ 
    \hline
  \end{tabular}
\end{center}



\begin{enumerate}
\item[a.] Saisir ces données dans une feuille de calcul, puis les représenter par différents types de graphiques : diagramme en bâtons, diagramme circulaire et courbe.
\item[b.] Laquelle de ces représentaions semble le mieux adaptée pour montrer l'évolution de la population de ce village ? Justifier. \\
\end{enumerate}

\Pointille[5] \\


\end{document}

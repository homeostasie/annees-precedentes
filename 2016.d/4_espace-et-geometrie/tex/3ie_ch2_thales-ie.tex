%%%%%%%%%%%%%%%%%%%%%%%%%%%%%%%%%%%%%%%%%
% LaTeX Template
% http://www.LaTeXTemplates.com
%
% Original author:
% Linux and Unix Users Group at Virginia Tech Wiki 
% (https://vtluug.org/wiki/Example_LaTeX_chem_lab_report)
%
% License:
% CC BY-NC-SA 3.0 (http://creativecommons.org/licenses/by-nc-sa/3.0/)
%
%%%%%%%%%%%%%%%%%%%%%%%%%%%%%%%%%%%%%%%%%

%----------------------------------------------------------------------------------------
%	PACKAGES AND DOCUMENT CONFIGURATIONS
%----------------------------------------------------------------------------------------

\documentclass[12pt]{article}
\usepackage{geometry} % Pour passer au format A4
\geometry{hmargin=1cm, vmargin=1cm} % 

\usepackage{graphicx} % Required for including pictures
\usepackage{float} % 

%Français
\usepackage[T1]{fontenc} 
\usepackage[english,francais]{babel}
\usepackage[utf8]{inputenc}
\usepackage{eurosym}
\usepackage{lmodern}
\usepackage{url}
\usepackage{multicol}
\usepackage{multido}

%Maths
\usepackage{amsmath,amsfonts,amssymb,amsthm}
%\usepackage[linesnumbered, ruled, vlined]{algorithm2e}
%\SetAlFnt{\small\sffamily}

%Autres
\linespread{1} % Line spacing
\setlength\parindent{0pt} % Removes all indentation from paragraphs

\renewcommand{\labelenumi}{\alph{enumi}.} % 
\pagestyle{empty}

\newcommand{\horrule}[1]{\rule{\linewidth}{#1}} 
\newcommand{\Pointille}[1][3]{\multido{}{#1}{ \makebox[\linewidth]{\dotfill}\\[\parskip]}}
%----------------------------------------------------------------------------------------
%	DOCUMENT INFORMATION
%----------------------------------------------------------------------------------------
\begin{document}

%\maketitle % Insert the title, author and date

\begin{multicols}{3}

  \begin{figure}[H]
    \centering
    \includegraphics[width=0.6\linewidth]{sources/1/3ie_ch2_thales-ie-1.pdf}
  \end{figure}

\textbf{Nom, Prénom :}\\
\textit{Les droites horizontales sont parallèles.} \textbf{Calculer les deux longueurs manquantes.}

\end{multicols}

\vspace{1cm}
\horrule{1px}
\vspace{1cm}

\begin{multicols}{3}

  \begin{figure}[H]
    \centering
    \includegraphics[width=0.6\linewidth]{sources/1/3ie_ch2_thales-ie-2.pdf}
  \end{figure}

\textbf{Nom, Prénom :}\\
\textit{Les droites horizontales sont parallèles.} \textbf{Calculer les deux longueurs manquantes.}

\end{multicols}

\vspace{1cm}
\horrule{1px}
\vspace{1cm}

\begin{multicols}{3}

  \begin{figure}[H]
    \centering
    \includegraphics[width=0.6\linewidth]{sources/1/3ie_ch2_thales-ie-3.pdf}
  \end{figure}

\textbf{Nom, Prénom :}\\
\textit{Les droites horizontales sont parallèles.} \textbf{Calculer les deux longueurs manquantes.}

\end{multicols}



\end{document}

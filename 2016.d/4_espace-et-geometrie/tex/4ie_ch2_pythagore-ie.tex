%%%%%%%%%%%%%%%%%%%%%%%%%%%%%%%%%%%%%%%%%
% LaTeX Template
% http://www.LaTeXTemplates.com
%
% Original author:
% Linux and Unix Users Group at Virginia Tech Wiki 
% (https://vtluug.org/wiki/Example_LaTeX_chem_lab_report)
%
% License:
% CC BY-NC-SA 3.0 (http://creativecommons.org/licenses/by-nc-sa/3.0/)
%
%%%%%%%%%%%%%%%%%%%%%%%%%%%%%%%%%%%%%%%%%

%----------------------------------------------------------------------------------------
%	PACKAGES AND DOCUMENT CONFIGURATIONS
%----------------------------------------------------------------------------------------

\documentclass[12pt]{article}
\usepackage{geometry} % Pour passer au format A4
\geometry{hmargin=1cm, vmargin=1cm} % 

\usepackage{graphicx} % Required for including pictures
\usepackage{float} % 

%Français
\usepackage[T1]{fontenc} 
\usepackage[english,francais]{babel}
\usepackage[utf8]{inputenc}
\usepackage{eurosym}
\usepackage{lmodern}
\usepackage{url}
\usepackage{multicol}

%Maths
\usepackage{amsmath,amsfonts,amssymb,amsthm}
%\usepackage[linesnumbered, ruled, vlined]{algorithm2e}
%\SetAlFnt{\small\sffamily}

%Autres
\linespread{1} % Line spacing
\setlength\parindent{0pt} % Removes all indentation from paragraphs

\renewcommand{\labelenumi}{\alph{enumi}.} % 
\newcommand{\horrule}[1]{\rule{\linewidth}{#1}} 
\newcommand{\Pointille}[1][3]{\multido{}{#1}{ \makebox[\linewidth]{\dotfill}\\[\parskip]}}
\pagestyle{empty}
%----------------------------------------------------------------------------------------
%	DOCUMENT INFORMATION
%----------------------------------------------------------------------------------------
\begin{document}

%\maketitle % Insert the title, author and date

\textbf{Nom, Prénom :} \hspace{8cm} \textbf{Classe :} \hspace{3cm} \textbf{Date :}\\
\textbf{Calculatrice :}

\begin{center}
\textit{Les mathématiques ne sont une moindre immensité que la mer.} - \textbf{Victor Hugo}
\end{center}


% Exercice 1 

\begin{multicols}{2}
\textbf{Exercice 1 - Calculer}

\begin{enumerate}
\item[a.] $3.4^2$
\item[b.] $\sqrt{8.9}$
\item[c.] $10^2 - 6^2 - 7^2$
\item[d.] $\sqrt{4^2 + 13^2}$
\item[e.] $5^2 + 2 \times \sqrt{3^2 + 4^2}$
\end{enumerate}
\end{multicols}


% Exercice 2

\textbf{Exercice 2} 
En rentrant chez moi, je m'aperçois que j'ai oublié mes clés. je sais que le bas de la fenêtre du premier étage se trouve à 5 m du sol et qu'elle est entrouverte. Un voisin me prête une échelle de 5,5 m de long. Pour grimper sans tomber, je suis obligé de poser les pieds de l'échelle à au moins 1,5 m du pied du mur. 

\begin{enumerate}
\item[a.] Faire un schéma de la situation. 
\item[b.] Pourrai-je atteindre le bas de ma fenêtre ?
\end{enumerate}

% Exercice 3

\textbf{Exercice 3} 
En sortant du métro, je me retrouve au pied d'un escalateur qui monte vers la sortie. J'ai le choix entre : 
\begin{itemize}
\item prendre celui-ci pour une distance de 52m.
\item marcher horizontalement sur 34m puis prendre un ascenseur qui me dépose en haut de l'escalateur.
\end{itemize} 

\begin{enumerate}
\item[a.] Faire un schéma de la situation.
\item[b.] Quel hauteur parcours l'ascenseur ?
\end{enumerate}

\vspace{1cm}
\horrule{1px}
\vspace{1cm}

\textbf{Nom, Prénom :} \hspace{8cm} \textbf{Classe :} \hspace{3cm} \textbf{Date :}\\
\textbf{Calculatrice :}

\begin{center}
\textit{Les mathématiques ne sont une moindre immensité que la mer.} - \textbf{Victor Hugo}
\end{center}


% Exercice 1 

\begin{multicols}{2}
\textbf{Exercice 1 - Calculer}

\begin{enumerate}
\item[a.] $3.4^2$
\item[b.] $\sqrt{8.7}$
\item[c.] $11^2 - 6^2 - 7^2$
\item[d.] $\sqrt{4^2 + 15^2}$
\item[e.] $6^2 + 3 \times \sqrt{3^2 + 4^2}$
\end{enumerate}
\end{multicols}


% Exercice 2

\textbf{Exercice 2} 
En rentrant chez moi, je m'aperçois que j'ai oublié mes clés. je sais que le bas de la fenêtre du premier étage se trouve à 5 m du sol et qu'elle est entrouverte. Un voisin me prête une échelle de 5,5 m de long. Pour grimper sans tomber, je suis obligé de poser les pieds de l'échelle à au moins 1,7 m du pied du mur. 

\begin{enumerate}
\item[a.] Faire un schéma de la situation. 
\item[b.] Pourrai-je atteindre le bas de ma fenêtre ?
\end{enumerate}

% Exercice 3

\textbf{Exercice 3} 
En sortant du métro, je me retrouve au pied d'un escalateur qui monte vers la sortie. J'ai le choix entre : 
\begin{itemize}
\item prendre celui-ci pour une distance de 54m.
\item marcher horizontalement sur 36m puis prendre un ascenseur qui me dépose en haut de l'escalateur.
\end{itemize} 

\begin{enumerate}
\item[a.] Faire un schéma de la situation.
\item[b.] Quel hauteur parcours l'ascenseur ?
\end{enumerate}

\end{document}

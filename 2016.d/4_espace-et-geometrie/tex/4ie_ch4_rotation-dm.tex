%----------------------------------------------------------------------------------------
%	PACKAGES AND DOCUMENT CONFIGURATIONS
%----------------------------------------------------------------------------------------

\documentclass[10pt,landscape]{article}
\usepackage{geometry} % Pour passer au format A4
\geometry{hmargin=1cm, vmargin=1cm} % 

\usepackage{graphicx} % Required for including pictures
\usepackage{float} % 

%Français
\usepackage[T1]{fontenc} 
\usepackage[english,francais]{babel}
\usepackage[utf8]{inputenc}
\usepackage{eurosym}
\usepackage{lmodern}
\usepackage{url}
\usepackage{multicol,multido}

%Maths
\usepackage{amsmath,amsfonts,amssymb,amsthm,gensymb}


%Autres
\linespread{1} % Line spacing
\setlength\parindent{0pt} % Removes all indentation from paragraphs

\renewcommand{\labelenumi}{\alph{enumi}.} % 
\pagestyle{empty}
\newcommand{\horrule}[1]{\rule{\linewidth}{#1}} % Create horizontal rule command with 1 argument of height
\newcommand{\Pointille}[1][3]{\multido{}{#1}{ \makebox[\linewidth]{\dotfill}\\[\parskip]}}

%----------------------------------------------------------------------------------------
%	DOCUMENT INFORMATION
%----------------------------------------------------------------------------------------
\begin{document}

%\maketitle % Insert the title, author and date

\textbf{Nom, Prénom :} \hspace{8cm} \textbf{Classe :} \hspace{3cm} \textbf{Date :}\\

\begin{center}
  \textit{On vit de ce que l’on obtient. On construit sa vie sur ce que l’on donne.}  - \textbf{Winston Churchill}
\end{center}

\setlength{\columnseprule}{1pt}

\begin{multicols}{2}

  \begin{figure}[H]
    \centering
    \includegraphics[width=0.9\linewidth]{sources/2/4ie_ch4_rot-dm-1.pdf}
  \end{figure}

  \textit{Il est demandé de laisser les traces de construction finement tracées au crayon à papier}

  \begin{enumerate}
  \item[a.] Placer le Mont Markhan : M1 qui est l'image du point A par la rotation de centre O, d'angle $90\degree$ dans le sens des aiguilles d'une montre. 
  \item[b.] Placer le Mont Vinson : M2 qui est l'image du point B par la rotation de centre O, d'angle $180\degree$ dans le sens inverse des aiguilles d'une montre.
  \item[c.] Placer la station Concordia : Sc qui est l'image du point C par la rotation de centre O, d'angle $35\degree$ dans le sens des aiguilles d'une montre. 
  \item[d.] Placer la station Mawson : Sm qui est l'image du point O par la rotation de centre C, d'angle $82\degree$ dans le sens des aiguilles d'une montre. 
  \end{enumerate}


  \setlength{\columnseprule}{0pt}

  \begin{multicols}{2}

    \subsection*{Bonus}
    \textit{Dessiner le drapeau}
    \begin{figure}[H]
      \centering
      \includegraphics[width=0.6\linewidth]{sources/2/4ie_ch4_rot-dm-2.pdf}
    \end{figure}
    
  \end{multicols}

  \subsection*{Continent}
  
  \begin{enumerate}
  \item[a.] Quel est le nom de ce continent ?\\
    \Pointille[1]
  \item[b.] Comment s'appelle le point O ?\\
    \Pointille[1] 
  \item[c.] Quelle est la superficie de ce continent ?\\
    \Pointille[1]
  \item[d.] $98\%$ de la surface sont recouverts de glace. L'épaisseur moyenne est de 1.6km. Quelle est le volume de glace sur ce continent ?\\
    \Pointille[4]
  \end{enumerate}

  \subsection*{Découverte}
  
  \textit{Quels sont les événements importants liés à cette date ?}
  \begin{itemize}
  \item 1577.\\
    \Pointille[1]
  \item 1772 \\
    \Pointille[1]
  \item 1959. \\
    \Pointille[1]
  \end{itemize}

  \begin{multicols}{2}
    \subsection*{Sciences}
    \textit{Donner le nom de 4 bases scientifiques se trouvant sur ce continent et les placer sur la carte.}
    \begin{itemize}
    \item \Pointille[1]
    \item \Pointille[1]
    \item \Pointille[1]
    \item \Pointille[1]
    \end{itemize}
  \end{multicols}

\end{multicols}

\end{document}

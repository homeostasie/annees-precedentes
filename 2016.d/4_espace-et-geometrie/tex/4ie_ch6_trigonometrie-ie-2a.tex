%----------------------------------------------------------------------------------------
%	PACKAGES AND DOCUMENT CONFIGURATIONS
%----------------------------------------------------------------------------------------

\documentclass[10pt]{article}
\usepackage{geometry} % Pour passer au format A4
\geometry{hmargin=1cm, vmargin=1cm} % 

\usepackage{graphicx} % Required for including pictures
\usepackage{float} % 

%Français
\usepackage[T1]{fontenc} 
\usepackage[english,francais]{babel}
\usepackage[utf8]{inputenc}
\usepackage{eurosym}
\usepackage{lmodern}
\usepackage{url}
\usepackage{multicol,multido}

%Maths
\usepackage{amsmath,amsfonts,amssymb,amsthm,gensymb}
\usepackage{siunitx}

%Autres
\linespread{1} % Line spacing
\setlength\parindent{0pt} % Removes all indentation from paragraphs

\renewcommand{\labelenumi}{\alph{enumi}.} % 
\pagestyle{empty}
\newcommand{\horrule}[1]{\rule{\linewidth}{#1}} % Create horizontal rule command with 1 argument of height
\newcommand{\Pointille}[1][3]{\multido{}{#1}{ \makebox[\linewidth]{\dotfill}\\[\parskip]}}

%----------------------------------------------------------------------------------------
%	DOCUMENT INFORMATION
%----------------------------------------------------------------------------------------
\begin{document}

%\maketitle % Insert the title, author and date

\textbf{Nom, Prénom :} \hspace{8cm} \textbf{Classe :} \hspace{3cm} \textbf{Date :}\\
\textbf{La calculatrice.}

\begin{center}
  \textit{L’essentiel est sans cesse menacé par l’insignifiant.}  - \textbf{René Char}
\end{center}

% Exercice 1 
\begin{multicols}{2}
  \begin{center}
    \begin{tabular}{| l || c | c | c |}
      \hline
      &  $AB$  &  $BC$ & $\widehat{ABC}$ \\ 
      \hline
      1. &  2cm  & 8cm &  \\
      \hline 
      2. &  10cm &     & 60\degree \\
      \hline
      3. &       & 7cm & 70\degree \\
      \hline
      4. &       & 34m &  32\degree \\
      \hline
      5. &  23cm &     &  43\degree\\
      \hline
      6. &  25cm &  2m &  \\
      \hline
    \end{tabular}
  \end{center}

  \subsection*{Calcul 1}
  
  \textit{Compléter le tableau ci-contre.}\\

  $ABC$ est un triangle rectangle en $A$.\\

\end{multicols}

\begin{multicols}{2}
  \subsection*{Calcul 2}
  On donne $PK = 4cm$, $RT = 8cm$, $\widehat{KPM} = 40\degree$ et $\widehat{PLR} = 30\degree$.

  \begin{enumerate}
  \item[1.] Calculer la longueur PL.
  \item[2.] Calculer la longueur RL.
  \item[3.] Calculer l'angle $\widehat{LRT}$. En déduire l'angle $\widehat{LTR}$.
  \end{enumerate}

  \begin{figure}[H]
    \centering
    \includegraphics[width=0.4\linewidth]{sources/5/4ie_ch6_calc.pdf}
  \end{figure}

\end{multicols}

\begin{multicols}{2}
  % Exercice 1 
  \subsection*{Exercice 1 - Titanic} 

  Un sous-marin $S$, navigue sur la mer du Nord. Visuellement, il apercoit un iceberg $I$. Son sonar lui indique une distance $SP$ de $628m$. Il veut plonger pour passer dessous.

  \begin{enumerate}
  \item[1.] Pour $1m$ au-dessus de l'eau, il y a environ $8m$ en-dessous. Calculer la hauteur de la partie immergée de l'iceberg puis sa hauteur totale.
  \item[2.] Calculer la mesure de l'angle $\widehat{ISP}$ de plongée.
  \end{enumerate}

  \begin{figure}[H]
    \centering
    \includegraphics[width=0.6\linewidth]{sources/5/4ie_ch6_sous.pdf}
  \end{figure}
\end{multicols}

\begin{multicols}{2}

  % Exercice 1 
  \subsection*{Exercice 2 - Natation} 

  \begin{enumerate}
  \item[1.] Un nageur est parti de $A$ pour traverser la rivière dont les rives sont parallèles. Il est emporté par le courant et arrive en $B$. Quelle distance a-t-il parcourue ?

  \item[2.]Quelle est la distance $VH$ entre le voilier et la côté ?
  \end{enumerate}

  \begin{figure}[H]
    \centering
    \includegraphics[width=0.5\linewidth]{sources/5/4ie_ch6_nage-1a.pdf}
  \end{figure}

  \begin{figure}[H]
    \centering
    \includegraphics[width=0.5\linewidth]{sources/5/4ie_ch6_nage-2a.pdf}
  \end{figure}
\end{multicols}

\begin{multicols}{2}
  % Exercice 1 
  \subsection*{Exercice 3 - Garage}

  On accède au garage situé au sous-sol d'une maison par une rampe $[AC]$.
  On sait que : $AC = 10.25 m$ ; $BC = 2.25 m$.

  \begin{enumerate}
  \item[1.] Calculer la distance $AB$ entre le portail et l'entrée.
  \item[2.] Calculer la mesure de l'angle $\widehat{BAC}$
  \end{enumerate}
  \textbf{Bonus} : Donner le nom de cinq candidats à l'élection présidentielle Française de 2017.

  \begin{figure}[H]
    \centering
    \includegraphics[width=0.8\linewidth]{sources/5/4ie_ch6_garage.pdf}
  \end{figure}

\end{multicols}



\end{document}

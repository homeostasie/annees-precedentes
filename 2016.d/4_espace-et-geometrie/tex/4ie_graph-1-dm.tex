%----------------------------------------------------------------------------------------
%	PACKAGES AND DOCUMENT CONFIGURATIONS
%----------------------------------------------------------------------------------------

\documentclass[10pt]{article}
\usepackage{geometry} % Pour passer au format A4
\geometry{hmargin=1cm, vmargin=1cm} % 

\usepackage{graphicx} % Required for including pictures
\usepackage{float} % 

%Français
\usepackage[T1]{fontenc} 
\usepackage[english,francais]{babel}
\usepackage[utf8]{inputenc}
\usepackage{eurosym}
\usepackage{lmodern}
\usepackage{url}
\usepackage{multicol,multido}

%Maths
\usepackage{amsmath,amsfonts,amssymb,amsthm,gensymb}

\usepackage{tikz}

%Autres
\linespread{1} % Line spacing
\setlength\parindent{0pt} % Removes all indentation from paragraphs

\renewcommand{\labelenumi}{\alph{enumi}.} % 
\pagestyle{empty}
\newcommand{\horrule}[1]{\rule{\linewidth}{#1}} % Create horizontal rule command with 1 argument of height
\newcommand{\Pointille}[1][3]{\multido{}{#1}{ \makebox[\linewidth]{\dotfill}\\[\parskip]}}

%----------------------------------------------------------------------------------------
%	DOCUMENT INFORMATION
%----------------------------------------------------------------------------------------
\begin{document}

%\maketitle % Insert the title, author and date

\textbf{Nom, Prénom :} \hspace{8cm} \textbf{Classe :} \hspace{3cm} \textbf{Date :}\\

\begin{center}
  \textit{L’essentiel est sans cesse menacé par l’insignifiant.}  - \textbf{René Char}
\end{center}

\vspace{-0.5cm}

\begin{figure}[H]
  \centering
  \begin{tikzpicture}
    \draw[color=gray, line width=.05pt] (0,0) grid[ step=0.2] (18,12);
    \draw (0,0) grid[thick] (18,12);
  \end{tikzpicture}
\end{figure}

\Pointille[1]

\textbf{Données}

Entre 1900 et 2000, on calcule la températures moyenne sur le siecle. On appelle cette température $T_0$. Chaque valeur de température dans le tableau correspond à l'écart (\textit{la différence}) entre cette température $T_0$ et la température moyenne annuelle.
\begin{center}
  \begin{tabular}{| c| c   |     c |   c |    c |    c |    c |    c |    c |    c |    c |    c |    c |    c |    c |    c |   c |}
    \hline
    année & 1880 &  1890 &  1900 &  1910 &  1920 &  1930 & 1940 &  1950 &  1960 & 1970 & 1980 & 1990 & 2000 & 2010 & 2015 & 2016 \\
    \hline
    temp & -0.20 & -0.37 & -0.09 & -0.43 & -0.27 & -0.15 & 0.08 & -0.18 & -0.02 & 0.02 & 0.27 & 0.44 & 0.42 & 0.71 & 0.86 & 1.12 \\
    \hline
  \end{tabular}
\end{center}

\noindent\hrulefill

\begin{multicols}{2}


  \textbf{1 - Préparer le graphique}

  \begin{itemize}
  \item Un titre en français, 
  \item Tracer les axes avec des noms. Ils doivent couvrir l'ensemble des données et être gradués. Il n'est pas nécéssaire de commencer  par 0  mais les graduations doivent être régulières.
  \item Placer les points sur le repère.
  \end{itemize}

  \textbf{2 - Tracer la représentation graphique}

  Le tracer doit être fait au crayon à papier. Les points ne doivent pas être relier à la règle mais à main levée. Il n'est pas obligé de passer par tous les points.

\end{multicols}

\noindent\hrulefill

\textbf{Interpretation}\\

\textit{Donner votre interprétation de ces données et de votre graphique en 2/3 lignes.}\\

\Pointille[4]

Source : \textit{Les données proviennent du site :} http://data.giss.nasa.gov/gistemp/ \\

\end{document}

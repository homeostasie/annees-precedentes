%----------------------------------------------------------------------------------------
%	PACKAGES AND DOCUMENT CONFIGURATIONS
%----------------------------------------------------------------------------------------

\documentclass[10pt]{article}
\usepackage{geometry} % Pour passer au format A4
\geometry{hmargin=1cm, vmargin=1cm} %

\usepackage{graphicx,pgf} % Required for including pictures
\usepackage{float} %

%Français
\usepackage[T1]{fontenc}
\usepackage[english,francais]{babel}
\usepackage[utf8]{inputenc}
\usepackage{eurosym}
\usepackage{lmodern}
\usepackage{url}
\usepackage{multicol,multido}

%Maths
\usepackage{amsmath,amsfonts,amssymb,amsthm,gensymb}
\usepackage{siunitx}

%Autres
\linespread{1} % Line spacing
\setlength\parindent{0pt} % Removes all indentation from paragraphs

\renewcommand{\labelenumi}{\alph{enumi}.}
\pagestyle{empty}
\newcommand{\horrule}[1]{\rule{\linewidth}{#1}} % Create horizontal rule command with 1 argument of height
\newcommand{\Pointille}[1][3]{\multido{}{#1}{ \makebox[\linewidth]{\dotfill}\\[\parskip]}}

%----------------------------------------------------------------------------------------
%	DOCUMENT INFORMATION
%----------------------------------------------------------------------------------------
\begin{document}

%\maketitle % Insert the title, author and date

\textbf{Nom, Prénom :} \hspace{8cm} \textbf{Classe :} \hspace{3cm} \textbf{Date :}\\

\vspace{-0.4cm}
\begin{center}
  \textit{L’essentiel est sans cesse menacé par l’insignifiant.}  - \textbf{René Char}
\end{center}

% Exercice 1

\begin{enumerate}
\item[1a.] Restituer la definition du périmètre d'une figure.\\
  \Pointille[2]
\item[1b.] Restituer la definition de l'aire d'une figure.\\
  \Pointille[2]
\end{enumerate}

% Exercice 1
\begin{minipage}{0.4\linewidth}

  \subsubsection*{Exercice 1}

  \textbf{Exprimer les aires en fonction de l'aire a et les périmètres en fonction des côtés h et c}.

  \begin{enumerate}
  \item[1.] Aire = \textbf{a}\\
    Périmère = \Pointille[1]
  \item[2.] Aire = \Pointille[1]
    Périmètre = \Pointille[1]
  \item[3.] Aire = \Pointille[1]
    Périmètre = \Pointille[1]
  \item[4.] Aire = \Pointille[1]
    Périmètre = \Pointille[1]
  \item[5.] Aire = \textbf{28a}\\
    Périmètre = \textbf{4h + 8c}
  \item[6.] Aire = \Pointille[1]
    Périmètre = \Pointille[1]
  \item[7.] Aire = \Pointille[1]
    Périmètre = \Pointille[1]
  \end{enumerate}
  
\end{minipage}
\begin{minipage}{0.7\linewidth}

  \begin{figure}[H]
    \centering
    \includegraphics[width=0.5\linewidth]{sources/7/5ie_ch7_octo-1a.pdf}
  \end{figure}

\end{minipage}

% Exercice 1

\begin{minipage}{0.3\linewidth}
  \subsection*{Exercice 2}
  \begin{figure}[H]
    \centering
    \includegraphics[width=\linewidth]{sources/7/5ie_ch7_piece-1a.pdf}
  \end{figure}
  
\end{minipage}
\begin{minipage}{0.7\linewidth}


  \begin{center}
    \begin{tabular}{| l | l | c |}
      \hline
      &    Méthode &   Calcul \\ 
      \hline
      Triangle $\phantom{\dfrac{\dfrac{0}{0}}{\dfrac{\dfrac{0}{0}}{0}}}$ 
		  & $A1 = \dfrac{base \times hauteur}{2}$           & \phantom{\hspace{6cm}}\\
      \hline
      Rectangle $\phantom{\dfrac{\dfrac{0}{0}}{\dfrac{\dfrac{0}{0}}{0}}}$ 
      & $A2 = base \times hauteur$                      & \\
      \hline 
      Demi-cercle $\phantom{\dfrac{\dfrac{0}{0}}{\dfrac{\dfrac{0}{0}}{0}}}$ 
      & $A3 = \dfrac{\pi \times Rayon \times Rayon}{2}$ & \\
      \hline
      Pièce $\phantom{\dfrac{\dfrac{0}{0}}{\dfrac{0}{0}}}$ 
      & Aire totale de la pièce :                              & \\
      & =   \\
      \hline
    \end{tabular}
  \end{center}


\end{minipage}


\end{document}

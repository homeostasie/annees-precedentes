
\documentclass[12pt]{article}
\usepackage{geometry} % Pour passer au format A4
\geometry{hmargin=1cm, vmargin=1cm} % 

% Page et encodage
\usepackage[T1]{fontenc} % Use 8-bit encoding that has 256 glyphs
\usepackage[english,francais]{babel} % Français et anglais
\usepackage[utf8]{inputenc} 

\usepackage{lmodern}
\setlength\parindent{0pt}

% Graphiques
\usepackage{graphicx} 
\usepackage{float}


% Maths et divers
\usepackage{amsmath,amsfonts,amssymb,amsthm,verbatim}
\usepackage{multicol,enumitem,url,eurosym}


% Sections
\usepackage{sectsty} % Allows customizing section commands
\allsectionsfont{\centering \normalfont\scshape} % Make all sections centered, the default font and small caps

% Tête et pied de page

\usepackage{fancyhdr} 
\pagestyle{fancyplain} 

\fancyhead{} % No page header
\fancyfoot{}
% \fancyfoot[C]{Triangles} % Empty center footer
% \fancyfoot[R]{\thepage} % Page numbering for right footer

\renewcommand{\headrulewidth}{0pt} % Remove header underlines
\renewcommand{\footrulewidth}{0pt} % Remove footer underlines

\newcommand{\horrule}[1]{\rule{\linewidth}{#1}} % Create horizontal rule command with 1 argument of height


%----------------------------------------------------------------------------------------
%	Début du document
%----------------------------------------------------------------------------------------

\begin{document}

%----------------------------------------------------------------------------------------
% RE-DEFINITION
%----------------------------------------------------------------------------------------
% MATHS
%-----------

\newtheorem{Definition}{Définition}
\newtheorem{Theorem}{Théorème}
\newtheorem{Proposition}{Propriété}

% MATHS
%-----------
\renewcommand{\labelitemi}{$\bullet$}
\renewcommand{\labelitemii}{$\circ$}
%----------------------------------------------------------------------------------------
%	Titre
%----------------------------------------------------------------------------------------

\setlength{\columnseprule}{1pt}

\horrule{1px}
\section*{Théorème de Pythagore}
\horrule{1px}

\begin{enumerate}
\item[1.] Calculer des longueurs dans un triangle rectangle.
\item[2.] Montrer qu'un triangle est rectangle.
\end{enumerate}

\horrule{1px}

\begin{multicols}{2}

  \subsection*{Théorème}

  \textbf{Dans un triangle rectangle, le carré de l’hypoténuse est égal à la somme des deux autres carrés.}

  Soit un triangle ABC rectangle en A. 
  $$AB^2 + AC^2 = BC^2$$

  \begin{figure}[H]
    \centering
    \includegraphics[width=0.6\linewidth]{sources/1/1_pytha-thm.pdf}
  \end{figure}

\end{multicols}

\vspace{-0.5cm}
\horrule{1px}
\vspace{-1cm}

\subsection*{Calcul d'un troisième côté}

On sait que le triangle est rectange et on connait la longueur de 2 côtés de ce triangle.

\begin{multicols}{2}

  \subsubsection*{Recherche de l'hypothénuse}

  \begin{multicols}{2}

    \begin{figure}[H]
      \centering
      \includegraphics[width=\linewidth]{sources/1/1_pytha-re-g.pdf}
    \end{figure}

    Dans le triangle MNO rectangle en O.\\
    D'après le théorème de Pythagore.
    \begin{eqnarray*}
      MN^2 &=& MO^2 + NO^2 \\
      MN^2 &=& 20^2 + 32^2 \\
      MN^2 &=&  1424\\
      MN   &=& \sqrt{1424} \\
      MN   &\approx& 37.7 
    \end{eqnarray*}
    La longueur MN mesure environ 27.7cm.

  \end{multicols}

  \subsubsection*{Recherche d'un petit côté}

  \begin{multicols}{2}


    \begin{figure}[H]
      \centering
      \includegraphics[width=\linewidth]{sources/1/1_pytha-re-p.pdf}
    \end{figure}

    Dans le triangle MNO rectangle en O.\\
    D'après le théorème de Pythagore.
    \begin{eqnarray*}
      MN^2 &=& MO^2 + NO^2 \\
      MO^2 &=& MN^2 - NO^2 \\
      MO^2 &=& 40^2 - 12^2 \\
      MO^2 &=& 1456\\
      MO   &=& \sqrt{1456} \\
      MO   &=& 38
    \end{eqnarray*}
    La longueur MO mesure 38cm.

  \end{multicols}
\end{multicols}

\vspace{-0.5cm}
\horrule{1px}
\vspace{-1cm}

\subsection*{Montrer qu'un triangle est rectangle.}

On a les trois côtés d'un triangle rectangle. On cherche à montrer si le triangle est rectangle ou non.

\begin{multicols}{2}

  \subsubsection*{Oui}

  \begin{multicols}{2}

    \begin{figure}[H]
      \centering
      \includegraphics[width=\linewidth]{sources/1/1_pytha-oui.pdf}
    \end{figure}

    Dans le triangle ABC, le plus grand côté est AB.
    \begin{eqnarray*}
      AB^2 &=& 15^2 \\
      AB^2 &=& 225
    \end{eqnarray*}

    \begin{eqnarray*}
      AC^2 + BC^2 &=& 9^2 + 12^2 \\
      AC^2 + BC^2 &=& 225
    \end{eqnarray*}

    Donc $AB^2 = AC^2 + BC^2$
    D'après la réciproque du théorème de Pythagore, le triangle est ABC est rectangle en C.

  \end{multicols}

  \subsubsection*{Non}

  \begin{multicols}{2}


    \begin{figure}[H]
      \centering
      \includegraphics[width=\linewidth]{sources/1/1_pytha-non.pdf}
    \end{figure}

    Le plus grand côté est AB.
    \begin{eqnarray*}
      AB^2 &=& 13^2 \\
      AB^2 &=& 169
    \end{eqnarray*}

    \begin{eqnarray*}
      AC^2 + BC^2 &=& 11^2 + 12^2 \\
      AC^2 + BC^2 &=& 265
    \end{eqnarray*}
    Donc $AB^2 \neq AC^2 + BC^2$
    D'après la réciproque du théorème de Pythagore, le triangle est ABC n'est pas rectangle.

  \end{multicols}
\end{multicols}
\end{document}

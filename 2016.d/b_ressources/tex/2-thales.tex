
\documentclass[12pt]{article}
\usepackage{geometry} % Pour passer au format A4
\geometry{hmargin=1cm, vmargin=1cm} % 

% Page et encodage
\usepackage[T1]{fontenc} % Use 8-bit encoding that has 256 glyphs
\usepackage[english,francais]{babel} % Français et anglais
\usepackage[utf8]{inputenc} 

\usepackage{lmodern}
\setlength\parindent{0pt}

% Graphiques
\usepackage{graphicx} 
\usepackage{float}


% Maths et divers
\usepackage{amsmath,amsfonts,amssymb,amsthm,verbatim}
\usepackage{multicol,enumitem,url,eurosym}


% Sections
\usepackage{sectsty} % Allows customizing section commands
\allsectionsfont{\centering \normalfont\scshape} % Make all sections centered, the default font and small caps

% Tête et pied de page

\usepackage{fancyhdr} 
\pagestyle{fancyplain} 

\fancyhead{} % No page header
\fancyfoot{}
% \fancyfoot[C]{Triangles} % Empty center footer
% \fancyfoot[R]{\thepage} % Page numbering for right footer

\renewcommand{\headrulewidth}{0pt} % Remove header underlines
\renewcommand{\footrulewidth}{0pt} % Remove footer underlines

\newcommand{\horrule}[1]{\rule{\linewidth}{#1}} % Create horizontal rule command with 1 argument of height


%----------------------------------------------------------------------------------------
%	Début du document
%----------------------------------------------------------------------------------------

\begin{document}

%----------------------------------------------------------------------------------------
% RE-DEFINITION
%----------------------------------------------------------------------------------------
% MATHS
%-----------

\newtheorem{Definition}{Définition}
\newtheorem{Theorem}{Théorème}
\newtheorem{Proposition}{Propriété}

% MATHS
%-----------
\renewcommand{\labelitemi}{$\bullet$}
\renewcommand{\labelitemii}{$\circ$}
%----------------------------------------------------------------------------------------
%	Titre
%----------------------------------------------------------------------------------------

\setlength{\columnseprule}{1pt}

\section*{Théorème de Thalès}

\begin{enumerate}
\item[1.] Calculer des longueurs manquantes avec des triangles.
\item[2.] Montrer que des droites sont parallèles.
\end{enumerate}

\horrule{1px}

\begin{multicols}{2}

  \subsection*{Théorème}

  \textbf{Lorsque des côtés de deux triangles sont parallèles, on a un tableau de proportionnalité entre les longueurs des côtés des deux triangles.}

  \begin{figure}[H]
    \centering
    \includegraphics[width=\linewidth]{sources/2/2_thales-thm.pdf}
  \end{figure}

\end{multicols}

\vspace{-0.5cm}
\horrule{1px}
\vspace{-1cm}

\setlength{\columnseprule}{0pt}
\begin{multicols}{2}

  \subsection*{Calculer une longueur manquante}

  On sait que les droites sont parallèles.

  Les droites (AB) et (MN) sont parallèles.


  \begin{tabular}{| l || c | c | c | }
    \hline
    ABC & AB = 22 & AC = 20 & BC = ?                        \\
    &         &         & $22 \times 16 \div 10 = 35.2$ \\
    \hline
    AMN & AM = 10 & AN = ?                     & MN = 16    \\
    &         & $20 \times 10 \div 22$ = 9 &            \\ 
    \hline
  \end{tabular}


  Dans certain cas un peu plus complexe, il faut poser une équation.

  \begin{figure}[H]
    \centering
    \includegraphics[width=.6\linewidth]{sources/2/2_thales-long.pdf}
  \end{figure}

\end{multicols}

\vspace{-0.5cm}
\horrule{1px}
\vspace{-1cm}

\subsection*{Chercher si des droites sont parallèles.}

On connait des longueurs, on vérifie si la tableau est de proportionnalité.

\setlength{\columnseprule}{1pt}

\begin{multicols}{2}

  \subsubsection*{Oui}
  
  \begin{center}
    \begin{tabular}{| l || c | c | }
      \hline
      ABC & AB = 25 & AC = 20 \\
      \hline
      AMN & AM = 10 & AN = 8  \\
      \hline
    \end{tabular}
  \end{center}
  
  \begin{multicols}{2}

    \begin{eqnarray*}
      \dfrac{AB}{AM} &=& \dfrac{25}{10} = 2.5\\
      \dfrac{AC}{AN} &=& \dfrac{20}{8} = 2.5
    \end{eqnarray*}
    
    Donc $\dfrac{AB}{AM} = \dfrac{AC}{AN}$. \\
    C'est un tableau de proportionnalité.\\
    Les droites (BC) et (MN) sont parallèles.

    \begin{figure}[H]
      \centering
      \includegraphics[width=.7\linewidth]{sources/2/2_thales-oui.pdf}
    \end{figure}

  \end{multicols}

  \subsubsection*{Non}
  
  \begin{center}
    \begin{tabular}{| l || c | c | }
      \hline
      ABC & AB = 9 & BC = 10 \\
      \hline
      AMN & AM = 16 & MN = 20  \\
      \hline
    \end{tabular}
  \end{center}
  
  \begin{multicols}{2}
  
    \begin{figure}[H]
      \centering
      \includegraphics[width=\linewidth]{sources/2/2_thales-non.pdf}
    \end{figure}

    \begin{eqnarray*}
      \dfrac{AB}{AM} &=& \dfrac{9}{16} \approx 0.56\\
      \dfrac{BC}{MN} &=& \dfrac{10}{20} = 0.5
    \end{eqnarray*}
    
    Donc $\dfrac{AB}{AM} \neq \dfrac{AC}{AN}$. \\
    Ce n''est pas un tableau de proportionnalité.\\
    Les droites (BC) et (MN) ne sont pas parallèles.
  \end{multicols}

\end{multicols}

\end{document}

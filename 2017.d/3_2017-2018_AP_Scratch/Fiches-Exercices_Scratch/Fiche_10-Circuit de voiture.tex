\documentclass[12pt,a4paper]{article} % pour paysage on met [landscape]
\usepackage[textheight=28cm,textwidth=19cm,headheight=1mm,footskip=1mm]{geometry}
\usepackage[UTF8]{inputenc}
\usepackage[T1]{fontenc}
\usepackage[french]{babel}
\usepackage{fourier}
\usepackage{amsmath,amsfonts,amssymb}
\usepackage{paralist,array} % paralist : pour les puces et numéros ; array pour les tableaux
\usepackage{pstricks-add} % pour les dessins PSTRICKS

\pagestyle{empty} % pas de numéro de page

\newcommand{\ent}[2]{\par \noindent \textbf{NOM - Prénom :} \dotfill \vspace{9pt} \\ Scratch $-$ Fiche #1 : #2.}
%fin de la commande \ent (entête)

\newcounter {exercice}

\newcommand{\exdeb}{\vspace{18pt} \par \noindent \stepcounter{exercice} \textbf{\underline{{Exercice }\theexercice\,: }}}

\begin{document}
\ent{10}{Le circuit de voiture.} \vspace{24pt} \\
Le but de l'exercice est de créer un circuit de formule 1 fonctionnel. \vspace{9pt} 
\begin{enumerate}[1{)}]
\item
\textbf{Création du parcours sur la scène.} \hfill \renewcommand{\arraystretch}{2}
\begin{tabular}{|p{5mm}|}
\hline
\\
\hline
\end{tabular} \vspace{-6pt}
% \hfill \renewcommand{\arraystretch}{2}
%\begin{tabular}{l|p{5mm}|p{0.5cm}l|p{5mm}|p{0.5cm}l|p{5mm}|p{0.5cm}l|p{5mm}|}
%\cline{2-2} \cline{5-5} \cline{8-8} \cline{11-11}
%a)&&&b)&&&c)&&&d)&\\
%\cline{2-2} \cline{5-5} \cline{8-8} \cline{11-11}
%\end{tabular} \vspace{-6pt}
\begin{enumerate}[a{)}]
\item
Sélectionner Scène, puis l'onglet arrière-plan.
\item
Colorier l'arrière-plan à l'aide de l'outil \og pot de peinture \fg{}.
\item
À l'aide de l'outil \og crayon \fg{}, trace le circuit. \\
Laisse assez d'espace pour pouvoir placer la voiture
\item
À l'aide de l'outil \og ligne \fg, trace le départ et l'arrivée, en choisissant des couleurs différentes.
\end{enumerate} \vspace{6pt}
\item
\textbf{Choix et taille de la voiture.} \hfill \renewcommand{\arraystretch}{2}
\begin{tabular}{|p{5mm}|}
\hline
\\
\hline
\end{tabular} \vspace{-6pt}
\begin{enumerate}[a{)}]
\item
Sélectionner costume, puis choisir un costume dans la bibliothèque et enfin \og car-bug \fg{}.
\item
Choisir un pourcentage de taille, de façon à ce que la voiture puisse circuler sans difficulté sur le circuit que vous avez tracé.
\item
Quand le drapeau vert est cliqué, la voiture doit se mettre à la bonne taille.
\item
Quand le drapeau vert est cliqué, la voiture doit se mettre entre la ligne de départ et la ligne d'arrivée.
\end{enumerate} \vspace{6pt}
\item
\textbf{Déplacement de la voiture.} \hfill \renewcommand{\arraystretch}{2}
\begin{tabular}{|p{5mm}|}
\hline
\\
\hline
\end{tabular} \vspace{-6pt} \\
Compléter le script commençant par \og Quand le drapeau vert est cliqué \fg{} de façon à ce que :
\begin{enumerate}[a{)}]
\item
La voiture avance toute seule de 3 pas en 3 pas.
\item
Si la touche \og flèche droite \fg{} est pressée, la voiture doit tourner à droite de 10°.
\item
Si la touche \og flèche gauche \fg{} est pressée, la voiture doit tourner à gauche de 10°.
\end{enumerate} \vspace{6pt}
\item
\textbf{La voiture doit rester sur le circuit.} \hfill \renewcommand{\arraystretch}{2}
\begin{tabular}{|p{5mm}|}
\hline
\\
\hline
\end{tabular} \vspace{-6pt}
%\hfill \renewcommand{\arraystretch}{2}
%\begin{tabular}{l|p{5mm}|p{0.5cm}l|p{5mm}|}
%\cline{2-2} \cline{5-5}
%a)&&&b)&\\
%\cline{2-2} \cline{5-5}
%\end{tabular}
\begin{enumerate}[a{)}]
\item
Dès que la voiture sort du circuit, un message de fin de jeu doit s'afficher.
\item
Dès que la voiture sort du circuit, le jeu doit s'arrêter. (On pourra utiliser \og stop tout \fg{} du menu contrôle.)
\end{enumerate} \vspace{6pt}
\item
\textbf{Gestion de l'arrivée.} \hfill \renewcommand{\arraystretch}{2}
\begin{tabular}{|p{5mm}|}
\hline
\\
\hline
\end{tabular} \vspace{-6pt}
\begin{enumerate}[a{)}]
\item
Dès que la ligne d'arrivée est touchée, un message doit s'afficher pour dire que vous avez gagné.
\item
Dès que la ligne d'arrivée est touchée, le jeu s'arrête.
\end{enumerate} \vspace{6pt}
\item
\textbf{Gestion du chronomètre.} \hfill \renewcommand{\arraystretch}{2}
\begin{tabular}{|p{5mm}|}
\hline
\\
\hline
\end{tabular} \vspace{-6pt}
\begin{enumerate}[a{)}]
\item
Dans le menu Données, créer une variable que vous appellerez chrono.
\item
La variable chrono doit être caché, au départ du jeu.
\item
Lorsque la ligne de départ est touchée, réinitialiser le chronomètre. (menu capteur)
\item
Lorsque la ligne d'arrivée est touchée, stocker chronomètre dans la variable chrono.
\item
Lorsque la ligne d'arrivée est touchée, afficher la variable chrono.
\end{enumerate}
\item
\textbf{Mise en place de niveaux de difficulté.} \hfill \renewcommand{\arraystretch}{2}
\begin{tabular}{|p{5mm}|}
\hline
\\
\hline
\end{tabular} \vspace{-3pt} \\
L'utilisateur doit choisir un niveau de difficulté entre 1 et 4. \\
La voiture avance alors de 1,5 fois le niveau de difficulté.
\end{enumerate}

\end{document}


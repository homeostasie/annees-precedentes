\documentclass[12pt,a4paper]{article}
\usepackage[textheight=28cm,textwidth=19cm,headheight=1mm,footskip=1mm]{geometry}
\usepackage[UTF8]{inputenc}
\usepackage[T1]{fontenc}
\usepackage[french]{babel}
\usepackage{fourier}
\usepackage{amsmath,amsfonts,amssymb}
\usepackage{paralist,array}
\usepackage{pgf,tikz}
\usetikzlibrary{shapes,snakes,arrows,patterns}
\usepackage{graphicx,multicol}
\usepackage{varwidth}

\pagestyle{empty}

\newcommand{\ent}[2]{\par \noindent \textbf{NOM - Prénom :} \dotfill \vspace{9pt} \\ Tableur $-$ Fiche #1 : #2.}
%fin de la commande \ent (entête)

\newcounter {exercice}

\newcommand{\Exdeb}{\par \noindent \stepcounter{exercice} \textbf{\underline{{Exercice }\theexercice\,: }}\vspace{3 pt}}
%fin de la commande \exdeb (entête d'exercice 1)

\newcommand{\exdeb}{\vspace{36pt} \par \noindent \stepcounter{exercice} \textbf{\underline{{Exercice }\theexercice\,: }}\vspace{3 pt}}

\begin{document}
\ent{1}{Tracés de graphiques}{} \vspace{24 pt}\\
\textbf{\underline{Exercice :}} %\vspace{3pt} \\
\begin{enumerate}[1{)}]
\item
Ouvrir le tableur de LibreOffice. \\
Pour cela 
\item
Le tableau suivant donne l'évolution de la population d'un village depuis 1990. \vspace{3pt} \\
\renewcommand{\arraystretch}{1.5}
\begin{tabular}{|l|>{\centering}p{1cm}|>{\centering}p{1cm}|>{\centering}p{1cm}|>{\centering}p{1cm}|>{\centering}p{1cm}|>{\centering}p{1cm}|l}
\cline{1-7}
Année&1990&1995&2000&2005&2010&2015&$\ $\\
\cline{1-7}
Effectif&672&815&910&985&1 058&1 354&$\ $\\
\cline{1-7}
\end{tabular}
\vspace{6pt} \\
\textbf{Saisir ces données dans une feuille de calcul.} \vspace{3pt}
\item
Représenter l'évolution de la population par : 
\begin{enumerate}[a{)}]
\item
un diagramme en barre ;
\item
un diagramme circulaire ;
\item
une courbe. \vspace{6pt}
\end{enumerate}
\textbf{Appeler le professeur} \vspace{12pt} \\
\hfill
\renewcommand{\arraystretch}{2}
\begin{tabular}{l|p{5mm}|p{0.5cm}l|p{5mm}|p{0.5cm}l|p{5mm}|p{0.5cm}l|p{5mm}|p{0.5cm}l|p{5mm}|}
\cline{2-2} \cline{5-5} \cline{8-8} \cline{11-11} \cline{14-14}
1)&&&2)&&&3a)&&&3b)&&&3c)&\\
\cline{2-2} \cline{5-5} \cline{8-8} \cline{11-11} \cline{14-14}
\end{tabular} \vspace{6pt}
\item
Laquelle de ces représentations semble la mieux adaptée pour montrer l'évolution de la population de ce village ? Justifier.\vspace{6pt} \\
.\dotfill \vspace{6pt} \\
.\dotfill \vspace{6pt} \\
.\dotfill \vspace{6pt}
\item
Refaire la représentation graphique sous la forme d'une courbe \og lissée \fg{}.
\item
Explorer différents autres types de graphique. \vspace{24pt} \\
\renewcommand{\arraystretch}{2}
\begin{tabular}{l|p{5mm}|p{0.5cm}l|p{5mm}|p{0.5cm}l|p{5mm}|}
\cline{2-2} \cline{5-5} \cline{8-8}
4)&&&5)&&&6)&\\
\cline{2-2} \cline{5-5} \cline{8-8}
\end{tabular} %\vspace{6pt}
\end{enumerate}


\end{document}


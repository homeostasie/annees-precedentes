\documentclass[12pt,a4paper]{article}
\usepackage[textheight=28cm,textwidth=19cm,headheight=1mm,footskip=1mm]{geometry}
\usepackage[UTF8]{inputenc}
\usepackage[T1]{fontenc}
\usepackage[french]{babel}
\usepackage{fourier}
\usepackage{amsmath,amsfonts,amssymb}
\usepackage{paralist,array}
\usepackage{pgf,tikz}
\usetikzlibrary{shapes,snakes,arrows,patterns}
\usepackage{graphicx,multicol}
\usepackage{varwidth}

\pagestyle{empty}

\newcommand{\ent}[2]{\par \noindent \textbf{NOM - Prénom :} \dotfill \vspace{9pt} \\ Tableur $-$ Fiche #1 : #2.}
%fin de la commande \ent (entête)

\newcounter {exercice}

\newcommand{\Exdeb}{\par \noindent \stepcounter{exercice} \textbf{\underline{{Exercice }\theexercice\,: }}\vspace{3 pt}}
%fin de la commande \exdeb (entête d'exercice 1)

\newcommand{\exdeb}{\vspace{36pt} \par \noindent \stepcounter{exercice} \textbf{\underline{{Exercice }\theexercice\,: }}\vspace{3 pt}}

\begin{document}
\ent{2}{utilisation des fonctions statistiques} \vspace{24 pt}
\Exdeb \\
Nous avons demandé à 50 personnes de lancé un dé à six faces. Nous avons noté à chaque fois la face qui est sortie. Voici les résultats obtenus : \vspace{3pt} \\
\renewcommand{\arraystretch}{1.5}
\begin{tabular}{|l|l|l|l|l|l|l|l|l|l|l|l|l|l|l|l|l|l|l|l|l|l|l|l|l|}
\hline
1&6&2&2&2&1&2&5&1&3&6&1&1&3&5&4&3&5&4&3&2&3&1&2&5\\
\hline
6&3&4&6&2&2&3&4&1&1&2&3&3&5&6&3&3&6&5&3&6&6&6&2&2\\
\hline
\end{tabular} \vspace{6pt} 
\begin{enumerate}[1{)}]
\item
Ouvrir le fichier Tableur-Seance2
\item
\begin{enumerate}[a{)}]
\item
Dans la cellule B7, faire calculer la moyenne de la série statistique. \textit{On pourra utiliser la fonction \og moyenne \fg{}.}
\item
Dans la cellule B9, faire calculer la médiane de la série statistique. \textit{On pourra utiliser la fonction \og mediane \fg{}.}
\end{enumerate}
\item
Compléter, directement sur la feuille de calcul, le tableau suivant : \vspace{3pt} \\
\renewcommand{\arraystretch}{1.5}
\begin{tabular}{|l|>{\centering}p{1cm}|>{\centering}p{1cm}|>{\centering}p{1cm}|>{\centering}p{1cm}|>{\centering}p{1cm}|>{\centering}p{1cm}|l|}
\hline
\textbf{Numéro de la face}&\textbf{1}&\textbf{2}&\textbf{3}&\textbf{4}&\textbf{5}&\textbf{6}&\textbf{Total}\\
\hline
\textbf{Effectif}&&&&&&&\\
\hline
\textbf{Fréquences en \%}&&&&&&&\\
\hline
\end{tabular}
\vspace{12pt} \\
\underline{Remarque :} Pour remplir la ligne des effectifs on pourra se servir de la fonction NB.SI
\vspace{9pt} \\
\textbf{Appeler le professeur.}
\vspace{12pt} \\
\renewcommand{\arraystretch}{2}
\begin{tabular}{l|p{5mm}|p{0.5cm}l|p{5mm}|p{0.5cm}l|p{5mm}|p{0.5cm}l|p{5mm}|p{0.5cm}l|p{5mm}|}
\cline{2-2} \cline{5-5} \cline{8-8} \cline{11-11} \cline{14-14}
1)&&&2a)&&&2b)&&&3a)&&&3b)&\\
\cline{2-2} \cline{5-5} \cline{8-8} \cline{11-11} \cline{14-14}
\end{tabular} \vspace{12pt}
\item
Tracer le diagramme en bâtons.
\item
Tracer le diagramme circulaire (avec les pourcentages).
\item
Que pouvez-vous dire de la répartition des résultats ?. \vspace{6pt} \\
.\dotfill  \vspace{6pt} \\
.\dotfill  \vspace{6pt} \\
.\dotfill  \vspace{12pt} \\
\renewcommand{\arraystretch}{2}
\begin{tabular}{l|p{5mm}|p{0.5cm}l|p{5mm}|p{0.5cm}l|p{5mm}|}
\cline{2-2} \cline{5-5} \cline{8-8}
4)&&&5)&&&6)&\\
\cline{2-2} \cline{5-5} \cline{8-8}
\end{tabular}
\end{enumerate}
\exdeb \\
Pendant un an, un site internet de vente de chaussures de sport enregistre le nombre de commandes de ses clients.
\begin{enumerate}[1{)}]
\item
Recopier et compléter le tableau suivant : \hfill 
\renewcommand{\arraystretch}{2}
\begin{tabular}{|p{5mm}|}
\hline
\\
\hline
\end{tabular} \vspace{-3pt} \\
\renewcommand{\arraystretch}{1.5}
\begin{tabular}{|l|>{\centering}p{1cm}|>{\centering}p{1cm}|>{\centering}p{1cm}|>{\centering}p{1cm}|>{\centering}p{1cm}|l|}
\hline
\textbf{Nombre de commandes}&\textbf{1}&\textbf{2}&\textbf{3}&\textbf{4}&\textbf{5}&\textbf{Total}\\
\hline
\textbf{Effectif}&180&288&360&276&96&\\
\hline
\textbf{Fréquences en \%}&&&&&&\\
\hline
\end{tabular}
\vspace{6pt}
\item
Tracer le diagramme circulaire (avec les pourcentages).
\hfill
\renewcommand{\arraystretch}{2}
\begin{tabular}{|p{5mm}|}
\hline
\\
\hline
\end{tabular}
\item	
\begin{enumerate}[a{)}]
\item
Quel est le pourcentage de clients qui ont effectués dans l'année au moins deux commandes ? \vspace{6pt} \\
.\dotfill \vspace{12pt} 
\item
Quel est le pourcentage de clients qui ont effectués dans l'année  moins de trois commandes ? \vspace{6pt} \\
.\dotfill \vspace{12pt} 
\item
Est-il vrai que plus d'un tiers des clients de ce site ont effectué au moins quatre commandes dans l'année ? Justifier.\vspace{6pt} \\
.\dotfill \vspace{6pt} \\
.\dotfill \vspace{6pt} \\
.\dotfill \vspace{12pt} \\
\end{enumerate}
\renewcommand{\arraystretch}{2}
\begin{tabular}{l|p{5mm}|p{0.5cm}l|p{5mm}|p{0.5cm}l|p{5mm}|}
\cline{2-2} \cline{5-5} \cline{8-8}
3a)&&&3b)&&&3c)&\\
\cline{2-2} \cline{5-5} \cline{8-8}
\end{tabular}
\end{enumerate}

\end{document}


\documentclass[11pt]{article}
\usepackage{geometry} % Pour passer au format A4
\geometry{hmargin=1cm, vmargin=1cm} % 

% Page et encodage
\usepackage[T1]{fontenc} % Use 8-bit encoding that has 256 glyphs
\usepackage[english,francais]{babel} % Français et anglais
\usepackage[utf8]{inputenc} 

\usepackage{lmodern}
\setlength\parindent{0pt}

% Graphiques
\usepackage{graphicx, float}
\usepackage{tikz,tkz-tab}

% Maths et divers
\usepackage{amsmath,amsfonts,amssymb,amsthm,verbatim}
\usepackage{multicol,enumitem,url,eurosym,gensymb}

% Sections
\usepackage{sectsty} % Allows customizing section commands
\allsectionsfont{\centering \normalfont\scshape}

% Tête et pied de page

\usepackage{fancyhdr} 
\pagestyle{fancyplain} 

\fancyhead{} % No page header
\fancyfoot{}

\renewcommand{\headrulewidth}{0pt} % Remove header underlines
\renewcommand{\footrulewidth}{0pt} % Remove footer underlines

\newcommand{\horrule}[1]{\rule{\linewidth}{#1}} % Create horizontal rule command with 1 argument of height

%----------------------------------------------------------------------------------------
%	Début du document
%----------------------------------------------------------------------------------------

\begin{document}

\textbf{Nom, Prénom :} \hspace{8cm} \textbf{Classe :} \hspace{3cm} \textbf{Date :}\\


\begin{center}
  \textit{Si nous faisions tout ce dont nous sommes capables, nous nous surprendrions vraiment.}  - \textbf{Thomas Edison}
\end{center}

\textit{La présentation, la rédaction et le soin général apportés à la copie sont sur 2 points.}

\horrule{1px}
\vspace{-1cm}

\begin{multicols}{2}

\subsubsection*{ex1}

  Déterminer la décomposition en produit de facteurs premiers de :
  \begin{enumerate}
    \item[1a.] $24$
    \item[1b.] $308$
    \item[1c.] $1200$
    \item[1d.] $750$
    \item[1e.] $2772$
    \item[1f.] $980$
  \end{enumerate} 

\subsubsection*{ex2}
\begin{enumerate}
  \item[2] Donner la définition d'un nombre permier.
\end{enumerate} 

Les nombres suivants sont-ils premiers. Justifier.

\begin{enumerate}
  \item[2a.] $27$
  \item[2b.] $41$
  \item[2c.] $1233$
  \item[2d.] $115$
\end{enumerate} 

\subsubsection*{ex3}
\textit{Justifier.}
\begin{enumerate}
  \item[3a.] 369 est un multiple de 15
  \item[3b.] Donner la liste des diviseurs de 32.
  \item[3c.] Donner les 5 premiers multiples de 23.
  \item[3d.] 6 est-il un diviseur de 2018 ?
\end{enumerate} 

\subsubsection*{ex4}

Un fleuriste veut faire des bouquets de 6 roses. Il reçoit 688 roses.
\begin{enumerate}
  \item[4a.]  Combien fera-t-il de bouquets ?
  \item[4b.] Restera-t-il des roses ?
\end{enumerate}

Un fleuriste veut faire des bouquets de 4 roses rouges et 3 roses blanches. Il reçoit 402 roses rouges et 307 roses blanches.
\begin{enumerate}
  \item[4c.]  Combien fera-t-il de bouquets ?
  \item[4d.] Restera-t-il des roses ?
\end{enumerate}

\subsubsection*{ex5}

\begin{enumerate}
  \item[5a.]  Calculer l'image des nombres $0, 2, 5, 12, -1$ et $\dfrac{2}{5}$ par la fonction $f : x \rightarrow 2 x^2 + x + 10$
  \item[5b.] Calculer $g(-1), g(1)$ et $g(11)$ par $g(x) = (x+3)^2$
\end{enumerate}

\subsubsection*{ex6}

\begin{center}
  \begin{tabular}{|c||c|c|c|c|c|}
  \hline
    x  & 1 & 2 & 3 & 4 & 5 \\
  \hline
  m(x) & 12 & 4 & 2 & 5 & 4 \\
  \hline
\end{tabular}
\end{center}

\begin{enumerate}
  \item[6a.] Donner les images de 2, 3 et 4.
  \item[6b.] Doner les antécédents de 2, 3 et 4.
\end{enumerate}

\end{multicols}

\begin{multicols}{2}
  
\begin{tikzpicture}
  \draw plot coordinates {(1,0) (2,2) (3,0) (4,6) (5,4) (6,7) (7,4) (8,3) (9,0) (10,1)};
  \draw (1,0) grid (10,7);
  \foreach \y in {1,2,...,7} \draw(1,\y)node[left]{\y};
  \foreach \x in {1,2,...,10} \draw(\x,0)node[below]{\x};
  \end{tikzpicture}
  \subsubsection*{ex7}
\begin{enumerate}
  \item[7a.] Quelle est l'ordonnée du point de la courbe d'abscisse 4.
  \item[7b.] Lire les image de 1,3, 4 et 7.
  \item[7c.] Donner les antécédents de la courbe de 1, 4, 6 et 8.
\end{enumerate}


\end{multicols}


\end{document}

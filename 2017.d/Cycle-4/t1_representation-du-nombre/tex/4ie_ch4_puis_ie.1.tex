\documentclass[11pt]{article}
\usepackage{geometry} % Pour passer au format A4
\geometry{hmargin=1cm, vmargin=1cm} % 

% Page et encodage
\usepackage[T1]{fontenc} % Use 8-bit encoding that has 256 glyphs
\usepackage[english,francais]{babel} % Français et anglais
\usepackage[utf8]{inputenc} 

\usepackage{lmodern}
\setlength\parindent{0pt}

% Graphiques
\usepackage{graphicx, float}


% Maths et divers
\usepackage{amsmath,amsfonts,amssymb,amsthm,verbatim}
\usepackage{multicol,enumitem,url,eurosym,gensymb}

% Sections
\usepackage{sectsty} % Allows customizing section commands
\allsectionsfont{\centering \normalfont\scshape}

% Tête et pied de page

\usepackage{fancyhdr} 
\pagestyle{fancyplain} 

\fancyhead{} % No page header
\fancyfoot{}

\renewcommand{\headrulewidth}{0pt} % Remove header underlines
\renewcommand{\footrulewidth}{0pt} % Remove footer underlines

\newcommand{\horrule}[1]{\rule{\linewidth}{#1}} % Create horizontal rule command with 1 argument of height

%----------------------------------------------------------------------------------------
%	Début du document
%----------------------------------------------------------------------------------------

\begin{document}

\textbf{Nom, Prénom :} \hspace{8cm} \textbf{Classe :} \hspace{3cm} \textbf{Date :}\\


\begin{center}
  \textit{Si nous faisions tout ce dont nous sommes capables, nous nous surprendrions vraiment.}  - \textbf{Thomas Edison}
\end{center}

\textit{La présentation, la rédaction et le soin général apportés à la copie sont sur 2 points.}
\begin{itemize}
\item \textsc{Calculer} : 
\item \textsc{Raisonner} : 
\item \textsc{Communiquer} : 
\item \textsc{Note} : 
\end{itemize}

\horrule{1px}
\vspace{-1cm}

\subsubsection*{Calculer - \textit{(/4)}}
\begin{multicols}{2}
Calculer les nombres suivants :\\
\begin{enumerate}
  \item[a.] $ 2^{10} $
  \item[b.] $ 10^4 $ 
  \item[c.] $ (-2)^8 $
  \item[d.] $ 7^4 + 89 $
  \item[e.] $ \dfrac{1}{5^4} $
  \item[f.] $ 2^{32} \times \dfrac{1}{32^2} $
  \item[g.] $ 3^{32} + 1$
  \item[h.] $ 3^{32}$
  \item[i.] $ 3^{-33}$
  \item[j.] $ \dfrac{1}{4^{-3}}$
\end{enumerate}
\end{multicols}

\subsubsection*{Ordonner - \textit{(/3)}}
Classer les deux séries de nombres suivant dans l'ordre croissant : du plus petit au plus grand. Les calculs ne sont pas demandés.
\begin{multicols}{2}

\begin{enumerate}
  \item[1a.] $ 2^{20} $
  \item[1b.] $ 5.2 \times 10^{16} $ 
  \item[1b.] $ 9.2 \times 10^{15} $ 
  \item[1d.] $ -6.4 \times 10^{18} $ 
  \item[1e.] $ 196.4 \times 10^{-8} $ 
\end{enumerate}
\begin{enumerate}
  \item[2a.] $ 20000 m $
  \item[2b.] $ 4576 km $ 
  \item[2c.] $ 2 \times 10^{14} mm $
  \item[2d.] 2 millions de m
  \item[2e.] $ 2 \times 10^{-14} km $
\end{enumerate}

\end{multicols}

\subsubsection*{Problème 1 - Terre - Mars  - \textit{(/3)}}
\begin{enumerate}
\item[1.] La lumière se propage à une vitesse de $3 \times 10^8 m/s$. Un rayon partant de la planete Mars arrive sur Terre au bout de $6 \text{ min } 20s$. \\
Quelle est la distance Terre - Mars ?
\end{enumerate}

\subsubsection*{Problème 2 - $CO_2$  - \textit{(/4)}}

\begin{enumerate}
  \item[2.] Une molécule de dioxyde de carbone est composée d'un atome de carbone et de deux atomes d'oxygène. La masse d'un atome de carbone est $2 \times 10^{-26}kg$ et la masse d'un atome d'oxygène est $1.8 \times 10^{-26}kg$. \\
  Combien trouve-t-on de molécule de dioxyde carbone dans 2kg ?
\end{enumerate}

\subsubsection*{Problème 3 -Grain de sables VS étoiles  - \textit{(/4)}}
\begin{enumerate}
\item[3.] Il y a $2^{32}$ galaxies dans l'univers. Dans chaque galaxie, il y a $2^{24}$ planètes.\\
Dans $1m^3$ de sable, il y a 5 millions de grains de sable. Or, il y a $2.56 \times 10 ^{10} m^3$ de sables à Feyzin.\\
Calculer puis comparer ses deux nombres.

\end{enumerate}

\textbf{Bonus : } \textit{Après avoir écrit votre modèle de calculatrice. Quel est le plus grand nombre que votre calculatrice peut calculer ? Quel est la précision de votre calculatrice : le nombre de chiffre sur lequel elle travaille ? Donner l'écriture binaire de 15. Donner l'écriture décimale de 1001.}

\end{document}

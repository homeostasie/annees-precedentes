\documentclass[11pt]{article}
\usepackage{geometry} % Pour passer au format A4
\geometry{hmargin=1cm, vmargin=1cm} % 

% Page et encodage
\usepackage[T1]{fontenc} % Use 8-bit encoding that has 256 glyphs
\usepackage[english,francais]{babel} % Français et anglais
\usepackage[utf8]{inputenc} 

\usepackage{lmodern}
\setlength\parindent{0pt}

% Graphiques
\usepackage{graphicx, float}


% Maths et divers
\usepackage{amsmath,amsfonts,amssymb,amsthm,verbatim}
\usepackage{multicol,enumitem,url,eurosym}

% Sections
\usepackage{sectsty} % Allows customizing section commands
\allsectionsfont{\centering \normalfont\scshape}

% Tête et pied de page

\usepackage{fancyhdr} 
\pagestyle{fancyplain} 

\fancyhead{} % No page header
\fancyfoot{}

\renewcommand{\headrulewidth}{0pt} % Remove header underlines
\renewcommand{\footrulewidth}{0pt} % Remove footer underlines

\newcommand{\horrule}[1]{\rule{\linewidth}{#1}} % Create horizontal rule command with 1 argument of height

%----------------------------------------------------------------------------------------
%	Début du document
%----------------------------------------------------------------------------------------

\begin{document}

%----------------------------------------------------------------------------------------
% RE-DEFINITION
%----------------------------------------------------------------------------------------
% MATHS
%-----------

\newtheorem{Definition}{Définition}
\newtheorem{Theorem}{Théorème}
\newtheorem{Proposition}{Propriété}

% MATHS
%-----------
\renewcommand{\labelitemi}{$\bullet$}
\renewcommand{\labelitemii}{$\circ$}
%----------------------------------------------------------------------------------------
%	Titre
%----------------------------------------------------------------------------------------

\setlength{\columnseprule}{1pt}

\horrule{2px}
\section{Théorème de Pythagore}
\horrule{2px}


  \subsection{Démonstration Chinoise}

\begin{multicols}{2}

  \begin{figure}[H]
    \centering
    \includegraphics[width=0.6\linewidth]{sources/ch2-thm-pythagore/1_pytha-chine.pdf}
  \end{figure}

  \Proposition{Un triangle rectangle de petits côtés 3cm et 4cm a un grand côté qui mesure 5cm. Ce plus grand côté s'appelle l'hypoténuse.}

  \paragraph{Démonstration} : Soient $A$ l'aire du carré jaune et $T$ l'aire des triangles de côtés 3 et 4. 
  \begin{eqnarray*}
    A &=&  4 \times T + 1\\
    A &=&  4 \times \dfrac{4 \times 3}{2} + 1\\
    A &=& 25
  \end{eqnarray*}
  Le côté du carré est 5.

\end{multicols}

\horrule{1px}

  \subsection{Nouvelles opérations : carré et racine carré}

  \Proposition{L'aire d'un carré de côté $a$ s'écrit $a^2$ et est égal à $a \times a$. On dit $a$ au carré ou $a$ puissance 2.}

  \Proposition{Le côté d'un carré d'aire b s'écrit $\sqrt{b}$. C'est l'unique nombre tel que $\sqrt{b}^2 = b$. On dit racine carré de 2. Il est facile de donner une valeur approchée de ce nombre à la calculatrice. Pour une valeur exacte, on conversera l'écriture avec $\sqrt{\phantom{abc}}$.}
 
 \horrule{1px}
   
\begin{multicols}{2}

  \subsection{Énoncés}

  \Theorem{Énoncé géométrique}\\
  
  Dans un triangle rectangle l'aire du carré issue de l'hypoténuse est égal à la somme des aires des carrés issues des deux autres côtés.\\

  \Theorem{Énoncé général}\\
  
  Dans un triangle recctangle, l'hypoténuse au carré est égal à la somme des autres côtés au carré. \\

  \Theorem{Énoncé exercices}\\
  
  \begin{figure}[H]
    \centering
    \includegraphics[width=0.5\linewidth]{sources/ch2-thm-pythagore/1_pytha-thm.pdf}
  \end{figure}
  
  \begin{enumerate}
  \item[1.] Dans le triangle ABC rectangle en A.
  \item[2.] D'après le théorème de Pythagore.
  \item[3.] $BC^2 = AB^2 + AC^2$
  \end{enumerate}


\end{multicols}

\newpage

\subsection{Modélisation}

\Proposition{Si on a dispose d'un triangle rectangle et qu'on connait la longueur de deux côtés, alors il est possible de calculer la longueur du troisième côté.}

\subsubsection{Recherche de l'hypoténuse }

\begin{multicols}{2}

  \begin{figure}[H]
    \centering
    \includegraphics[width=0.9\linewidth]{sources/ch2-thm-pythagore/1_pytha-re-g.pdf}
  \end{figure}

  Dans le triangle MNO rectangle en O.\\
  D'après le théorème de Pythagore.
   \begin{eqnarray*}
    MN^2 &=& MO^2 + NO^2 \\
    MN^2 &=& 20^2 + 32^2 \\
    MN^2 &=&  1424\\
    MN   &=& \sqrt{1424} \\
    MN   &\approx& 37.7 
  \end{eqnarray*}
  
  La longueur MN mesure environ 27.7cm. Elle est plus grande que les deux autres côtés.
\end{multicols}


\subsubsection{Recherche d'un petit côté}

\begin{multicols}{2}

  \begin{figure}[H]
    \centering
    \includegraphics[width=0.9\linewidth]{sources/ch2-thm-pythagore/1_pytha-re-p.pdf}
  \end{figure}

  Dans le triangle MNO rectangle en O.\\
  D'après le théorème de Pythagore.
  \begin{eqnarray*}
    MN^2 &=& MO^2 + NO^2 \\
    MO^2 &=& MN^2 - NO^2 \\
    MO^2 &=& 40^2 - 12^2 \\
    MO^2 &=& 1456\\
    MO   &=& \sqrt{1456} \\
    MO   &=& 38
  \end{eqnarray*}
  
  La longueur MO mesure 38cm. Elle n'est pas le plus grand côté.
\end{multicols}
\end{document}

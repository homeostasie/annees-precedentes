\documentclass[11pt]{article}
\usepackage{geometry} % Pour passer au format A4
\geometry{hmargin=1cm, vmargin=1cm} % 

% Page et encodage
\usepackage[T1]{fontenc} % Use 8-bit encoding that has 256 glyphs
\usepackage[english,french]{babel} % Français et anglais
\usepackage[utf8]{inputenc} 

\usepackage{lmodern}
\setlength\parindent{0pt}

% Graphiques
\usepackage{graphicx,float,grffile}

% Maths et divers
\usepackage{amsmath,amsfonts,amssymb,amsthm,verbatim}
\usepackage{multicol,enumitem,url,eurosym,gensymb}

% Sections
\usepackage{sectsty} % Allows customizing section commands
\allsectionsfont{\centering \normalfont\scshape}

% Tête et pied de page

\usepackage{fancyhdr} 
\pagestyle{fancyplain} 

\fancyhead{} % No page header
\fancyfoot{}

\renewcommand{\headrulewidth}{0pt} % Remove header underlines
\renewcommand{\footrulewidth}{0pt} % Remove footer underlines

\newcommand{\horrule}[1]{\rule{\linewidth}{#1}} % Create horizontal rule command with 1 argument of height

%----------------------------------------------------------------------------------------
%	Début du document
%----------------------------------------------------------------------------------------

\begin{document}

%----------------------------------------------------------------------------------------
% RE-DEFINITION
%----------------------------------------------------------------------------------------
% MATHS
%-----------

\newtheorem{Definition}{Définition}
\newtheorem{Theorem}{Théorème}
\newtheorem{Proposition}{Propriété}

% MATHS
%-----------
\renewcommand{\labelitemi}{$\bullet$}
\renewcommand{\labelitemii}{$\circ$}
%----------------------------------------------------------------------------------------
%	Titre
%----------------------------------------------------------------------------------------

\setlength{\columnseprule}{0pt}

\section*{Comment attaquer le brevet de maths}
\horrule{2px} 
\textbf{Règle 1 : Avoir son matériel : calculatrice, compas, equerre, rapporteur, crayon à papier.}


\section*{Conseils de base}

\begin{itemize}
\item Regarder le nombre d'exercices. Diviser 2h = 120min par ce nombre pour avoir une idée moyenne du temps disponible par exercices.
\item Commencer par les exercices qu'on pense savoir traiter \textit{facilement}. 
\item Quand on change d'exercice, on change de page.
\item Ce n'est pas parce qu'on ne fait pas la première question d'un exercice qu'on ne peut pas faire les autres.
\item Quand onn fait un calcul à la calculatrice, on l'écrit aussi sur la copie.
\item Quand on conclut une question, on fait une phrase réponse.
\end{itemize}

\section*{Conseils de maths}

Ces dernières années, on distringue trois types d'exercices.
\begin{itemize}
\item Exercice de maths pures. (Entre 1 et 2). Ce sont les exercices qui font appel aux théorèmes de maths. Ce sont les "plus durs" car il faut les bonnes connaissances.
\item Exercice technique. Souvent 1, souvent de type QCM. Il faut bien lire la consigne : 1 seule réponse juste ou plusieurs réponses possibles. Justifier ou pas. 
\item Problème. Le reste, 4/5. Il faut bien lire les questions puis les documents. Souvent des formules sont données. Des questions sont parfois très simples. Il faut par contre faire attention aux unités et autres petits pièges.                                                                                          
\end{itemize}

\horrule{2px} 

\section*{Comment attaquer le brevet de maths}
\horrule{2px} 
\textbf{Règle 1 : Avoir son matériel : calculatrice, compas, equerre, rapporteur, crayon à papier.}


\section*{Conseils de base}

\begin{itemize}
\item Regarder le nombre d'exercices. Diviser 2h = 120min par ce nombre pour avoir une idée moyenne du temps disponible par exercices.
\item Commencer par les exercices qu'on pense savoir traiter \textit{facilement}. 
\item Quand on change d'exercice, on change de page.
\item Ce n'est pas parce qu'on ne fait pas la première question d'un exercice qu'on ne peut pas faire les autres.
\item Quand onn fait un calcul à la calculatrice, on l'écrit aussi sur la copie.
\item Quand on conclut une question, on fait une phrase réponse.
\end{itemize}

\section*{Conseils de maths}

Ces dernières années, on distringue trois types d'exercices.
\begin{itemize}
\item Exercice de maths pures. (Entre 1 et 2). Ce sont les exercices qui font appel aux théorèmes de maths. Ce sont les "plus durs" car il faut les bonnes connaissances.
\item Exercice technique. Souvent 1, souvent de type QCM. Il faut bien lire la consigne : 1 seule réponse juste ou plusieurs réponses possibles. Justifier ou pas. 
\item Problème. Le reste, 4/5. Il faut bien lire les questions puis les documents. Souvent des formules sont données. Des questions sont parfois très simples. Il faut par contre faire attention aux unités et autres petits pièges.                                                                                          
\end{itemize}

\end{document}

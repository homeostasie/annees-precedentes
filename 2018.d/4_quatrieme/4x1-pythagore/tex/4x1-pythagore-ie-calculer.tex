\documentclass[12pt]{article}
\usepackage{geometry} % Pour passer au format A4
\geometry{hmargin=1cm, vmargin=1cm} % 

% Page et encodage
\usepackage[T1]{fontenc} % Use 8-bit encoding that has 256 glyphs
\usepackage[english,french]{babel} % Français et anglais
\usepackage[utf8]{inputenc} 

\usepackage{lmodern}
\setlength\parindent{0pt}

% Graphiques
\usepackage{graphicx,float,grffile}

% Maths et divers
\usepackage{amsmath,amsfonts,amssymb,amsthm,verbatim}
\usepackage{multicol,enumitem,url,eurosym,gensymb}

% Sections
\usepackage{sectsty} % Allows customizing section commands
\allsectionsfont{\centering \normalfont\scshape}

% Tête et pied de page

\usepackage{fancyhdr} 
\pagestyle{fancyplain} 

\fancyhead{} % No page header
\fancyfoot{}

\renewcommand{\headrulewidth}{0pt} % Remove header underlines
\renewcommand{\footrulewidth}{0pt} % Remove footer underlines

\newcommand{\horrule}[1]{\rule{\linewidth}{#1}} % Create horizontal rule command with 1 argument of height

%----------------------------------------------------------------------------------------
%	Début du document
%----------------------------------------------------------------------------------------

\begin{document}

%----------------------------------------------------------------------------------------
% RE-DEFINITION
%----------------------------------------------------------------------------------------
% MATHS
%-----------

\newtheorem{Definition}{Définition}
\newtheorem{Theorem}{Théorème}
\newtheorem{Proposition}{Propriété}

% MATHS
%-----------
\renewcommand{\labelitemi}{$\bullet$}
\renewcommand{\labelitemii}{$\circ$}
%----------------------------------------------------------------------------------------
%	Titre
%----------------------------------------------------------------------------------------

\setlength{\columnseprule}{1pt}

\section*{ie 1 - Carré et racine carré}
\begin{center}
\textit{Philip K. Dick - La réalité, c'est ce qui refuse de disparaître quand on cesse d'y croire.}
\end{center}
\horrule{2px}

\begin{multicols}{2}
\begin{itemize}
\item S'organiser : Avoir sa calculatrice.
\item Communiquer : Soin général.
\item Calculer.
\item Connaissances : Opération Carré.
\item Connaissances : Opération Racine carré.
\end{itemize}
\end{multicols}

\textit{Calculer à l'aide de la calculatrice et arrondir si besoin au deuxième chiffre après la virgule.}

\begin{multicols}{2}

\begin{itemize}
\item $2^2$
\item $9^2$
\item $15^2$
\item $7.2^2$
\item $3^2 + 4^2$
\item $11^2 -10^2$
\item $(8+8)^2$
\item $1001^2$
\item $\left( \dfrac{3}{4} \right)^2$
\item $\sqrt{4}^2$
\end{itemize}

\begin{itemize}
\item $\sqrt{9}$
\item $\sqrt{25}$
\item $\sqrt{42}$
\item $\sqrt{54.5}$
\item $\sqrt{1+9}$
\item $\sqrt{1^2 +10^2}$
\item $\sqrt{5} + \sqrt{5}$
\item $\sqrt{3^2 + 4^2}$
\item $\sqrt{10} - 2$
\item $\sqrt{-5}$
\end{itemize}

\end{multicols}

\vspace{0.5cm}

\section*{ie 1 - Carré et racine carré}
\begin{center}
\textit{Philip K. Dick - La réalité, c'est ce qui refuse de disparaître quand on cesse d'y croire.}
\end{center}
\horrule{2px}

\begin{multicols}{2}
\begin{itemize}
\item S'organiser : Avoir sa calculatrice.
\item Communiquer : Soin général.
\item Calculer.
\item Connaissances : Opération Carré.
\item Connaissances : Opération Racine carré.
\end{itemize}
\end{multicols}

\textit{Calculer à l'aide de la calculatrice et arrondir si besoin au deuxième chiffre après la virgule.}

\begin{multicols}{2}

\begin{itemize}
\item $2^2$
\item $9^2$
\item $14^2$
\item $6.2^2$
\item $4^2 + 5^2$
\item $12^2 -11^2$
\item $(6+6)^2$
\item $1002^2$
\item $\left( \dfrac{3}{5} \right)^2$
\item $\sqrt{5}^2$
\end{itemize}


\begin{itemize}
\item $\sqrt{4}$
\item $\sqrt{100}$
\item $\sqrt{32}$
\item $\sqrt{54.1}$
\item $\sqrt{12+8}$
\item $\sqrt{1^2 +11^2}$
\item $\sqrt{6} + \sqrt{6}$
\item $\sqrt{3^2 + 4^2}$
\item $\sqrt{9} - 2$
\item $\sqrt{-4}$
\end{itemize}

\end{multicols}





\end{document}

\documentclass[12pt]{article}
\usepackage{geometry} % Pour passer au format A4
\geometry{hmargin=1cm, vmargin=1cm} % 

% Page et encodage
\usepackage[T1]{fontenc} % Use 8-bit encoding that has 256 glyphs
\usepackage[english,french]{babel} % Français et anglais
\usepackage[utf8]{inputenc} 

\usepackage{lmodern}
\setlength\parindent{0pt}

% Graphiques
\usepackage{graphicx,float,grffile}

% Maths et divers
\usepackage{amsmath,amsfonts,amssymb,amsthm,verbatim}
\usepackage{multicol,enumitem,url,eurosym,gensymb}

% Sections
\usepackage{sectsty} % Allows customizing section commands
\allsectionsfont{\centering \normalfont\scshape}

% Tête et pied de page

\usepackage{fancyhdr} 
\pagestyle{fancyplain} 

\fancyhead{} % No page header
\fancyfoot{}

\renewcommand{\headrulewidth}{0pt} % Remove header underlines
\renewcommand{\footrulewidth}{0pt} % Remove footer underlines

\newcommand{\horrule}[1]{\rule{\linewidth}{#1}} % Create horizontal rule command with 1 argument of height

%----------------------------------------------------------------------------------------
%	Début du document
%----------------------------------------------------------------------------------------

\begin{document}

%----------------------------------------------------------------------------------------
% RE-DEFINITION
%----------------------------------------------------------------------------------------
% MATHS
%-----------

\newtheorem{Definition}{Définition}
\newtheorem{Theorem}{Théorème}
\newtheorem{Proposition}{Propriété}

% MATHS
%-----------
\renewcommand{\labelitemi}{$\bullet$}
\renewcommand{\labelitemii}{$\circ$}
%----------------------------------------------------------------------------------------
%	Titre
%----------------------------------------------------------------------------------------

\setlength{\columnseprule}{1pt}

\horrule{2px}
\section*{Chapitre 2 - Trigonometrie}
\horrule{2px}

\subsection*{Programme des exercices}


\subsection*{Triangle}

\begin{itemize}
	\item \textit{On rappelle la propriété : la somme des angles dans un triangle fait $180 \degree$.}
	\item \textit{On signale un angle dans un triangle, on écrit le nom des côtés : hypoténuse, adjacent, opposé.}
\end{itemize}

\subsection*{Modéliser}

\begin{itemize}
	\item \textit{On donne des situations de problèmes, on dit si on peut utiliser la trigonométrie.}
	\item \textit{On donne un angle, trois côtés, on écrit les trois relations trigonométriques.}
	\item \textit{On donne un angle, un côté, on cherche un autre côté, on dit quelle relation trigonométrique utiliser.}
\end{itemize}


\subsection*{Calculer}

\begin{itemize}
	\item \textit{On calcule des cosinus, sinus, tangente d'angle remarquable.}
	\item \textit{On calcule des côtés avec un angle et un côté.}
\end{itemize}

\subsection*{Problèmes}

\end{document}

\documentclass[12pt]{article}
\usepackage{geometry} % Pour passer au format A4
\geometry{hmargin=1cm, vmargin=1cm} % 

% Page et encodage
\usepackage[T1]{fontenc} % Use 8-bit encoding that has 256 glyphs
\usepackage[english,french]{babel} % Français et anglais
\usepackage[utf8]{inputenc} 

\usepackage{lmodern}
\setlength\parindent{0pt}

% Graphiques
\usepackage{graphicx,float,grffile}

% Maths et divers
\usepackage{amsmath,amsfonts,amssymb,amsthm,verbatim}
\usepackage{multicol,enumitem,url,eurosym,gensymb}

% Sections
\usepackage{sectsty} % Allows customizing section commands
\allsectionsfont{\centering \normalfont\scshape}

% Tête et pied de page

\usepackage{fancyhdr} 
\pagestyle{fancyplain} 

\fancyhead{} % No page header
\fancyfoot{}

\renewcommand{\headrulewidth}{0pt} % Remove header underlines
\renewcommand{\footrulewidth}{0pt} % Remove footer underlines

\newcommand{\horrule}[1]{\rule{\linewidth}{#1}} % Create horizontal rule command with 1 argument of height

%----------------------------------------------------------------------------------------
%	Début du document
%----------------------------------------------------------------------------------------

\begin{document}

%----------------------------------------------------------------------------------------
% RE-DEFINITION
%----------------------------------------------------------------------------------------
% MATHS
%-----------

\newtheorem{Definition}{Définition}
\newtheorem{Theorem}{Théorème}
\newtheorem{Proposition}{Propriété}

% MATHS
%-----------
\renewcommand{\labelitemi}{$\bullet$}
\renewcommand{\labelitemii}{$\circ$}
%----------------------------------------------------------------------------------------
%	Titre
%----------------------------------------------------------------------------------------

\setlength{\columnseprule}{1pt}

\horrule{2px}
\section*{Chapitre 3 - Puissances}
\horrule{2px}


\subsection*{Problème 1 – Distance Professeurs - Mars}


La distance Terre-Mars est 76 millions de kilomètres.
M. Lafond mesure 1m65.

\begin{itemize}
\item[1.] Combien faut-il de \textsc{M. Lafond} pour atteindre Mars ?
\item[2.] Il faut 40 milliards de \textsc{M. Allilouche} pour atteindre Mars. Quelle taille fait-il ?
\end{itemize}



\subsection*{Problème 2 – Casa de Papel}


\textit{\og El Professeur \fg{} } vient de dérober 2 milliards d’euros. \\
Les billets de banque ont une épaisseur de $80 \times 10^{-6} m$. (On dit 80 micromètres)

\begin{itemize}
\item[1.] Quelle hauteur atteindrait une pile de billets de banque de 50 \euro{} représentant cette somme ?
\item[2.] Quelle hauteur atteindrait une pile de billets de banque de 10 \euro{} représentant cette somme ?
\end{itemize}



\subsection*{Problème 3 - Distance Collège- Soleil}

La lumière se propage à une vitesse de $3 \times 10^8 m/s$. \\
Un rayon partant du Soleil arrive au collège Frédéric Mistral au bout de 8 min 20s.


\begin{itemize}
\item[1.] Convertir 8 min 20s en un temps uniquement en seconde.
\item[2.] Quelle est la distance Collège- Soleil ? 
\end{itemize}


\end{document}

\documentclass[12pt]{article}
\usepackage{geometry} % Pour passer au format A4
\geometry{hmargin=1cm, vmargin=1cm} % 

% Page et encodage
\usepackage[T1]{fontenc} % Use 8-bit encoding that has 256 glyphs
\usepackage[english,french]{babel} % Français et anglais
\usepackage[utf8]{inputenc} 

\usepackage{lmodern}
\setlength\parindent{0pt}

% Graphiques
\usepackage{graphicx,float,grffile}
\usepackage{tikz}

% Maths et divers
\usepackage{amsmath,amsfonts,amssymb,amsthm,verbatim}
\usepackage{multicol,enumitem,url,eurosym,gensymb}
\usepackage{multido}

% Sections
\usepackage{sectsty} % Allows customizing section commands
\allsectionsfont{\centering \normalfont\scshape}

% Tête et pied de page
\usepackage{fancyhdr} 
\pagestyle{fancyplain} 
\fancyhead{} % No page header
\fancyfoot{}

\renewcommand{\headrulewidth}{0pt} % Remove header underlines
\renewcommand{\footrulewidth}{0pt} % Remove footer underlines

\newcommand{\horrule}[1]{\rule{\linewidth}{#1}} % Create horizontal rule command with 1 argument of height

\newcommand{\Pointilles}[1]{%
  \par\nobreak
  \noindent\rule{0pt}{1.5\baselineskip}% Provides a larger gap between the preceding paragraph and the dots
  \multido{}{#1}{\noindent\makebox[\linewidth]{\dotfill}\endgraf}% ... dotted lines ...
  \bigskip% Gap between dots and next paragraph
}
%----------------------------------------------------------------------------------------
%	Début du document
%----------------------------------------------------------------------------------------

\begin{document}

%----------------------------------------------------------------------------------------
% RE-DEFINITION
%----------------------------------------------------------------------------------------
% MATHS
%-----------

\newtheorem{Definition}{Définition}
\newtheorem{Theorem}{Théorème}
\newtheorem{Proposition}{Propriété}

% MATHS
%-----------
\renewcommand{\labelitemi}{$\bullet$}
\renewcommand{\labelitemii}{$\circ$}
%----------------------------------------------------------------------------------------
%	Titre
%----------------------------------------------------------------------------------------

\setlength{\columnseprule}{1pt}

\textbf{Nom, Prénom : }

\section*{ie 4 - Puissances}
\begin{center}
  \textit{Théophraste - La plus coûteuse des dépenses, c’est la perte de temps.}
\end{center}
\horrule{2px}

\subsection*{Ex1 - Calculer}

\begin{multicols}{3}
  \begin{itemize}
  \item[a =] $7^2 + 1 =  \dotfill $
  \item[b =] $10^{5+3} =  \dotfill $
  \item[c =] $3^4 \times 4^3 =  \dotfill $
  \item[d =] $(\sqrt{2})^{20} =  \dotfill $
  \item[e =] $1^0 \times 2^1 \times 3^2 =  \dotfill $
  \item[f =] $123^0 =  \dotfill $
  \end{itemize}

\end{multicols}

\begin{multicols}{2}

  $G = \dfrac{0,35 \times 10^{-3} \times 2,7 \times 10^{2}}{900 \times (10^5)^2} =  \dotfill $\\
  $H = \dfrac{80 \times 10^{-10} \times 0,18 \times 10^{-3}}{3,6 \times (10^{-7})^5} =  \dotfill $ \\
  $I = \dfrac{1800 \times 10^{9} \times 3600 \times 10^{8}}{1,2 \times (10^{-7})^2} =  \dotfill $\\
  $J = \dfrac{490 \times 10^{3} \times 3,6 \times 10^{2}}{25,2 \times (10^{10})^5} =  \dotfill $ 

\end{multicols}

\subsection*{Ex2 - Raisonner}

Compléter par un nombre de la forme $a^n$ avec $a$ et $n$ entiers :

\begin{multicols}{4}
  \begin{enumerate}
  \item[1.] $(11^{10})^{8} = \dotfill$
  \item[2.] $5^{4}  \times  3^{4}  =  \dotfill$
  \item[3.] $5^{6} \times 5^{5} = \dotfill$
  \item[4.] $\dfrac{11^{11}}{11^{6}} = \dotfill$
  \item[5.] $\dfrac{11^{9}}{11^{6}} = \dotfill$
  \item[6.] $5^{6} \times 5^{3} = \dotfill$
  \item[7.] $5^{7}  \times  7^{7}  =  \dotfill$
  \item[8.] $(10^{10})^{7} = \dotfill$
  \item[9.] $\dfrac{5^{11}}{5^{4}} = \dotfill$
  \item[10.] $(11^{3})^{5} = \dotfill$
  \item[11.] $(9^{8})^{7} = \dotfill$
  \item[12.] $9^{5} \times 9^{2} = \dotfill$
  \item[13.] $7^{4}  \times  2^{4}  =  \dotfill$
  \item[14.] $3^{7}  \times  5^{7}  =  \dotfill$
  \item[15.] $5^{9} \times 5^{7} = \dotfill$
  \item[16.] $\dfrac{11^{11}}{11^{8}} = \dotfill$
  \end{enumerate}
\end{multicols}


\subsection*{Ex3 - Raisonner}
Compléter par le nombre qui convient :

\begin{multicols}{3}

  \begin{enumerate}
  \item[1.] $6{,}098 \times \dotfill = 0{,}000\,006\,098$
  \item[2.] $6\,602 = 6{,}602 \times \dotfill$
  \item[3.] $0{,}004\,027 = 4{,}027 \times \dotfill$
  \item[4.] $0{,}060\,04 = 6{,}004 \times \dotfill$
  \item[5.] $610{,}9 = 6{,}109 \times \dotfill$
  \item[6.] $0{,}150\,2 = 1{,}502 \times \dotfill$
  \item[7.] $0{,}000\,630\,7 = 6{,}307 \times \dotfill$
  \item[8.] $7{,}306 \times \dotfill = 7\,306$
  \item[9.] $6{,}034 \times \dotfill = 60\,340\,000$
  \item[10.] $50{,}09 = 5{,}009 \times \dotfill$
  \item[11.] $0{,}206\,9 = 2{,}069 \times \dotfill$
  \item[12.] $40\,170 = 4{,}017 \times \dotfill$
  \end{enumerate}
\end{multicols}


\subsection*{Ex4 - Sciences}

Avec vos mots et un schéma, expliquer pourquoi à la calculatrice : $1.6 + 10^{-16} = 1.6$ 
\Pointilles{8}

\newpage

\subsection*{Ex5 - Problème}

\textit{\og El Professeur \fg{} } vient de dérober 4 milliards d’euros. \\
Les billets de banque ont une épaisseur de $60 \times 10^{-6} m$. (On dit 60 micromètres)

\begin{itemize}
\item[1.] Quelle hauteur atteindrait une pile de billets de banque de 50 \euro{} représentant cette somme ?
\item[2.] Quelle hauteur atteindrait une pile de billets de banque de 10 \euro{} représentant cette somme ?
\end{itemize}

\Pointilles{20}

\subsection*{Bonus}

\textit{Dessinez un chat. Racontez une blague avec un chat.}


%----------------------------------------------------------------------------------------
%
% SUJET 2
%
%----------------------------------------------------------------------------------------

\newpage

%----------------------------------------------------------------------------------------
%
% SUJET 2
%
%----------------------------------------------------------------------------------------
\textbf{Nom, Prénom : }

\section*{ie 4 - Puissances}
\begin{center}
  \textit{Théophraste - La plus coûteuse des dépenses, c’est la perte de temps.}
\end{center}
\horrule{2px}

\subsection*{Ex1 - Calculer}

\begin{multicols}{3}
  \begin{itemize}
  \item[a =] $7^2 + 1 =  \dotfill $
  \item[b =] $10^{4+3} =  \dotfill $
  \item[c =] $4^5 \times 5^4 =  \dotfill $
  \item[d =] $(\sqrt{2})^{20} =  \dotfill $
  \item[e =] $1^0 + 2^1 + 3^2 =  \dotfill $
  \item[f =] $1234^0 =  \dotfill $
  \end{itemize}

\end{multicols}

\begin{multicols}{2}

  $G = \dfrac{1,35 \times 10^{-4} \times 1,7 \times 10^{2}}{600 \times (10^5)^2} =  \dotfill $\\
  $H = \dfrac{60 \times 10^{-8} \times 0,58 \times 10^{-3}}{9,6 \times (10^{-7})^4} =  \dotfill $ \\
  $I = \dfrac{1600 \times 10^{6} \times 3200 \times 10^{8}}{21,2 \times (10^{-7})^2} =  \dotfill $\\
  $J = \dfrac{460 \times 10^{3} \times 13,6 \times 10^{4}}{5,2 \times (10^{10})^4} =  \dotfill $ 

\end{multicols}

\subsection*{Ex2 - Raisonner}

Compléter par un nombre de la forme $a^n$ avec $a$ et $n$ entiers :

\begin{multicols}{4}
  \begin{enumerate}
  \item[1.] $(5^{3})^{8} = \dotfill$
  \item[2.] $11^{6} \times 5^{6}  =  \dotfill$
  \item[3.] $10^{2}\times10^{5} = \dotfill$
  \item[4.] $4^{7}\times4^{11} = \dotfill$
  \item[5.] $\dfrac{10^{11}}{10^{4}} = \dotfill$
  \item[6.] $\dfrac{11^{9}}{11^{3}} = \dotfill$
  \item[7.] $(2^{6})^{3} = \dotfill$
  \item[8.] $9^{5} \times 8^{5}  =  \dotfill$
  \item[9.] $7^{3}\times7^{6} = \dotfill$
  \item[10.] $11^{4} \times 2^{4}  =  \dotfill$
  \item[11.] $\dfrac{11^{9}}{11^{2}} = \dotfill$
  \item[12.] $\dfrac{3^{5}}{3^{2}} = \dotfill$
  \item[13.] $(6^{10})^{3} = \dotfill$
  \item[14.] $9^{11} \times 6^{11}  =  \dotfill$
  \item[15.] $(2^{11})^{2} = \dotfill$
  \item[16.] $7^{9}\times7^{8} = \dotfill$
  \end{enumerate}
\end{multicols}


\subsection*{Ex3 - Raisonner}

Compléter par le nombre qui convient :

\begin{multicols}{3}

  \begin{enumerate}
  \item[1.] $28\,000 = 2{,}8 \times \dotfill$
  \item[2.] $0{,}000\,006\,701 = 6{,}701 \times \dotfill$
  \item[3.] $2{,}807 \times \dotfill = 0{,}000\,280\,7$
  \item[4.] $2\,106\,000 = 2{,}106 \times \dotfill$
  \item[5.] $7{,}06 \times \dotfill = 70\,600$
  \item[6.] $5{,}057 \times \dotfill = 0{,}000\,505\,7$
  \item[7.] $3{,}7 \times \dotfill = 0{,}000\,003\,7$
  \item[8.] $7\,701\,000 = 7{,}701 \times \dotfill$
  \item[9.] $9\,026\,000 = 9{,}026 \times \dotfill$
  \item[10.] $270{,}6 = 2{,}706 \times \dotfill$
  \item[11.] $4{,}069 \times \dotfill = 406{,}9$
  \item[12.] $9{,}079 \times \dotfill = 9\,079$
  \end{enumerate}
\end{multicols}


\subsection*{Ex4 - Sciences}

Avec vos mots et un schéma, expliquer pourquoi à la calculatrice : $1.6 + 10^{16} = 10^{16}$ 
\Pointilles{8}

\newpage

\subsection*{Ex5 - Problème}

\textit{\og El Professeur \fg{} } vient de dérober 6 milliards d’euros. \\
Les billets de banque ont une épaisseur de $80 \times 10^{-6} m$. (On dit 80 micromètres)

\begin{itemize}
\item[1.] Quelle hauteur atteindrait une pile de billets de banque de 50 \euro{} représentant cette somme ?
\item[2.] Quelle hauteur atteindrait une pile de billets de banque de 10 \euro{} représentant cette somme ?
\end{itemize}

\Pointilles{20}

\subsection*{Bonus}

\textit{Dessinez un chat. Racontez une blague avec un chat.}


\end{document}

\documentclass[12pt]{article}
\usepackage{geometry} % Pour passer au format A4
\geometry{hmargin=1cm, vmargin=1cm} % 

% Page et encodage
\usepackage[T1]{fontenc} % Use 8-bit encoding that has 256 glyphs
\usepackage[english,french]{babel} % Français et anglais
\usepackage[utf8]{inputenc} 

\usepackage{lmodern}
\setlength\parindent{0pt}

% Graphiques
\usepackage{graphicx,float,grffile}
\usepackage{pst-eucl, pst-plot} 

% Maths et divers
\usepackage{amsmath,amsfonts,amssymb,amsthm,verbatim}
\usepackage{multicol,enumitem,url,eurosym,gensymb}

% Sections
\usepackage{sectsty} % Allows customizing section commands
\allsectionsfont{\centering \normalfont\scshape}

% Tête et pied de page

\usepackage{fancyhdr} 
\pagestyle{fancyplain} 

\fancyhead{} % No page header
\fancyfoot{}

\renewcommand{\headrulewidth}{0pt} % Remove header underlines
\renewcommand{\footrulewidth}{0pt} % Remove footer underlines

\newcommand{\horrule}[1]{\rule{\linewidth}{#1}} % Create horizontal rule command with 1 argument of height
\newcommand{\Pointille}[1][3]{\multido{}{#1}{ \makebox[\linewidth]{\dotfill}\\[\parskip]}}

%----------------------------------------------------------------------------------------
%	Début du document
%----------------------------------------------------------------------------------------

\begin{document}

%----------------------------------------------------------------------------------------
% RE-DEFINITION
%----------------------------------------------------------------------------------------
% MATHS
%-----------

\newtheorem{Definition}{Définition}
\newtheorem{Theorem}{Théorème}
\newtheorem{Proposition}{Propriété}

% MATHS
%-----------
\renewcommand{\labelitemi}{$\bullet$}
\renewcommand{\labelitemii}{$\circ$}
%----------------------------------------------------------------------------------------
%	Titre
%----------------------------------------------------------------------------------------

\setlength{\columnseprule}{1pt}

\section*{ie - Équations - A}
\begin{center}
  \textit{Philip K. Dick - La réalité, c'est ce qui refuse de disparaître quand on cesse d'y croire.}
\end{center}
\textbf{Nom, Prenom :}\\
\horrule{2px}

\subsection*{ex1 - Calculs  à trou}
\textit{Trouver le nombre manquant.}

  \begin{multicols}{3}\noindent
    \begin{enumerate}
    \item $7 + 7 = \ldots\ldots$
    \item $-20 \div \ldots\ldots = -10$
    \item $-1 - \left( -4\right) = \ldots\ldots$
    \item $\ldots\ldots \div 10 = -7$
    \item $8 + \left( -10\right) = \ldots\ldots$
    \item $7 \times 4 = \ldots\ldots$
    \item $2 + \left( -4\right) = \ldots\ldots$
    \item $30 \div \left( -5\right) = \ldots\ldots$
    \item $5 - 1 = \ldots\ldots$
    \item $5 + 4 = \ldots\ldots$
    \item $-17 - \ldots\ldots = -9$
    \item $2 \times 6 = \ldots\ldots$
    \item $7 \div \left( -7\right) = \ldots\ldots$
    \item $\ldots\ldots \times \left( -6\right) = 6$
    \item $7 - 1 = \ldots\ldots$
    \item $\ldots\ldots \times \left( -3\right) = -24$
    \item $-15 - \left( -7\right) = \ldots\ldots$
    \item $35 \div \ldots\ldots = -7$
    \item $1 + 8 = \ldots\ldots$
    \item $8 \times \ldots\ldots = -24$
    \end{enumerate}
  \end{multicols}

    \section*{Équations}
  \textit{Résoudre les équations.} \textbf{Écrire les étapes.} Rédiger soigneusement.

  \begin{multicols}{2}
    \subsection*{ex2 - Équations 1}

    \begin{eqnarray*}
      & a) & x + 14 = 46  \\
      & b) & 3x + 20 = 40  \\
      & c) & 4x - 6  = 46  \\
      & d) & 8x - 24 = -46 \\
      & e) & -24x + 24 = 204 \\
      & f) & -26x - 26 = -44 \\
      & g) & -6x - 44 = 34  \\
      & h) & 20 + 6x = 44  \\
      & i) & 80x + 24 = 24 \\
      & j) & \sqrt{4} x + 24 = 40 
    \end{eqnarray*}


    \subsection*{ex3 - Équations 2}

    \begin{eqnarray*}
      & a) & 4x + 4 = 4x + 24    \\
      & b) & 8x + 24 = 6x - 4     \\
      & c) & 44x - 44 = 6x + 24    \\
      & d) & -24x + 23 = 4x -40     \\
      & e) & -8x - 20 = -24x - 44  \\
      & f) & 6,4x + 24 = 9x + 36    \\
    \end{eqnarray*}
  \end{multicols}

\newpage

\section*{ie - Équations - B}
\begin{center}
  \textit{Philip K. Dick - La réalité, c'est ce qui refuse de disparaître quand on cesse d'y croire.}
\end{center}
\textbf{Nom, Prenom :}\\
\horrule{2px}

\subsection*{ex1 - Calculs  à trou}
\textit{Trouver le nombre manquant.}

  \begin{multicols}{3}\noindent
    \begin{enumerate}
    \item $-12 \div \ldots\ldots = -4$
    \item $-3 \times \ldots\ldots = -24$
    \item $-2 \times 2 = \ldots\ldots$
    \item $7 + \left( -9\right) = \ldots\ldots$
    \item $-16 \div 2 = \ldots\ldots$
    \item $3 + \left( -10\right) = \ldots\ldots$
    \item $-30 \div \left( -10\right) = \ldots\ldots$
    \item $-1 \times \ldots\ldots = -9$
    \item $-5 \times \left( -3\right) = \ldots\ldots$
    \item $\ldots\ldots - \left( -9\right) = 2$
    \item $4 - \ldots\ldots = 7$
    \item $3 + \left( -2\right) = \ldots\ldots$
    \item $60 \div \ldots\ldots = 6$
    \item $-28 \div 7 = \ldots\ldots$
    \item $\ldots\ldots - \left( -8\right) = -8$
    \item $6 + \left( -1\right) = \ldots\ldots$
    \item $\ldots\ldots \times 5 = 10$
    \item $10 + \left( -7\right) = \ldots\ldots$
    \item $2 - 1 = \ldots\ldots$
    \item $\ldots\ldots - \left( -10\right) = -10$
    \end{enumerate}
  \end{multicols}


    \section*{Équations}
  \textit{Résoudre les équations.} \textbf{Écrire les étapes.} Rédiger soigneusement.

  \begin{multicols}{2}
    \subsection*{ex2 - Équations 1}

    \begin{eqnarray*}
      & a) & x + 22 = 35  \\
      & b) & 3x + 10 = 40  \\
      & c) & 3x - 5  = 36  \\
      & d) & 8x - 23 = -45 \\
      & e) & -23x + 24 = 203 \\
      & f) & -25x - 26 = -34 \\
      & g) & -6x - 34 = 33  \\
      & h) & 20 + 5x = 33  \\
      & i) & 80x + 23 = 23 \\
      & j) & \sqrt{3} x + 23 = 40 
    \end{eqnarray*}


    \subsection*{ex3 - Équations 2}

    \begin{eqnarray*}
      & a) & 4x + 3 = 3x + 23    \\
      & b) & 8x + 23 = 6x - 4     \\
      & c) & 34x - 33 = 5x + 23    \\
      & d) & -23x + 23 = 4x -30     \\
      & e) & -8x - 20 = -23x - 34  \\
      & f) & 4,3x + 24 = 5x + 36    \\
    \end{eqnarray*}
  \end{multicols}

  \newpage

  \section*{ie - Équations - C}
\begin{center}
  \textit{Philip K. Dick - La réalité, c'est ce qui refuse de disparaître quand on cesse d'y croire.}
\end{center}
\textbf{Nom, Prenom :}\\
\horrule{2px}

\subsection*{ex1 - Calculs  à trou}
\textit{Trouver le nombre manquant.}

  \begin{multicols}{3}\noindent
    \begin{enumerate}
    \item $\ldots\ldots + 2 = 3$
    \item $45 \div 9 = \ldots\ldots$
    \item $-7 \times \ldots\ldots = -42$
    \item $4 + \ldots\ldots = 5$
    \item $42 \div 7 = \ldots\ldots$
    \item $4 \times \left( -10\right) = \ldots\ldots$
    \item $-1 - \left( -10\right) = \ldots\ldots$
    \item $27 \div \left( -9\right) = \ldots\ldots$
    \item $-72 \div \ldots\ldots = 8$
    \item $\ldots\ldots \div \left( -9\right) = 3$
    \item $-8 - \ldots\ldots = 1$
    \item $8 + \left( -2\right) = \ldots\ldots$
    \item $-2 + \left( -2\right) = \ldots\ldots$
    \item $-1 \times \ldots\ldots = -6$
    \item $\ldots\ldots + \left( -1\right) = -9$
    \item $-1 - \left( -7\right) = \ldots\ldots$
    \item $-5 \times \left( -1\right) = \ldots\ldots$
    \item $-9 - \ldots\ldots = -1$
    \item $\ldots\ldots \times \left( -2\right) = -20$
    \item $-8 - \left( -9\right) = \ldots\ldots$
    \end{enumerate}
  \end{multicols}

    \section*{Équations}
  \textit{Résoudre les équations.} \textbf{Écrire les étapes.} Rédiger soigneusement.

  \begin{multicols}{2}
    \subsection*{ex2 - Équations 1}

    \begin{eqnarray*}
      & a) & x + 28 = 85  \\
      & b) & 3x + 20 = 40  \\
      & c) & 8x - 5  = 86  \\
      & d) & 8x - 28 = -45 \\
      & e) & -28x + 24 = 208 \\
      & f) & -25x - 26 = -84 \\
      & g) & -6x - 84 = 38  \\
      & h) & 20 + 5x = 88  \\
      & i) & 80x + 28 = 28 \\
      & j) & \sqrt{8} x + 28 = 40 
    \end{eqnarray*}


    \subsection*{ex3 - Équations 2}

    \begin{eqnarray*}
      & a) & 4x + 8 = 8x + 28    \\
      & b) & 8x + 28 = 6x - 4     \\
      & c) & 84x - 88 = 5x + 28    \\
      & d) & -28x + 23 = 4x -80     \\
      & e) & -8x - 20 = -28x - 84  \\
      & f) & 4,8x + 14 = 5x + 36    \\
    \end{eqnarray*}
  \end{multicols}

  \newpage

  \section*{ie - Équations - D}
\begin{center}
  \textit{Philip K. Dick - La réalité, c'est ce qui refuse de disparaître quand on cesse d'y croire.}
\end{center}
\textbf{Nom, Prenom :}\\
\horrule{2px}

\subsection*{ex1 - Calculs  à trou}
\textit{Trouver le nombre manquant.}

  \begin{multicols}{3}\noindent
    \begin{enumerate}
    \item $\ldots\ldots \times 6 = 60$
    \item $-36 \div 6 = \ldots\ldots$
    \item $\ldots\ldots + \left( -10\right) = -2$
    \item $18 \div 6 = \ldots\ldots$
    \item $\ldots\ldots + 10 = 0$
    \item $-6 + 4 = \ldots\ldots$
    \item $-7 + \ldots\ldots = 0$
    \item $5 - 3 = \ldots\ldots$
    \item $100 \div \left( -10\right) = \ldots\ldots$
    \item $\ldots\ldots - \left( -8\right) = 6$
    \item $\ldots\ldots - 4 = -8$
    \item $-4 + \left( -5\right) = \ldots\ldots$
    \item $6 \times \left( -8\right) = \ldots\ldots$
    \item $\ldots\ldots \times 3 = -18$
    \item $\ldots\ldots \div 1 = -5$
    \item $16 - \ldots\ldots = 6$
    \item $12 - 8 = \ldots\ldots$
    \item $-3 \times \ldots\ldots = 24$
    \item $1 \times \left( -2\right) = \ldots\ldots$
    \item $\ldots\ldots \div 10 = 5$
    \end{enumerate}
  \end{multicols}

    \section*{Équations}
  \textit{Résoudre les équations.} \textbf{Écrire les étapes.} Rédiger soigneusement.

  \begin{multicols}{2}
    \subsection*{ex2 - Équations 1}

    \begin{eqnarray*}
      & a) & x + 13 = 35  \\
      & b) & 3x + 10 = 40  \\
      & c) & 3x - 5  = 36  \\
      & d) & 8x - 43 = -45 \\
      & e) & -43x + 44 = 403 \\
      & f) & -45x - 46 = -34 \\
      & g) & -6x - 34 = 33  \\
      & h) & 40 + 5x = 33  \\
      & i) & 80x + 43 = 43 \\
      & j) & \sqrt{3} x + 43 = 40 
    \end{eqnarray*}


    \subsection*{ex3 - Équations 3}

    \begin{eqnarray*}
      & a) & 4x + 3 = 3x + 43    \\
      & b) & 8x + 43 = 6x - 4     \\
      & c) & 34x - 33 = 5x + 43    \\
      & d) & -43x + 43 = 4x -30     \\
      & e) & -8x - 40 = -43x - 34  \\
      & f) & 4,3x + 44 = 5x + 36    \\
    \end{eqnarray*}
  \end{multicols}

\end{document}

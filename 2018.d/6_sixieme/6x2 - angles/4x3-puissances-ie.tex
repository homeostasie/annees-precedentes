\documentclass[11pt]{article}
\usepackage{geometry} % Pour passer au format A4
\geometry{hmargin=1cm, vmargin=1cm} % 

% Page et encodage
\usepackage[T1]{fontenc} % Use 8-bit encoding that has 256 glyphs
\usepackage[english,french]{babel} % Français et anglais
\usepackage[utf8]{inputenc} 

\usepackage{lmodern}
\setlength\parindent{0pt}

% Graphiques
\usepackage{graphicx,float,grffile}
\usepackage{tikz}

% Maths et divers
\usepackage{amsmath,amsfonts,amssymb,amsthm,verbatim}
\usepackage{multicol,enumitem,url,eurosym,gensymb}
\usepackage{multido}

% Sections
\usepackage{sectsty} % Allows customizing section commands
\allsectionsfont{\centering \normalfont\scshape}

% Tête et pied de page
\usepackage{fancyhdr} 
\pagestyle{fancyplain} 
\fancyhead{} % No page header
\fancyfoot{}

\renewcommand{\headrulewidth}{0pt} % Remove header underlines
\renewcommand{\footrulewidth}{0pt} % Remove footer underlines

\newcommand{\horrule}[1]{\rule{\linewidth}{#1}} % Create horizontal rule command with 1 argument of height

\newcommand{\Pointilles}[1]{%
  \par\nobreak
  \noindent\rule{0pt}{1.5\baselineskip}% Provides a larger gap between the preceding paragraph and the dots
  \multido{}{#1}{\noindent\makebox[\linewidth]{\dotfill}\endgraf}% ... dotted lines ...
  \bigskip% Gap between dots and next paragraph
}

\usepackage{fourier}
\usepackage{amsmath,amsfonts,amssymb}
\usepackage{paralist,array}
\usepackage{pgf,tikz}
\usetikzlibrary{shapes,snakes,arrows,patterns}
\usepackage{graphicx,multicol}
\usepackage{varwidth}


%----------------------------------------------------------------------------------------
%	Début du document
%----------------------------------------------------------------------------------------

\begin{document}

%----------------------------------------------------------------------------------------
% RE-DEFINITION
%----------------------------------------------------------------------------------------
% MATHS
%-----------

\newtheorem{Definition}{Définition}
\newtheorem{Theorem}{Théorème}
\newtheorem{Proposition}{Propriété}

% MATHS
%-----------
\renewcommand{\labelitemi}{$\bullet$}
\renewcommand{\labelitemii}{$\circ$}
%----------------------------------------------------------------------------------------
%	Titre
%----------------------------------------------------------------------------------------

\setlength{\columnseprule}{1pt}

\textbf{Nom, Prénom : }

\section*{Eval - angles}
\begin{center}
  \textit{Théophraste - La plus coûteuse des dépenses, c’est la perte de temps.}
\end{center}
\horrule{2px}

\begin{multicols}{3}
\begin{itemize}
\item[o] S'organiser
\item[o] Restituer
\item[o] Représenter
\end{itemize}
\end{multicols}

\subsection*{Ex1 - Mesurer les cinq angles ci-dessous.}

\begin{tikzpicture}[line cap=round,line join=round,>=triangle 45,x=1.0cm,y=1.0cm,scale=0.8]
\clip(-1,-2) rectangle (17.23,10.24);
\draw (3.67,6.29)-- (0,4.7)-- (0.25,8.69);
\draw [shift={(0,4.7)}] (23.43:0.73) arc (23.43:86.43:0.73);
\draw (3,5.9) node[anchor=north]{A}; \draw (-0.1,4.7) node[anchor=north]{O}; \draw (0.1,7.96) node[anchor=east]{B};
\draw (7.3,7.3)-- (10.3,7.3)-- (7.81,5.62);
\draw [shift={(10.3,7.3)}] (180:0.73) arc (180:214:0.73);
\draw (7.7,7.4) node[anchor=south]{C}; \draw (10.4,7.3) node[anchor=west]{D}; \draw (8.56,6.03) node[anchor=north]{E};
\draw (16.92,7.6)-- (13.4,9.5)-- (14.35,5.61);
\draw [shift={(13.4,9.5)}] (-76.3:0.73) arc (-76.3:-28.3:0.73);
\draw (16,8.1) node[anchor=south west]{F}; \draw (13.3,9.5) node[anchor=south]{G}; \draw (14,6.63) node[anchor=east]{H};
\draw (2.39,2.33)-- (5,0)-- (1.71,1.19);
\draw [shift={(5,0)}] (138.18:0.73) arc (138.18:160.18:0.73);
\draw (3.2,1.8) node[anchor=south]{I}; \draw (5.1,0) node[anchor=west]{J}; \draw (2.6,0.76) node[anchor=north]{K};
\draw (7,3.2)-- (10.5,3.2)-- (13.4,1.24);
\draw [shift={(10.5,3.2)}] (180:0.73) arc (180:326:0.73);
\draw (7.7,3.3) node[anchor=south]{L}; \draw (10.5,3.3) node[anchor=south]{M}; \draw (12.92,1.83) node[anchor=west]{N};
\begin{scriptsize}
\draw (3,6)++(-3pt,-3pt) -- ++(6.0pt,6.0pt) ++(-6.0pt,0) -- ++(6.0pt,-6.0pt); % A
\draw (0.2,7.96)++(-3pt,-3pt) -- ++(6.0pt,6.0pt) ++(-6.0pt,0) -- ++(6.0pt,-6.0pt); % B
\draw (7.7,7.3)++(-3pt,-3pt) -- ++(6.0pt,6.0pt) ++(-6.0pt,0) -- ++(6.0pt,-6.0pt); % C
\draw (8.56,6.13)++(-3pt,-3pt) -- ++(6.0pt,6.0pt) ++(-6.0pt,0) -- ++(6.0pt,-6.0pt);% E
\draw (16,8.1)++(-3pt,-3pt) -- ++(6.0pt,6.0pt) ++(-6.0pt,0) -- ++(6.0pt,-6.0pt);% F
\draw (14.1,6.63)++(-3pt,-3pt) -- ++(6.0pt,6.0pt) ++(-6.0pt,0) -- ++(6.0pt,-6.0pt);%H
\draw (3.1,1.7)++(-3pt,-3pt) -- ++(6.0pt,6.0pt) ++(-6.0pt,0) -- ++(6.0pt,-6.0pt); %I
\draw (2.6,0.86)++(-3pt,-3pt) -- ++(6.0pt,6.0pt) ++(-6.0pt,0) -- ++(6.0pt,-6.0pt); %K
\draw (7.7,3.2)++(-3pt,-3pt) -- ++(6.0pt,6.0pt) ++(-6.0pt,0) -- ++(6.0pt,-6.0pt);% L
\draw (12.82,1.63)++(-3pt,-3pt) -- ++(6.0pt,6.0pt) ++(-6.0pt,0) -- ++(6.0pt,-6.0pt); % N
\end{scriptsize}
\end{tikzpicture}

\subsection*{Ex2 - Construire les angles suivants.}
$\widehat{\text{ABC}} $ = 48° \quad ; \quad $\widehat{\text{DEF}} $ = 96° \quad ; \quad $\widehat{\text{GHI}} $ = 143°\\

\begin{tikzpicture}[line cap=round,line join=round,>=triangle 45,x=1.0cm,y=1.0cm,scale=0.8]
\clip(-0.96,-0.34) rectangle (18,8);
\draw (0,2.6)-- (2.39,4.73);
\draw (-0.1,2.6) node[anchor=east]{B}; \draw (1.8,4.2) node[anchor=north west]{A};
\draw (16,3.8)-- (12.78,4.52);
\draw (16.1,3.8) node[anchor=west]{E}; \draw (13.3,4.4) node[anchor=north east]{D};
\draw (8.2,1.6)-- (11.27,0);
\draw (10.5,0.5) node[anchor=south]{G}; \draw (8.2,1.5) node[anchor=north]{H};
\begin{scriptsize}
\draw (0,2.6) ++(-3pt,-3pt) -- ++(6.0pt,6.0pt) ++(-6.0pt,0) -- ++(6.0pt,-6.0pt);
\draw (1.8,4.2) ++(-3pt,-3pt) -- ++(6.0pt,6.0pt) ++(-6.0pt,0) -- ++(6.0pt,-6.0pt);
\draw (13.3,4.4) ++(-3pt,-3pt) -- ++(6.0pt,6.0pt) ++(-6.0pt,0) -- ++(6.0pt,-6.0pt);
\draw (16,3.8) ++(-3pt,-3pt) -- ++(6.0pt,6.0pt) ++(-6.0pt,0) -- ++(6.0pt,-6.0pt);
\draw (8.2,1.6) ++(-3pt,-3pt) -- ++(6.0pt,6.0pt) ++(-6.0pt,0) -- ++(6.0pt,-6.0pt);
\draw (10.5,0.4) ++(-3pt,-3pt) -- ++(6.0pt,6.0pt) ++(-6.0pt,0) -- ++(6.0pt,-6.0pt);
\end{scriptsize}
\end{tikzpicture}
\newpage
\subsection*{Ex3 - Savoir mesurer les angles dans une figure}

% Figure 3
\begin{minipage}{0.5\linewidth}
\begin{tikzpicture}[line cap=round,line join=round,>=triangle 45,x=1.0cm,y=1.0cm,scale=0.7]
\clip(-0.5,-0.36) rectangle (6.41,4.66);
\draw (0,0)-- (5.3,0)-- (4.92,2.37)-- (5.8,4.17) -- (1.63,4.24)-- cycle;
\draw [shift={(0,0)}] (0:0.55) arc (0:69:0.55);
\draw [shift={(5.3,0)}] (99:0.55) arc (99:180:0.55);
\draw [shift={(4.92,2.37)}] (64:0.55) arc (64:279:0.55);
\draw [shift={(5.8,4.17)}] (179:0.55) arc (179:244:0.55);
\draw [shift={(1.63,4.24)}] (-111:0.55) arc (-111:-1:0.55);
\draw (0,0) node [anchor=east] {S};
\draw (5.3,0) node [anchor=west] {C};
\draw (4.92,2.37) node [anchor=west] {U};
\draw (5.8,4.17) node [anchor=west] {R};
\draw (1.63,4.24) node [anchor=east] {T};
\end{tikzpicture}
\end{minipage}
\begin{minipage}{0.5\linewidth}
\vspace{18pt}
. \dotfill \vspace{12pt} \\
. \dotfill \vspace{12pt} \\
. \dotfill \vspace{12pt} \\
. \dotfill \vspace{12pt} \\
. \dotfill \vspace{12pt} \\
. \dotfill \vspace{12pt} \\
\end{minipage}


\subsection*{Ex4 - Reproduire une figure à partir d'angles}
Reproduire en vraie grandeur les figures suivantes.

% Figure 1
\begin{minipage}{0.5\linewidth}
\begin{tikzpicture}[line cap=round,line join=round,>=triangle 45,x=1.0cm,y=1.0cm]
\clip(-1,-1) rectangle (6,3.5);
\draw (0,0)-- (3.19,2.4)-- (5,0)-- cycle;
\draw [shift={(0,0)}] (0:0.64) arc (0:37:0.64) ;
\draw [shift={(5,0)}] (127:0.64) arc (127:180:0.64);
\draw (0,0) node [anchor=east] {A};
\draw (5,0) node [anchor=west] {B};
\draw (3.19,2.4) node [anchor=south] {C};
\draw (1,0.3) node {37°};
\draw (4.15,0.43) node {53°};
\draw (2.5,-.3) node {6 cm};
\end{tikzpicture}
\end{minipage}
\begin{minipage}{0.5\linewidth}
$\ $
\end{minipage}
\\
% Figure 2
\begin{minipage}{0.5\linewidth}
\begin{tikzpicture}[line cap=round,line join=round,>=triangle 45,x=1.0cm,y=1.0cm]
\clip(-1,-1) rectangle (6,4);
\draw (0,0)-- (4.3,0)-- (3.87,2.46)-- (0.8,3)--cycle;
\draw [shift={(3.87,2.46)}] (170:0.51) arc (170:280:0.51);
\draw [shift={(0,0)}] (0:0.51) arc (0:75:0.51);
\draw [shift={(0.8,3)}] (-105:0.51) arc (-105:-10:0.51);
\draw (0,0) node [anchor=east] {A};
\draw (4.3,0) node [anchor=west] {B};
\draw (3.87,2.46) node [anchor=west] {C};
\draw (0.8,3) node [anchor=east] {D};
\draw (3.12,1.8) node {110°};
\draw (0.8,0.47) node {75°};
\draw (1.2,2.3) node {95°};
\draw (2.5,3) node {4 cm};
\draw (-.3,1.5) node {4,5 cm};
\end{tikzpicture}
\end{minipage}
\begin{minipage}{0.5\linewidth}
$\ $
\end{minipage}

\subsection*{Restituer}
\begin{minipage}{\linewidth}
\vspace{18pt}
. \dotfill \vspace{12pt} \\
. \dotfill \vspace{12pt} \\
. \dotfill \vspace{12pt} \\
. \dotfill \vspace{12pt} \\
. \dotfill \vspace{12pt} \\
. \dotfill \vspace{12pt} \\
\end{minipage}


\end{document}

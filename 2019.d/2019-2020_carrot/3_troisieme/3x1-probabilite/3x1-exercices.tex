%%!TEX TS-program = latex
\documentclass[a4paper,11pt]{article}
\usepackage[utf8]{inputenc} % UTF-8
\usepackage[T1]{fontenc}
\usepackage[frenchb]{babel} % francisation
\usepackage[fleqn]{amsmath} % aligne le mode maths à gauche
\usepackage{amssymb} % the amsfont symbols
\usepackage[table, usenames, svgnames]{xcolor} % Couleurs
\usepackage{multicol} % Multi-colonnes
\usepackage{fancyhdr} % Mise en page, en-tête et pied de page
\usepackage{calc} % Opérations
\usepackage{marvosym} % Martin Vogels Symbole (\EUR)
\usepackage{cancel} % draw diagonal lines
\usepackage{units} % typesetting units and nice fractions
\usepackage[autolanguage]{numprint} % écrituredes virgules
\usepackage{tabularx} % creates a paragraph-like column whose width
% automatically expands
\usepackage{wrapfig} % allows figures or tables to have text wrapped around
\usepackage{pst-eucl, pst-plot} % figures géométriques
\usepackage{wasysym} % Symbole Euro
%\usepackage{textcomp}
\input{/usr/share/pyromaths/packages/tabvar.tex}

\usepackage[a4paper, dvips, left=1.5cm, right=1.5cm, top=2cm,%
bottom=2cm, marginpar=5mm, marginparsep=5pt]{geometry}
\newcounter{exo}
\setlength{\headheight}{18pt}
\setlength{\fboxsep}{1em}
\setlength\parindent{0em}
\setlength\mathindent{0em}
\setlength{\columnsep}{30pt}
\usepackage[bookmarks=true, bookmarksnumbered=true, ps2pdf, pagebackref=true,%
colorlinks=true,linkcolor=blue,plainpages=true]{hyperref}
\hypersetup{pdfauthor={Jérôme Ortais},pdfsubject={Exercices de
    mathématiques},pdftitle={Exercices créés par Pyromaths, un logiciel libre
    en Python sous licence GPL}}
\makeatletter
\newcommand\styleexo[1][]{
  \renewcommand{\theenumi}{\arabic{enumi}}
  \renewcommand{\labelenumi}{$\blacktriangleright$\textbf{\theenumi.}}
  \renewcommand{\theenumii}{\alph{enumii}}
  \renewcommand{\labelenumii}{\textbf{\theenumii)}}
  {\fontfamily{pag}\fontseries{b}\selectfont \underline{#1 \theexo}}
  \par\@afterheading\vspace{0.5\baselineskip minus 0.2\baselineskip}}
\newcommand*\exercice{%
  \psset{unit=1cm, dash=4pt 4pt, PointName=default,linecolor=Maroon,
    dotstyle=x, linestyle=solid, hatchcolor=Peru, gridcolor=Olive,
    subgridcolor=Olive, fillcolor=Peru}
  %\ifthenelse{\equal{\theexo}{0}}{}{\filbreak}
  \refstepcounter{exo}%
  \stepcounter{nocalcul}%
  \par\addvspace{1.5\baselineskip minus 1\baselineskip}%
  \@ifstar%
  {\penalty-130\styleexo[Corrigé de l'exercice]}%
  {\penalty-130\styleexo[Exercice]}%
  }
\makeatother
\newlength{\ltxt}
\newcounter{fig}
\newcommand{\figureadroite}[2]{
  \setlength{\ltxt}{\linewidth}
  \setbox\thefig=\hbox{#1}
  \addtolength{\ltxt}{-\wd\thefig}
  \addtolength{\ltxt}{-10pt}
  \begin{minipage}{\ltxt}
    #2
  \end{minipage}
  \hfill
  \begin{minipage}{\wd\thefig}
    #1
  \end{minipage}
  \refstepcounter{fig}
  }
\count1=\year \count2=\year
\ifnum\month<8\advance\count1by-1\else\advance\count2by1\fi
\pagestyle{fancy}
\cfoot{\textsl{\footnotesize{Année \number\count1/\number\count2}}}
\rfoot{\textsl{\tiny{http://www.pyromaths.org}}}
\lhead{\textsl{\footnotesize{Page \thepage/ \pageref{LastPage}}}}
\chead{\Large{\textsc{Fiche de révisions}}}
\rhead{\textsl{\footnotesize{Classe de 3\ieme}}}
\DecimalMathComma
\begin{document}
  \currentpdfbookmark{Les énoncés des exercices}{Énoncés}
  \newcounter{nocalcul}[exo]
  \renewcommand{\thenocalcul}{\Alph{nocalcul}}
  \raggedcolumns
  \setlength{\columnseprule}{0.5pt}

  \exercice
  Dans une urne, il y a 3 boules vertes (V), 5 boules marrons (M) et 1 boule
  orange (O), indiscernables au toucher. On tire successivement et sans remise
  deux boules.
  \begin{enumerate}
  \item Quelle est la probabilité de tirer une boule marron au premier tirage?
  \item Construire un arbre des probabilités décrivant l'expérience aléatoire.
  \item Quelle est la probabilité que la première boule soit orange et la
    deuxième soit marron?
  \item Quelle est la probabilité que la deuxième boule soit verte ?
  \end{enumerate}

  \exercice
  Dans une urne, il y a 1 boule bleue (B), 1 boule marron (M) et 1 boule orange
  (O), indiscernables au toucher. On tire successivement et sans remise deux
  boules.
  \begin{enumerate}
  \item Quelle est la probabilité de tirer une boule marron au premier tirage?
  \item Construire un arbre des probabilités décrivant l'expérience aléatoire.
  \item Quelle est la probabilité que la première boule soit orange et la
    deuxième soit marron?
  \item Quelle est la probabilité que la deuxième boule soit bleue ?
  \end{enumerate}

  \exercice
  Dans une urne, il y a 5 boules rouges (R), 1 boule orange (O) et 4 boules
  bleues (B), indiscernables au toucher. On tire successivement et sans remise
  deux boules.
  \begin{enumerate}
  \item Quelle est la probabilité de tirer une boule orange au premier tirage?
  \item Construire un arbre des probabilités décrivant l'expérience aléatoire.
  \item Quelle est la probabilité que la première boule soit bleue et la
    deuxième soit orange?
  \item Quelle est la probabilité que la deuxième boule soit rouge ?
  \end{enumerate}

  \exercice
  Dans une urne, il y a 4 boules jaunes (J), 2 boules vertes (V) et 1 boule
  marron (M), indiscernables au toucher. On tire successivement et sans remise
  deux boules.
  \begin{enumerate}
  \item Quelle est la probabilité de tirer une boule verte au premier tirage?
  \item Construire un arbre des probabilités décrivant l'expérience aléatoire.
  \item Quelle est la probabilité que la première boule soit marron et la
    deuxième soit verte?
  \item Quelle est la probabilité que la deuxième boule soit jaune ?
  \end{enumerate}
  \label{LastPage}
  \newpage
  \currentpdfbookmark{Le corrigé des
    exercices}{Corrigé}\lhead{\textsl{\footnotesize{Page \thepage/
        \pageref{LastCorPage}}}}
  \setcounter{page}{1} \setcounter{exo}{0}

  \exercice*
  Dans une urne, il y a 3 boules vertes (V), 5 boules marrons (M) et 1 boule
  orange (O), indiscernables au toucher. On tire successivement et sans remise
  deux boules.
  \begin{enumerate}
  \item Quelle est la probabilité de tirer une boule marron au premier
    tirage?\par
    Il y a 9 boules dans l'urne dont 5 boules marrons. \par
    La probabilité de tirer une boule marron au premier tirage est donc
    $\dfrac{5}{9}$.
  \item Construire un arbre des probabilités décrivant l'expérience
    aléatoire.\\ [0,3cm]
    \psset{unit=1 mm}
    \psset{linewidth=0.3,dotsep=1,hatchwidth=0.3,hatchsep=1.5,shadowsize=1,dimen=middle}
    \psset{dotsize=0.7 2.5,dotscale=1 1,fillcolor=black}
    \psset{arrowsize=1 2,arrowlength=1,arrowinset=0.25,tbarsize=0.7
      5,bracketlength=0.15,rbracketlength=0.15}
    \begin{pspicture}(0,0)(80,53)
      \psline(0,3)(10,23)
      \psline(10,3)(10,23)
      \psline(20,3)(10,23)
      \psline(30,3)(40,23)
      \psline(40,3)(40,23)
      \psline(50,3)(40,23)
      \psline(60,3)(70,23)
      \psline(80,3)(70,23)
      \psline(70,3)(70,23)
      \psline(15,28)(40,53)
      \psline(40,53)(65,28)
      \psline(40,53)(40,28)
      \rput(20,40){$\dfrac{3}{9}$} \rput(37,40){$\dfrac{5}{9}$}
      \rput(60,40){$\dfrac{1}{9}$}
      \rput(10,26){V} \rput(40,26){M} \rput(70,26){O}
      \rput(0,10){$\dfrac{2}{8}$} \rput(7,10){$\dfrac{5}{8}$}
      \rput(20,10){$\dfrac{1}{8}$}
      \rput(0,0){V} \rput(10,0){M} \rput(20,0){O}
      \rput(30,10){$\dfrac{3}{8}$} \rput(37,10){$\dfrac{4}{8}$}
      \rput(50,10){$\dfrac{1}{8}$}
      \rput(30,0){V} \rput(40,0){M} \rput(50,0){O}
      \rput(60,10){$\dfrac{3}{8}$} \rput(67,10){$\dfrac{5}{8}$}
      \rput(80,10){$\dfrac{0}{8}$}
      \rput(60,0){V} \rput(70,0){M} \rput(80,0){O}
    \end{pspicture}
    \vspace{0.3cm}
  \item Quelle est la probabilité que la première boule soit orange et la
    deuxième soit marron?\par
    On utilise l'arbre construit précédemment.\par
    $p(O,M)=\dfrac{1}{9} \times \dfrac{5}{8} = \dfrac{5}{72}$\par
    La probabilité que la première boule soit orange et la deuxième soit marron
    est égale à $\dfrac{5}{72}$.
  \item Quelle est la probabilité que la deuxième boule soit verte ?\par
    On note (?, V) l'évènement: la deuxième boule tirée est verte. \par
    $p(?,V)=p(V,V)+p(M,V)+p(O,V,)=\dfrac{3}{9}\times
    \dfrac{2}{8}+\dfrac{5}{9}\times \dfrac{3}{8}+\dfrac{1}{9}\times
    \dfrac{3}{8}=\dfrac{24}{72}$
  \end{enumerate}

  \exercice*
  Dans une urne, il y a 1 boule bleue (B), 1 boule marron (M) et 1 boule orange
  (O), indiscernables au toucher. On tire successivement et sans remise deux
  boules.
  \begin{enumerate}
  \item Quelle est la probabilité de tirer une boule marron au premier
    tirage?\par
    Il y a 3 boules dans l'urne dont 1 boule marron. \par
    La probabilité de tirer une boule marron au premier tirage est donc
    $\dfrac{1}{3}$.
  \item Construire un arbre des probabilités décrivant l'expérience
    aléatoire.\\ [0,3cm]
    \psset{unit=1 mm}
    \psset{linewidth=0.3,dotsep=1,hatchwidth=0.3,hatchsep=1.5,shadowsize=1,dimen=middle}
    \psset{dotsize=0.7 2.5,dotscale=1 1,fillcolor=black}
    \psset{arrowsize=1 2,arrowlength=1,arrowinset=0.25,tbarsize=0.7
      5,bracketlength=0.15,rbracketlength=0.15}
    \begin{pspicture}(0,0)(80,53)
      \psline(0,3)(10,23)
      \psline(10,3)(10,23)
      \psline(20,3)(10,23)
      \psline(30,3)(40,23)
      \psline(40,3)(40,23)
      \psline(50,3)(40,23)
      \psline(60,3)(70,23)
      \psline(80,3)(70,23)
      \psline(70,3)(70,23)
      \psline(15,28)(40,53)
      \psline(40,53)(65,28)
      \psline(40,53)(40,28)
      \rput(20,40){$\dfrac{1}{3}$} \rput(37,40){$\dfrac{1}{3}$}
      \rput(60,40){$\dfrac{1}{3}$}
      \rput(10,26){B} \rput(40,26){M} \rput(70,26){O}
      \rput(0,10){$\dfrac{0}{2}$} \rput(7,10){$\dfrac{1}{2}$}
      \rput(20,10){$\dfrac{1}{2}$}
      \rput(0,0){B} \rput(10,0){M} \rput(20,0){O}
      \rput(30,10){$\dfrac{1}{2}$} \rput(37,10){$\dfrac{0}{2}$}
      \rput(50,10){$\dfrac{1}{2}$}
      \rput(30,0){B} \rput(40,0){M} \rput(50,0){O}
      \rput(60,10){$\dfrac{1}{2}$} \rput(67,10){$\dfrac{1}{2}$}
      \rput(80,10){$\dfrac{0}{2}$}
      \rput(60,0){B} \rput(70,0){M} \rput(80,0){O}
    \end{pspicture}
    \vspace{0.3cm}
  \item Quelle est la probabilité que la première boule soit orange et la
    deuxième soit marron?\par
    On utilise l'arbre construit précédemment.\par
    $p(O,M)=\dfrac{1}{3} \times \dfrac{1}{2} = \dfrac{1}{6}$\par
    La probabilité que la première boule soit orange et la deuxième soit marron
    est égale à $\dfrac{1}{6}$.
  \item Quelle est la probabilité que la deuxième boule soit bleue ?\par
    On note (?, B) l'évènement: la deuxième boule tirée est bleue. \par
    $p(?,B)=p(B,B)+p(M,B)+p(O,B,)=\dfrac{1}{3}\times
    \dfrac{0}{2}+\dfrac{1}{3}\times \dfrac{1}{2}+\dfrac{1}{3}\times
    \dfrac{1}{2}=\dfrac{2}{6}$
  \end{enumerate}

  \exercice*
  Dans une urne, il y a 5 boules rouges (R), 1 boule orange (O) et 4 boules
  bleues (B), indiscernables au toucher. On tire successivement et sans remise
  deux boules.
  \begin{enumerate}
  \item Quelle est la probabilité de tirer une boule orange au premier
    tirage?\par
    Il y a 10 boules dans l'urne dont 1 boule orange. \par
    La probabilité de tirer une boule orange au premier tirage est donc
    $\dfrac{1}{10}$.
  \item Construire un arbre des probabilités décrivant l'expérience
    aléatoire.\\ [0,3cm]
    \psset{unit=1 mm}
    \psset{linewidth=0.3,dotsep=1,hatchwidth=0.3,hatchsep=1.5,shadowsize=1,dimen=middle}
    \psset{dotsize=0.7 2.5,dotscale=1 1,fillcolor=black}
    \psset{arrowsize=1 2,arrowlength=1,arrowinset=0.25,tbarsize=0.7
      5,bracketlength=0.15,rbracketlength=0.15}
    \begin{pspicture}(0,0)(80,53)
      \psline(0,3)(10,23)
      \psline(10,3)(10,23)
      \psline(20,3)(10,23)
      \psline(30,3)(40,23)
      \psline(40,3)(40,23)
      \psline(50,3)(40,23)
      \psline(60,3)(70,23)
      \psline(80,3)(70,23)
      \psline(70,3)(70,23)
      \psline(15,28)(40,53)
      \psline(40,53)(65,28)
      \psline(40,53)(40,28)
      \rput(20,40){$\dfrac{5}{10}$} \rput(37,40){$\dfrac{1}{10}$}
      \rput(60,40){$\dfrac{4}{10}$}
      \rput(10,26){R} \rput(40,26){O} \rput(70,26){B}
      \rput(0,10){$\dfrac{4}{9}$} \rput(7,10){$\dfrac{1}{9}$}
      \rput(20,10){$\dfrac{4}{9}$}
      \rput(0,0){R} \rput(10,0){O} \rput(20,0){B}
      \rput(30,10){$\dfrac{5}{9}$} \rput(37,10){$\dfrac{0}{9}$}
      \rput(50,10){$\dfrac{4}{9}$}
      \rput(30,0){R} \rput(40,0){O} \rput(50,0){B}
      \rput(60,10){$\dfrac{5}{9}$} \rput(67,10){$\dfrac{1}{9}$}
      \rput(80,10){$\dfrac{3}{9}$}
      \rput(60,0){R} \rput(70,0){O} \rput(80,0){B}
    \end{pspicture}
    \vspace{0.3cm}
  \item Quelle est la probabilité que la première boule soit bleue et la
    deuxième soit orange?\par
    On utilise l'arbre construit précédemment.\par
    $p(B,O)=\dfrac{4}{10} \times \dfrac{1}{9} = \dfrac{4}{90}$\par
    La probabilité que la première boule soit bleue et la deuxième soit orange
    est égale à $\dfrac{4}{90}$.
  \item Quelle est la probabilité que la deuxième boule soit rouge ?\par
    On note (?, R) l'évènement: la deuxième boule tirée est rouge. \par
    $p(?,R)=p(R,R)+p(O,R)+p(B,R,)=\dfrac{5}{10}\times
    \dfrac{4}{9}+\dfrac{1}{10}\times \dfrac{5}{9}+\dfrac{4}{10}\times
    \dfrac{5}{9}=\dfrac{45}{90}$
  \end{enumerate}

  \exercice*
  Dans une urne, il y a 4 boules jaunes (J), 2 boules vertes (V) et 1 boule
  marron (M), indiscernables au toucher. On tire successivement et sans remise
  deux boules.
  \begin{enumerate}
  \item Quelle est la probabilité de tirer une boule verte au premier
    tirage?\par
    Il y a 7 boules dans l'urne dont 2 boules vertes. \par
    La probabilité de tirer une boule verte au premier tirage est donc
    $\dfrac{2}{7}$.
  \item Construire un arbre des probabilités décrivant l'expérience
    aléatoire.\\ [0,3cm]
    \psset{unit=1 mm}
    \psset{linewidth=0.3,dotsep=1,hatchwidth=0.3,hatchsep=1.5,shadowsize=1,dimen=middle}
    \psset{dotsize=0.7 2.5,dotscale=1 1,fillcolor=black}
    \psset{arrowsize=1 2,arrowlength=1,arrowinset=0.25,tbarsize=0.7
      5,bracketlength=0.15,rbracketlength=0.15}
    \begin{pspicture}(0,0)(80,53)
      \psline(0,3)(10,23)
      \psline(10,3)(10,23)
      \psline(20,3)(10,23)
      \psline(30,3)(40,23)
      \psline(40,3)(40,23)
      \psline(50,3)(40,23)
      \psline(60,3)(70,23)
      \psline(80,3)(70,23)
      \psline(70,3)(70,23)
      \psline(15,28)(40,53)
      \psline(40,53)(65,28)
      \psline(40,53)(40,28)
      \rput(20,40){$\dfrac{4}{7}$} \rput(37,40){$\dfrac{2}{7}$}
      \rput(60,40){$\dfrac{1}{7}$}
      \rput(10,26){J} \rput(40,26){V} \rput(70,26){M}
      \rput(0,10){$\dfrac{3}{6}$} \rput(7,10){$\dfrac{2}{6}$}
      \rput(20,10){$\dfrac{1}{6}$}
      \rput(0,0){J} \rput(10,0){V} \rput(20,0){M}
      \rput(30,10){$\dfrac{4}{6}$} \rput(37,10){$\dfrac{1}{6}$}
      \rput(50,10){$\dfrac{1}{6}$}
      \rput(30,0){J} \rput(40,0){V} \rput(50,0){M}
      \rput(60,10){$\dfrac{4}{6}$} \rput(67,10){$\dfrac{2}{6}$}
      \rput(80,10){$\dfrac{0}{6}$}
      \rput(60,0){J} \rput(70,0){V} \rput(80,0){M}
    \end{pspicture}
    \vspace{0.3cm}
  \item Quelle est la probabilité que la première boule soit marron et la
    deuxième soit verte?\par
    On utilise l'arbre construit précédemment.\par
    $p(M,V)=\dfrac{1}{7} \times \dfrac{2}{6} = \dfrac{2}{42}$\par
    La probabilité que la première boule soit marron et la deuxième soit verte
    est égale à $\dfrac{2}{42}$.
  \item Quelle est la probabilité que la deuxième boule soit jaune ?\par
    On note (?, J) l'évènement: la deuxième boule tirée est jaune. \par
    $p(?,J)=p(J,J)+p(V,J)+p(M,J,)=\dfrac{4}{7}\times
    \dfrac{3}{6}+\dfrac{2}{7}\times \dfrac{4}{6}+\dfrac{1}{7}\times
    \dfrac{4}{6}=\dfrac{24}{42}$
  \end{enumerate}
  \label{LastCorPage}
\end{document}
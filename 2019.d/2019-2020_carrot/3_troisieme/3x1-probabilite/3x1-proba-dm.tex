\documentclass[11pt]{article}
\usepackage{geometry} % Pour passer au format A4
\geometry{hmargin=1cm, vmargin=1cm} % 

% Page et encodage
\usepackage[T1]{fontenc} % Use 8-bit encoding that has 256 glyphs
\usepackage[english,french]{babel} % Français et anglais
\usepackage[utf8]{inputenc} 

\usepackage{lmodern}
\setlength\parindent{0pt}

% Graphiques
\usepackage{graphicx,float,grffile}

% Maths et divers
\usepackage{amsmath,amsfonts,amssymb,amsthm,verbatim}
\usepackage{multicol,enumitem,url,eurosym,gensymb,multido}

% Sections
\usepackage{sectsty} % Allows customizing section commands
\allsectionsfont{\centering \normalfont\scshape}

% Tête et pied de page

\usepackage{fancyhdr} 
\pagestyle{fancyplain} 

\fancyhead{} % No page header
\fancyfoot{}

\renewcommand{\headrulewidth}{0pt} % Remove header underlines
\renewcommand{\footrulewidth}{0pt} % Remove footer underlines

\newcommand{\horrule}[1]{\rule{\linewidth}{#1}} % Create horizontal rule command with 1 argument of height

%----------------------------------------------------------------------------------------
%	Début du document
%----------------------------------------------------------------------------------------

\begin{document}

%----------------------------------------------------------------------------------------
% RE-DEFINITION
%----------------------------------------------------------------------------------------
% MATHS
%-----------

\newtheorem{Definition}{Définition}
\newtheorem{Theorem}{Théorème}
\newtheorem{Proposition}{Propriété}

% MATHS
%-----------
\renewcommand{\labelitemi}{$\bullet$}
\renewcommand{\labelitemii}{$\circ$}

\newcommand{\Pointilles}[1][3]{%
\multido{}{#1}{\makebox[\linewidth]{\dotfill}\\[\parskip]
}}

%----------------------------------------------------------------------------------------
%	Titre
%----------------------------------------------------------------------------------------

\setlength{\columnseprule}{1pt}

\textbf{Nom, Prénom :} \hspace{8cm} \textbf{Classe :} \hspace{3cm} \textbf{Date :}\\
\vspace{-0.8cm}

\section*{DM Probabilités - La loi de Murphy}

La loi de Murphy a été énoncée par Edward A. Murphy Jr, un ingénieur aérospatial américain vers les années 1950 lorsqu'il travaillait sur la résistance des pilotes dans les avions de chasse.

\subsection*{Énoncés}

\begin{center}
    \textbf{\og Tout ce qui est susceptible d'aller mal, ira mal un jour donné.\fg}
\end{center}

\begin{multicols}{2}
\subsection*{Interprétation}

\textbf{Humoristique}. \textbf{\og Le pire est toujours certain.\fg} S'il y a une tuile, vous allez vous la prendre. La \og loi de l’emmerdement maximum\fg \: est assez proche de la loi de Murphy et est très souvent confondue avec celle-ci. Cette loi nous explique que quand quelque chose tourne mal, même si c'est grave, le pire est encore à venir. \vspace{0.3cm} \\ 
\textbf{Réele}. \textbf{\og Si on s'attend au pire, autant s'y préparer.\fg} On doit s'attendre aux problèmes les plus courants mais aussi les plus improbables car les deux arriveront. 


\subsection*{Émotions}

\begin{itemize}
    \item Tout râte de manière imprévisible. 
    \item Même s'il n'y a qu'une toute petite chance de rater, il existera une personne qui va rater.  
    \item Quelque chose qui rate marque plus que quelque chose qui réussit. 
    \item Si une première erreur est faite, alors on stresse et on va en faire d'autres... qui sont tout aussi improbables.
\end{itemize}

\end{multicols}

\subsection*{En savoir plus}

\begin{itemize}
    \item Wikipédia (présentation): \url{https://fr.wikipedia.org/wiki/Loi_de_Murphy}
    \item Angèle - La Loi de Murphy (26 millions de vues) : \url{https://www.youtube.com/watch?v=zGyThu7EAHQ}
    \item Interstellar (film) : \url{http://www.allocine.fr/film/fichefilm_gen_cfilm=114782.html}
\end{itemize}

\subsection*{Devoir Maison}

\textbf{Rédiger au présent (entre 5 et 10 lignes) une courte expérience faisant intervenir la loi de Murphy.}
\textit{(Attention à l'orthographe.)}\\

\textit{Ce matin vous sortez de chez vous de très bonne humeur pour vous dirigez au collège Faubert... Cela avait vraiment peu de chance de se produire... mais c'est arrivé.}\\

\Pointilles[18]


\end{document}

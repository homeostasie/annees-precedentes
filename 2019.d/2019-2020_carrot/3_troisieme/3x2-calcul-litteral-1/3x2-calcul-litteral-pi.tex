\documentclass[11pt]{article}
\usepackage{geometry} % Pour passer au format A4
\geometry{hmargin=1cm, vmargin=1cm} % 

% Page et encodage
\usepackage[T1]{fontenc} % Use 8-bit encoding that has 256 glyphs
\usepackage[english,french]{babel} % Français et anglais
\usepackage[utf8]{inputenc} 

\usepackage{lmodern}
\setlength\parindent{0pt}

% Graphiques
\usepackage{graphicx,float,grffile}

% Maths et divers
\usepackage{amsmath,amsfonts,amssymb,amsthm,verbatim}
\usepackage{multicol,enumitem,url,eurosym,gensymb,multido}

% Sections
\usepackage{sectsty} % Allows customizing section commands
\allsectionsfont{\centering \normalfont\scshape}

% Tête et pied de page

\usepackage{fancyhdr} 
\pagestyle{fancyplain} 

\fancyhead{} % No page header
\fancyfoot{}

\renewcommand{\headrulewidth}{0pt} % Remove header underlines
\renewcommand{\footrulewidth}{0pt} % Remove footer underlines

\newcommand{\horrule}[1]{\rule{\linewidth}{#1}} % Create horizontal rule command with 1 argument of height

%----------------------------------------------------------------------------------------
%	Début du document
%----------------------------------------------------------------------------------------

\begin{document}

%----------------------------------------------------------------------------------------
% RE-DEFINITION
%----------------------------------------------------------------------------------------

\newcommand{\Pointilles}[1][3]{%
  \multido{}{#1}{\makebox[\linewidth]{\dotfill}\\[\parskip]
}}

%----------------------------------------------------------------------------------------
%	Titre
%----------------------------------------------------------------------------------------

\setlength{\columnseprule}{1pt}

\textbf{Nom, Prénom :} \hspace{8cm} \textbf{Classe :} \hspace{3cm} \textbf{Date :}\\
\vspace{-0.8cm}

\section*{DM $\pi$}
\begin{center}
3.1415926535897932384626433832795028841971693993751058209749445923078164062862089986280348253421\newline
170679821480865132823066470938446095505822317253594081284811174502841027019385211055596446229489\newline
549303819644288109756659334461284756482337867831652712019091456485669234603486104543266482133936\newline
072602491412737245870066063155881748815209209628292540917153643678925903600113305305488204665213\newline
841469519415116094330572703657595919530921861173819326117931051185480744623799627495673518857527\newline
248912279381830119491298336733624406566430860213949463952247371907021798609437027705392171762931\newline
767523846748184676694051320005681271452635608277857713427577896091736371787214684409012249534301\newline
465495853710507922796892589235420199561121290219608640344181598136297747713099605187072113499999\newline
983729780499510597317328160963185950244594553469083026425223082533446850352619311881710100031378\newline
387528865875332083814206171776691473035982534904287554687311595628638823537875937519577818577805\newline
321712268066130019278766111959092164201989380952572010654858632788659361533818279682303019520353\newline
018529689957736225994138912497217752834791315155748572424541506959508295331168617278558890750983\newline
817546374649393192550604009277016711390098488240128583616035637076601047101819429555961989467678\newline
374494482553797747268471040475346462080466842590694912933136770289891521047521620569660240580381\newline
501935112533824300355876402474964732639141992726042699227967823547816360093417216412199245863150\newline
302861829745557067498385054945885869269956909272107975093029553211653449872027559602364806654991\newline
198818347977535663698074265425278625518184175746728909777727938000816470600161452491921732172147\newline
723501414419735685481613611573525521334757418494684385233239073941433345477624168625189835694855\newline
620992192221842725502542568876717904946016534668049886272327917860857843838279679766814541009538\newline
837863609506800642251252051173929848960841284886269456042419652850222106611863067442786220391949\newline
450471237137869609563643719172874677646575739624138908658326459958133904780275900994657640789512\newline
694683983525957098258226205224894077267194782684826014769909026401363944374553050682034962524517\newline
493996514314298091906592509372216964615157098583874105978859597729754989301617539284681382686838\newline
689427741559918559252459539594310499725246808459872736446958486538367362226260991246080512438843\newline
904512441365497627807977156914359977001296160894416948685558484063534220722258284886481584560285\newline
060168427394522674676788952521385225499546667278239864565961163548862305774564980355936345681743\newline
241125150760694794510965960940252288797108931456691368672287489405601015033086179286809208747609\newline
178249385890097149096759852613655497818931297848216829989487226588048575640142704775551323796414\newline
515237462343645428584447952658678210511413547357395231134271661021359695362314429524849371871101\newline
457654035902799344037420073105785390621983874478084784896833214457138687519435064302184531910484\newline
810053706146806749192781911979399520614196634287544406437451237181921799983910159195618146751426\newline
912397489409071864942319615679452080951465502252316038819301420937621378559566389377870830390697\newline
920773467221825625996615014215030680384477345492026054146659252014974428507325186660021324340881\newline
907104863317346496514539057962685610055081066587969981635747363840525714591028970641401109712062\newline
804390397595156771577004203378699360072305587631763594218731251471205329281918261861258673215791\newline
984148488291644706095752706957220917567116722910981690915280173506712748583222871835209353965725\newline
121083579151369882091444210067510334671103141267111369908658516398315019701651511685171437657618\newline
351556508849099898599823873455283316355076479185358932261854896321329330898570642046752590709154\newline
814165498594616371802709819943099244889575712828905923233260972997120844335732654893823911932597\newline
463667305836041428138830320382490375898524374417029132765618093773444030707469211201913020330380\newline
197621101100449293215160842444859637669838952286847831235526582131449576857262433441893039686426\newline
243410773226978028073189154411010446823252716201052652272111660396665573092547110557853763466820\newline
653109896526918620564769312570586356620185581007293606598764861179104533488503461136576867532494\newline
416680396265797877185560845529654126654085306143444318586769751456614068007002378776591344017127\newline
494704205622305389945613140711270004078547332699390814546646458807972708266830634328587856983052\newline
358089330657574067954571637752542021149557615814002501262285941302164715509792592309907965473761\newline
255176567513575178296664547791745011299614890304639947132962107340437518957359614589019389713111\newline
790429782856475032031986915140287080859904801094121472213179476477726224142548545403321571853061\newline
422881375850430633217518297986622371721591607716692547487389866549494501146540628433663937900397\newline
6926567214638530673609657120918076383271664162748888007869256029022847210403172118608204190004...
\end{center}
\newpage

\subsection*{1 - Histoire et formules}
$\pi$ est un nombre qui permet de calculer le périmiètre d'un cercle et l'aire d'un disque à partir de son rayon. Archimède l'utilise pour la première fois vers -250 dans son ouvrage \og De la mesure du cercle \fg. L’usage de la lettre grecque $\pi$, première lettre du mots périmètre en grec ancien est apparu vers 1600 par William Oughtred. \\

Le nombre $\pi$ possède un nombre infini de chiffres après la virgule. On appelle ces chiffres les décimales de $\pi$. Il est écrit au dos les 4700 premières. Le record actuel de décimales connues date du 14 Mars 2019. Il est détenu par Emma Haruka Iwao ingénieur chez Google. Elle a calculé : 31 415 926 535 897 décimales.

\begin{enumerate}
\item[1.] \textbf{Chercher dans les décimales de $\pi$ votre date de naissance et sa position} sur le site \url{http://www.facade.com/legacy/amiinpi/}. Par exemple, Je suis né le 23 décembre, si je tape 2312, j'obtiens ma date de naissance dans $\pi$ à la 34033 décimales.

\item[2.] De nombreuses formules, de physique, d’ingénierie et de mathématiques utilisent $\pi$. Elle est une des constantes les plus importantes des mathématiques. Le nombre $\pi$ est irrationnel. On ne peut pas l'écrire simplement comme une fraction. \textbf{Chercher et écrire les formules suivantes :} 

  \begin{multicols}{3}
  \begin{enumerate}
  \item Périmètre du cercle
  \item Aire du disque
  \item Volume de la sphère 
  \end{enumerate}
  \end{multicols}

\end{enumerate}

\vspace{-0.4cm}
\subsection*{2 - Approximation de $\pi$}

En écrivant $\pi \approx 3.14$, on donne deux décimales de $\pi$ : 1 et 4. On dit que la précision est $10^{-2}$.\\
En écrivant $\pi \approx 3.14159$, on donne 5 décimales de $\pi$. On dit que la précision est $10^{-5}$.

\begin{enumerate}
\item[1.] \textbf{Faire les calculs suivants. Comparer avec le nombre $\pi$ et donner le nombre de décimales correctes. Donner la précision.}

  \begin{multicols}{3}
  \begin{enumerate}
    \item $\dfrac{22}{7} $
    \item $\dfrac{355}{113} $
    \item $\dfrac{208341}{66317} $
    \item $\dfrac{9}{5} + \sqrt{\dfrac{9}{5}} $
    \item $\dfrac {9801{\sqrt {2}}}{4412}$
    \item $\dfrac{7^7}{4^9} $
    \item $\sqrt{2} + \sqrt{3} $
    \item $\dfrac {19\sqrt{7}}{16} $
  \end{enumerate}
  \end{multicols}

\item[2.] \textbf{Donner le nom de votre calculatrice. Écrire la valeur de $\pi$ qu'elle vous propose. Donner la précision.}
\end{enumerate}

\vspace{-0.4cm}
\subsection*{3 - Srinivasa Ramanujan}

\begin{enumerate}
  \item[1.] \textbf{Donner le lieu et la date de naissance du mathématicien Srinivasa Ramanujan.}
  \item[2.] \textbf{Préciser ses trois grands domaines d'études mathématiques.}
  \item[3.] \textbf{Anecdote du taxi :}

    {\fontfamily{jkplos}\selectfont 
     \og Je me souviens d'avoir pris un taxi avec Srinivasa Ramanujan. Le taxi portait le numéro 1729. Il me dit alors, c'est un nombre très intéressant. 1729 peut s'écrire comme la somme de deux cubes de deux manières différentes. Je me dit alors qu'il donnait l'impression que chaque entier était un de ses amis personnels.\fg
    }
    Calculer $9^{3}+10^{3}$ et $1^{3}+12^{3}$. Avait-il raison ?
 \end{enumerate}

\vspace{-0.4cm}
 \subsection*{4 - sujet d'invention - (Bonus)}

\textbf{Inventer un nombre. Donner lui un symbole original que vous dessinerez en grand. Inventer une situation où l'utiliser (même si cela est faux).}

\vspace{-0.4cm}
\subsection*{En savoir plus}

\begin{itemize}
  \item Person of Interest S2E11 : \url{https://www.youtube.com/watch?v=a7SfWlqQ0yU}
  \item Wikipedia - $\pi$: \url{https://fr.wikipedia.org/wiki/Pi}
  \item Maths et tiques : \url{https://www.maths-et-tiques.fr/index.php/histoire-des-maths/nombres/le-nombre-pi}
  \item Wikipedia - Srinivasa Ramanujan : \url{https://fr.wikipedia.org/wiki/Srinivasa_Ramanujan}
  \item Wikipedia - Approximation : \url{https://fr.wikipedia.org/wiki/Approximation_de_%CF%80}
\end{itemize}

\end{document}
\documentclass[11pt]{article}
\usepackage{geometry} % Pour passer au format A4
\geometry{hmargin=1cm, vmargin=2cm} % 

% Page et encodage
\usepackage[T1]{fontenc} % Use 8-bit encoding that has 256 glyphs
\usepackage[english,french]{babel} % Français et anglais
\usepackage[utf8]{inputenc} 

\usepackage{lmodern}
\setlength\parindent{0pt}


\usepackage{fourier}
%\usepackage[scaled=0.875]{helvet} 
\renewcommand{\ttdefault}{lmtt} 
\usepackage{amsmath,amssymb,makeidx}
\usepackage[normalem]{ulem}
\usepackage{fancybox,graphicx}
\usepackage{tabularx,booktabs}
\usepackage{pifont}
\usepackage{ulem}
\usepackage{dcolumn}
\usepackage{textcomp}
\usepackage{diagbox}
\usepackage{tabularx}
\usepackage{multirow,multicol}
\usepackage{lscape}
\newcommand{\euro}{\eurologo{}}

%Tapuscrit : François Kriegk
\usepackage{pst-eucl}
\usepackage{diagbox}% à mettre après pst-eucl
\usepackage{graphicx,pstricks,pst-plot,pst-grad,pst-node,pst-text,pstricks-add}
\usepackage{pgf,tikz,pgfplots}
\usetikzlibrary{patterns,calc,decorations.pathmorphing}
\setlength\paperheight{297mm}
\setlength\paperwidth{210mm}
\setlength{\textheight}{25cm}
\newcommand{\R}{\mathbb{R}}
\newcommand{\N}{\mathbb{N}}
\newcommand{\D}{\mathbb{D}}
\newcommand{\Z}{\mathbb{Z}}
\newcommand{\Q}{\mathbb{Q}}
\newcommand{\C}{\mathbb{C}}

\renewcommand{\theenumi}{\textbf{\arabic{enumi}}}
\renewcommand{\labelenumi}{\textbf{\theenumi.}}
\renewcommand{\theenumii}{\textbf{\alph{enumii}}}
\renewcommand{\labelenumii}{\textbf{\theenumii.}}

\newcommand{\vect}[1]{\mathchoice%
  {\overrightarrow{\displaystyle\mathstrut#1\,\,}}%
  {\overrightarrow{\textstyle\mathstrut#1\,\,}}%
  {\overrightarrow{\scriptstyle\mathstrut#1\,\,}}%
  {\overrightarrow{\scriptscriptstyle\mathstrut#1\,\,}}}
\def\Oij{$\left(\text{O},~\vect{\imath},~\vect{\jmath}\right)$}
\def\Oijk{$\left(\text{O},~\vect{\imath},~\vect{\jmath},~\vect{k}\right)$}
\def\Ouv{$\left(\text{O},~\vect{u},~\vect{v}\right)$}
\setlength{\voffset}{-1,5cm}

\usepackage{fancyhdr} 
\pagestyle{fancyplain} 

\fancyhead{} % No page header
\fancyfoot{}

\renewcommand{\headrulewidth}{0pt} % Remove header underlines
\renewcommand{\footrulewidth}{0pt} % Remove footer underlines

\newcommand{\horrule}[1]{\rule{\linewidth}{#1}} % Create horizontal rule command with 1 argument of height

\newcommand{\tempsexo}[1]{\textit{\textbf{(#1)}}}
%----------------------------------------------------------------------------------------
%   Début du document
%----------------------------------------------------------------------------------------

\begin{document}

%----------------------------------------------------------------------------------------
% RE-DEFINITION
%----------------------------------------------------------------------------------------
% MATHS
%-----------

\newtheorem{Definition}{Définition}
\newtheorem{Theorem}{Théorème}
\newtheorem{Proposition}{Propriété}

% MATHS
%-----------
\renewcommand{\labelitemi}{$\bullet$}
\renewcommand{\labelitemii}{$\circ$}
%----------------------------------------------------------------------------------------
%   Titre
%----------------------------------------------------------------------------------------

\setlength{\columnseprule}{1pt}

\horrule{2px}
\section*{Troisième}
\horrule{2px}

\section*{S1 : Semaine du 16/03 au 22/03 - Exercice complémentaire}

\begin{itemize}
  \item Brevet 2019 - Centres étrangers
  \item 14 points
  \item 10 / 30 minutes pour l'exercice
  \item 5 / 10 minutes pour la lecture et la compréhension de la correction
\end{itemize}

 \horrule{2px}

Une famille a effectué une randonnée en montagne. Le graphique ci-dessous donne la distance parcourue en km en fonction du temps en heures.

\begin{center}
\psset{xunit=1.5cm,yunit=0.3cm}
\begin{pspicture*}(-0.5,-4)(7.4,27)
\uput[r](0,26){Distance en km}
\uput[d](6.5,-2){Temps en heures}
\psgrid[gridlabels=0,subgriddiv=5](0,0)(8,25)
\psaxes[linewidth=1pt,Dy=5](0,0)(0,0)(8,25)
\psline[linewidth=1.25pt](0,0)(1,4)(2,7)(3,8)(4,15)(5,15)(6,18)(7,20)
\end{pspicture*}
\end{center}
\medskip

\begin{enumerate}
\item Ce graphique traduit-il une situation de proportionnalité ? Justifier la réponse.
\item On utilisera le graphique pour répondre aux questions suivantes. Aucune justification n'est
demandée.
	\begin{enumerate}
		\item Quelle est la durée totale de cette randonnée?
		\item Quelle distance cette famille a-t-elle parcourue au total?
		\item Quelle est la distance parcourue au bout de $6$~h de marche?
		\item Au bout de combien de temps ont-ils parcouru les $8$ premiers km ?
		\item Que s'est-il passé entre la $4$\up{e} et la $5$\up{e} heure de randonnée?
	\end{enumerate}
\item  Un randonneur expérimenté marche à une vitesse moyenne de $4$~km/h sur toute la randonnée.
Cette famille est-elle expérimentée? Justifier la réponse.
\end{enumerate}

\newpage

\section*{Correction}

	\begin{enumerate}
		\item[1.] Les points du graphique ne sont pas alignés. Il ne s’agit doncpas d’une situation de proportionnalité.
    \item[2.]
    \begin{enumerate}
		  \item[a.] La randonnée a duré 7 heures.
      \item[b.] La famille a parcouru 20 km.
      \item[c.] Le point d’abscisse 6 a une ordonnée de 18 : au bout de six heures la famille a parcouru 18 km.
      \item[d.] Le point d’ordonnée 8 a pour abscisse 3 : la famille a parcouru8 km en 3 heures.
      \item[e.] Entre la 4eet la 5eheure la distance parcourue n’a pas augmenté : ceci signifie que la famille s’est arrêtée.
      \end{enumerate}
    \item[3] 3.Un randonneur expérimenté parcourt $7 \times 4=28 km$ en 7 heures. La famille n’en a fait que 20 : elle n’est pas expérimentée.
    \end{enumerate}

\end{document}
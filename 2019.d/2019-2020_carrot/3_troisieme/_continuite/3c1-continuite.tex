\documentclass[11pt]{article}
\usepackage{geometry} % Pour passer au format A4
\geometry{hmargin=1cm, vmargin=1cm} % 

% Page et encodage
\usepackage[T1]{fontenc} % Use 8-bit encoding that has 256 glyphs
\usepackage[english,french]{babel} % Français et anglais
\usepackage[utf8]{inputenc} 

\usepackage{lmodern}
\setlength\parindent{0pt}

% Graphiques
\usepackage{graphicx,float,grffile}

% Maths et divers
\usepackage{amsmath,amsfonts,amssymb,amsthm,verbatim}
\usepackage{multicol,enumitem,url,eurosym,gensymb}

% Sections
\usepackage{sectsty} % Allows customizing section commands
\allsectionsfont{\centering \normalfont\scshape}

% Tête et pied de page

\usepackage{fancyhdr} 
\pagestyle{fancyplain} 

\fancyhead{} % No page header
\fancyfoot{}

\renewcommand{\headrulewidth}{0pt} % Remove header underlines
\renewcommand{\footrulewidth}{0pt} % Remove footer underlines

\newcommand{\horrule}[1]{\rule{\linewidth}{#1}} % Create horizontal rule command with 1 argument of height

\newcommand{\tempsexo}[1]{\textit{\textbf{(#1)}}}
%----------------------------------------------------------------------------------------
%   Début du document
%----------------------------------------------------------------------------------------

\begin{document}

%----------------------------------------------------------------------------------------
% RE-DEFINITION
%----------------------------------------------------------------------------------------
% MATHS
%-----------

\newtheorem{Definition}{Définition}
\newtheorem{Theorem}{Théorème}
\newtheorem{Proposition}{Propriété}

% MATHS
%-----------
\renewcommand{\labelitemi}{$\bullet$}
\renewcommand{\labelitemii}{$\circ$}
%----------------------------------------------------------------------------------------
%   Titre
%----------------------------------------------------------------------------------------

\setlength{\columnseprule}{1pt}

\horrule{2px}
\section*{Troisième}
\horrule{2px}

\section{S1 : Semaine du 16/03 au 22/03}

\subsection{Travail sur le chapitre - Fonctions linéaires}

\begin{enumerate}
\item[1.] \huge{Lire} \\ \normalsize 
Lire p120, p121 \tempsexo{15 minutes}

\item[2.] \huge{Écrire} \\ \normalsize 
  Dans le cahier, côté cours. \tempsexo{40 minutes}

	\textsc{Chapitre - Fonctions linéaires}
	
	Recopier la page 120 - 1 - fonctions linéaires. Les trois parties A, B et C avec les définition et les propriétés.

\item[3.] \huge{Faire} \\ \normalsize 
Dans le cahier côté exercices. 

	\textsc{Chapitre - Fonctions linéaires}
	
	Faire les quatre exercices :
	\begin{itemize}
  	\item p122 ex 1 \tempsexo{5 minutes}
	\item p122 ex 3 \tempsexo{10 minutes}
	\item p122 ex 4 \tempsexo{10 minutes}
	\item p133 ex 79 \tempsexo{30 minutes}
  \end{itemize}

\end{enumerate}		

\subsection{Travail graphique}

\begin{enumerate}
\item[1.] \huge{Faire} \\ \normalsize 
  Rappel Translation \tempsexo{30 minutes} \\
	Faire autant d'essaie que nécéssaire pour réussir avec de bonnes notes à l'activité.

	\begin{itemize} 
    \item \url{https://www.geogebra.org/classic/EdZVhFZU} 
  \end{itemize}
  
  Pour vous aidez si besoin : \url{https://www.maths-et-tiques.fr/telech/19Transfo1.pdf}



\item[2.] \huge{Faire} \\ \normalsize 
Activité de géométrie sur géogebra en ligne. Rappel programme de construction \tempsexo{10 minutes} 

	\begin{itemize} 
	  \item \url{https://www.geogebra.org/classic/axs8vZD4}
	  \item \url{https://www.geogebra.org/classic/?id=2448135}
  \end{itemize}

\end{enumerate}		
\subsection{Travail complémentaire}

\begin{enumerate}
\item[1.] \huge{Lire} \\ \normalsize 
  Qui a inventé l'algèbre \tempsexo{15 minutes}

	Lecture Al Khwarizmi : \url{https://www.maths-et-tiques.fr/index.php/histoire-des-maths/mathematiciens-celebres/al-khwarizmi}
	
\item[2.] \huge{Faire} \\ \normalsize 

  S'entrainer au brevet \tempsexo{30 minutes exercices et 15 minutes corrections}

	Exercice brevet 1 : présent dans  collection troisième sur pearltree (avec sa correction sur la deuxième page).
\end{enumerate}		

\newpage
\section{Evaluation S1 }

Les documents textes, ou photos du cahier sont à déposer dans la collection que vous partagez avec moi dans pearltrees.

La réussite des exercices sans erreur n'est pas nécéssaire pour avoir les points.

	\begin{itemize} 
	  \item \tempsexo{3 points} Les exercices sont fait et déposés sur Pearltrees avant le 23 / 03 \\	
    \textbf{p122 ex 1, 3 et 4}
	
	  \item \tempsexo{3 points} L'exercice est fait et déposé sur Pearltrees avant le 30 / 03 \\
	\textbf{p133 ex 79}

	  \item \tempsexo{2 points} Avoir écrit le cours, une photo du cours écrit dans le cahier.
		\textbf{Cours : Chapitre - Fonctions linéaires}
\end{itemize}	
\end{document}

\documentclass[11pt]{article}
\usepackage{geometry} % Pour passer au format A4
\geometry{hmargin=1cm, vmargin=2cm} % 

% Page et encodage
\usepackage[T1]{fontenc} % Use 8-bit encoding that has 256 glyphs
\usepackage[english,french]{babel} % Français et anglais
\usepackage[utf8]{inputenc} 

\usepackage{lmodern}
\setlength\parindent{0pt}


\usepackage{fourier}
%\usepackage[scaled=0.875]{helvet} 
\renewcommand{\ttdefault}{lmtt} 
\usepackage{amsmath,amssymb,makeidx}
\usepackage[normalem]{ulem}
\usepackage{fancybox,graphicx}
\usepackage{tabularx,booktabs}
\usepackage{pifont}
\usepackage{ulem}
\usepackage{dcolumn}
\usepackage{textcomp}
\usepackage{diagbox}
\usepackage{tabularx}
\usepackage{multirow,multicol}
\usepackage{lscape}
\newcommand{\euro}{\eurologo{}}

%Tapuscrit : François Kriegk
\usepackage{pst-eucl}
\usepackage{diagbox}% à mettre après pst-eucl
\usepackage{graphicx,pstricks,pst-plot,pst-grad,pst-node,pst-text,pstricks-add}
\usepackage{pgf,tikz,pgfplots}
\usetikzlibrary{patterns,calc,decorations.pathmorphing}
\setlength\paperheight{297mm}
\setlength\paperwidth{210mm}
\setlength{\textheight}{25cm}
\newcommand{\R}{\mathbb{R}}
\newcommand{\N}{\mathbb{N}}
\newcommand{\D}{\mathbb{D}}
\newcommand{\Z}{\mathbb{Z}}
\newcommand{\Q}{\mathbb{Q}}
\newcommand{\C}{\mathbb{C}}

\renewcommand{\theenumi}{\textbf{\arabic{enumi}}}
\renewcommand{\labelenumi}{\textbf{\theenumi.}}
\renewcommand{\theenumii}{\textbf{\alph{enumii}}}
\renewcommand{\labelenumii}{\textbf{\theenumii.}}

\newcommand{\vect}[1]{\mathchoice%
  {\overrightarrow{\displaystyle\mathstrut#1\,\,}}%
  {\overrightarrow{\textstyle\mathstrut#1\,\,}}%
  {\overrightarrow{\scriptstyle\mathstrut#1\,\,}}%
  {\overrightarrow{\scriptscriptstyle\mathstrut#1\,\,}}}
\def\Oij{$\left(\text{O},~\vect{\imath},~\vect{\jmath}\right)$}
\def\Oijk{$\left(\text{O},~\vect{\imath},~\vect{\jmath},~\vect{k}\right)$}
\def\Ouv{$\left(\text{O},~\vect{u},~\vect{v}\right)$}
\setlength{\voffset}{-1,5cm}

\usepackage{fancyhdr} 
\pagestyle{fancyplain} 

\fancyhead{} % No page header
\fancyfoot{}

\renewcommand{\headrulewidth}{0pt} % Remove header underlines
\renewcommand{\footrulewidth}{0pt} % Remove footer underlines

\newcommand{\horrule}[1]{\rule{\linewidth}{#1}} % Create horizontal rule command with 1 argument of height

\newcommand{\tempsexo}[1]{\textit{\textbf{(#1)}}}
%----------------------------------------------------------------------------------------
%   Début du document
%----------------------------------------------------------------------------------------

\begin{document}

%----------------------------------------------------------------------------------------
% RE-DEFINITION
%----------------------------------------------------------------------------------------
% MATHS
%-----------

\newtheorem{Definition}{Définition}
\newtheorem{Theorem}{Théorème}
\newtheorem{Proposition}{Propriété}

% MATHS
%-----------
\renewcommand{\labelitemi}{$\bullet$}
\renewcommand{\labelitemii}{$\circ$}
%----------------------------------------------------------------------------------------
%   Titre
%----------------------------------------------------------------------------------------

\setlength{\columnseprule}{1pt}

\section*{S2 : Semaine du 23/03 au 29/03 - Exercice complémentaire}

\begin{itemize}
  \item Brevet 2019 - Amérique du Nord
  \item 12 points
  \item 10 / 30 minutes pour l'exercice
  \item 5 / 10 minutes pour la lecture et la compréhension de la correction
\end{itemize}

 \horrule{2px}

\begin{center}
	\begin{tikzpicture}[y=0.125mm,x=4mm]% l'échelle ridicule proposée ici est celle du sujet original
		\draw [xstep=24,ystep=20,gray!50,line width=0.5pt] (0,0) grid (24,580);
		\foreach \a in {0,4,...,24}{
		\draw[gray!50,line width=0.5pt] (\a,0)--(\a,-4pt);}	
		\draw [xstep=24.1,ystep=100,black,line width=0.7pt] (0,0) grid (24,580);	
		\foreach \o in {0,100,...,500}{
			\node at (-1.2,\o) {\o};}
		
		\draw[fill=black!65] ( 1,0) rectangle (3 ,545);	
		\node[rotate=65,left] at(2,-30) {Pays A};
		\draw[fill=black!65] ( 5,0) rectangle (7 ,215);
		\node[rotate=65,left] at(6,-30) {Pays B};
		\draw[fill=black!65] ( 9,0) rectangle (11,150);	
		\node[rotate=65,left] at(10,-30) {Pays C};
		\draw[fill=black!65] (13,0) rectangle (15,137.5);
		\node[rotate=65,left] at(14,-30) {Pays D};
		\draw[fill=black!65] (17,0) rectangle (19,130);	
		\node[rotate=65,left] at(18,-30) {Pays E};
		\draw[fill=black!65] (21,0) rectangle (23,110);
		\node[rotate=65,left] at(22,-30) {Pays F};
		\node at (12,610){Quantité de nourriture gaspillée en kg par habitant en 2010};
	\end{tikzpicture}
\end{center}

Le diagramme ci-contre représente, pour six pays, la quantité de nourriture gaspillée (en kg) par habitant en 2010.

\begin{enumerate}
	\item Donner approximativement la quantité de nourriture gaspillée par un habitant du pays D en 2010.	
	\item Peut-on affirmer que le gaspillage de nourriture d'un habitant du pays F représente environ un cinquième du gaspillage de nourriture d'un habitant du pays A ?	
	\item On veut rendre compte de la quantité de nourriture gaspillée pour d'autres pays. On réalise alors le tableau ci-dessous à l'aide d'un tableur. \hfill \textit{Rappel :} 1 tonne = 1000kg.	
	\begin{tabular}{|m{5mm}| p{1.5cm} | p{3.5cm} | p{3.5cm} | p{3.5cm}|} \hline
	
      & \textsf{A} & \textsf{B} & \textsf{C} & \textsf{D}\\ \hline
	1 &            &\parbox{\linewidth}{\rule{0pt}{10pt}Quantité de nourriture gaspillée par habitant en 2010 (en kg)\rule[-4pt]{0pt}{1pt}}
		                        &\parbox{\linewidth}{Nombre d'habitants en 2010 (en millions)} 
		                                     &\parbox{\linewidth}{Quantité totale de nourriture gaspillée (en tonnes)}\\ \hline	
	2 & Pays X & 345 & 10,9 & 3760500\\ \hline	
	3 & Pays Y & 212 & 9,4  &             \\ \hline
	4 & Pays Z & 135 & 46,6 &             \\ \hline
	\end{tabular}

	\begin{enumerate}
		\item  Quelle est la quantité totale de nourriture gaspillée par les habitants du pays X en 2010 ?		
		\item Voici trois propositions de formule, recopier sur votre copie celle qu'on a saisie dans
		la cellule \textsf{D2} avant de l'étirer jusqu'en \textsf{D4}.
		
		\renewcommand{\arraystretch}{1.5}
		\begin{tabular}{| p{3.5cm} | p{3.5cm} | p{3.5cm} |} \hline
			\textbf{Proposition 1}&\textbf{Proposition 2}&\textbf{Proposition 3}\\ \hline
			\textsf{=B2*C2*1000000}&\textsf{=B2*C2}& \textsf{=B2*C2*1000}\\ \hline
		\end{tabular}
	\end{enumerate}
\end{enumerate}

\newpage

\section*{Correction}

\begin{enumerate}
	\item On lit approximativement 130~kg.
	\item On lit pour un habitant du pays F à peu près 110 et pour un habitant du pays A un peu plus de 540~kg. Comme $5 \times 110 = 550$ l'affirmation est correcte.	
	\item 
	\begin{enumerate}
		\item  Le résultat est dans le tableau. On peut le justifier :
        
        La quantité totale pour les habitants du pays X est :
		
        $345 \times 10,9 \times 10^6 = 3760500000 kg$ soit $3760500 tonnes$.	
		\item \textsf{=B2*C2*1000}
	\end{enumerate}
\end{enumerate}


\end{document}
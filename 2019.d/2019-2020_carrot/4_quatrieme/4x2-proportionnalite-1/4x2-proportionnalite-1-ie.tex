\documentclass[12pt]{article}
\usepackage{geometry} % Pour passer au format A4
\geometry{hmargin=1cm, vmargin=1cm} % 

% Page et encodage
\usepackage[T1]{fontenc} % Use 8-bit encoding that has 256 glyphs
\usepackage[english,french]{babel} % Français et anglais
\usepackage[utf8]{inputenc} 

\usepackage{lmodern}
\setlength\parindent{0pt}

% Graphiques
\usepackage{graphicx,float,grffile,tikz}

% Maths et divers
\usepackage{amsmath,amsfonts,amssymb,amsthm,verbatim}
\usepackage{multicol,enumitem,url,eurosym,gensymb,multido}

% Sections
\usepackage{sectsty} % Allows customizing section commands
\allsectionsfont{\centering \normalfont\scshape}

% Tête et pied de page

\usepackage{fancyhdr} 
\pagestyle{fancyplain} 

\fancyhead{} % No page header
\fancyfoot{}

\renewcommand{\headrulewidth}{0pt} % Remove header underlines
\renewcommand{\footrulewidth}{0pt} % Remove footer underlines

\newcommand{\horrule}[1]{\rule{\linewidth}{#1}} % Create horizontal rule command with 1 argument of height

%----------------------------------------------------------------------------------------
%	Début du document
%----------------------------------------------------------------------------------------

\begin{document}

% MATHS
%-----------
\newcommand{\Pointilles}[1][3]{%
  \multido{}{#1}{\makebox[\linewidth]{\dotfill}\\[\parskip]
}}
%----------------------------------------------------------------------------------------
%	Titre
%----------------------------------------------------------------------------------------

\setlength{\columnseprule}{1pt}

\textbf{Nom, Prénom :} \hspace{8cm} \textbf{Classe :} \hspace{3cm} \textbf{Date :}\\

\begin{center}
  \textit{If you do the work, you get rewarded. There are no shortcuts in life.}  - \textbf{Michael Jordan}
\end{center}


\subsection*{ex1 - Calcul}

\begin{multicols}{3}\noindent

  \begin{enumerate}
  \item $1 + \left( -1\right) = \ldots\ldots$
  \item $10 \div \ldots\ldots = -5$
  \item $-7 \times \left( -1\right) = \ldots\ldots$
  \item $32 \div 4 = \ldots\ldots$
  \item $\ldots\ldots - \left( -5\right) = 6$
  \item $-6 \times \left( -8\right) = \ldots\ldots$
  \item $-8 + \left( -2\right) = \ldots\ldots$
  \item $-7 - \ldots\ldots = -9$
  \item $-3 + \left( -1\right) = \ldots\ldots$
  \item $-24 \div \ldots\ldots = -4$
  \item $2 \times \left( -10\right) = \ldots\ldots$
  \item $-7 + \left( -6\right) = \ldots\ldots$
  \item $-2 - \left( -8\right) = \ldots\ldots$
  \item $\ldots\ldots \div 10 = 5$
  \item $-16 \div \ldots\ldots = -8$
  \item $1 + \ldots\ldots = 4$
  \item $-9 - \left( -3\right) = \ldots\ldots$
  \item $-1 - \left( -10\right) = \ldots\ldots$
  \item $-4 \times \left( -2\right) = \ldots\ldots$
  \item $8 \times \ldots\ldots = -40$
  \end{enumerate}
  
\end{multicols}

\subsection*{ex2 - Proportionnalité}

\begin{enumerate}
\item Remplir les tableaux de proportionnalité. Écrire le calcul effectuté en dessous.

  \begin{multicols}{4}\noindent
    \begin{center}
      \begin{tabular}{|c|c|}
        \hline
        1 & 32\\  \hline
        14,5 & $\phantom{azertyuiop}$\\  \hline
      \end{tabular}
    \end{center}
    \Pointilles[1]
    \begin{center}
      \begin{tabular}{|c|c|}
        \hline
        12 & $\phantom{azertyuiop}$\\  \hline
        1,5 & 24\\  \hline
      \end{tabular}
    \end{center}
    \Pointilles[1]
    \begin{center}
      \begin{tabular}{|c|c|}
        \hline
        $\phantom{azertyuiop}$  & 24\\  \hline
        8 & 68\\  \hline
      \end{tabular}
    \end{center}
    \Pointilles[1]
    \begin{center}
      \begin{tabular}{|c|c|}
        \hline
        12 & 46\\  \hline
        $\phantom{azertyuiop}$ & 24\\  \hline
      \end{tabular}
    \end{center}
    \Pointilles[1]
  \end{multicols}

\item Dans une épicerie, le prix des fruits est proportionnel à la masse achetée. 

  \begin{center}
    \begin{tabular}{|c|c|c|c|c|c|c|}
      \hline
      Masse en Kg.&  0.8 & 1,1 & 1,6 & $\phantom{azertyuiop}$   & $\phantom{azertyuiop}$   & 3 \\  \hline
      Prix en \euro & 2.16 & $\phantom{azertyuiop}$    & $\phantom{azertyuiop}$    & 16 & 25 & $\phantom{azertyuiop}$   \\  \hline
    \end{tabular}
  \end{center}

\item Les tableaux suivant sont-ils proportionnels. Justifier. S'ils ne le sont pas, corriger.

  \begin{multicols}{2}\noindent
    \vspace{0.4cm}
    \begin{center}
      \begin{tabular}{|c|c|c|}
        \hline
        2 & 8 & 26 \\  \hline
        24 & 96 & 212\\  \hline
      \end{tabular}
    \end{center}
    \vspace{1cm}
    \Pointilles[5]
  \end{multicols}

  \begin{multicols}{2}\noindent
    \vspace{0.4cm}
    \begin{center}
      \begin{tabular}{|c|c|c|}
        \hline
        12 & 18 & 6 \\  \hline
        30 & 45 & 15\\  \hline
      \end{tabular}
    \end{center}
    \vspace{1cm}
    \Pointilles[5]
  \end{multicols}

\end{enumerate}
\newpage
\subsection*{ex3 - Pourcentages}

\begin{enumerate}
\item Une entreprise a produit 650 tonnes d'écrous et de vis. Elle a vendu un quart de sa production sur le marché national, 20 \% sur le marché européen, 15 \% sur le marché américain et le reste sur le marché asiatique. Dans chaque cas, calculer la production en tonnes.
\item Un commerçant a accordé un rabais de 15 \% sur un article qui coûtait initialement 250 \euro. Calculer le prix de la remise. Calculer le nouveau prix de vente.
\item Un collège de 720 élèves compte 372 élèves demi-pensionnaires. Calculer le pourcentage d'élèves demi-pensionnaires de ce collège. Cacluler le pourcentage d'externe. 
\end{enumerate}
\Pointilles[17]
\subsection*{ex4 - Vitesse}

\begin{enumerate}
\item Écrire les trois relations reliant la vitesse, la distance et le temps.
\item Je mets 12 minutes pour aller chercher mon pain à vélo à la boulangerie qui se situe à 4,2 km de chez moi. Si je pouvais maintenir cette allure de
  manière constante, quelle distance aurais-je parcourue en 1 h 22 min ?
\end{enumerate}
\Pointilles[15]

\newpage

\textbf{Nom, Prénom :} \hspace{8cm} \textbf{Classe :} \hspace{3cm} \textbf{Date :}\\

\begin{center}
  \textit{If you do the work, you get rewarded. There are no shortcuts in life.}  - \textbf{Michael Jordan}
\end{center}


\subsection*{ex1 - Calcul}

\begin{multicols}{3}\noindent

    \begin{enumerate}
    \item $6 \times 7 = \ldots\ldots$
    \item $12 - 4 = \ldots\ldots$
    \item $-6 \times \left( -8\right) = \ldots\ldots$
    \item $-1 + 7 = \ldots\ldots$
    \item $-5 \div \ldots\ldots = 5$
    \item $-6 \times \ldots\ldots = 30$
    \item $-1 + \left( -9\right) = \ldots\ldots$
    \item $-8 \div 2 = \ldots\ldots$
    \item $\ldots\ldots + \left( -7\right) = -5$
    \item $\ldots\ldots + \left( -2\right) = 5$
    \item $1 - 4 = \ldots\ldots$
    \item $1 - \ldots\ldots = 4$
    \item $-35 \div \left( -5\right) = \ldots\ldots$
    \item $2 - \left( -6\right) = \ldots\ldots$
    \item $-1 + \left( -7\right) = \ldots\ldots$
    \item $\ldots\ldots \div 1 = 3$
    \item $-9 - \left( -5\right) = \ldots\ldots$
    \item $-24 \div \ldots\ldots = -4$
    \item $-8 \times 4 = \ldots\ldots$
    \item $\ldots\ldots \times \left( -5\right) = -45$
    \end{enumerate}
  
\end{multicols}

\subsection*{ex2 - Proportionnalité}

\begin{enumerate}
\item Remplir les tableaux de proportionnalité. Écrire le calcul effectuté en dessous.

  \begin{multicols}{4}\noindent
    \begin{center}
      \begin{tabular}{|c|c|}
        \hline
        1 & 22\\  \hline
        24,5 & $\phantom{azertyuiop}$\\  \hline
      \end{tabular}
    \end{center}
    \Pointilles[1]
    \begin{center}
      \begin{tabular}{|c|c|}
        \hline
        12 & $\phantom{azertyuiop}$\\  \hline
        2,5 & 26\\  \hline
      \end{tabular}
    \end{center}
    \Pointilles[1]
    \begin{center}
      \begin{tabular}{|c|c|}
        \hline
        $\phantom{azertyuiop}$  & 44\\  \hline
        18 & 62\\  \hline
      \end{tabular}
    \end{center}
    \Pointilles[1]
    \begin{center}
      \begin{tabular}{|c|c|}
        \hline
        22 & 44\\  \hline
        $\phantom{azertyuiop}$ & 24\\  \hline
      \end{tabular}
    \end{center}
    \Pointilles[1]
  \end{multicols}

\item Dans une épicerie, le prix des fruits est proportionnel à la masse achetée. 

  \begin{center}
    \begin{tabular}{|c|c|c|c|c|c|c|}
      \hline
      Masse en Kg.&  0.8 & 1,2 & 1,8 & $\phantom{azertyuiop}$   & $\phantom{azertyuiop}$   & 4 \\  \hline
      Prix en \euro & 3.16 & $\phantom{azertyuiop}$    & $\phantom{azertyuiop}$    & 18 & 35 & $\phantom{azertyuiop}$   \\  \hline
    \end{tabular}
  \end{center}

\item Les tableaux suivant sont-ils proportionnels. Justifier. S'ils ne le sont pas, corriger.

  \begin{multicols}{2}\noindent
    \vspace{0.4cm}
    \begin{center}
      \begin{tabular}{|c|c|c|}
        \hline
        2 & 8 & 26 \\  \hline
        26 & 104 & 338\\  \hline
      \end{tabular}
    \end{center}
    \vspace{1cm}
    \Pointilles[5]
  \end{multicols}

  \begin{multicols}{2}\noindent
    \vspace{0.4cm}
    \begin{center}
      \begin{tabular}{|c|c|c|}
        \hline
        12 & 18 & 6 \\  \hline
        60 & 80 & 30\\  \hline
      \end{tabular}
    \end{center}
    \vspace{1cm}
    \Pointilles[5]
  \end{multicols}

\end{enumerate}
\newpage
\subsection*{ex3 - Pourcentages}

\begin{enumerate}
\item Une entreprise a produit 550 tonnes d'écrous et de vis. Elle a vendu un demi de sa production sur le marché national, 10 \% sur le marché européen, 15 \% sur le marché américain et le reste sur le marché asiatique. Dans chaque cas, calculer la production en tonnes.
\item Un commerçant a accordé un rabais de 25 \% sur un article qui coûtait initialement 210 \euro. Calculer le prix de la remise. Calculer le nouveau prix de vente.
\item Un collège de 660 élèves compte 272 élèves demi-pensionnaires. Calculer le pourcentage d'élèves demi-pensionnaires de ce collège. Cacluler le pourcentage d'externe. 
\end{enumerate}
\Pointilles[17]
\subsection*{ex4 - Vitesse}

\begin{enumerate}
\item Écrire les trois relations reliant la vitesse, la distance et le temps.
\item Je mets 14 minutes pour aller chercher mon pain à vélo à la boulangerie qui se situe à 3,8 km de chez moi. Si je pouvais maintenir cette allure de
  manière constante, quelle distance aurais-je parcourue en 1 h 24 min ?
\end{enumerate}
\Pointilles[15]
\end{document}

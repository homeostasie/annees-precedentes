\documentclass[10pt]{article}
\usepackage{geometry} % Pour passer au format A4
\geometry{hmargin=1cm, vmargin=1cm} % 

% Page et encodage
\usepackage[T1]{fontenc} % Use 8-bit encoding that has 256 glyphs
\usepackage[english,french]{babel} % Français et anglais
\usepackage[utf8]{inputenc} 

\usepackage{lmodern}
\setlength\parindent{0pt}

% Graphiques
\usepackage{graphicx,float,grffile}

% Maths et divers
\usepackage{amsmath,amsfonts,amssymb,amsthm,verbatim}
\usepackage{multicol,enumitem,url,eurosym,gensymb}

% Sections
\usepackage{sectsty} % Allows customizing section commands
\allsectionsfont{\centering \normalfont\scshape}

% Tête et pied de page

\usepackage{fancyhdr} 
\pagestyle{fancyplain} 

\fancyhead{} % No page header
\fancyfoot{}

\renewcommand{\headrulewidth}{0pt} % Remove header underlines
\renewcommand{\footrulewidth}{0pt} % Remove footer underlines

\newcommand{\horrule}[1]{\rule{\linewidth}{#1}} % Create horizontal rule command with 1 argument of height

%----------------------------------------------------------------------------------------
%	Début du document
%----------------------------------------------------------------------------------------

\begin{document}

%----------------------------------------------------------------------------------------
% RE-DEFINITION
%----------------------------------------------------------------------------------------
% MATHS
%-----------

\newtheorem{Definition}{Définition}
\newtheorem{Theorem}{Théorème}
\newtheorem{Proposition}{Propriété}

% MATHS
%-----------
\renewcommand{\labelitemi}{$\bullet$}
\renewcommand{\labelitemii}{$\circ$}
%----------------------------------------------------------------------------------------
%	Titre
%----------------------------------------------------------------------------------------

\setlength{\columnseprule}{1pt}

\textbf{Nom, Prénom :} \hspace{8cm} \textbf{Classe :} \hspace{3cm} \textbf{Date :}\\

\begin{center}
  \textit{If you do the work, you get rewarded. There are no shortcuts in life}  - \textbf{Michael Jordan}
\end{center}

\subsubsection*{Ex 1 : Calculer}

\begin{multicols}{3}
  \begin{itemize}
  \item[a =] $7^2 + 1 =  \dotfill $
  \item[b =] $10^5+3 =  \dotfill $
  \item[c =] $5^4 \times 4^5 =  \dotfill $
  \item[d =] $(-2)^{12} =  \dotfill $
  \item[e =] $1^0 \times 2^1 \times 3^2 =  \dotfill $
  \item[f =] $0^{24} =  \dotfill $
  \end{itemize}

\end{multicols}

\subsubsection*{Ex 2 : Calculer une grande expression }


\begin{multicols}{2}

  $G = \dfrac{0,15 \times 10^{-4} \times 2,7 \times 10^{2}}{500 \times (10^5)^2} =  \dotfill $\\
  $H = \dfrac{60 \times 10^{-11} \times 0,18 \times 10^{-3}}{4,8 \times (10^{-7})^5} =  \dotfill $ 

\end{multicols}


\subsubsection*{Ex 3 : Comprendre la notation puissance}

Remplacer la notation puissance par autant de signes $\times$ qu'il le faut. 

\begin{multicols}{3}
  \begin{itemize}
  \item[a =] $6^2 =  \dotfill $
  \item[b =] $8^{10} =  \dotfill $
  \item[c =] $-5^4  =  \dotfill $
  \item[d =] $(-5)^{4} =  \dotfill $
  \item[e =] $\dfrac{1}{3^4} =  \dotfill $
  \item[f =] $ 4^{-3} =  \dotfill $
  \item[g =] $ \dfrac{2^3}{5^{4}} =  \dotfill $
  \item[h =] $ \dfrac{1}{10^{3}} =  \dotfill $
  \end{itemize}
\end{multicols}

\subsubsection*{Ex 4 : utiliser les règles de simplifications.}

Compléter par un nombre de la forme $a^n$ avec $a$ et $n$ entiers :

\begin{multicols}{4}
  \begin{enumerate}
  \item[1.] $6^{7}  \times  6^{8}  =  \dotfill$
  \item[2.] $\dfrac{10^{12}}{10^{12}} = \dotfill$
  \item[3.] $5^{6} \times 5^{5} = \dotfill$
  \item[4.] $\dfrac{11^{11}}{11^{6}} = \dotfill$
  \item[5.] $\dfrac{11^{9}}{11^{6}} = \dotfill$
  \item[6.] $5^{6} \times 5^{3} = \dotfill$
  \item[7.] $(12^{10})^{8} = \dotfill$
  \item[8.] $(10^{10})^{7} = \dotfill$
  \end{enumerate}
\end{multicols}


\subsubsection*{Ex 5 : Compléter par le nombre qui convient :}

\begin{multicols}{3}

  \begin{enumerate}
  \item[1.] $6{,}098 \times \dotfill = 0{,}000\,006\,098$
  \item[2.] $6\,602 = 6{,}602 \times \dotfill$
  \item[3.] $0{,}004\,027 = 4{,}027 \times \dotfill$
  \item[4.] $0{,}060\,04 = 6{,}004 \times \dotfill$
  \item[5.] $610{,}9 = 6{,}109 \times \dotfill$
  \item[6.] $0{,}150\,2 = 1{,}502 \times \dotfill$
  \item[7.] $0{,}000\,630\,7 = 6{,}307 \times \dotfill$
  \item[8.] $7{,}306 \times \dotfill = 7\,306$
  \item[9.] $2{,}36 \times \dotfill = 236\,000$
  \end{enumerate}
\end{multicols}

\subsubsection*{Problème 1 – Distance Vous - Mars}

La distance Terre-Mars est 76 milliards de mètres. 

\begin{itemize}
\item[1.] Donner vitre taille en m. (ex M. Lafond : 1,65m) Combien faut-il de \textsc{Vous} pour atteindre Mars ?
\end{itemize}

\subsubsection*{Problème 2 – Casa de Papel}

\textit{\og El Professeur \fg{} } vient de dérober 12 millions d’euros. \\
Les billets de banque ont une épaisseur de $60 \times 10^{-6} m$. (On dit 60 micromètres)

\begin{itemize}
\item[2.] Quelle hauteur atteindrait une pile de billets de banque de 50 \euro{} représentant cette somme ?
\end{itemize}


\subsubsection*{Problème 3 - Étoiles Vs Grain de sables.}

Il y a $2^{36}$ galaxies dans notre univers. Chaque galaxies contient $2^{41}$ étoiles.  \\
On estime le volume de sable sur Terre à $1\,000 \text{ milliards de } m^3$. Chaque $m^3$ contient environ $100 \text{milliards}$ de grains de sable. 

\begin{itemize}
\item[3.] Inès affirme qu'il y a autant de grain de sable sur Terre que d'étoiles dans l'univers. A-t-elle raison ? 
\end{itemize}



\subsubsection*{Problème 4 - $CO_2$  - \textit{(/4)}}

\begin{enumerate}
  \item[4.] Une molécule de dioxyde de carbone est composée d'un atome de carbone et de deux atomes d'oxygène. La masse d'un atome de carbone est $2 \times 10^{-26}kg$ et la masse d'un atome d'oxygène est $1.8 \times 10^{-26}kg$. \\
  Combien trouve-t-on de molécule de dioxyde carbone dans 2kg ?
\end{enumerate}

%--------------------------------------------------------------------------------------------
%--------------------------------------------------------------------------------------------

\newpage

%--------------------------------------------------------------------------------------------
%--------------------------------------------------------------------------------------------


\textbf{Nom, Prénom :} \hspace{8cm} \textbf{Classe :} \hspace{3cm} \textbf{Date :}\\

\begin{center}
  \textit{If you do the work, you get rewarded. There are no shortcuts in life}  - \textbf{Michael Jordan}
\end{center}

\subsubsection*{Ex 1 : Calculer}

\begin{multicols}{3}
  \begin{itemize}
  \item[a =] $2^7 + 1 =  \dotfill $
  \item[b =] $12^{5+3} =  \dotfill $
  \item[c =] $3^4 \times 4^3 =  \dotfill $
  \item[d =] $(-2)^{20} =  \dotfill $
  \item[e =] $13^2 \times 12^1 + 3^2 =  \dotfill $
  \item[f =] $123^0 =  \dotfill $
  \end{itemize}

\end{multicols}

\subsubsection*{Ex 2 : Calculer une grande expression }


\begin{multicols}{2}

  $G = \dfrac{0,35 \times 10^{-3} \times 2,7 \times 10^{2}}{900 \times (10^5)^2} =  \dotfill $\\
  $H = \dfrac{80 \times 10^{-10} \times 0,18 \times 10^{-3}}{3,6 \times (10^{-7})^5} =  \dotfill $ 
\end{multicols}


\subsubsection*{Ex 3 : Comprendre la notation puissance}

Remplacer la notation puissance par autant de signes $\times$ qu'il le faut. 

\begin{multicols}{3}
  \begin{itemize}
  \item[a =] $7^2 =  \dotfill $
  \item[b =] $10^{8} =  \dotfill $
  \item[c =] $-3^4  =  \dotfill $
  \item[d =] $(-3)^{4} =  \dotfill $
  \item[e =] $\dfrac{1}{4^4} =  \dotfill $
  \item[f =] $ 5^{-3} =  \dotfill $
  \item[g =] $ \dfrac{2^4}{5^{3}} =  \dotfill $
  \item[h =] $ \dfrac{1}{10^{4}} =  \dotfill $
  \end{itemize}
\end{multicols}

\subsubsection*{Ex 4 : utiliser les règles de simplifications.}

Compléter par un nombre de la forme $a^n$ avec $a$ et $n$ entiers :

\begin{multicols}{4}
  \begin{enumerate}
  \item[9.] $\dfrac{5^{11}}{5^{4}} = \dotfill$
  \item[10.] $(11^{3})^{5} = \dotfill$
  \item[11.] $(9^{8})^{7} = \dotfill$
  \item[12.] $9^{5} \times 9^{2} = \dotfill$
  \item[13.] $7^{4}  \times  2^{4}  =  \dotfill$
  \item[14.] $3^{7}  \times  5^{7}  =  \dotfill$
  \item[15.] $5^{9} \times 5^{7} = \dotfill$
  \item[16.] $\dfrac{11^{11}}{11^{8}} = \dotfill$
  \end{enumerate}
\end{multicols}


\subsubsection*{Ex 5 : Compléter par le nombre qui convient :}

\begin{multicols}{3}

  \begin{enumerate}
  \item[1.] $6{,}098 \times \dotfill = 0{,}000\,006\,098$
  \item[2.] $6\,602 = 6{,}602 \times \dotfill$
  \item[3.] $0{,}004\,027 = 4{,}027 \times \dotfill$
  \item[4.] $0{,}060\,04 = 6{,}004 \times \dotfill$
  \item[5.] $610{,}9 = 6{,}109 \times \dotfill$
  \item[6.] $0{,}150\,2 = 1{,}502 \times \dotfill$
  \item[7.] $0{,}000\,630\,7 = 6{,}307 \times \dotfill$
  \item[8.] $7{,}306 \times \dotfill = 7\,306$
  \item[9.] $2{,}36 \times \dotfill = 236\,000$
  \end{enumerate}
\end{multicols}

\subsubsection*{Problème 1 – Distance Vous - Mars}

La distance Terre-Mars est 76 milliards de mètres. 

\begin{itemize}
\item[1.] Donner vitre taille en m. (ex M. Lafond : 1,65m) Combien faut-il de \textsc{Vous} pour atteindre Mars ?
\end{itemize}

\subsubsection*{Problème 2 – Casa de Papel}

\textit{\og El Professeur \fg{} } vient de dérober 20 millions d’euros. \\
Les billets de banque ont une épaisseur de $80 \times 10^{-6} m$. (On dit 80 micromètres)

\begin{itemize}
\item[2.] Quelle hauteur atteindrait une pile de billets de banque de 50 \euro{} représentant cette somme ?
\end{itemize}


\subsubsection*{Problème 3 - Étoiles Vs Grain de sables.}

Il y a $2^{32}$ galaxies dans notre univers. Chaque galaxies contient $2^{43}$ étoiles.  \\
On estime le volume de sable sur Terre à $1\,000 \text{ milliards de } m^3$. Chaque $m^3$ contient environ $100 \text{milliards}$ de grains de sable. 

\begin{itemize}
\item[3.] Inès affirme qu'il y a autant de grain de sable sur Terre que d'étoiles dans l'univers. A-t-elle raison ? 
\end{itemize}


\subsubsection*{Problème 4 - $CO_2$  - \textit{(/4)}}

\begin{enumerate}
  \item[4.] Une molécule de dioxyde de carbone est composée d'un atome de carbone et de deux atomes d'oxygène. La masse d'un atome de carbone est $5 \times 10^{-26}kg$ et la masse d'un atome d'oxygène est $2.8 \times 10^{-26}kg$. \\
  Combien trouve-t-on de molécule de dioxyde carbone dans 4kg ?
\end{enumerate}

\end{document}

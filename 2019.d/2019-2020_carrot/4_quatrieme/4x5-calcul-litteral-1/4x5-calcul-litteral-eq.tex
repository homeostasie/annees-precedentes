\documentclass[12pt]{article}
\usepackage{geometry} % Pour passer au format A4
\geometry{hmargin=1cm, vmargin=1cm} % 

% Page et encodage
\usepackage[T1]{fontenc} % Use 8-bit encoding that has 256 glyphs
\usepackage[english,french]{babel} % Français et anglais
\usepackage[utf8]{inputenc} 

\usepackage{lmodern}
\setlength\parindent{0pt}

% Graphiques
\usepackage{graphicx,float,grffile}

% Maths et divers
\usepackage{amsmath,amsfonts,amssymb,amsthm,verbatim}
\usepackage{multicol,enumitem,url,eurosym,gensymb}

% Sections
\usepackage{sectsty} % Allows customizing section commands
\allsectionsfont{\centering \normalfont\scshape}

% Tête et pied de page

\usepackage{fancyhdr} 
\pagestyle{fancyplain} 

\fancyhead{} % No page header
\fancyfoot{}

\renewcommand{\headrulewidth}{0pt} % Remove header underlines
\renewcommand{\footrulewidth}{0pt} % Remove footer underlines

\newcommand{\horrule}[1]{\rule{\linewidth}{#1}} % Create horizontal rule command with 1 argument of height

%----------------------------------------------------------------------------------------
%   Début du document
%----------------------------------------------------------------------------------------

\begin{document}

%----------------------------------------------------------------------------------------
% RE-DEFINITION
%----------------------------------------------------------------------------------------
% MATHS
%-----------

\newtheorem{Definition}{Définition}
\newtheorem{Theorem}{Théorème}
\newtheorem{Proposition}{Propriété}

% MATHS
%-----------
\renewcommand{\labelitemi}{$\bullet$}
\renewcommand{\labelitemii}{$\circ$}
%----------------------------------------------------------------------------------------
%   Titre
%----------------------------------------------------------------------------------------

\setlength{\columnseprule}{1pt}

\horrule{2px}
\section*{DM - Modélisation du monde}
\horrule{2px}

Après avoir entourer votre équation. 
\textbf{Écrire l'équation en grand sur une feuille A4. Coller des images en lien / dessiner pour l'illustrer. Dire une ou deux phrases pour résumer \og de quoi elle parle \fg}

\textit{Les équations avec (*) sont un peu plus complexes à illustrer.}
\begin{multicols}{2}

\subsection*{Équation de la chaleur}

$$ \frac {\partial \phi }{\partial t} +\nabla \cdot (\phi \mathbf {V} )=S $$

\subsection*{Équation des ondes}

$$ \nabla ^{2}{\vec {E}} = {\frac {1}{c^{2}}}{\frac {\partial ^{2}{\vec {E}}}{\partial t^{2}}} $$

\subsection*{Loi des gaz parfaits (*)}

$$ PV = nRT $$

\subsection*{Équation de Schrödinger (**)}

$$ {\frac {{\hat {\vec {\mathbf {p} }}}^{2}}{2m}}|\Psi (t)\rangle + V{\Bigl (}{\hat {\vec {\mathbf {r} }}},t{\Bigr )}|\Psi (t)\rangle =i\hbar {\partial \over \partial t}|\Psi (t) \rangle $$

\subsection*{Équation d'Einstein (*)}

$$ E = mc^2 $$

\subsection*{Chute libre}

$$  \frac {\mathrm {d} ^{2}x(t)}{\mathrm {d} t^{2}} = -g $$

\subsection*{Chute libre avec frottement}

$$ mg + {\frac {1}{2}}\,C_{x}\,\rho \,S\,v^{2} = ma $$

\subsection*{Circuit électrique RLC (*)}

$$ L\,C\, {\frac {\mathrm {d} ^{2}u_{C}}{\mathrm {d} t^{2}}}+R_{t}\,C\,{\frac {\mathrm {d} u_{C}}{\mathrm {d} t}}+u_{C} = E $$

\subsection*{Désintégration radioactive}

$$ N(t)=N_{0}\,e^{-{\lambda }t}$$

\subsection*{Équation de Saint Venant}

$$ {\frac {\partial ^{2}}{\partial t^{2}}}(s{\sqrt {g}})={\frac {\partial }{\partial x}}\left(c^{2}\,{\frac {\partial }{\partial x}}(s{\sqrt {g}}\,)\right)$$

\subsection*{Théorème Central Limite (**)}

$$ \lim _{n\to \infty }\mathbb {P} (Z_{n}\leq z)=\Phi (z)$$

\subsection*{Navier Stokes}

$$ \rho \left({\dfrac {\partial e}{\partial t}}+\mathbf {V} \cdot \mathbf {\nabla } e\right)={\mathsf {P}}:\mathbf {\nabla } \mathbf {V} +\mathbf {\nabla } \cdot \mathbf {q} +\mathbf {\nabla } \cdot \mathbf {q} _{R}$$

\subsection*{Lois de Kepler}
$$A(t) = {\frac{ \| {\vec {L} \|}}{2 \cdot m} \cdot t}$$

\subsection*{Loi de Darcy}

$$ Q = K A \dfrac{\Delta H}{L}$$
\end{multicols}


\end{document}

\documentclass[11pt]{article}
\usepackage{geometry} % Pour passer au format A4
\geometry{hmargin=1cm, vmargin=2cm} % 

% Page et encodage
\usepackage[T1]{fontenc} % Use 8-bit encoding that has 256 glyphs
\usepackage[english,french]{babel} % Français et anglais
\usepackage[utf8]{inputenc} 

\usepackage{lmodern}
\setlength\parindent{0pt}


\usepackage{fourier}
%\usepackage[scaled=0.875]{helvet} 
\renewcommand{\ttdefault}{lmtt} 
\usepackage{amsmath,amssymb,makeidx}
\usepackage[normalem]{ulem}
\usepackage{fancybox,graphicx}
\usepackage{tabularx,booktabs}
\usepackage{pifont}
\usepackage{ulem}
\usepackage{dcolumn}
\usepackage{textcomp}
\usepackage{diagbox}
\usepackage{tabularx}
\usepackage{multirow,multicol}
\usepackage{lscape}
\newcommand{\euro}{\eurologo{}}

%Tapuscrit : François Kriegk
\usepackage{pst-eucl}
\usepackage{diagbox}% à mettre après pst-eucl
\usepackage{graphicx,pstricks,pst-plot,pst-grad,pst-node,pst-text,pstricks-add}
\usepackage{pgf,tikz,pgfplots}
\usetikzlibrary{patterns,calc,decorations.pathmorphing}
\setlength\paperheight{297mm}
\setlength\paperwidth{210mm}
\setlength{\textheight}{25cm}
\newcommand{\R}{\mathbb{R}}
\newcommand{\N}{\mathbb{N}}
\newcommand{\D}{\mathbb{D}}
\newcommand{\Z}{\mathbb{Z}}
\newcommand{\Q}{\mathbb{Q}}
\newcommand{\C}{\mathbb{C}}

\renewcommand{\theenumi}{\textbf{\arabic{enumi}}}
\renewcommand{\labelenumi}{\textbf{\theenumi.}}
\renewcommand{\theenumii}{\textbf{\alph{enumii}}}
\renewcommand{\labelenumii}{\textbf{\theenumii.}}

\newcommand{\vect}[1]{\mathchoice%
  {\overrightarrow{\displaystyle\mathstrut#1\,\,}}%
  {\overrightarrow{\textstyle\mathstrut#1\,\,}}%
  {\overrightarrow{\scriptstyle\mathstrut#1\,\,}}%
  {\overrightarrow{\scriptscriptstyle\mathstrut#1\,\,}}}
\def\Oij{$\left(\text{O},~\vect{\imath},~\vect{\jmath}\right)$}
\def\Oijk{$\left(\text{O},~\vect{\imath},~\vect{\jmath},~\vect{k}\right)$}
\def\Ouv{$\left(\text{O},~\vect{u},~\vect{v}\right)$}
\setlength{\voffset}{-1,5cm}

\usepackage{fancyhdr} 
\pagestyle{fancyplain} 

\fancyhead{} % No page header
\fancyfoot{}

\renewcommand{\headrulewidth}{0pt} % Remove header underlines
\renewcommand{\footrulewidth}{0pt} % Remove footer underlines

\newcommand{\horrule}[1]{\rule{\linewidth}{#1}} % Create horizontal rule command with 1 argument of height

\newcommand{\tempsexo}[1]{\textit{\textbf{(#1)}}}
%----------------------------------------------------------------------------------------
%   Début du document
%----------------------------------------------------------------------------------------

\begin{document}

%----------------------------------------------------------------------------------------
% RE-DEFINITION
%----------------------------------------------------------------------------------------
% MATHS
%-----------

\newtheorem{Definition}{Définition}
\newtheorem{Theorem}{Théorème}
\newtheorem{Proposition}{Propriété}

% MATHS
%-----------
\renewcommand{\labelitemi}{$\bullet$}
\renewcommand{\labelitemii}{$\circ$}
%----------------------------------------------------------------------------------------
%   Titre
%----------------------------------------------------------------------------------------

\setlength{\columnseprule}{1pt}

\horrule{2px}
\section*{Troisième}
\horrule{2px}

\section*{S1 : Semaine du 16/03 au 22/03 - Exercice complémentaire}

\begin{itemize}
  \item Brevet 2019 - Centres Amérique du nord
  \item 15 points
  \item 10 / 30 minutes pour l'exercice
  \item 5 / 10 minutes pour la lecture et la compréhension de la correction
\end{itemize}

 \horrule{2px}


Pour ranger les boulets de canon, les soldats du XVI\up{e} siècle utilisaient souvent un type d'empilement pyramidal à base carrée, comme le montrent les dessins suivants :

\hspace{-12mm}
\begin{tabular}{>{\centering \arraybackslash} p{2.5cm}
		>{\centering \arraybackslash} p{3.5cm}
		>{\centering \arraybackslash} p{4.5cm}
		>{\centering \arraybackslash} p{5.5cm}}
\begin{tikzpicture}[x={(6:8mm)},y=(145:4mm),z={(90:8.50mm)},baseline={(current bounding box.center)}]
\foreach \x in {2.5,3.5}{
	\shade[ball color=gray!30] (\x,3.5,1.697) circle (4mm);}
\shade[ball color=gray!30] (3,3,2.263) circle (4mm);
\foreach \x in {2.5,3.5}{
	\shade[ball color=gray!30] (\x,2.5,1.697) circle (4mm);}
\end{tikzpicture}
&
 \begin{tikzpicture}[x={(6:8mm)},y=(145:4mm),z={(90:8.50mm)},baseline={(current bounding box.center)}]
\foreach \x in {2,3,4}{
	\shade[ball color=gray!30] (\x,4,1.131) circle (4mm);}
\foreach \x in {2.5,3.5}{
	\shade[ball color=gray!30] (\x,3.5,1.697) circle (4mm);}
\foreach \x in {2,3,4}{
	\shade[ball color=gray!30] (\x,3,1.131) circle (4mm);}
\shade[ball color=gray!30] (3,3,2.263) circle (4mm);
\foreach \x in {2.5,3.5}{
	\shade[ball color=gray!30] (\x,2.5,1.697) circle (4mm);}
\foreach \x in {2,3,4}{
	\shade[ball color=gray!30] (\x,2,1.131) circle (4mm);}
\end{tikzpicture}
&
\begin{tikzpicture}[x={(6:8mm)},y=(145:4mm),z={(90:8.50mm)},baseline={(current bounding box.center)}]
\foreach \x in {1.5,2.5,...,4.5}{
	\shade[ball color=gray!30] (\x,4.5,0.5657) circle (4mm);}
\foreach \x in {2,3,4}{
	\shade[ball color=gray!30] (\x,4,1.131) circle (4mm);}
\foreach \x in {1.5,2.5,...,4.5}{
	\shade[ball color=gray!30] (\x,3.5,0.5657) circle (4mm);}
\foreach \x in {2.5,3.5}{
	\shade[ball color=gray!30] (\x,3.5,1.697) circle (4mm);}
\foreach \x in {2,3,4}{
	\shade[ball color=gray!30] (\x,3,1.131) circle (4mm);}
\shade[ball color=gray!30] (3,3,2.263) circle (4mm);
\foreach \x in {1.5,2.5,...,4.5}{
	\shade[ball color=gray!30] (\x,2.5,0.5657) circle (4mm);}
\foreach \x in {2.5,3.5}{
	\shade[ball color=gray!30] (\x,2.5,1.697) circle (4mm);}
\foreach \x in {2,3,4}{
	\shade[ball color=gray!30] (\x,2,1.131) circle (4mm);}
\foreach \x in {1.5,2.5,...,4.5}{
	\shade[ball color=gray!30] (\x,1.5,0.5657) circle (4mm);}
\end{tikzpicture}
&
 \begin{tikzpicture}[x={(6:8mm)},y=(145:4mm),z={(90:8.50mm)},baseline={(current bounding box.center)}]
 \foreach \x in {1,...,5}{
 	\shade[ball color=gray!30] (\x,5,0) circle (4mm);}
 \foreach \x in {1.5,2.5,...,4.5}{
	\shade[ball color=gray!30] (\x,4.5,0.5657) circle (4mm);}
 \foreach \x in {1,...,5}{
 	\shade[ball color=gray!30] (\x,4,0) circle (4mm);}
 \foreach \x in {2,3,4}{
	\shade[ball color=gray!30] (\x,4,1.131) circle (4mm);}
 \foreach \x in {1.5,2.5,...,4.5}{
	\shade[ball color=gray!30] (\x,3.5,0.5657) circle (4mm);}
 \foreach \x in {2.5,3.5}{
	\shade[ball color=gray!30] (\x,3.5,1.697) circle (4mm);}
 \foreach \x in {1,...,5}{
	\shade[ball color=gray!30] (\x,3,0) circle (4mm);}
 \foreach \x in {2,3,4}{
	\shade[ball color=gray!30] (\x,3,1.131) circle (4mm);}
	\shade[ball color=gray!30] (3,3,2.263) circle (4mm);
 \foreach \x in {1.5,2.5,...,4.5}{
	\shade[ball color=gray!30] (\x,2.5,0.5657) circle (4mm);}
 \foreach \x in {2.5,3.5}{
	\shade[ball color=gray!30] (\x,2.5,1.697) circle (4mm);}
 \foreach \x in {1,...,5}{
	\shade[ball color=gray!30] (\x,2,0) circle (4mm);}
 \foreach \x in {2,3,4}{
	\shade[ball color=gray!30] (\x,2,1.131) circle (4mm);}
 \foreach \x in {1.5,2.5,...,4.5}{
	\shade[ball color=gray!30] (\x,1.5,0.5657) circle (4mm);}
  \foreach \x in {1,...,5}{
 	\shade[ball color=gray!30] (\x,1,0) circle (4mm);}
 \end{tikzpicture}\\ [1.8cm]
 Empilement \linebreak à 2 niveaux&Empilement à 3 niveaux&
 Empilement à 4 niveaux&Empilement à 5 niveaux
\end{tabular}

\begin{enumerate}
	\item Combien de boulets contient l'empilement à 2 niveaux ?	
	\item Expliquer pourquoi l'empilement à 3 niveaux contient 14 boulets.	
	\item On range 55 boulets de canon selon cette méthode. Combien de niveaux comporte alors l'empilement obtenu ?	
	\item Ces boulets sont en fonte; la masse volumique de cette fonte est de $7300 kg/m^3$.
	
On modélise un boulet de canon par une boule de rayon 6cm.
	
Montrer que l'empilement à 3 niveaux de ces boulets pèse 92 kg, au kg près.
	
\emph{Rappels:}
\begin{itemize}
		\item $\emph{volume d'une boule} = \dfrac{4}{3}\times \pi \times \text{\emph{rayon}} \times \text{\emph{rayon}} \times \text{\emph{rayon}}$.	
		\item une masse volumique de $7300 kg/m^3$ signifie que $1 m^3$ pèse $7300 kg$.
\end{itemize}
\end{enumerate}

\newpage

\section*{Correction}

\begin{enumerate}
	\item L'empilement à 2 niveaux contient $4 + 1 = 5$~boulets.	
	\item L'empilement à 3 niveaux contient $9 + 4 + 1 = 14$~boulets.
	\item Avec 4 niveaux on peut ranger $16 + 9 + 4 + 1 = 30$~boulets. Il faut donc un niveau de plus de $5 \times 5 = 25$ boulets.

Sur 5 niveaux il y aura $25 + 16 + 9 + 4 + 1 = 55$~boulets exactement.
	\item -- Volume d'un boulet : $\dfrac{4}{3} \times \pi \times 6 \times 6 \times 6 = 288\pi$~cm$^3$.

-- L'empilement à 3 niveaux contient 14 boulets qui ont un volume de $14 \times 288 \pi = 4032 \pi cm^3$.

$1 m^3$ de fonte a une masse de $7300 kg/m^3$, donc $1 dm^3$ de fonte a une masse de $7,3 k$ et $1 cm^3$ de fonte a une masse de $0,0073 kg$, donc les 14 boulets ont une masse de :

$ 4032\pi \times 0,0073 = 29,4336 \pi \approx 92,46 kg$, soit 92kg au kilogramme près.
	
\end{enumerate}
 
\end{document}
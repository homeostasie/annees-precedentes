\documentclass[11pt]{article}
\usepackage{geometry} % Pour passer au format A4
\geometry{hmargin=1cm, vmargin=1cm} % 

% Page et encodage
\usepackage[T1]{fontenc} % Use 8-bit encoding that has 256 glyphs
\usepackage[english,french]{babel} % Français et anglais
\usepackage[utf8]{inputenc} 

\usepackage{lmodern}
\setlength\parindent{0pt}

% Graphiques
\usepackage{graphicx,float,grffile}

% Maths et divers
\usepackage{amsmath,amsfonts,amssymb,amsthm,verbatim}
\usepackage{multicol,enumitem,url,eurosym,gensymb}

% Sections
\usepackage{sectsty} % Allows customizing section commands
\allsectionsfont{\centering \normalfont\scshape}

% Tête et pied de page

\usepackage{fancyhdr} 
\pagestyle{fancyplain} 

\fancyhead{} % No page header
\fancyfoot{}

\renewcommand{\headrulewidth}{0pt} % Remove header underlines
\renewcommand{\footrulewidth}{0pt} % Remove footer underlines

\newcommand{\horrule}[1]{\rule{\linewidth}{#1}} % Create horizontal rule command with 1 argument of height

\newcommand{\tempsexo}[1]{\textit{\textbf{(#1)}}}
%----------------------------------------------------------------------------------------
%   Début du document
%----------------------------------------------------------------------------------------

\begin{document}

%----------------------------------------------------------------------------------------
% RE-DEFINITION
%----------------------------------------------------------------------------------------
% MATHS
%-----------

\newtheorem{Definition}{Définition}
\newtheorem{Theorem}{Théorème}
\newtheorem{Proposition}{Propriété}
\newtheorem{Exo}{Éxercice}

% MATHS
%-----------
\renewcommand{\labelitemi}{$\bullet$}
\renewcommand{\labelitemii}{$\circ$}
%----------------------------------------------------------------------------------------
%   Titre
%----------------------------------------------------------------------------------------

\setlength{\columnseprule}{1pt}

\section{S1 : Correction - Semaine du 16/03 au 22/03}

\subsection{Travail sur le chapitre - Calcul littéral}

\Exo{p102 ex15}\\

\textit{L'objectif ici est de mieux comprendre comment se combinent les additions et les multiplications.}\\
On pouvait / devait aussi vérifier les calculs avec la calculatrice.

\begin{enumerate}
    \item[a.] $ 101 \times 17= (100 + 1) \times 17 = 100 \times 17 + 1 \times 17 = 1700 + 17 = 1717$
    \item[b.] $ 96 \times 5 + 4 \times 5 = (96 + 4) \times 5 = 100 \times 5 = 500$
    \item[c.] $ 99 \times 13= (100 - 1) \times 13 = 100 \times 13 - 1 \times 13 = 1300 - 13 = 1287$
    \item[d.] $ 5.4 \times 7 + 5.4 \times 3 = 5.4 \times (7 + 3) = 5.4 \times 10 = 54$
    \item[e.] $ 1002 \times 14= (1000 + 2) \times 14 = 1000 \times 14 + 2 \times 14 = 14000 + 28 = 14028$
    \item[f.] $ 8 \times 19 + 2 \times 19 = (8 + 2) \times 19 = 10 \times 19 = 190$
    \item[g.] $ 98 \times 22 = (100 - 2) \times 22 = 100 \times 22 - 2 \times 22 = 2200 - 44 = 2156$
    \item[h.] $ 13 \times 103 - 3 \times 13 = (103 - 3) \times 13 = 100 \times 13 = 1300$
\end{enumerate}

\Exo{p102 ex17}\\

\begin{enumerate}
    \item[a.] $5 \times (2x + 3) = 5 \times 2x + 5 \times 3 = 10x + 15$
    \item[b.] $5 + (2x + 3)$ : \textbf{NON}\\
    On peut réduire, mais on ne peut pas développer car il n'y a pas de $\times$ devant la parenthèse.\\
    $5 + (2x + 3) =  5 + 2x + 3 = 2x + 8$
    \item[c.] $(5 + 2x) \times 3 = 5  \times 3 + 2x \times 3 = 15 + 6x$
    \item[d.] $4 \times (5x - 2) = 4 \times 5x - 4 \times 2 = 20x - 8$
    \item[e.] $4 \times (5x \times 2)$ : \textbf{NON}\\
    On peut réduire, mais on ne peut pas développer car il n'y a pas de $+$ dans la parenthèse.\\
    $4 \times (5x \times 2) = 4 \times 5x \times 2 = 40x$
    \item[f.] $4 \times (3 \times x + 2) = 4 \times 3x + 4 \times 2 = 12x + 8$
\end{enumerate}


\Exo{p102 ex19}\\

\begin{enumerate}
    \item[a.] $3 \times x + 3 \times 7 = 3 \times (x+7)$
    \item[b.] $y \times 9 + y \times y = y \times (9+y)$
    \item[c.] $2.5x^2 - 0.3x^2 = 2.2x^2$
    \item[d.] $9 - 3 \times 4 \times N = 3 \times 3 - 3 \times 4 \times N = 3 \times (3 - 4N)$
    \item[e.] $3 \times x \times 4 \times x = 12 \times x \times x = 12 x^2$
    \item[f.] $x - x^2 = x \times 1 - x \times x = x \times (1 - x)$
\end{enumerate}   
\end{document}
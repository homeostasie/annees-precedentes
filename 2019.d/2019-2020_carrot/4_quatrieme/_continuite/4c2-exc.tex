\documentclass[11pt]{article}
\usepackage{geometry} % Pour passer au format A4
\geometry{hmargin=1cm, vmargin=2cm} % 

% Page et encodage
\usepackage[T1]{fontenc} % Use 8-bit encoding that has 256 glyphs
\usepackage[english,french]{babel} % Français et anglais
\usepackage[utf8]{inputenc} 

\usepackage{lmodern}
\setlength\parindent{0pt}


\usepackage{fourier}
%\usepackage[scaled=0.875]{helvet} 
\renewcommand{\ttdefault}{lmtt} 
\usepackage{amsmath,amssymb,makeidx}
\usepackage[normalem]{ulem}
\usepackage{fancybox,graphicx}
\usepackage{tabularx,booktabs}
\usepackage{pifont}
\usepackage{ulem}
\usepackage{dcolumn}
\usepackage{textcomp}
\usepackage{diagbox}
\usepackage{tabularx}
\usepackage{multirow,multicol}
\usepackage{lscape}
\newcommand{\euro}{\eurologo{}}

%Tapuscrit : François Kriegk
\usepackage{pst-eucl}
\usepackage{diagbox}% à mettre après pst-eucl
\usepackage{graphicx,pstricks,pst-plot,pst-grad,pst-node,pst-text,pstricks-add}
\usepackage{pgf,tikz,pgfplots}
\usetikzlibrary{patterns,calc,decorations.pathmorphing}
\setlength\paperheight{297mm}
\setlength\paperwidth{210mm}
\setlength{\textheight}{25cm}
\newcommand{\R}{\mathbb{R}}
\newcommand{\N}{\mathbb{N}}
\newcommand{\D}{\mathbb{D}}
\newcommand{\Z}{\mathbb{Z}}
\newcommand{\Q}{\mathbb{Q}}
\newcommand{\C}{\mathbb{C}}

\renewcommand{\theenumi}{\textbf{\arabic{enumi}}}
\renewcommand{\labelenumi}{\textbf{\theenumi.}}
\renewcommand{\theenumii}{\textbf{\alph{enumii}}}
\renewcommand{\labelenumii}{\textbf{\theenumii.}}

\newcommand{\vect}[1]{\mathchoice%
  {\overrightarrow{\displaystyle\mathstrut#1\,\,}}%
  {\overrightarrow{\textstyle\mathstrut#1\,\,}}%
  {\overrightarrow{\scriptstyle\mathstrut#1\,\,}}%
  {\overrightarrow{\scriptscriptstyle\mathstrut#1\,\,}}}
\def\Oij{$\left(\text{O},~\vect{\imath},~\vect{\jmath}\right)$}
\def\Oijk{$\left(\text{O},~\vect{\imath},~\vect{\jmath},~\vect{k}\right)$}
\def\Ouv{$\left(\text{O},~\vect{u},~\vect{v}\right)$}
\setlength{\voffset}{-1,5cm}

\usepackage{fancyhdr} 
\pagestyle{fancyplain} 

\fancyhead{} % No page header
\fancyfoot{}

\renewcommand{\headrulewidth}{0pt} % Remove header underlines
\renewcommand{\footrulewidth}{0pt} % Remove footer underlines

\newcommand{\horrule}[1]{\rule{\linewidth}{#1}} % Create horizontal rule command with 1 argument of height

\newcommand{\tempsexo}[1]{\textit{\textbf{(#1)}}}
%----------------------------------------------------------------------------------------
%   Début du document
%----------------------------------------------------------------------------------------

\begin{document}

%----------------------------------------------------------------------------------------
% RE-DEFINITION
%----------------------------------------------------------------------------------------
% MATHS
%-----------

\newtheorem{Definition}{Définition}
\newtheorem{Theorem}{Théorème}
\newtheorem{Proposition}{Propriété}

% MATHS
%-----------
\renewcommand{\labelitemi}{$\bullet$}
\renewcommand{\labelitemii}{$\circ$}
%----------------------------------------------------------------------------------------
%   Titre
%----------------------------------------------------------------------------------------

\setlength{\columnseprule}{1pt}

\section*{S2 : Semaine du 23/03 au 29/03 - Exercice complémentaire}

\begin{itemize}
  \item Brevet 2019 - Grèce - (une question en moins)
  \item 10 points
  \item 5 / 15 minutes pour l'exercice
  \item 5 minutes pour la lecture et la compréhension de la correction
\end{itemize}

 \horrule{2px}


Marc et Jim, deux amateurs de course à pied, s'entraînent sur une piste d'athlétisme dont la longueur du tour mesure $400$~m. 

Marc fait un temps moyen de $2$ minutes par tour. 

Marc commence son entrainement par un échauffement d'une longueur d'un kilomètre. 

Le schéma ci-dessous représente la piste d'athlétisme de Marc constituée de deux segments [AB] et [CD] et de deux demi-cercles de diamètre [AD] et [BC]. 

(\emph{Le schéma n'est pas à l'échelle et les longueurs indiquées sont arrondies à l'unité.}) 

\medskip

\begin{center}
\psset{unit=0.75cm}
\begin{pspicture}(18,6)
\psline(3,0.5)(9.7,0.5)\psline(3,5.7)(9.7,5.7)
\psline[linestyle=dotted](3,0.5)(3,5.7)
\psline[linestyle=dotted](9.7,0.5)(9.7,5.7)
\psarc(3,3.1){2.6}{90}{270}
\psarc(9.7,3.1){2.6}{-90}{90}
\uput[u](3,5.7){A} \uput[u](9.7,5.7){B} \uput[d](9.7,0.5){C} \uput[d](3,0.5){D} 
\psframe(3,5.7)(3.2,5.5)\psframe(9.7,5.7)(9.5,5.5)\psframe(9.7,0.5)(9.5,0.7)
\psframe(3,0.5)(3.2,0.7)
\psline[linestyle=dashed]{<->}(3,5.9)(9.7,5.9)\uput[u](6.35,5.9){90 m}
\psline[linestyle=dashed]{<->}(3.4,5.7)(3.4,0.5)\uput[r](3.4,3.1){70 m}
\rput(15.5,3){ABCD est un rectangle}
\rput(15.5,2.2){AB = 90 m et AD = 70 m}
\end{pspicture}
\end{center}

\begin{enumerate}
\item Combien de temps durera l'échauffement de Marc? 
\item Quelle est la vitesse moyenne de course de Marc en km/h ? 
\end{enumerate}


\newpage

\section*{Correction}

\begin{enumerate}
\item En supposant que Marc court  à la vitesse de 2 minutes pour faire 400 m, il mettra 1 minute pour faire 200 m, donc 5~minutes pour faire $5 \times 200 = 1000$~m.
\item 1~km en 5~min représente une vitesse de $12 \times 1 = 12$~(km/h) en $12 \times 5 = 60$~min = 1 h. 
\end{enumerate}

 
\end{document}
\documentclass[11pt]{article}
\usepackage{geometry} % Pour passer au format A4
\geometry{hmargin=0.8cm, vmargin=0.8cm} % 

% Page et encodage
\usepackage[T1]{fontenc} % Use 8-bit encoding that has 256 glyphs
\usepackage[english,french]{babel} % Français et anglais
\usepackage[utf8]{inputenc} 

\usepackage{lmodern}
\setlength\parindent{0pt}

% Graphiques
\usepackage{graphicx,float,grffile}

% Maths et divers
\usepackage{amsmath,amsfonts,amssymb,amsthm,verbatim}
\usepackage{multicol,enumitem,url,eurosym,gensymb}

% Sections
\usepackage{sectsty} % Allows customizing section commands
\allsectionsfont{\centering \normalfont\scshape}

% Tête et pied de page

\usepackage{fancyhdr} 
\pagestyle{fancyplain} 

\fancyhead{} % No page header
\fancyfoot{}

\renewcommand{\headrulewidth}{0pt} % Remove header underlines
\renewcommand{\footrulewidth}{0pt} % Remove footer underlines

\newcommand{\horrule}[1]{\rule{\linewidth}{#1}} % Create horizontal rule command with 1 argument of height

%----------------------------------------------------------------------------------------
%   Début du document
%----------------------------------------------------------------------------------------

\begin{document}

\setlength{\columnseprule}{1pt}

\section*{Méthode de travail}
\horrule{2px}

Afin de réussir au collège, il convient d'aquérir des \textbf{méthodes de travail} efficaces. Il faut distinguer \textbf{les évaluations} et \textbf{le travail personnel}. Il faut comprendre que le travail personnel \underline{n'est pas noté} directement mais contribue à avoir \underline{de bonnes notes}. Un apprentissage est efficace sur la durée alors qu'une révision uniquement une veille d'évaluation n'est pas très efficace. Il n'est pas rare de voir des élèves être déçus d'avoir râté une évaluation alors qu'ils ont révisé \og plusieurs heures la vieille \fg. \textbf{Il faut travailler chaque soir pour apprendre plutôt que juste réviser pour les éval.} On peut chaque soir se poser une série de questions pour le lendemain. 

\subsection*{Évaluations}
\horrule{1px}

Les premières questions à se poser restent sur le travail évalué. L'élève ne doit pas hésiter à demander une confirmation à son groupe \og d'amis \fg.  Il doit réviser de manière profonde une évaluation en relisant le cours correspondant et essayant de refaire les exercices fait en classe. \textbf{30 min} de révision pour une évaluation reste un minimum. Il doit s'assurer de bien avoir son matériel pour l'évaluation.  \textit{Apprendre le vocabulaire n'est jamais perdu car si des définitions ne sont pas toujours demandées, les mots de vocabulaires peuvent être utilisés dans les consignes. }

\begin{multicols}{2}

\paragraph{Questions à poser}

\begin{itemize}
  \item As-tu une évaluation ? 
  \item As-tu à rendre un devoir maison ?
  \item As-tu bien tout le matériel pour ton évaluation ? (crayon de couleur, livre, calculatrice,...)
  \item Montre-moi ta leçon; tes exercices, sur le livre ou sur le cahier correspondant. 
  \item As-tu besoin d'une feuille ? si oui, en as-tu ? Est-elle déjà préparée ? 
  \item As-tu bien mis ton nom sur ton devoir maison ?
  \item Montre-moi ton devoir maison. Il faut être soigné sur la présentation.
  \item Connais-tu la définition de : ... ? (Prenre un mot dans le cahier ou le livre)
 \end{itemize} 
\end{multicols}

\subsection*{Travail personnel}
\horrule{1px}

Souvent et c'est là une erreur, l'élève s'arrête aux questions d'évaluation. Le travail personnel n'a pas encore été abordé. Il reste des questions importantes à se poser. \textbf{Il faut que l'élève d'arrive dans chaque cours avec le souvenir de ce qui a été fait la fois précédente}. On appelle cela préparer un cours. L'élève doit relire la partie du cours qui a été écrite ou collée. Si des feuilles n'ont pas été collées, que des titres n'ont pas été soulignés, c'est le moment de le faire. Il doit également reprendre ses exercices pour voir ce qui avait été demandé. Si, l'élève n'arrive à répondre rapidement et avec suffisamment de détail à ces questions alors, il doit relire encore une fois le travail fait lors de la séance précédente. Il est important de faire ça pour toutes les matières. L'élève ne doit pas négliger les matières qu'il a qu'une seule fois par semaine car le dernier cours étant plus loin, il a plus de chance d'avoir oublié ce qui a été fait la séance précédente. Au minimum; l'élève doit passer \textbf{au moins 5 min} et ce \textbf{PAR} matière du lendemain.

\begin{multicols}{2}
\paragraph{Questions à poser}

\begin{itemize}
  \item Quels cours as-tu demain ?
  \item Quelles étaient les contenus des cours précédents dans ses matières?
  \item Montre-moi la dernière chose écrite dans ce cours ? 
  \item Montre-moi ton cahier, tes derniers exercices; les pages du livre, ou ta dernière activité. 
 \end{itemize} 
\end{multicols}

\subsection*{Administratif}
\horrule{1px}

Il reste une question importante à se poser qui est celle de la responsabilité.
\begin{multicols}{2}
\paragraph{Questions à poser}

\begin{itemize}
  \item M'as-tu donné tous les papiers qui ont été distribués ?
  \item Dois-je signer quelque chose ?
  \item Montre-moi ton carnet ? (au moins une fois par semaine)
\end{itemize} 
\end{multicols}
\end{document}

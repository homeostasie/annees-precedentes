\documentclass[11pt]{article}
\usepackage{geometry} % Pour passer au format A4
\geometry{hmargin=1cm, vmargin=1cm} % 

% Page et encodage
\usepackage[T1]{fontenc} % Use 8-bit encoding that has 256 glyphs
\usepackage[english,french]{babel} % Français et anglais
\usepackage[utf8]{inputenc} 

\usepackage{lmodern}
\setlength\parindent{0pt}

% Graphiques
\usepackage{graphicx,float,grffile}

% Maths et divers
\usepackage{amsmath,amsfonts,amssymb,amsthm,verbatim}
\usepackage{multicol,enumitem,url,eurosym,gensymb}

% Sections
\usepackage{sectsty} % Allows customizing section commands
\allsectionsfont{\centering \normalfont\scshape}

% Tête et pied de page

\usepackage{fancyhdr} 
\pagestyle{fancyplain} 

\fancyhead{} % No page header
\fancyfoot{}

\renewcommand{\headrulewidth}{0pt} % Remove header underlines
\renewcommand{\footrulewidth}{0pt} % Remove footer underlines

\newcommand{\horrule}[1]{\rule{\linewidth}{#1}} % Create horizontal rule command with 1 argument of height

%----------------------------------------------------------------------------------------
%   Début du document
%----------------------------------------------------------------------------------------

\begin{document}

%----------------------------------------------------------------------------------------
% RE-DEFINITION
%----------------------------------------------------------------------------------------
% MATHS
%-----------

\newtheorem{Definition}{Définition}
\newtheorem{Theorem}{Théorème}
\newtheorem{Proposition}{Propriété}

% MATHS
%-----------
\renewcommand{\labelitemi}{$\bullet$}
\renewcommand{\labelitemii}{$\circ$}
%----------------------------------------------------------------------------------------
%   Titre
%----------------------------------------------------------------------------------------

\setlength{\columnseprule}{1pt}

\horrule{2px}
\section*{Chapitre 5 - Expression Littérale}
\horrule{2px}

\section*{1 - Modéliser - Le domaine des mathématiques}

Trouver, créer et utiliser des outils mathématiques pour répondre à des questions. On peut \textbf{modéliser} le monde avec des mathématiques. En mathématiques, on ne travaille pas toujours avec des choses connues. On ne connaît pas directement la réponse. C'est pourquoi, on travaille avec des des inconnues. 

\begin{center}
{\fontfamily{jkplos}\selectfont 
\og En mathématiques, on ne comprend pas les choses, on s'y habitue \fg - John Von Neumann
}
\end{center}

\begin{itemize}
  \item Ex : Équations de Navier-Stokes.
  \item Ex : DM - Équations, Scientifiques, Informatiques.
\end{itemize}

\section*{2 - Travailler avec une inconnue}

\begin{itemize}
  \item Ex : Petit Prince.
\end{itemize}

\section*{3 - Mettre en équation}

\begin{itemize}
  \item Ex : Mettre en équations.
  \item Ex et COR : pb bâton puis pb immeuble.   
\end{itemize}


\begin{enumerate}
  \item[1.] Choix de l'inconnue : La lecture attentive de l'énoncé du problème et de la question posée permet de choisir l'inconnue. 
  \item[2.] Mise en équation du problème : On exprime les données du problème en fonction de l'inconnue choisie. On obtient une \textbf{équation}.
  \item[3.] Résolution de l'équation.
  \item[4.] Vérification des résultats. On reporte les résultats trouvés dans l'énoncé et on vérifie notre solution.
\end{enumerate}

On ne s'intéresse plus au résultat d'un calcul, on ne peut pas toujours avoir la réponse directement. On cherche à écrire une équation avec des \textbf{expressions}. La résolution vient quand cela est possible dans un second temps. \textbf{Il n'est pas possible de résoudre directement des équations avec une calculatrice collège.}

\section*{4 - Résolution d'équations}

On apprend au collège uniquement à résoudre les équations du premier degré. On utilise la méthode du rajout et de l'équilibre.

\begin{itemize}
  \item Ex 1 - puis COR : pb rentrée scolaire.
  \item Ex 2 à 7.
  \item Ex : DM - Al Kwarismi. 
  \item Ex : DRILL - équations du premier degré.
\end{itemize}

\section*{5 - Vérification d'une équation}

Pour vérifier une équation, on évalue les expressions de gauche et de droite. \textbf{Il est possible d'évaluer une expression avec une calculatrice collège.}
\end{document}

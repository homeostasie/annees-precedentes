\documentclass[11pt]{article}
\usepackage{geometry,marginnote} % Pour passer au format A4
\geometry{hmargin=1cm, vmargin=1cm} % 

% Page et encodage
\usepackage[T1]{fontenc} % Use 8-bit encoding that has 256 glyphs
\usepackage[english,french]{babel} % Français et anglais
\usepackage[utf8]{inputenc} 

\usepackage{lmodern}
\setlength\parindent{0pt}

% Graphiques
\usepackage{graphicx,float,grffile}
\usepackage{pst-eucl, pst-plot,units} 

% Maths et divers
\usepackage{amsmath,amsfonts,amssymb,amsthm,verbatim}
\usepackage{multicol,enumitem,url,eurosym,gensymb}
\DeclareUnicodeCharacter{20AC}{\euro}

% Sections
\usepackage{sectsty} % Allows customizing section commands
\allsectionsfont{\centering \normalfont\scshape}

% Tête et pied de page

\usepackage{fancyhdr} 
\pagestyle{fancyplain} 

\fancyhead{} % No page header
\fancyfoot{}

\renewcommand{\headrulewidth}{0pt} % Remove header underlines
\renewcommand{\footrulewidth}{0pt} % Remove footer underlines

\newcommand{\horrule}[1]{\rule{\linewidth}{#1}} % Create horizontal rule command with 1 argument of height

%----------------------------------------------------------------------------------------
%   Début du document
%----------------------------------------------------------------------------------------

\begin{document}

\setlength{\columnseprule}{1pt}

\horrule{2px}
\section*{Chapitre 4 - Factoriser - \texttt{(T)}}
\horrule{2px}

Le facteur distribue du courrier. Factoriser en mathématiques revient à distribuer des nombres.

\section*{1 - La multiplication, un retour dans le passé}

Au primaire, pour apprendre à multiplier, on apprend à factoriser... \textit{mais sans nous le dire}.

\begin{multicols}{2}
  \begin{align*}
       123 & \\
  \times 6 & \\
  \text{\rule{2cm}{1px}} & \\
        18 & \\
    +  120 & \\
    +  600 & \\
  \text{\rule{2cm}{1px}} & \\
      738
  \end{align*}

  \begin{align*}
  123 \times 6 &= (100 + 20 + 3) \times 6 \\
               &= 100 \times 6 + 20 \times 6 + 3 \times 6 \\
               &= 600 + 120 + 18\\
               &= 738
  \end{align*}
\end{multicols}

\section*{2 - Le calcul astucieux}

Le calcul astucieux utilisé en sixième repose sur la factorisation. 

\begin{multicols}{2}
  \begin{align*}
  1\,003 \times 12 &= (1\,000 + 3) \times 12 \\
                  &= 1\,000 \times 12 + 3 \times 12 \\
                  &= 12\, 000 + 36\\
                  &= 12\, 036
  \end{align*}

  \begin{align*}
  98 \times 6 &= (100 - 2) \times 6 \\
              &= 100 \times 6 - 2 \times 6 \\
              &= 600 - 12\\
              &= 588
  \end{align*} 
\end{multicols}

\section*{3 - Le passage au français}

La factorisation existe en français. Il faut par contre conjuguer.

\begin{center}
  {\fontfamily{jkplos}\selectfont 
  Yassine et Nassime sont forts en mathématiques. 
  }
\end{center}

$$(12 + 13) \times 18$$

\begin{center}
  {\fontfamily{jkplos}\selectfont 
  Yassine est fort en mathématiques et Nassime est fort en mathématiques. 
  }
\end{center}

$$12 \times 18 + 13 \times 18$$

... et on peut créer des phrases bien longues... 

\begin{center}
  {\fontfamily{jkplos}\selectfont 
  Inès et Yasmine sont fortes en mathématiques et en anglais. \\
  Inès est forte en mathématiques et en anglais et Yasmine est forte en mathématiques et en anglais. \\
  Inès est forte en mathématiques et elle (Inès) est forte en anglais et Yasmine est forte en mathématiques et elle (Yasmine) est forte en anglais.  
  }
\end{center}


\section*{4 - La distribution simple}

Lorsqu'on utilise des expressions littérales, on a besoin de savoir factoriser. On va distribuer le facteur. Le nombre qui multiplie la parenthèse va être distribuer ($\times$) à chacun des nombres dans la parenthèse.
\begin{multicols}{2}
\begin{align*}
  2(x + 5 ) &= 2 \times (x + 5 ) \\
            &= 2 \times x + 2 \times 5 \\
            &= 2x + 10
\end{align*}

\begin{align*}
  5(12 -2y) &= 5 \times (12 -2y) \\
            &= 5 \times 12 - 5 \times 2y \\
            &= 60 - 10y
\end{align*}
\end{multicols}

0n trouve parfois les formes généralisées : 

$$\fbox{$k(a + b ) = ka + kb$} \text{ et } \fbox{$k(a - b ) = ka - kb$}$$

\section*{5 - La distribution double}

La distribution double s'appuie sur la même méthode mais est un peu plus technique... 
On reprend l'exemple avec Inès et Yasmine. 

\begin{align*}
  (x + 3)(5 + 2x) &= (x + 3) \times (5 + 2x) \\
                  &= x \times (5 + 2x) + 3 \times (5 + 2x) \\
                  &= x \times 5 + x \times 2x + 3 \times 5 + 3 \times 2x \\
                  &= 5x + 2x^2 + 15 + 6x \\
                  &= 2x^2 + 11x + 15     
\end{align*}

Et attention aux erreurs de signes qui arrivent vites... Conseil : mettre les nombres négatifs entre parenthèses.

\begin{align*}
  (x - 4)(-6 - 2x) &= (x - 4) \times (-6 - 2x) \\
                   &= (x + (-4)) \times ((-6) + (-2x)) \\
                  &= x \times ((-6) + (-2x)) + (-4) \times ((-6) + (-2x)) \\
                  &= x \times (-6) + x \times (-2x) + (-4)  \times (-6) + (-4) \times (-2x) \\
                  &= -6x - 2x^2 + 24 + 8x \\
                  &= -2x^2 + 2x + 24     
\end{align*}

0n trouve parfois la forme généralisée : 

$$\fbox{(a + b )(x + y) = ax + ay + bx + by}$$

\section*{6 - Les égalités remarquables}

En quatrième, on croise la route de l'opération carré. On rappelle que : $x^2 = x \times x$ et donc $(x+3)^2 = (x+3) \times (x+3)$... 

$$\fbox{$(a + b)^2 = a^2 + b^2 + 2ab$}$$
$$\fbox{$(a - b)^2 = a^2 + b^2 - 2ab$}$$
$$\fbox{$(a - b)(a+b)= a^2 - b^2$}$$

\section*{Apprendre tout ça...}

Il faut s'entraîner, s'entraîner... Ce chapitre se repris en début d'année de seconde... mais cela facilite l'année d'être familier avec la factorisation.

\end{document}

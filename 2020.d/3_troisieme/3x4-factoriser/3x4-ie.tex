\documentclass[11pt]{article}
\usepackage{geometry} % Pour passer au format A4
\geometry{hmargin=1cm, vmargin=1cm} % 

% Page et encodage
\usepackage[T1]{fontenc} % Use 8-bit encoding that has 256 glyphs
\usepackage[english,french]{babel} % Français et anglais
\usepackage[utf8]{inputenc} 

\usepackage{lmodern}
\setlength\parindent{0pt}

% Graphiques
\usepackage{graphicx,float,grffile}

% Maths et divers
\usepackage{amsmath,amsfonts,amssymb,amsthm,verbatim}
\usepackage{multicol,enumitem,url,eurosym,gensymb,multido}

% Sections
\usepackage{sectsty} % Allows customizing section commands
\allsectionsfont{\centering \normalfont\scshape}

% Tête et pied de page

\usepackage{fancyhdr} 
\pagestyle{fancyplain} 

\fancyhead{} % No page header
\fancyfoot{}

\renewcommand{\headrulewidth}{0pt} % Remove header underlines
\renewcommand{\footrulewidth}{0pt} % Remove footer underlines

\newcommand{\horrule}[1]{\rule{\linewidth}{#1}} % Create horizontal rule command with 1 argument of height

\newcommand{\Pointilles}[1][3]{%
  \multido{}{#1}{\makebox[\linewidth]{\dotfill}\\[\parskip]
}}

\begin{document}

\setlength{\columnseprule}{1pt}

\textbf{Nom, Prénom :} \hspace{8cm} \textbf{Classe :} \hspace{3cm} \textbf{Date :}\\
\vspace{-0.8cm}
\begin{center}
  \textit{Les mathématiques ne sont une moindre immensité que la mer.}  - \textbf{Victor Hugo}
\end{center}
\vspace{-0.8cm}

\subsection*{Ex1 : Simplifier}

\begin{multicols}{3}

\begin{enumerate}
  \item[a.] $4y - 2y$
  \item[b.] $2 \times 5 x$
  \item[c.] $7x + 3$
  \item[d.] $x + 2x^2$
  \item[e.] $8x + 2$
  \item[f.] $4y + 6y$
  \item[g.] $2 \times 2x \times 5$
  \item[h.] $4x \times 2x + 2x^2$
  \item[i.] $4x + 6y + 10$    
\end{enumerate}

\end{multicols}

 \Pointilles[8] 

 \subsection*{Ex2 : Distribuer}

\begin{multicols}{3}


\begin{enumerate}
    \item[a.] $(x+1)^2$ 
    \item[b.] $(a-5)^2$
    \item[c.] $(2x + 10)^2$
    \item[d.] $(12 - x)^2$
    \item[e.] $(5 - x)(5 + x)$   
    \item[f.] $(6 + 2y)(6 - 2y)$
    \item[g.] $(x + x^2)^2$
    \item[h.] $(4x + 0,1)^2$
    \item[i.] $(\dfrac{x}{2} - 1)^2$ 
\end{enumerate}

\end{multicols}

\Pointilles[18] 

\subsection*{Bonus : Ok Boomer}
Deux profs de maths, Jason et Thomas discutent entre eux... Le premier se plaint d'être trop vieux : \og Tu t'imagines, cette année, j'ai le double de mes élèves de secondes. \fg Le second lui répond : \og Ne m'en parle pas, cette année, j'ai le triple de mes sixièmes et ils ont 11ans. \fg \\
Qui a raison ? Justifier (un peu).

\Pointilles[3] \newpage

\textbf{Nom, Prénom :} \hspace{8cm} \textbf{Classe :} \hspace{3cm} \textbf{Date :}\\
\vspace{-0.8cm}
\begin{center}
  \textit{Les mathématiques ne sont une moindre immensité que la mer.}  - \textbf{Victor Hugo}
\end{center}
\vspace{-0.8cm}

\subsection*{Ex1 : Simplifier}

\begin{multicols}{3}

\begin{enumerate}
  \item[a.] $6y - 2y$
  \item[b.] $3 \times 5 x$
  \item[c.] $7x + 3x$
  \item[d.] $x + 5x^2$
  \item[e.] $8x - 2$
  \item[f.] $4y - 6y$
  \item[g.] $2 \times 3x \times 5x$
  \item[h.] $4 \times 2x + 2x$
  \item[i.] $2x + 8y + 10$    
\end{enumerate}

\end{multicols}

 \Pointilles[8] 

 \subsection*{Ex2 : Distribuer}

\begin{multicols}{3}


\begin{enumerate}
    \item[a.] $(x+1)^2$ 
    \item[b.] $(a-4)^2$
    \item[c.] $(3x + 10)^2$
    \item[d.] $(11 - x)^2$
    \item[e.] $(6 - x)(6 + x)$   
    \item[f.] $(4 + 2y)(4 - 2y)$
    \item[g.] $(x^2 + x)^2$
    \item[h.] $(5x + 0,1)^2$
    \item[i.] $(\dfrac{x}{2} + 1)^2$ 
\end{enumerate}

\end{multicols}

\Pointilles[18] 

\subsection*{Bonus : Ok Boomer}
Deux profs de maths, Jason et Thomas discutent entre eux... Le premier se plaint d'être trop vieux : \og Tu t'imagines, cette année, j'ai le double de mes élèves de secondes. \fg Le second lui répond : \og Ne m'en parle pas, cette année, j'ai le triple de mes sixièmes et ils ont 11ans. \fg \\
Qui a raison ? Justifier (un peu).

\Pointilles[3] 



\end{document}

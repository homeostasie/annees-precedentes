\documentclass[11pt]{article}
\usepackage{geometry} % Pour passer au format A4
\geometry{hmargin=1cm, vmargin=1cm} % 

% Page et encodage
\usepackage[T1]{fontenc} % Use 8-bit encoding that has 256 glyphs
\usepackage[english,french]{babel} % Français et anglais
\usepackage[utf8]{inputenc} 

\usepackage{lmodern}
\setlength\parindent{0pt}

% Graphiques
\usepackage{graphicx,float,grffile}
\usepackage{pgf,tikz}
\usetikzlibrary{arrows}

% Maths et divers
\usepackage{amsmath,amsfonts,amssymb,amsthm,verbatim}
\usepackage{multicol,multido,enumitem,url,eurosym,gensymb}

% Sections
\usepackage{sectsty} % Allows customizing section commands
\allsectionsfont{\centering \normalfont\scshape}

% Tête et pied de page

\usepackage{fancyhdr} 
\pagestyle{fancyplain} 

\fancyhead{} % No page header
\fancyfoot{}

\renewcommand{\headrulewidth}{0pt} % Remove header underlines
\renewcommand{\footrulewidth}{0pt} % Remove footer underlines

\newcommand{\horrule}[1]{\rule{\linewidth}{#1}} % Create horizontal rule command with 1 argument of height

%----------------------------------------------------------------------------------------
%	Début du document
%----------------------------------------------------------------------------------------

\begin{document}

%----------------------------------------------------------------------------------------
% RE-DEFINITION
%----------------------------------------------------------------------------------------
% MATHS
%-----------

\newtheorem{Definition}{Définition}
\newtheorem{Theorem}{Théorème}
\newtheorem{Proposition}{Propriété}

% MATHS
%-----------
\renewcommand{\labelitemi}{$\bullet$}
\renewcommand{\labelitemii}{$\circ$}
\newcommand{\Pointilles}[1][3]{%
  \multido{}{#1}{\makebox[\linewidth]{\dotfill}\\[\parskip]
}}

%----------------------------------------------------------------------------------------
%	Titre
%----------------------------------------------------------------------------------------

\setlength{\columnseprule}{1pt}

\textbf{Nom, Prénom :} \hspace{8cm} \textbf{Classe :} \hspace{3cm} \textbf{Date :}\\
\vspace{-0.2cm}
\begin{center}
  \textit{On vit de ce que l’on obtient. On construit sa vie sur ce que l’on donne.}  - \textbf{Winston Churchill}
\end{center}
\vspace{-0.2cm}


\subsection*{ex1 -  la longueur manquante}
\textbf{Écrire le calcul et le résultat.}
  
  \begin{center}
  \definecolor{zzttqq}{rgb}{0.6,0.2,0}
  \begin{tikzpicture}[line cap=round,line join=round,>=triangle 45,x=1.0cm,y=1.0cm,scale=0.5]
  \clip(-1.73,-4.29) rectangle (25.18,5.94);
  \fill[color=zzttqq,fill=zzttqq,fill opacity=0.1] (-0.24,2.54) -- (4.28,2.54) -- (-0.22,4.76) -- cycle;
  \fill[color=zzttqq,fill=zzttqq,fill opacity=0.1] (6.86,2.48) -- (10.7,2.44) -- (10.72,5.14) -- cycle;
  \fill[color=zzttqq,fill=zzttqq,fill opacity=0.1] (13.18,5.06) -- (13.51,-1.46) -- (17.57,-1.24) -- cycle;
  \fill[color=zzttqq,fill=zzttqq,fill opacity=0.1] (0.55,0.81) -- (-0.21,-2.41) -- (7.23,-3.72) -- cycle;
  \fill[color=zzttqq,fill=zzttqq,fill opacity=0.1] (19.07,4.46) -- (23.57,4.43) -- (23.46,-4.05) -- cycle;
  \draw [color=zzttqq] (-0.24,2.54)-- (4.28,2.54);
  \draw [color=zzttqq] (4.28,2.54)-- (-0.22,4.76);
  \draw [color=zzttqq] (-0.22,4.76)-- (-0.24,2.54);
  \draw [color=zzttqq] (6.86,2.48)-- (10.7,2.44);
  \draw [color=zzttqq] (10.7,2.44)-- (10.72,5.14);
  \draw [color=zzttqq] (10.72,5.14)-- (6.86,2.48);
  \draw [color=zzttqq] (13.18,5.06)-- (13.51,-1.46);
  \draw [color=zzttqq] (13.51,-1.46)-- (17.57,-1.24);
  \draw [color=zzttqq] (17.57,-1.24)-- (13.18,5.06);
  \draw [color=zzttqq] (0.55,0.81)-- (-0.21,-2.41);
  \draw [color=zzttqq] (-0.21,-2.41)-- (7.23,-3.72);
  \draw [color=zzttqq] (7.23,-3.72)-- (0.55,0.81);
  \draw [color=zzttqq] (19.07,4.46)-- (23.57,4.43);
  \draw [color=zzttqq] (23.57,4.43)-- (23.46,-4.05);
  \draw [color=zzttqq] (23.46,-4.05)-- (19.07,4.46);
  \end{tikzpicture}
\end{center}

\Pointilles[5]

\subsection*{ex2 - \textbf{RÉDIGER} - Rectangle ?}

\begin{enumerate}
  \item[3a.]Soit WEQ un triangle tel que : EQ = 9,1 cm, QW = 3,5 cm et EW = 8,4 cm.\\
  \textbf{Quelle est la nature du triangle WEQ ?}

  \item[3b.]Soit HZM un triangle tel que : MH = 9,6 cm, ZH = 11 cm et ZM = 14,8 cm. \\
  \textbf{Quelle est la nature du triangle HZM ?}
\end{enumerate}

\Pointilles[20]
\newpage

\subsection*{ex3 - \textbf{RÉDIGER} - Calculer une longueur}


\begin{enumerate}
  \item[2a.]Soit PVR un triangle rectangle en V tel que : RV = 17,5 cm et RP = 18,5 cm. \\
  \textbf{Calculer la longueur PV.}

  \item[2b.]Soit LQF un triangle rectangle en F tel que : QF = 4,8 cm et LF = 2 cm. \\
  \textbf{Calculer la longueur QL.}

  \item[2c.]Soit JMR un triangle rectangle en R tel que : MJ = 2,9 cm et MR = 2,1 cm. \\
  \textbf{Calculer la longueur JR.}

  \item[2d.]Soit ABC un triangle rectangle en A tel que : AB = 12,9 cm et AC = 12,1 cm. \\
  \textbf{Calculer la longueur BC.}

\end{enumerate}

\Pointilles[43]




\end{document}

\documentclass[12pt]{article}
\usepackage{geometry} % Pour passer au format A4
\geometry{hmargin=1cm, vmargin=1cm} % 

% Page et encodage
\usepackage[T1]{fontenc} % Use 8-bit encoding that has 256 glyphs
\usepackage[english,french]{babel} % Français et anglais
\usepackage[utf8]{inputenc} 

\usepackage{lmodern}
\setlength\parindent{0pt}

% Graphiques
\usepackage{graphicx,float,grffile}
\usepackage{pgf,tikz}
\usetikzlibrary{arrows}

% Maths et divers
\usepackage{amsmath,amsfonts,amssymb,amsthm,verbatim}
\usepackage{multicol,multido,enumitem,url,eurosym,gensymb}

% Sections
\usepackage{sectsty} % Allows customizing section commands
\allsectionsfont{\centering \normalfont\scshape}

% Tête et pied de page

\usepackage{fancyhdr} 
\pagestyle{fancyplain} 

\fancyhead{} % No page header
\fancyfoot{}

\renewcommand{\headrulewidth}{0pt} % Remove header underlines
\renewcommand{\footrulewidth}{0pt} % Remove footer underlines


\begin{document}

Pour demain,

\textbf{L'épreuve} : L'après-midi, 2h, 100 points, influence sur le deuxième trimestre, influence pour le lsu point de CC pour le brevet.

\subsection*{Le matériel}

\begin{itemize}
    \item Calculatrice : pas de portable, pas de prêt de calculatrice, qu'on connaît, pas casser, avec des piles,\dots
    \item Crayon à papier : taillé, pour les graphiques.
    \item Stylo : pour écrire le reste.
    \item Règle : pour souligner les résultats, les exercices, faire des figures, tracer des droites, \dots
    \item Équerre, Rapporteur, Compas : pas forcément utile demain, mais à avoir le jour du brevet. 
\end{itemize}

\subsection*{La rédaction}

\begin{itemize}
    \item Sur l'annexe, je ne mets que mon numéro de candidat et je rends l'annexe avec ma copie.
    \item Sauf s'il est écrit dans l'exercice de ne pas justifier : il faut justifier.
    \item On n'écrit pas juste le résultat : on écrit toujours le ou les calculs qu'on a fait sur la calculatrice et qui ont permis d'obtenir le résultat.
    \item On évite de mettre un gros pâté de blanc... Vos deux correcteurs n'apprécient pas. En vrai, c'est une perte de temps, ce n'est pas esthétique : il faut mieux une petite rayure et continuer en dessous.
    \item On évite de passer / perdre 30 min sur une question qu'on ne sait pas faire... perte de temps = perte de points.
    \item On peut passer une question... (et y revenir après si on a du temps)
    \item On peut faire les exercices dans le désordre.
    \item Il y a des questions faciles.
    \item Il y a des points de rédactions
\end{itemize}

\section*{Les exercices}

Il y a 7 exercices. (15 / 20 min max par exo en fonction du barème... Certains sont faciles, d'autres sont plus durs. 

\textit{bon courage !!!}


\newpage


\begin{multicols}{2}

\subsection*{Pourcentages}

\vspace{5.5cm}

\subsection*{Vitesse}
    
\vspace{5.5cm}

\subsection*{Pythagore}

\vspace{5.5cm}

\subsection*{Thalès} 

\vspace{5.5cm}

\columnbreak

\subsection*{Programme de calcul}

\vspace{5.5cm}

\subsection*{Fonctions}

\vspace{5.5cm}

\subsection*{Problèmes ouverts}

\vspace{5.5cm}

\subsection*{Scratch}

\vspace{5.5cm}

\end{multicols}

\end{document}

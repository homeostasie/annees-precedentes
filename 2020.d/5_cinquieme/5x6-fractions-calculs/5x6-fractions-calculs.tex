\documentclass[12pt]{article}
\usepackage{geometry} % Pour passer au format A4
\geometry{hmargin=1cm, vmargin=1cm} % 

% Page et encodage
\usepackage[T1]{fontenc} % Use 8-bit encoding that has 256 glyphs
\usepackage[english,french]{babel} % Français et anglais
\usepackage[utf8]{inputenc} 

\usepackage{lmodern}
\setlength\parindent{0pt}

% Graphiques
\usepackage{graphicx,float,grffile}

% Maths et divers
\usepackage{amsmath,amsfonts,amssymb,amsthm,verbatim}
\usepackage{multicol,enumitem,url,eurosym,gensymb}

% Sections
\usepackage{sectsty} % Allows customizing section commands
\allsectionsfont{\centering \normalfont\scshape}

% Tête et pied de page

\usepackage{fancyhdr} 
\pagestyle{fancyplain} 

\fancyhead{} % No page header
\fancyfoot{}

\renewcommand{\headrulewidth}{0pt} % Remove header underlines
\renewcommand{\footrulewidth}{0pt} % Remove footer underlines

\newcommand{\horrule}[1]{\rule{\linewidth}{#1}} % Create horizontal rule command with 1 argument of height

%----------------------------------------------------------------------------------------
%   Début du document
%----------------------------------------------------------------------------------------

\begin{document}

%----------------------------------------------------------------------------------------
% RE-DEFINITION
%----------------------------------------------------------------------------------------
% MATHS
%-----------

\newtheorem{Definition}{Définition}
\newtheorem{Theorem}{Théorème}
\newtheorem{Proposition}{Propriété}

% MATHS
%-----------
\renewcommand{\labelitemi}{$\bullet$}
\renewcommand{\labelitemii}{$\circ$}
%----------------------------------------------------------------------------------------
%   Titre
%----------------------------------------------------------------------------------------

\setlength{\columnseprule}{1pt}

\horrule{2px}
\section*{Chapitre 5 - Calculer avec des fractions}
\horrule{2px}

\section*{Comprendre l'heure et l'addition de fractions}

\subsection*{30 minutes}

un quart d'heure plus un quart d'heure égale deux quarts d'heure.
$$\dfrac{1}{4} + \dfrac{1}{4} = \dfrac{2}{4}$$

un quart d'heure plus un quart d'heure égale une demi-heure.
$$\dfrac{1}{4} + \dfrac{1}{4} = \dfrac{1}{2}$$

\subsection*{45 minutes}

un quart d'heure plus un quart d'heure plus un quart d'heure égale trois quarts d'heure.
$$\dfrac{1}{4} + \dfrac{1}{4} + \dfrac{1}{4} = \dfrac{3}{4}$$
une demi-heure plus un quart d'heure égale trois quarts d'heure.
$$\dfrac{1}{2} + \dfrac{1}{4} = \dfrac{3}{4}$$

\subsection*{60 minutes}

un quart d'heure plus un quart d'heure plus un quart d'heure égale quatre quarts d'heure.
$$\dfrac{1}{4} + \dfrac{1}{4} + \dfrac{1}{4} + \dfrac{1}{4}= \dfrac{4}{4}$$

une demi-heure plus une demi-heure égale deux demi-heures.
$$\dfrac{1}{2} + \dfrac{1}{2} = \dfrac{2}{2}$$

une demi-heure plus une demi-heure égale une heure.
$$\dfrac{1}{2} + \dfrac{1}{2} = \dfrac{2}{2}$$

\newpage
\subsection*{On formalise}

Pour additionner des fractions, il faut le même dénominateur. Alors on additionne les numérateurs. 

$$\dfrac{1}{5} + \dfrac{2}{5} = \dfrac{3}{5}$$

Et si on n'a pas les mêmes dénominateurs. Pas le choix, on passe les fractions sur le même dénominateurs. (Fractions égales)

\begin{align*}
\dfrac{5}{12} + \dfrac{4}{3} &= \dfrac{5}{12} + \dfrac{4 \times 4}{3 \times 4}\\
							 &= \dfrac{5}{12} + \dfrac{16}{12} \\
                             &= \dfrac{21}{12}								 
\end{align*}


Parfois, après le résultat, on peut simplifier la fraction.

\begin{align*}
\dfrac{6}{2} + \dfrac{10}{4} &= \dfrac{12}{4} + \dfrac{10}{4}\\
							 &= \dfrac{22}{4} \\
							 &= \dfrac{11}{2}					 
\end{align*}


Et il est même possible que le résultat soit un nombre entier.

\begin{align*}
\dfrac{5}{3} + \dfrac{4}{3} &= \dfrac{9}{3}\\
							&= 3			 
\end{align*}

Parfois on additionne un nombre entier et une fraction.

\begin{align*}
5 + \dfrac{42}{4} &=  \dfrac{5}{1} + \dfrac{42}{4}\\
				  &= \dfrac{20}{4} + \dfrac{42}{4}\\
				  &= \dfrac{62}{4} \\
				  &= \dfrac{31}{2}					 
\end{align*}

On peut utiliser la calculatrice, mais il faudra rédiger :

\begin{itemize}
	\item Le passage au même dénominateur.
	\item La simplification de fraction.
\end{itemize}

\newpage

\section*{Comprendre la soustraction de fractions}

La soustraction de fractions se passe exactement pareil que l'addition.

\section*{Comprendre la multiplication de fractions}

Lors d'un partage, on a tendance à dire : \textbf{de, des, du}. Cela signifie \textbf{fois} : $\times$.


\subsection*{La fraction d'une quantité}

La moitié des 24 élèves la classe correspond à 12 élèves. 
$$\dfrac{1}{2} \times 24 = 12$$


\subsection*{La fraction d'une fraction}

La moitié des deux sixièmes de la classe est un sixième
$$\dfrac{1}{2} \times \dfrac{2}{6} = \dfrac{1}{6}$$

\subsection*{On formalise}

Pour multiplier des fractions, on multiplie les numérateurs entre eux et les dénominateurs entre eux.

\begin{align*}
\dfrac{2}{5} \times \dfrac{4}{3} &= \dfrac{2 \times 4}{5 \times 3}  \\
								 &= \dfrac{8}{15}								 
\end{align*}

Parfois on multiplie un nombre entier et une fraction.

\begin{align*}
5 \times \dfrac{2}{7} &=  \dfrac{5}{1} \times \dfrac{2}{7}\\
				      &= \dfrac{5 \times 2}{7}\\
				      &= \dfrac{10}{7} \\			 
\end{align*}

On peut utiliser la calculatrice, mais il faudra rédiger :

\begin{itemize}
	\item La multiplication dans la fraction.
	\item La simplification de fraction.
\end{itemize}


\newpage

\section*{On simplifie}

Il est intéressant dans ce chapitre d'utiliser sa calculatrice, notamment pour vérifier les calculs. Les touches intéressantes sont : 

\begin{itemize}
	\item Fraction (affichage maths)
	\item Division (parenthèse)
	\item Valeur approchée
\end{itemize}

On remarque que le résultat n'est pas forcément le même que lorsqu'on rédige le calcul. Pourquoi ? Réponse : \textbf{Les fractions égales}.\\

Une calculatrice propose toujours une \textbf{fraction simplifiée}.

Pour simplifier une fraction, on cherche une fraction égale écrit avec des nombres plus petits. On utilise les critières de divisibilités :

\begin{itemize}
	\item 2 : Les deux nombres se terminent pas un chiffre pair : 0, 2, 4, 6 ou 8
	\item 3 : La somme des chiffres pour chaque nombre est un multiple de 3. 
	\item 5 : Les deux nombres se terminent pas un chiffre pair : 0 ou 5
\end{itemize}

\end{document}

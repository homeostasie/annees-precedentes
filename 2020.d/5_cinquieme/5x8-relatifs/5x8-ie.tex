%%!TEX TS-program = latex
\documentclass[a4paper,11pt]{article}
\usepackage[utf8]{inputenc} % UTF-8
\usepackage[T1]{fontenc}
\usepackage{lmodern} % Prévient un bug d'affichage evince lié à [T1]{fontenc}
\usepackage[frenchb]{babel} % francisation
\usepackage{adjustbox}
\usepackage[fleqn]{amsmath} % aligne le mode maths à gauche
\usepackage{amssymb} % the amsfont symbols
\usepackage[table, usenames, svgnames]{xcolor} % Couleurs
\usepackage{multicol} % Multi-colonnes
\usepackage{fancyhdr} % Mise en page, en-tête et pied de page
\usepackage{calc} % Opérations
\usepackage{marvosym} % Martin Vogels Symbole (\EUR)
\usepackage{cancel} % draw diagonal lines
\usepackage{units} % typesetting units and nice fractions
\usepackage[autolanguage]{numprint} % écrituredes virgules
\usepackage{tabularx} % creates a paragraph-like column whose width
                          % automatically expands
\usepackage{wrapfig} % allows figures or tables to have text wrapped around
\usepackage{pst-eucl, pst-plot} % figures géométriques
\usepackage{enumitem}
\usepackage{interval}
\usepackage{wasysym} % Symbole Euro
\usepackage{mathtools} % Encadrement dans align*
\usepackage[inline]{asymptote}
\usepackage{tkz-tab}
\usetikzlibrary{external} % set up externalization
\tikzexternalize[shell escape=-enable-write18] % activate externalisation
\tikzset{%
  external/system call={%
    latex \tikzexternalcheckshellescape -halt-on-error
    -interaction=batchmode -jobname "\image" "\texsource" &&
    dvips -o "\image".ps "\image".dvi &&
    ps2eps -f "\image.ps"
  }
}
\intervalconfig{ separator symbol={\,;\,}}

%\usepackage{textcomp}

\usepackage[a4paper, dvips, left=1cm, right=1cm, top=1cm,%
  bottom=1cm, marginpar=5mm, marginparsep=5pt]{geometry}
\newcounter{exo}
\frenchbsetup{StandardItemLabels} % remet \textbullet pour les listes
\setlength{\headheight}{18pt}
\setlength{\fboxsep}{5pt}
\setlength\parindent{0em}
\setlength\mathindent{0em}
\setlength{\columnsep}{30pt}
\usepackage[bookmarks=true, bookmarksnumbered=true, ps2pdf, pagebackref=true,%
            colorlinks=true,linkcolor=blue,plainpages=true,unicode]{hyperref}
\hypersetup{pdfauthor={Jérôme Ortais},pdfsubject={Exercices de
    mathématiques},pdftitle={Exercices créés par Pyromaths, un logiciel libre
    en Python sous licence GPL}}

\def\pshlabel#1{\psframebox*[fillcolor=White,framearc=.2]{\footnotesize $#1$}}
\def\psvlabel#1{\psframebox*[fillcolor=White,framearc=.2]{\footnotesize $#1$}}

\makeatletter
\newcommand\styleexo[1][]{
  \renewcommand{\theenumi}{\arabic{enumi}}
  \renewcommand{\labelenumi}{$\blacktriangleright$\textbf{\theenumi.}}
  \renewcommand{\theenumii}{\alph{enumii}}
  \renewcommand{\labelenumii}{\textbf{\theenumii)}}
  {\fontfamily{pag}\fontseries{b}\selectfont \underline{#1 \theexo}}
  \par\@afterheading\vspace{0.5\baselineskip minus 0.2\baselineskip}}
\newcommand*\exercice{%
  \psset{unit=1cm, dash=4pt 4pt, PointName=default,linecolor=Maroon,
    dotstyle=x, linestyle=solid, hatchcolor=Peru, gridcolor=Olive,
    subgridcolor=Olive, fillcolor=Peru}
  %\ifthenelse{\equal{\theexo}{0}}{}{\filbreak}
  \refstepcounter{exo}%
  \stepcounter{nocalcul}%
  \par\addvspace{1.5\baselineskip minus 1\baselineskip}%
  \@ifstar%
  {\penalty-130\styleexo[Corrigé de l'exercice]}%
  {\penalty-130\styleexo[Exercice]}%
  }
\newcommand{\checkedbox}[0]{\makebox[0pt][l]{$\square$}\raisebox{.15ex}{\hspace{0pt}$\checkmark$}}
\makeatother
\newlength{\ltxt}
\newsavebox{\mybox}
\newlength{\wdofmybox}
\newcommand{\figureadroite}[2]{%
  \setlength{\ltxt}{\linewidth}
  \sbox{\mybox}{\hbox{#1}}
  \settowidth{\wdofmybox}{\usebox{\mybox}}
  \addtolength{\ltxt}{-\wdofmybox}
  \addtolength{\ltxt}{-10pt}
  \begin{minipage}{\ltxt}
    #2
  \end{minipage}
  \hfill
  \begin{minipage}{\wdofmybox}
    #1
  \end{minipage}
}

\usepackage{fancyhdr} 
\pagestyle{fancyplain} 

\fancyhead{} % No page header
\fancyfoot{}

\renewcommand{\headrulewidth}{0pt} % Remove header underlines
\renewcommand{\footrulewidth}{0pt} % Remove footer underlines

\newcommand{\horrule}[1]{\rule{\linewidth}{#1}} % Create horizontal rule command with 1 argument of height

\newcommand{\Pointilles}[1][3]{%
  \multido{}{#1}{\makebox[\linewidth]{\dotfill}\\[\parskip]
}}



\begin{document}

\textbf{Nom, Prénom :} \hspace{8cm} \textbf{Classe :} \hspace{3cm} \textbf{Date :}\\

\begin{center}
  \textit{Les propositions mathématiques sont reçues comme vraies parce que personne n'a intérêt qu'elles soient fausses.}  - \textbf{Montesquieu}
\end{center}

\subsection*{Ex1 - Cours}

\begin{enumerate}
  \item[1.] Donner la définition d'opposé. \\
  \Pointilles[2]
  \item[2.] Donner la définition de la soustraction d'un nombre.\\
  \Pointilles[2]
\end{enumerate}

\subsection*{Ex2 - Transformation soustraction en addition}

Transformer les soustractions en des additions sans changer le résultat.

\begin{multicols}{2}\noindent
  \begin{enumerate}
    \item[a.] $4 - 6 = \dotfill$
    \item[b.] $-4 - 6 = \dotfill$
    \item[c.] $4 - (-6) = \dotfill$
    \item[d.] $-4 - (-6) = \dotfill$    
    \item[e.] $14 - 5 = \dotfill$
    \item[f.] $-14 - 5 = \dotfill$    
    \item[g.] $14 - (-5) = \dotfill$
    \item[h.] $-14 - (-5) = \dotfill$    

  \end{enumerate}
\end{multicols}

\subsection*{Ex3 - Calculs relatifs}

\begin{multicols}{3}\noindent
  \begin{enumerate}
    \item[a.] $4 + \ldots\ldots\ldots = 6$
    \item[b.] $6 + \left( -1\right) = \ldots\ldots\ldots$
    \item[c.] $\ldots\ldots\ldots + 7 = 11$
    \item[d.] $2 + \ldots\ldots\ldots = -8$
    \item[e.] $-6 + 2 = \ldots\ldots\ldots$
    \item[f.] $\ldots\ldots\ldots + 9 = 7$
    \item[g.] $6 + \left( -1\right) = \ldots\ldots\ldots$
    \item[h.] $-7 + \ldots\ldots\ldots = -11$
    \item[i.] $\ldots\ldots\ldots - \left( -5\right) = -1$
    \item[j.] $\ldots\ldots\ldots + \left( -10\right) = -8$
    \item[k.] $\ldots\ldots\ldots + \left( -6\right) = -11$
    \item[l.] $4 + 10 = \ldots\ldots\ldots$
    \item[m.] $-18 - \left( -10\right) = \ldots\ldots\ldots$
    \item[n.] $7 - \left( -1\right) = \ldots\ldots\ldots$
    \item[o.] $\ldots\ldots\ldots + \left( -8,1\right) = -8,6$
    \item[p.] $5,4 + \ldots\ldots\ldots = 9$
    \item[q.] $-11,6 - \left( -3,6\right) = \ldots\ldots\ldots$
    \item[r.] $-11,6 - \left( -8\right) = \ldots\ldots\ldots$
    \item[s.] $-7,4 - \ldots\ldots\ldots = -2,4$
    \item[t.] $14,4 - \ldots\ldots\ldots = 6,7$
  \end{enumerate}
\end{multicols}

\subsection*{Ex4 - Repérage}

\parbox{0.4\linewidth}{
\begin{enumerate}
  \item Donner les coordonnées des points A, B, F, G, J et K.\\
  \Pointilles[3]
  \item Placer dans le repère les points M, S, T, V, W et Y de coordonnées respectives \hbox{$(4{,}5~;~0)$}, \hbox{$(2{,}5~;~1{,}5)$}, \hbox{$(0~;~-2)$}, \hbox{$(-1~;~-0{,}5)$}, \hbox{$(-4~;~2{,}5)$} et \hbox{$(2{,}5~;~-1{,}5)$}. 
  \item Placer dans le repère le point Z d'ordonnée -1,5 et d'abscisse 1,5
\end{enumerate}}\hfill 
\parbox{0.55\linewidth}{
\psset{unit=0.8cm}
\begin{pspicture}(-4.95,-4.95)(4.95,4.95)
  \psgrid[subgriddiv=2, subgridcolor=lightgray, gridlabels=8pt](0,0)(-5,-5)(5,5)
  \psline[linewidth=1.2pt]{->}(-5,0)(5,0)
  \psline[linewidth=1.2pt]{->}(0,-5)(0,5)
  \pstGeonode[PointSymbol=x,PosAngle={45,-135,45,-45,45,135,},PointNameSep=0.4](0, 0.0){A}(-2.0, -2.5){B}(2.5, 0){F}(2.0, -2.5){G}(4.5, 2.5){J}(-2.0, 4.0){K}
\end{pspicture}}


\newpage


\textbf{Nom, Prénom :} \hspace{8cm} \textbf{Classe :} \hspace{3cm} \textbf{Date :}\\

\begin{center}
  \textit{Les propositions mathématiques sont reçues comme vraies parce que personne n'a intérêt qu'elles soient fausses.}  - \textbf{Montesquieu}
\end{center}

\subsection*{Ex1 - Cours}

\begin{enumerate}
  \item[1.] Donner la définition d'opposé. \\
  \Pointilles[2]
  \item[2.] Donner la définition de la soustraction d'un nombre.\\
  \Pointilles[2]
\end{enumerate}

\subsection*{Ex2 - Transformation soustraction en addition}

Transformer les soustractions en des additions sans changer le résultat.

\begin{multicols}{2}\noindent
  \begin{enumerate}
    \item[a.] $6 - 8 = \dotfill$
    \item[b.] $-6 - 8 = \dotfill$
    \item[c.] $6 - (-8) = \dotfill$
    \item[d.] $-6 - (-8) = \dotfill$    
    \item[e.] $13 - 4 = \dotfill$
    \item[f.] $-13 - 4 = \dotfill$    
    \item[g.] $13 - (-4) = \dotfill$
    \item[h.] $-13 - (-4) = \dotfill$    

  \end{enumerate}
\end{multicols}

\subsection*{Ex3 - Calculs relatifs}

\begin{multicols}{3}\noindent
  \begin{enumerate}
    \item $5 + \left( -6\right) = \ldots\ldots\ldots$
    \item $\ldots\ldots\ldots + 2 = 8$
    \item $9 + \left( -6\right) = \ldots\ldots\ldots$
    \item $20 + \ldots\ldots\ldots = 30$
    \item $-6 + \ldots\ldots\ldots = 3$
    \item $-8 + 2 = \ldots\ldots\ldots$
    \item $\ldots\ldots\ldots - \left( -7\right) = 2$
    \item $\ldots\ldots\ldots - 4 = -7$
    \item $5 + \ldots\ldots\ldots = 3$
    \item $9 + 2 = \ldots\ldots\ldots$
    \item $\ldots\ldots\ldots - \left( -2\right) = -9$
    \item $-4 - \left( -5\right) = \ldots\ldots\ldots$
    \item $10 - \ldots\ldots\ldots = 5$
    \item $\ldots\ldots\ldots + 10 = 3$
    \item $-9,1 - \left( -1,7\right) = \ldots\ldots\ldots$
    \item $4,4 + 4,2 = \ldots\ldots\ldots$
    \item $4,4 + \ldots\ldots\ldots = 12,9$
    \item $\ldots\ldots\ldots + 7,6 = 1,3$
    \item $7,3 - \left( -2,7\right) = \ldots\ldots\ldots$
    \item $-10,8 - \ldots\ldots\ldots = -4,7$
  \end{enumerate}
\end{multicols}

\subsection*{Ex4 - Repérage}


\parbox{0.4\linewidth}{
\begin{enumerate}
\item Donner les coordonnées des points B, C, F, G, H et J. \\
\Pointilles[3]
\item Placer dans le repère les points L, O, P, R, S et T de coordonnées respectives \hbox{$(0~;~-0{,}5)$}, \hbox{$(4{,}5~;~4)$}, \hbox{$(-1{,}5~;~0)$}, \hbox{$(-3{,}5~;~-3)$}, \hbox{$(1{,}5~;~-0{,}5)$} et \hbox{$(-4~;~4{,}5)$}. 
\item Placer dans le repère le point W d'ordonnée -2 et d'abscisse -4
\end{enumerate}}\hfill 
\parbox{0.55\linewidth}{
\psset{unit=0.8cm}
\begin{pspicture}(-4.95,-4.95)(4.95,4.95)
\psgrid[subgriddiv=2, subgridcolor=lightgray, gridlabels=8pt](0,0)(-5,-5)(5,5)
\psline[linewidth=1.2pt]{->}(-5,0)(5,0)
\psline[linewidth=1.2pt]{->}(0,-5)(0,5)
\pstGeonode[PointSymbol=x,PosAngle={-45,45,-135,45,45,135,},PointNameSep=0.4](2.5, -1.0){B}(4.0, 0){C}(-1.5, -4.0){F}(0, 1.0){G}(3.0, 1.5){H}(-0.5, 2.0){J}
\end{pspicture}}
\end{document}
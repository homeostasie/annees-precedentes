\documentclass[12pt]{article}
\usepackage{geometry} % Pour passer au format A4
\geometry{hmargin=1cm, vmargin=1cm} % 

% Page et encodage
\usepackage[T1]{fontenc} % Use 8-bit encoding that has 256 glyphs
\usepackage[english,french]{babel} % Français et anglais
\usepackage[utf8]{inputenc} 

\usepackage{lmodern}
\setlength\parindent{0pt}
\usepackage{xcolor}

% Graphiques
\usepackage{graphicx,float,grffile}

% Maths et divers
\usepackage{amsmath,amsfonts,amssymb,amsthm,verbatim}
\usepackage{multicol,enumitem,url,eurosym,gensymb}

% Sections
\usepackage{sectsty} % Allows customizing section commands
\allsectionsfont{\centering \normalfont\scshape}

% Tête et pied de page

\usepackage{fancyhdr} 
\pagestyle{fancyplain} 

\fancyhead{} % No page header
\fancyfoot{}

\renewcommand{\headrulewidth}{0pt} % Remove header underlines
\renewcommand{\footrulewidth}{0pt} % Remove footer underlines

\newcommand{\horrule}[1]{\rule{\linewidth}{#1}} % Create horizontal rule command with 1 argument of height

%----------------------------------------------------------------------------------------
%   Début du document
%----------------------------------------------------------------------------------------

\begin{document}


\setlength{\columnseprule}{1pt}

\horrule{2px}
\section*{Progression 2020 - 2021 : Cinquième}
\horrule{2px}

\begin{multicols}{2}

\subsection*{\textcolor{red}{1 - Calculer avec les nombres décimaux}}

\begin{itemize}
\item Addition, soustraction, multiplication, division
\item Priorités de calculs 
\item Calculer astucieusement (factorisation, développement)
\item Conventions d’écritures
\item Résolution de problèmes (expression à poser en une seule ligne)
\item Programmes de calculs
\end{itemize}

\subsection*{\textcolor{green}{2 - Construire des triangles}}

\begin{itemize}
\item Inégalité triangulaire
\item Constructions de triangles
\item Droites remarquable du triangle (hauteur et médiatrice obligatoires, médiane vue en exemple)
\item Angles et triangles (somme des angles d’un triangle, triangles particuliers et angles)
+ Cercle circonscrit (construction vue comme un exemple d'approfondissement dans le cours + méthode avec Geogebra)
\end{itemize}

\subsection*{\textcolor{blue}{3 - Déterminer et calculer des aires}}

\begin{itemize}
\item Unité d’aire
\item Aire de figures usuelles (carré, rectangle, triangle rectangle, triangle quelconque, disque) + parallélogrammes
\item Méthodes par décomposition, soustraction ou découpage/collage
\end{itemize}

\subsection*{\textcolor{red}{4 - Découvrir et comprendre les nombres rationnels}}

\begin{itemize}
\item Quotient en tant que nombre, en tant que proportion (fréquence, pourcentage)
\item Définition d'une fraction décimale
\item Fractions égales
\item Ordre (ranger et comparer des quotients)
\end{itemize}

\subsection*{\textcolor{green}{5 - Comprendre et utiliser les symétries}}

\begin{itemize}
\item Comprendre l’effet d'une symétrie axiale sur une figure
\item Comprendre l'effet d'une symétrie centrale sur une figure
\item Centre et axe de symétrie
\end{itemize}

\subsection*{\textcolor{red}{6 - Calculer avec des fractions}}

\begin{itemize}

\item Opérations (addition, soustraction, multiplication)
\item Fractions de fractions
\item Qu'est-ce que prendre une fraction d'une quantité ?
\item Problèmes
\end{itemize}

\subsection*{\textcolor{red}{7 - Comprendre la proportionnalité}}

\begin{itemize}
\item Reconnaître une situation de proportionnalité
\item Méthodes : multiplicative, additive, coefficient de proportionnalité (quotient vu en tant qu’opérateur)
\item Travailler sur plusieurs registres : formules, graphiques, tableaux
\end{itemize}

\end{multicols}

\newpage

\begin{multicols}{2}

\subsection*{\textcolor{green}{8 - Angles et parallélisme}}

\begin{itemize}
\item Vocabulaire des angles (opposés par le sommet, alternes-internes, correspondants).
\item Angles formés par deux droites parallèles et une droite sécante
\item Reconnaître des droites parallèles
\end{itemize}

\subsection*{\textcolor{blue}{9 - Statistiques}}

\begin{itemize}
\item Calculer des effectifs, des fréquences
\item Regrouper des données en classes
\item Diagramme en bâtons, diagramme circulaire, histogramme
\item Recueillir des données, les organiser
\item Lire des données sous forme de données brutes, de tableau, de graphique
\end{itemize}

\subsection*{\textcolor{red}{10 - Découvrir les nombres relatifs}}

\begin{itemize}
\item Utilité et histoire
\item De quoi s'agit-il ? Situations concrètes
\item Notion de nombres opposés
\item Distance à zéro
\item Repérage sur une droite graduée et dans un repère du plan
\item Comparaison de nombres relatifs
\end{itemize}

\subsection*{\textcolor{green}{11 - Les parallélogrammes}}

\begin{itemize}
\item Intro : le parallélogramme est une figure obtenue par symétrie centrale
\item Définition
\item Propriétés relatives aux côtés et aux diagonales
\item Parallélogrammes particuliers
\item Reconnaître des quadrilatères particuliers
\end{itemize}

\subsection*{\textcolor{red}{12 - Calculer avec les nombres relatifs}}

\begin{itemize}
\item Addition, soustraction
\item Distance entre deux points
\item Somme algébrique
\item Gestion des parenthèse, simplifiée
\item Problèmes
\end{itemize}

\columnbreak

\subsection*{\textcolor{red}{13 - Découvrir le calcul littéral}}

\begin{itemize}
\item Approche expérimentale (situation où l'on a besoin d'une formule pour généraliser, carreaux colorés, allumettes, etc...)
\item Introduction de la lettre au travers d'exemples concrets (Périmètre, programme de calcul, etc).
\item Tester une égalité et utiliser une égalité
\item Distributivité simple
\end{itemize}

\subsection*{\textcolor{green}{14 - Prismes et Cylindres}}

\begin{itemize}
\item Définition
\item Volumes
\item Problèmes
\end{itemize}

\subsection*{\textcolor{blue}{15 - Hasard et Probabilités}}

\begin{itemize}
\item Décrire une expérience aléatoire, qu’est-ce que le hasard ?
\item Modèle d’équiprobabilité
\item Exprimer la probabilité d’un évènement
\item Utilisation du tableur pour simuler une expérience aléatoire
\end{itemize}

\end{multicols}

\end{document}


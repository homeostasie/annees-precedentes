\documentclass[11pt]{article}
\usepackage{geometry,marginnote} % Pour passer au format A4
\geometry{hmargin=1cm, vmargin=1cm} % 

% Page et encodage
\usepackage[T1]{fontenc} % Use 8-bit encoding that has 256 glyphs
\usepackage[english,french]{babel} % Français et anglais
\usepackage[utf8]{inputenc} 

\usepackage{lmodern,numprint}
\setlength\parindent{0pt}

% Graphiques
\usepackage{graphicx,float,grffile,units}
\usepackage{tikz,pst-eucl,pst-plot,pstricks,pst-node,pstricks-add,pst-fun,pgfplots} 

% Maths et divers
\usepackage{amsmath,amsfonts,amssymb,amsthm,verbatim}
\usepackage{multicol,enumitem,url,eurosym,gensymb,tabularx}

\DeclareUnicodeCharacter{20AC}{\euro}



% Sections
\usepackage{sectsty} % Allows customizing section commands
\allsectionsfont{\centering \normalfont\scshape}

% Tête et pied de page
\usepackage{fancyhdr} \pagestyle{fancyplain} \fancyhead{} \fancyfoot{}

\renewcommand{\headrulewidth}{0pt} % Remove header underlines
\renewcommand{\footrulewidth}{0pt} % Remove footer underlines

\newcommand{\horrule}[1]{\rule{\linewidth}{#1}} % Create horizontal rule command with 1 argument of height

\newcommand{\Pointilles}[1][3]{%
  \multido{}{#1}{\makebox[\linewidth]{\dotfill}\\[\parskip]
}}

\newtheorem{Definition}{Définition}

\usepackage{siunitx}
\sisetup{
    detect-all,
    output-decimal-marker={,},
    group-minimum-digits = 3,
    group-separator={~},
    number-unit-separator={~},
    inter-unit-product={~}
}

\setlength{\columnseprule}{1pt}

\begin{document}

\textbf{Nom, Prénom :} \hspace{8cm} \textbf{Classe :} \hspace{3cm} \textbf{Date :}\\
\vspace{-0.8cm}
\begin{center}
  \textit{Je n'aime pas le travail, nul ne l'aime ; mais j'aime ce qui est dans le travail l'occasion de se découvrir soi-même.}  - \textbf{Joseph Conrad}
\end{center}
\vspace{-0.8cm}

\subsection*{Exercice 1 - Calculer}

\begin{multicols}{3}\noindent
    \begin{enumerate}
        \item $-50 \div \ldots\ldots = -10$
        \item $\ldots\ldots \div \left( -4\right) = -1$
        \item $\ldots\ldots + 4 = -5$
        \item $\ldots\ldots + \left( -10\right) = -1$
        \item $-40 \div 10 = \ldots\ldots$
        \item $-80 \div \ldots\ldots = 10$
        \item $-3 - 3 = \ldots\ldots$
        \item $\ldots\ldots \times 10 = -60$
        \item $9 - 1 = \ldots\ldots$
        \item $-4 \times 6 = \ldots\ldots$
        \item $\ldots\ldots - \left( -1\right) = -8$
        \item $1 \div 1 = \ldots\ldots$
        \item $-1 \times \left( -4\right) = \ldots\ldots$
        \item $-10 - \left( -7\right) = \ldots\ldots$
        \item $\ldots\ldots - 4 = -3$
        \item $-8 + \ldots\ldots = -12$
        \item $2 \times 7 = \ldots\ldots$
        \item $4 \times \left( -10\right) = \ldots\ldots$
        \item $\ldots\ldots + \left( -4\right) = 0$
        \item $6 + \ldots\ldots = 14$
    \end{enumerate}
  \end{multicols}

\subsection*{Exercice 2 - signe d'un produit}

\begin{enumerate}
    \item[1.] \textbf{Question de cours. Comment peut-on connaître le signe d'un produit ?} \\
    \Pointilles[3]

    \item[2.] \textbf{Positif ou Négatif ? Justifier} 
    \begin{enumerate}
        \item $50 \times 4$ \dotfill
        \item $9 \times (-6) \times 7$ \dotfill
        \item $-2 \times (-7)$ \dotfill
        \item $4 \times 7 \times (-2) \times 5$ \dotfill
        \item $9 \times (-1) \times (-8)$ \dotfill
        \item $ -(-9)$ \dotfill
        \item $9 \times 6 \times (-6) \times (-5)$ \dotfill
        \item $-10 \times (-7) \times -1 \times - 4 \times -3 \times 500$ \dotfill
    \end{enumerate}
\end{enumerate}

\newpage

\textbf{Nom, Prénom :} \hspace{8cm} \textbf{Classe :} \hspace{3cm} \textbf{Date :}\\
\vspace{-0.8cm}
\begin{center}
  \textit{Je n'aime pas le travail, nul ne l'aime ; mais j'aime ce qui est dans le travail l'occasion de se découvrir soi-même.}  - \textbf{Joseph Conrad}
\end{center}
\vspace{-0.8cm}

\subsection*{Exercice 1 - Calculer}

\begin{multicols}{3}\noindent
    \begin{enumerate}
        \item $70 \div 10 = \ldots\ldots$
        \item $10 \times \ldots\ldots = -100$
        \item $-3 \times 9 = \ldots\ldots$
        \item $-1 - \ldots\ldots = 1$
        \item $\ldots\ldots \times 4 = -8$
        \item $4 + \left( -1\right) = \ldots\ldots$
        \item $\ldots\ldots - \left( -8\right) = -2$
        \item $\ldots\ldots \times \left( -6\right) = 48$
        \item $13 - 4 = \ldots\ldots$
        \item $7 \div \left( -7\right) = \ldots\ldots$
        \item $6 \div 3 = \ldots\ldots$
        \item $-1 + 7 = \ldots\ldots$
        \item $1 - \left( -6\right) = \ldots\ldots$
        \item $-8 + 5 = \ldots\ldots$
        \item $\ldots\ldots \times 7 = -42$
        \item $5 - 7 = \ldots\ldots$
        \item $-50 \div \left( -10\right) = \ldots\ldots$
        \item $\ldots\ldots + \left( -9\right) = -14$
        \item $-6 + \left( -4\right) = \ldots\ldots$
        \item $40 \div \ldots\ldots = 4$
    \end{enumerate}
  \end{multicols}

\subsection*{Exercice 2 - signe d'un produit}

\begin{enumerate}
    \item[1.] \textbf{Question de cours. Comment peut-on connaître le signe d'un produit ?} \\
    \Pointilles[3]

    \item[2.] \textbf{Positif ou Négatif ? Justifier} 
    \begin{enumerate}
        \item $50 \times (-4)$ \dotfill
        \item $9 \times 6 \times 7$ \dotfill
        \item $ -(-12)$ \dotfill
        \item $12 \times 6 \times (-6) \times (-5)$ \dotfill
        \item $-3 \times (-7) \times 4$ \dotfill
        \item $-11 \times (-7) \times -10 \times -1 \times 400$ \dotfill
        \item $2 \times -7 \times (-2) \times 5$ \dotfill
        \item $9 \times (-1) \times (-8)$ \dotfill
    \end{enumerate}
\end{enumerate}

\newpage

\textbf{Nom, Prénom :} \hspace{8cm} \textbf{Classe :} \hspace{3cm} \textbf{Date :}\\
\vspace{-0.8cm}
\begin{center}
  \textit{Je n'aime pas le travail, nul ne l'aime ; mais j'aime ce qui est dans le travail l'occasion de se découvrir soi-même.}  - \textbf{Joseph Conrad}
\end{center}
\vspace{-0.8cm}

\subsection*{Exercice 1 - Calculer}

\begin{multicols}{3}\noindent
    \begin{enumerate}
        \item $-10 - \left( -9\right) = \ldots\ldots$
        \item $0 - 4 = \ldots\ldots$
        \item $\ldots\ldots + \left( -4\right) = 0$
        \item $3 + \left( -6\right) = \ldots\ldots$
        \item $-3 \times \left( -8\right) = \ldots\ldots$
        \item $12 \div 6 = \ldots\ldots$
        \item $10 \div 5 = \ldots\ldots$
        \item $-42 \div \left( -6\right) = \ldots\ldots$
        \item $-3 \times \left( -3\right) = \ldots\ldots$
        \item $-8 + \left( -4\right) = \ldots\ldots$
        \item $\ldots\ldots \times \left( -8\right) = 32$
        \item $-3 - \ldots\ldots = 1$
        \item $10 \div \left( -1\right) = \ldots\ldots$
        \item $-8 \times \ldots\ldots = -72$
        \item $10 \times 8 = \ldots\ldots$
        \item $9 \div \ldots\ldots = -9$
        \item $4 + \ldots\ldots = 13$
        \item $5 + \ldots\ldots = -3$
        \item $-5 - \ldots\ldots = -2$
        \item $3 - \left( -5\right) = \ldots\ldots$
    \end{enumerate}
  \end{multicols}

\subsection*{Exercice 2 - signe d'un produit}

\begin{enumerate}
    \item[1.] \textbf{Question de cours. Comment peut-on connaître le signe d'un produit ?} \\
    \Pointilles[3]

    \item[2.] \textbf{Positif ou Négatif ? Justifier} 
    \begin{enumerate}
        \item $50 \times (-4)$ \dotfill
        \item $9 \times 6 \times 7$ \dotfill
        \item $ -(-12)$ \dotfill
        \item $12 \times 6 \times (-6) \times (-5)$ \dotfill
        \item $-3 \times (-7) \times 4$ \dotfill
        \item $-11 \times (-7) \times -10 \times 1 \times 700$ \dotfill
        \item $2 \times -7 \times (-2) \times 5$ \dotfill
        \item $9 \times (-1) \times (-8)$ \dotfill
    \end{enumerate}
\end{enumerate}

\newpage

\textbf{Nom, Prénom :} \hspace{8cm} \textbf{Classe :} \hspace{3cm} \textbf{Date :}\\
\vspace{-0.8cm}
\begin{center}
  \textit{Je n'aime pas le travail, nul ne l'aime ; mais j'aime ce qui est dans le travail l'occasion de se découvrir soi-même.}  - \textbf{Joseph Conrad}
\end{center}
\vspace{-0.8cm}

\subsection*{Exercice 1 - Calculer}

\begin{multicols}{3}\noindent
    \begin{enumerate}
        \item $-6 \times \ldots\ldots = -60$
        \item $-12 \div \left( -6\right) = \ldots\ldots$
        \item $-6 \div \left( -3\right) = \ldots\ldots$
        \item $\ldots\ldots \times 6 = 60$
        \item $6 - 4 = \ldots\ldots$
        \item $8 - \ldots\ldots = 2$
        \item $\ldots\ldots \div 4 = -8$
        \item $9 \times \ldots\ldots = -27$
        \item $5 + 10 = \ldots\ldots$
        \item $-6 + \left( -6\right) = \ldots\ldots$
        \item $\ldots\ldots - \left( -9\right) = 9$
        \item $-42 \div \ldots\ldots = 7$
        \item $6 \times \ldots\ldots = 48$
        \item $-3 \times \ldots\ldots = -6$
        \item $10 + \ldots\ldots = 16$
        \item $-10 + 8 = \ldots\ldots$
        \item $\ldots\ldots + 9 = 3$
        \item $60 \div \left( -6\right) = \ldots\ldots$
        \item $0 - 1 = \ldots\ldots$
        \item $11 - 9 = \ldots\ldots$
    \end{enumerate}
  \end{multicols}

\subsection*{Exercice 2 - signe d'un produit}

\begin{enumerate}
    \item[1.] \textbf{Question de cours. Comment peut-on connaître le signe d'un produit ?} \\
    \Pointilles[3]

    \item[2.] \textbf{Positif ou Négatif ? Justifier} 
    \begin{enumerate}
        \item $-10 \times 10$ \dotfill
        \item $9 \times (-6) \times -7$ \dotfill
        \item $-2 \times (-7)$ \dotfill
        \item $3 \times 7 \times (-2) \times 5$ \dotfill
        \item $8 \times 2 \times (-7)$ \dotfill
        \item $ -(-10)$ \dotfill
        \item $8 \times -5 \times (-10) \times (-5)$ \dotfill
        \item $-11 \times (-7) \times -1 \times -3 \times 200$ \dotfill
    \end{enumerate}
\end{enumerate}

\end{document}
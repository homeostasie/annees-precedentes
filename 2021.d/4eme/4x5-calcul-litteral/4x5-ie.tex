\documentclass[11pt]{article}
\usepackage{geometry,marginnote} % Pour passer au format A4
\geometry{hmargin=1cm, vmargin=1cm} % 

% Page et encodage
\usepackage[T1]{fontenc} % Use 8-bit encoding that has 256 glyphs
\usepackage[english,french]{babel} % Français et anglais
\usepackage[utf8]{inputenc} 

\usepackage{lmodern,numprint}
\setlength\parindent{0pt}

% Graphiques
\usepackage{graphicx,float,grffile,units}
\usepackage{tikz,pst-eucl,pst-plot,pstricks,pst-node,pstricks-add,pst-fun,pgfplots} 

% Maths et divers
\usepackage{amsmath,amsfonts,amssymb,amsthm,verbatim}
\usepackage{multicol,enumitem,url,eurosym,gensymb,tabularx}

\DeclareUnicodeCharacter{20AC}{\euro}



% Sections
\usepackage{sectsty} % Allows customizing section commands
\allsectionsfont{\centering \normalfont\scshape}

% Tête et pied de page
\usepackage{fancyhdr} \pagestyle{fancyplain} \fancyhead{} \fancyfoot{}

\renewcommand{\headrulewidth}{0pt} % Remove header underlines
\renewcommand{\footrulewidth}{0pt} % Remove footer underlines

\newcommand{\horrule}[1]{\rule{\linewidth}{#1}} % Create horizontal rule command with 1 argument of height

\newcommand{\Pointilles}[1][3]{%
  \multido{}{#1}{\makebox[\linewidth]{\dotfill}\\[\parskip]
}}

\newtheorem{Definition}{Définition}

\usepackage{siunitx}
\sisetup{
    detect-all,
    output-decimal-marker={,},
    group-minimum-digits = 3,
    group-separator={~},
    number-unit-separator={~},
    inter-unit-product={~}
}

\setlength{\columnseprule}{1pt}

\begin{document}

\textbf{Nom, Prénom :} \hspace{8cm} \textbf{Classe :} \hspace{3cm} \textbf{Date :}\\

\begin{center}
  \textit{Les mathématiques, science de l’éternel et de l’immuable, sont la science de l’irréel.}  - \textbf{Ernest Renan}
\end{center}

\subsubsection*{Cours}

\begin{enumerate}
	\item[1.] Sens du signe = : \dotfill 
	\item[2.] Sens du x :  \dotfill  
\end{enumerate}

\subsubsection*{Ex1 : Calculer}

On pose $a = 15$, $b = -2$ et $c = 0.2$.

\begin{enumerate}
  \item[a.] $a + 10$ = \dotfill 
  \item[b.] $a \times c + 25$ = \dotfill 
  \item[c.] $42 + 8 \times a$ = \dotfill 
  \item[d.] $(a + b)\times c$ = \dotfill 
  \item[e.] $2 \times b \times 5$ = \dotfill 
  \item[f.] $(2a + 10b)^5 - 100c$ = \dotfill 
\end{enumerate}


\subsubsection*{Ex2 : Démontrer}

Démontrer que : $2x + 3x = 5x$.
\Pointilles[2]

\subsubsection*{Ex3 : Réduire}

\begin{enumerate}
  \item[g.] $12x + 20x$ = \dotfill 
  \item[h.] $4x + 6 + 40x$ = \dotfill 
  \item[i.] $8x - x$ = \dotfill 
  \item[j.] $4x - 3x + 12$ = \dotfill 
  \item[k.] $4 \times x \times 10$ = \dotfill 
  \item[l.] $7 \times x \times 0 + 13$ = \dotfill 
\end{enumerate}

\subsubsection*{Équations : Résoudre}
\textbf{Écrire les étapes}

\begin{itemize}[label={$\bullet$}]
\item $x + 14 = 250$
\item $x - 24 = 34$
\item $6 \times x = 49$
\item $404 = 5x$
\item $2x + 3 = 32$
\item $8x - 204 = -32$
\item $-24 - 2x = -48$
\item $\SI{54000}{} = \SI{37000}{} + 540x$
\end{itemize}

\newpage 


\textbf{Nom, Prénom :} \hspace{8cm} \textbf{Classe :} \hspace{3cm} \textbf{Date :}\\

\begin{center}
  \textit{Les mathématiques, science de l’éternel et de l’immuable, sont la science de l’irréel.}  - \textbf{Ernest Renan}
\end{center}

\subsubsection*{Cours}

\begin{enumerate}
	\item[1.] Sens du signe = : \dotfill 
	\item[2.] Sens du x :  \dotfill  
\end{enumerate}

\subsubsection*{Ex1 : Calculer}

On pose $a = 25$, $b = -4$ et $c = 0.4$.

\begin{enumerate}
  \item[a.] $a + 10$ = \dotfill 
  \item[b.] $a \times c + 25$ = \dotfill 
  \item[c.] $42 + 8 \times a$ = \dotfill 
  \item[d.] $(a + b)\times c$ = \dotfill 
  \item[e.] $2 \times b \times 5$ = \dotfill 
  \item[f.] $(2a + 10b)^5 - 100c$ = \dotfill 
\end{enumerate}


\subsubsection*{Ex2 : Démontrer}

Démontrer que : $4x + 2x = 6x$.
\Pointilles[2]

\subsubsection*{Ex3 : Réduire}

\begin{enumerate}
  \item[g.] $10x + 25x$ = \dotfill 
  \item[h.] $3x + 7 + 30x$ = \dotfill 
  \item[i.] $9x - x$ = \dotfill 
  \item[j.] $6x - 5x + 11$ = \dotfill 
  \item[k.] $10 \times x \times 6$ = \dotfill 
  \item[l.] $9 \times x \times 0 + 11$ = \dotfill 
\end{enumerate}

\subsubsection*{Équations : Résoudre}
\textbf{Écrire les étapes}

\begin{itemize}[label={$\bullet$}]
\item $x + 12 = 140$
\item $x - 16 = 48$
\item $4 \times x = 37$
\item $604 = 7x$
\item $4x + 6 = 42$
\item $6x - 204 = -52$
\item $-12 - 4x = -46$
\item $\SI{64000}{} = \SI{47000}{} + 580x$
\end{itemize}

\end{document}
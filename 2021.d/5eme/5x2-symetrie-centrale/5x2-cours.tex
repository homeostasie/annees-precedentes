\input{../doc-class-cours.tex}

\begin{document}

\horrule{2px}
\section*{Chapitre 2 - Symétrie}
\horrule{2px}

\section*{La droite et le cercle}

\begin{Definition}{Droite parallèles}\\
  Deux droites sont parallèles si la distance entre celles-ci ne change jamais. 
\end{Definition}

\begin{Definition}{Cercle}\\
  Un cercle est  un ensemble des points situés à une même distance d'un centre. On appelle cette distance le rayon.
\end{Definition}

\section*{Sur quadrillage}


\section*{Symétrie centrale}

\begin{Definition}{Symétrique d'un point}\\
  \label{def:ch2-1.sympt}
  $M^{'}$ est le symétrique du point M par rapport au centre O est équivalent à O est le milieu de de [M $M^{'}$].
\end{Definition}

\end{document}

\documentclass[11pt]{article}
\usepackage{geometry,marginnote} % Pour passer au format A4
\geometry{hmargin=1cm, vmargin=1cm} % 

% Page et encodage
\usepackage[T1]{fontenc} % Use 8-bit encoding that has 256 glyphs
\usepackage[english,french]{babel} % Français et anglais
\usepackage[utf8]{inputenc} 

\usepackage{lmodern,numprint}
\setlength\parindent{0pt}

% Graphiques
\usepackage{graphicx,float,grffile,units}
\usepackage{tikz,pst-eucl,pst-plot,pstricks,pst-node,pstricks-add,pst-fun,pgfplots} 

% Maths et divers
\usepackage{amsmath,amsfonts,amssymb,amsthm,verbatim}
\usepackage{multicol,enumitem,url,eurosym,gensymb,tabularx}

\DeclareUnicodeCharacter{20AC}{\euro}



% Sections
\usepackage{sectsty} % Allows customizing section commands
\allsectionsfont{\centering \normalfont\scshape}

% Tête et pied de page
\usepackage{fancyhdr} \pagestyle{fancyplain} \fancyhead{} \fancyfoot{}

\renewcommand{\headrulewidth}{0pt} % Remove header underlines
\renewcommand{\footrulewidth}{0pt} % Remove footer underlines

\newcommand{\horrule}[1]{\rule{\linewidth}{#1}} % Create horizontal rule command with 1 argument of height

\newcommand{\Pointilles}[1][3]{%
  \multido{}{#1}{\makebox[\linewidth]{\dotfill}\\[\parskip]
}}

\newtheorem{Definition}{Définition}

\usepackage{siunitx}
\sisetup{
    detect-all,
    output-decimal-marker={,},
    group-minimum-digits = 3,
    group-separator={~},
    number-unit-separator={~},
    inter-unit-product={~}
}

\setlength{\columnseprule}{1pt}

\begin{document}

\section*{Programme de l’éval : Calcul Littéral}

\textbf{Date : } \hspace{4cm} \textit{Calculatrice autorisée}

\subsection*{Cours - (2pts)}


\begin{itemize}[label={$\bullet$}]
  \item Sens du signe = : Ce qui est à gauche = Ce qui est à droite. 
  \item $x ?$ : Le nombre x est inconnu et solution de l’équation. 
\end{itemize}


\subsection*{Exercices Types}


\subsubsection*{Ex1. Calculer  - (3pts)}

On pose $x = 12$ et $b = 2,5$
\begin{itemize}[label={$\bullet$}]
  \item  $x + 2 - b = 11,5$
  \item  $5 \times x + 14 = 74$
  \item  $(2x - 4b) \times 4 = 136$
\end{itemize}

\subsubsection*{Ex2. Remplacer les $\times$ par des $+$ - (3pts)}

\begin{itemize}[label={$\bullet$}]
  \item  $2x + 3x = x+x + x+x+x$
  \item  $1 + 2x = 1 + x + x$        
\end{itemize}

\subsubsection*{Ex3. Réduire - (3pts)}

\begin{itemize}[label={$\bullet$}]
  \item  $2x + 3x = 5x$
  \item  $4x - x + 10 = 3x + 10$
\end{itemize}

\subsection*{Mise en équations - (9pts)}

\begin{itemize}[label={$\bullet$}]
  \item  Soit $x$ l’inconnu et la solution de mon problème.
  \item  Écrire l’égalité du problème.
\end{itemize}

Ex : Rania fait de la course à pied. Elle commence par courir 300m pour aller au stade. Elle fait 4 tours de terrain puis s’arrête. Son portable lui signale qu’elle a couru 3400m. Quelle est la longueur d’un tour de terrain ?

\begin{itemize}[label={$\bullet$}]
  \item  Soit $x$ la longueur d’un tour de terrain.
  \item  $300 + 4 \times x = 3400$
\end{itemize}

\end{document}
\documentclass[11pt]{article}
\usepackage{geometry,marginnote} % Pour passer au format A4
\geometry{hmargin=1cm, vmargin=1cm} % 

% Page et encodage
\usepackage[T1]{fontenc} % Use 8-bit encoding that has 256 glyphs
\usepackage[english,french]{babel} % Français et anglais
\usepackage[utf8]{inputenc} 

\usepackage{lmodern,numprint}
\setlength\parindent{0pt}

% Graphiques
\usepackage{graphicx,float,grffile,units}
\usepackage{tikz,pst-eucl,pst-plot,pstricks,pst-node,pstricks-add,pst-fun,pgfplots} 

% Maths et divers
\usepackage{amsmath,amsfonts,amssymb,amsthm,verbatim}
\usepackage{multicol,enumitem,url,eurosym,gensymb,tabularx}

\DeclareUnicodeCharacter{20AC}{\euro}



% Sections
\usepackage{sectsty} % Allows customizing section commands
\allsectionsfont{\centering \normalfont\scshape}

% Tête et pied de page
\usepackage{fancyhdr} \pagestyle{fancyplain} \fancyhead{} \fancyfoot{}

\renewcommand{\headrulewidth}{0pt} % Remove header underlines
\renewcommand{\footrulewidth}{0pt} % Remove footer underlines

\newcommand{\horrule}[1]{\rule{\linewidth}{#1}} % Create horizontal rule command with 1 argument of height

\newcommand{\Pointilles}[1][3]{%
  \multido{}{#1}{\makebox[\linewidth]{\dotfill}\\[\parskip]
}}

\newtheorem{Definition}{Définition}

\usepackage{siunitx}
\sisetup{
    detect-all,
    output-decimal-marker={,},
    group-minimum-digits = 3,
    group-separator={~},
    number-unit-separator={~},
    inter-unit-product={~}
}

\setlength{\columnseprule}{1pt}

\begin{document}

\setlength{\columnseprule}{0pt}

\horrule{2px}
\section*{Chapitre 7 - Fractions}
\horrule{2px}

\section*{Les fractions}

\textbf{À Retenir : }

\textbf{Une fraction est un partage et un nombre. }

\begin{itemize}
  \item $\dfrac{1}{5}$ est le partage d'une unité en 5.
  \item $5 \times \dfrac{1}{5} = 1$ : $\dfrac{1}{5}$ est le nombre qui multiplier par 5 donne 1.
\end{itemize}

Dans la fraction $\dfrac{4}{5}$, on a "quatre cinquième".

$$\dfrac{4}{5} = 4 \times \dfrac{1}{5} = \dfrac{1}{5} + \dfrac{1}{5} + \dfrac{1}{5} + \dfrac{1}{5}$$

\section*{La multiplication}

\textbf{Démonstration : }

\begin{flalign*}
  3 \times \dfrac{2}{5} &= \dfrac{2}{5} + \dfrac{2}{5} + \dfrac{2}{5} \\ 
                        &= \dfrac{1}{5} + \dfrac{1}{5} + \dfrac{1}{5} + \dfrac{1}{5} + \dfrac{1}{5} + \dfrac{1}{5} \\
                        &= 6 \times \dfrac{1}{5} \\
                        &= \dfrac{6}{5}
\end{flalign*}

\textbf{À Retenir : }

$$ 3 \times \dfrac{2}{5} = \dfrac{3 \times 2}{5} = \dfrac{6}{5} $$

\section*{L'addition}

\textbf{Démonstration : }

\begin{flalign*}
  \dfrac{4}{5} + \dfrac{2}{5} &= \dfrac{1}{5} + \dfrac{1}{5} + \dfrac{1}{5} + \dfrac{1}{5} + \dfrac{1}{5} + \dfrac{1}{5}\\
                              &= 6 \times \dfrac{1}{5} \\
                              &= \dfrac{6}{5}
\end{flalign*}

\textbf{À Retenir : }

$$ \dfrac{4}{5} + \dfrac{2}{5} = \dfrac{4+2}{5}  = \dfrac{6}{5}$$

\textbf{On ne peut pas additionner deux fractions qui n'ont pas le même dénominateur.}

\section*{Les fractions égales}

\textbf{Fractions égales} : Deux fractions égales donnent le même nombre. 

$$\dfrac{1}{2} = \dfrac{2}{4} = \dfrac{3}{6} = \dfrac{250}{500} = \dfrac{400}{800} = ... = 0,5$$

\textbf{On peux avoir à changer le dénominateur d'une fraction pour faire certaines opérations.}

\textbf{À Retenir : }
Deux fractions sont égales si on peut passer de l'une à l'autre en multipliant le numérateur et le dénominateur par un même nombre.

$$\dfrac{1}{3} = \dfrac{1 \times 7}{3 \times 7} = \dfrac{7}{21}$$

\end{document}
\documentclass[11pt]{article}
\usepackage{geometry,marginnote} % Pour passer au format A4
\geometry{hmargin=1cm, vmargin=1cm} % 

% Page et encodage
\usepackage[T1]{fontenc} % Use 8-bit encoding that has 256 glyphs
\usepackage[english,french]{babel} % Français et anglais
\usepackage[utf8]{inputenc} 

\usepackage{lmodern,numprint}
\setlength\parindent{0pt}

% Graphiques
\usepackage{graphicx,float,grffile,units}
\usepackage{tikz,pst-eucl,pst-plot,pstricks,pst-node,pstricks-add,pst-fun,pgfplots} 

% Maths et divers
\usepackage{amsmath,amsfonts,amssymb,amsthm,verbatim}
\usepackage{multicol,enumitem,url,eurosym,gensymb,tabularx}

\DeclareUnicodeCharacter{20AC}{\euro}



% Sections
\usepackage{sectsty} % Allows customizing section commands
\allsectionsfont{\centering \normalfont\scshape}

% Tête et pied de page
\usepackage{fancyhdr} \pagestyle{fancyplain} \fancyhead{} \fancyfoot{}

\renewcommand{\headrulewidth}{0pt} % Remove header underlines
\renewcommand{\footrulewidth}{0pt} % Remove footer underlines

\newcommand{\horrule}[1]{\rule{\linewidth}{#1}} % Create horizontal rule command with 1 argument of height

\newcommand{\Pointilles}[1][3]{%
  \multido{}{#1}{\makebox[\linewidth]{\dotfill}\\[\parskip]
}}

\newtheorem{Definition}{Définition}

\usepackage{siunitx}
\sisetup{
    detect-all,
    output-decimal-marker={,},
    group-minimum-digits = 3,
    group-separator={~},
    number-unit-separator={~},
    inter-unit-product={~}
}

\setlength{\columnseprule}{1pt}

\begin{document}

\textbf{Nom, Prénom :} \hspace{8cm} \textbf{Classe :} \hspace{3cm} \textbf{Date :}\\
\vspace{-0.8cm}
\begin{center}
  \textit{La normalité est une route pavée : on y marche aisément mais les fleurs n’y poussent pas.} - \textbf{Vincent Van Gogh}
\end{center}

\textbf{Ex1 - Pièces}

\begin{multicols}{2}\noindent 

Chaque pièce n'a que 2 côtés : Pile (\textbf{P}) et Face (\textbf{F}).

\begin{enumerate}
  \item[1a.] On jette 2 pièces. Écrire toutes les possibilités. \\ \Pointilles[2]  \columnbreak 
  \item[1b.] On jette 3 pièces. Écrire toutes les possibilités. \\ \Pointilles[4]
\end{enumerate}  \end{multicols} 
\begin{enumerate}
  \item[1c.] On jette 20 pièces. Combien peut-on compter d'élément dans l'univers ? \\ \Pointilles[4]
\end{enumerate}  

\textbf{Ex2 - Dés}

\begin{multicols}{2}\noindent
On a deux dés équilibrés à 6 faces. 

\begin{itemize}[label={$\bullet$}]
  \item Dé 1 : 2, 2, 4, 4, 6, 6
  \item Dé 2 : 1, 1, 3, 3, 5, 5 
\end{itemize} 

\textbf{On additionne les deux faces.} 
\begin{enumerate}
  \item[2a.] Remplir le tableau des réalisations.
\end{enumerate}  
\columnbreak 

\begin{center}\begin{tabular}{|c|c|c|c|c|c|c|} \hline
  dés & 2 & 2 & 4 & 4 & 6 & 6 \\  \hline
    1 &   &   &   &   &   &   \\  \hline
    1 &   &   &   &   &   &   \\  \hline
    3 &   &   &   &   &   &   \\  \hline
    3 &   &   &   &   &   &   \\  \hline
    5 &   &   &   &   &   &   \\  \hline
    5 &   &   &   &   &   &   \\  \hline
\end{tabular}\end{center}

\end{multicols}

\begin{enumerate}
  \item[2b.] Quel est l'univers ? \\ \Pointilles[2] 
  \item[2c.] Quelle est la probabilité de faire 6 ? \\ \Pointilles[2] 
  \item[2d.] Quelle est la probabilité de faire plus grand ou égal à 7 ? \\ \Pointilles[3]  
\end{enumerate}  

\textbf{Ex3 - Dé}

On a un dé équilibré à 6 faces : 1, 1, 2, 2, 3, 4.

\begin{enumerate}
  \item[3a.] Quel est l'univers ? \\ \Pointilles[2]
  \item[3b.] Quels sont les probabilités associées à chaque élément ? \\ \Pointilles[3]
\end{enumerate}  

\textbf{Ex4 - Dé et Urnes}

On a un dé équilibrés à 6 faces : 1, 1, 6, 6, 6, 6. On lance le dé. 

\begin{multicols}{2}\noindent
\begin{itemize}[label={$\bullet$}]
  \item Si on obtient 1 avec le dé, on pioche une balle dans l'urne rouge.
  Urne rouge : 14 balles rouges et 10 balles noires.

  \item Si on obtient 6 avec le dé, on pioche une balle dans l'urne verte.
  Urne verte : 8 balles vertes et 12 balles noires.
\end{itemize}
\end{multicols}

\begin{enumerate}
  \item[4a.] Quel est l'univers ? \\ \Pointilles[2] 
  \item[4b.] On fait 6 avec le dé. Quel est la probabilité de piocher une balle noire ? \\ \Pointilles[2] 
  \item[4c.] Quel est la probabilité de piocher dans l'urne verte ? \\ \Pointilles[2] 
  \item[4d.] On sait que la probabilité de piocher une balle noire est $\dfrac{2}{6} \times \dfrac{10}{24} + \dfrac{4}{6} \times \dfrac{12}{20}$. \\
  Calculer cette probabilité. A-t-on plus d'une chance sur deux de tirer une balle noire ? \\ \Pointilles[3] 
\end{enumerate}

\textbf{Ex5 - Pierre, Feuille, Ciseaux}

Dans le jeu pierre–feuille–ciseaux, deux joueurs choisissent en même temps l’un des trois coups suivants : \textbf{Pierre, Feuille ou Ciseaux}.

\begin{itemize}[label={$\bullet$}]
  \item La pierre bat les ciseaux.
  \item Les ciseaux battent la feuille. 
  \item La feuille bat la pierre.
  \item Il y a match nul si les deux joueurs choisissent le même coup.
\end{itemize}

\begin{enumerate}
  \item[5a.] Je joue \textbf{Pierre}. Quel est la probabilité de gagner ? \\ \Pointilles[2] 
  \item[5b.] Je gagne la première partie. Quelle est la probabilité que je gagne la deuxième ? \\ \Pointilles[4] 
  \item[5c.] Quelle est la probabilité de gagner deux parties de suite ? \\ \Pointilles[4] 
\end{enumerate}

\newpage


\textbf{Nom, Prénom :} \hspace{8cm} \textbf{Classe :} \hspace{3cm} \textbf{Date :}\\
\vspace{-0.8cm}
\begin{center}
  \textit{La normalité est une route pavée : on y marche aisément mais les fleurs n’y poussent pas.} - \textbf{Vincent Van Gogh}
\end{center}

\textbf{Ex1 - Pièces}

\begin{multicols}{2}\noindent 

Chaque pièce n'a que 2 côtés : Pile (\textbf{P}) et Face (\textbf{F}).

\begin{enumerate}
  \item[1a.] On jette 2 pièces. Écrire toutes les possibilités. \\ \Pointilles[2]  \columnbreak 
  \item[1b.] On jette 3 pièces. Écrire toutes les possibilités. \\ \Pointilles[4]
\end{enumerate}  \end{multicols} 
\begin{enumerate}
  \item[1c.] On jette 12 pièces. Combien peut-on compter d'élément dans l'univers ? \\ \Pointilles[4]
\end{enumerate}  

\textbf{Ex2 - Dés}

\begin{multicols}{2}\noindent
On a deux dés équilibrés à 6 faces. 

\begin{itemize}[label={$\bullet$}]
  \item Dé 1 : 2, 2, 5, 5, 8, 8
  \item Dé 2 : 1, 1, 4, 4, 6, 6 
\end{itemize} 

\textbf{On additionne les deux faces.} 
\begin{enumerate}
  \item[2a.] Remplir le tableau des réalisations.
\end{enumerate}  
\columnbreak 

\begin{center}\begin{tabular}{|c|c|c|c|c|c|c|} \hline
  dés & 2 & 2 & 5 & 5 & 8 & 8 \\  \hline
    1 &   &   &   &   &   &   \\  \hline
    1 &   &   &   &   &   &   \\  \hline
    4 &   &   &   &   &   &   \\  \hline
    4 &   &   &   &   &   &   \\  \hline
    6 &   &   &   &   &   &   \\  \hline
    6 &   &   &   &   &   &   \\  \hline
\end{tabular}\end{center}

\end{multicols}

\begin{enumerate}
  \item[2b.] Quel est l'univers ? \\ \Pointilles[2] 
  \item[2c.] Quelle est la probabilité de faire 6 ? \\ \Pointilles[2] 
  \item[2d.] Quelle est la probabilité de faire plus grand ou égal à 9 ? \\ \Pointilles[3]  
\end{enumerate}  

\textbf{Ex3 - Dé}

On a un dé équilibré à 6 faces : 1, 1, 1, 2, 2, 6.

\begin{enumerate}
  \item[3a.] Quel est l'univers ? \\ \Pointilles[2]
  \item[3b.] Quels sont les probabilités associées à chaque élément ? \\ \Pointilles[3]
\end{enumerate}  

\textbf{Ex4 - Dé et Urnes}

On a un dé équilibrés à 6 faces : 1, 1, 6, 6, 6, 6. On lance le dé. 

\begin{multicols}{2}\noindent
\begin{itemize}[label={$\bullet$}]
  \item Si on obtient 1 avec le dé, on pioche une balle dans l'urne rouge.
  Urne rouge : 16 balles rouges et 12 balles noires.

  \item Si on obtient 6 avec le dé, on pioche une balle dans l'urne verte.
  Urne verte : 6 balles vertes et 14 balles noires.
\end{itemize}
\end{multicols}

\begin{enumerate}
  \item[4a.] Quel est l'univers ? \\ \Pointilles[2] 
  \item[4b.] On fait 6 avec le dé. Quel est la probabilité de piocher une balle noire ? \\ \Pointilles[2] 
  \item[4c.] Quel est la probabilité de piocher dans l'urne verte ? \\ \Pointilles[2] 
  \item[4d.] On sait que la probabilité de piocher une balle noire est $\dfrac{2}{6} \times \dfrac{12}{28} + \dfrac{4}{6} \times \dfrac{14}{20}$. \\
  Calculer cette probabilité. A-t-on plus d'une chance sur deux de tirer une balle noire ? \\ \Pointilles[3] 
\end{enumerate}

\textbf{Ex5 - Pierre, Feuille, Ciseaux}

Dans le jeu pierre–feuille–ciseaux, deux joueurs choisissent en même temps l’un des trois coups suivants : \textbf{Pierre, Feuille ou Ciseaux}.

\begin{itemize}[label={$\bullet$}]
  \item La pierre bat les ciseaux.
  \item Les ciseaux battent la feuille. 
  \item La feuille bat la pierre.
  \item Il y a match nul si les deux joueurs choisissent le même coup.
\end{itemize}

\begin{enumerate}
  \item[5a.] Je joue \textbf{Feuille}. Quel est la probabilité de gagner ? \\ \Pointilles[2] 
  \item[5b.] Je gagne la première partie. Quelle est la probabilité que je gagne la deuxième ? \\ \Pointilles[4] 
  \item[5c.] Quelle est la probabilité de gagner deux parties de suite ? \\ \Pointilles[4] 
\end{enumerate}

\end{document}
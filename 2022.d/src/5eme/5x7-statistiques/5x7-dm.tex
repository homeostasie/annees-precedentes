\documentclass[11pt]{article}
\usepackage{geometry,marginnote} % Pour passer au format A4
\geometry{hmargin=1cm, vmargin=1cm} % 

% Page et encodage
\usepackage[T1]{fontenc} % Use 8-bit encoding that has 256 glyphs
\usepackage[english,french]{babel} % Français et anglais
\usepackage[utf8]{inputenc} 

\usepackage{lmodern,numprint}
\setlength\parindent{0pt}

% Graphiques
\usepackage{graphicx,float,grffile,units}
\usepackage{tikz,pst-eucl,pst-plot,pstricks,pst-node,pstricks-add,pst-fun,pgfplots} 

% Maths et divers
\usepackage{amsmath,amsfonts,amssymb,amsthm,verbatim}
\usepackage{multicol,enumitem,url,eurosym,gensymb,tabularx}

\DeclareUnicodeCharacter{20AC}{\euro}



% Sections
\usepackage{sectsty} % Allows customizing section commands
\allsectionsfont{\centering \normalfont\scshape}

% Tête et pied de page
\usepackage{fancyhdr} \pagestyle{fancyplain} \fancyhead{} \fancyfoot{}

\renewcommand{\headrulewidth}{0pt} % Remove header underlines
\renewcommand{\footrulewidth}{0pt} % Remove footer underlines

\newcommand{\horrule}[1]{\rule{\linewidth}{#1}} % Create horizontal rule command with 1 argument of height

\newcommand{\Pointilles}[1][3]{%
  \multido{}{#1}{\makebox[\linewidth]{\dotfill}\\[\parskip]
}}

\newtheorem{Definition}{Définition}

\usepackage{siunitx}
\sisetup{
    detect-all,
    output-decimal-marker={,},
    group-minimum-digits = 3,
    group-separator={~},
    number-unit-separator={~},
    inter-unit-product={~}
}

\setlength{\columnseprule}{1pt}

\begin{document}

\textbf{Nom, Prénom :} \hspace{8cm} \textbf{Classe :} \hspace{3cm} \textbf{Date :}\\

\vspace{-0.5cm} \begin{center}
  \textit{L'intelligence est insipide sans altruisme.}  - \textbf{Michel Bouthot}
\end{center}

\subsection*{Ex 1 - Calculer les moyennes}

\begin{enumerate}
  \item[1a.] 8 ; 12 ; 18 ; 16 ; 4 ; 10 ; 11 ; 16 \\ \Pointilles[4]
  \item[1b.] 120 ; 250 ; 45 ; 342 ; 5600 ; 8 \\ \Pointilles[4]
  \item[1c.] -12 ; 42 ; -250 ; 130 ; 37 ; 0 ; 32 \\ \Pointilles[4]
\end{enumerate} 


\begin{multicols}{2}

  \subsection*{Ex 2 - Calculer les moyennes pondérées}

  \begin{enumerate}
  \item[2a.] On cherche à calculer la moyenne de Jean Neymar. Ses notes :\\
  \begin{itemize}[label={$\bullet$}]
    \item 8 coefficient 2
    \item 12 coefficient 1
    \item 14 coefficient 4
    \item 9 coefficient 2
  \end{itemize} 

  \item[2b.]  On cherche à calculer la moyenne d'âge des élèves de Faubert. \\
  \begin{itemize}[label={$\bullet$}]
    \item 12 élèves ont 10 ans
    \item 22 élèves ont 11 ans
    \item 49 élèves ont 12 ans
    \item 36 élèves ont 13 ans
    \item 52 élèves ont 14 ans
    \item 18 élèves ont 15 ans
\end{itemize} 

\end{enumerate} 

\columnbreak

\Pointilles[26]

\end{multicols}

\newpage

\begin{multicols}{2}

\subsection*{Pb1 - Les notes}

L'élève Manon Messi cherche à calculer ses notes pour le troisième trimestre.

\begin{itemize}[label={$\bullet$}]
  \item 13 coefficient 2
  \item 14 coefficient 4
  \item 12 coefficient 2
  \item 16 coefficient 1
\end{itemize} 

\begin{enumerate}
  \item[3a.] Démontrer que sa moyenne est de 13,5.
  \item[3b.] Comme ses copains, elle jette ses copies et ne retient que sa moyenne. \\
  Peut-elle retrouver exactement ses notes à partir de sa moyenne ? 
  \item[3c.] Il reste une dernière évaluation coefficient 1 mais Manon Messi veut faire le moins d'effort possible. \\
  Quelle est la plus basse note lui permettant d'obtenir 14 de moyenne sur le trimestre ?
\end{enumerate} 

\columnbreak

\Pointilles[18]

\end{multicols}

\subsection*{Pb2 - Le match}

Pour rien au monde, vous n'auriez loupé le classico : Barcelone vs Réal de Madrid. Mais malheureusement, votre professeur de maths vous a donné un DM. Vos parents vous autorisent à regarder le match une fois le DM fini. Trop lent pour finir votre DM, vous devez vous contenter de regarder les statistiques du match. 

\begin{center}
  \begin{tabular}{|c|c|c|} \hline
                        & BARCELONE & REAL DE MADRID \\  \hline
    Buts                &         1 &              4 \\  \hline
    Tirs cadrés         &        16 & 8 \\  \hline
    Tirs sur le poteau  &         5 & 1  \\  \hline
    Possession de balle &     70 \% & 30 \%  \\ \hline
    Fautes commises     &        13 & 5 \\  \hline
    Carton jaune        &         5 & 5 \\  \hline
    Remplacements       &         3 & 1 \\  \hline
  \end{tabular}
\end{center}

\begin{enumerate}
  \item[4a.] Les  différentes lignes : But, Tirs, Possession de balle, Fautes,... sont des \dotfill
  \item[4b.] Recopier la définition : \og Les indicateurs sont des méthodes numériques qui permettent d’interpréter et de comparer les données. \fg \dotfill \\ \Pointilles[1]
  \item[4c.] À partir des statistiques, faire un résumé en 5/6 lignes du match.
  \textit{Quelques astuces : Commencer par donner les équipes et le score ; La possession de balle donne une information sur le type de jeu ; Le nombre de fautes également ; Le nombre de tirs cadrés également}

  \Pointilles[10]
\end{enumerate} 
\end{document}

\documentclass[11pt]{article}
\usepackage{geometry,marginnote} % Pour passer au format A4
\geometry{hmargin=1cm, vmargin=1cm} % 

% Page et encodage
\usepackage[T1]{fontenc} % Use 8-bit encoding that has 256 glyphs
\usepackage[english,french]{babel} % Français et anglais
\usepackage[utf8]{inputenc} 

\usepackage{lmodern,numprint}
\setlength\parindent{0pt}

% Graphiques
\usepackage{graphicx,float,grffile,units}
\usepackage{tikz,pst-eucl,pst-plot,pstricks,pst-node,pstricks-add,pst-fun,pgfplots} 

% Maths et divers
\usepackage{amsmath,amsfonts,amssymb,amsthm,verbatim}
\usepackage{multicol,enumitem,url,eurosym,gensymb,tabularx}

\DeclareUnicodeCharacter{20AC}{\euro}



% Sections
\usepackage{sectsty} % Allows customizing section commands
\allsectionsfont{\centering \normalfont\scshape}

% Tête et pied de page
\usepackage{fancyhdr} \pagestyle{fancyplain} \fancyhead{} \fancyfoot{}

\renewcommand{\headrulewidth}{0pt} % Remove header underlines
\renewcommand{\footrulewidth}{0pt} % Remove footer underlines

\newcommand{\horrule}[1]{\rule{\linewidth}{#1}} % Create horizontal rule command with 1 argument of height

\newcommand{\Pointilles}[1][3]{%
  \multido{}{#1}{\makebox[\linewidth]{\dotfill}\\[\parskip]
}}

\newtheorem{Definition}{Définition}

\usepackage{siunitx}
\sisetup{
    detect-all,
    output-decimal-marker={,},
    group-minimum-digits = 3,
    group-separator={~},
    number-unit-separator={~},
    inter-unit-product={~}
}

\setlength{\columnseprule}{1pt}

\begin{document}

\textbf{Nom, Prénom :} \hspace{8cm} \textbf{Classe :} \hspace{3cm} \textbf{Date :}\\

\begin{center}
  \textit{La vie c’est comme une bicyclette, il faut avancer pour ne pas perdre l’équilibre.} - \textbf{Albert Einstein}
\end{center}


\subsection*{Ex1 - Définir un pourcentage}

\textit{Donner l'écriture fractionnaire et l'écriture décimale.}

\begin{multicols}{2}
\begin{itemize}[label={$\bullet$}]
  \item 55 \% = \dotfill \\ \Pointilles[1]
  \item 34 \% = \dotfill \\ \Pointilles[1] \columnbreak 
  \item 12 \% = \dotfill \\ \Pointilles[1]
  \item  5 \% = \dotfill \\ \Pointilles[1]
\end{itemize} 
\end{multicols}

\subsection*{Ex2 - Appliquer un pourcentage}

\textit{Écrire le calcul et le résultat.}

\begin{multicols}{2}
  \begin{itemize}[label={$\bullet$}]
    \item 55 \% de 1250 = \dotfill \\ \Pointilles[1]
    \item 34 \% de 134 = \dotfill \\ \Pointilles[1] \columnbreak 
    \item 12 \% de 60 = \dotfill \\ \Pointilles[1]
    \item  5 \% de 1110 = \dotfill \\ \Pointilles[1]
  \end{itemize} 
  \end{multicols}

\subsection*{Ex3 - Calculer un pourcentage}

\textit{Écrire le calcul et le résultat.}
  
\begin{multicols}{2}
  \begin{itemize}[label={$\bullet$}]
    \item 144 par rapport à 576 : \dotfill \\ \Pointilles[1]
    \item 25 par rapport à 475 : \dotfill \\ \Pointilles[1] \columnbreak 
    \item 132 par rapport à 190 : \dotfill \\ \Pointilles[1]
    \item 17 par rapport à 170 : \dotfill \\ \Pointilles[1]
  \end{itemize} 
\end{multicols}

\subsection*{Ex4 - Compléter un pourcentage}

\textit{Quel pourcentage reste-t-il pour le bleu ?}

\begin{multicols}{2}
\begin{itemize}[label={$\bullet$}]
  \item 40 \% est vert.
  \item 20 \% est jaune.
  \item 17 \% est rouge.
  \item Le reste est bleu. 
\end{itemize} \columnbreak 
\Pointilles[6]
\end{multicols}

\newpage

\subsection*{problème 1}

Une pastèque est composée à 70\% d'eau. Une pastèque pèse 1250g. Quelle quantité d'eau est contenue dans une pastèque ? \\

\Pointilles[4]

\subsection*{problème 2}

Pour obtenir la couleur souhaitée, Jasmine doit mélanger : 

\begin{itemize}[label={$\bullet$}]
  \item 75 \% de vert.
  \item 10 \% de jaune.
  \item Le reste de blanc.
\end{itemize}

Quelles sont les quantités nécessaires de vert, jaune et blanc pour faire 20L de peinture  ? \\
\Pointilles[8]

\subsection*{problème 3}

Le collège Faubert est composé de 254 garçons et de 276 filles. Quelles sont les pourcentages de garçon et de fille au collège Faubert ? \\

\Pointilles[4]

\subsection*{problème 4}

Un élève très attentif pendant le cours de mathématiques compte le nombre de véhicules qu'il voit passer. Il compte 150 véhicules : 

\begin{multicols}{3}
\begin{itemize}[label={$\bullet$}]
  \item 75 voitures.
  \item 36 camions.
  \item 22 vélo.
  \item 12 trottinettes.
  \item 5 moto.
\end{itemize}
\end{multicols}

Quell est le pourcentage de vélo ? \\

\Pointilles[4]

\newpage


\textbf{Nom, Prénom :} \hspace{8cm} \textbf{Classe :} \hspace{3cm} \textbf{Date :}\\

\begin{center}
  \textit{La vie c’est comme une bicyclette, il faut avancer pour ne pas perdre l’équilibre.} - \textbf{Albert Einstein}
\end{center}


\subsection*{Ex1 - Définir un pourcentage}

\textit{Donner l'écriture fractionnaire et l'écriture décimale.}

\begin{multicols}{2}
\begin{itemize}[label={$\bullet$}]
  \item 65 \% = \dotfill \\ \Pointilles[1]
  \item 24 \% = \dotfill \\ \Pointilles[1] \columnbreak 
  \item 18 \% = \dotfill \\ \Pointilles[1]
  \item  7 \% = \dotfill \\ \Pointilles[1]
\end{itemize} 
\end{multicols}

\subsection*{Ex2 - Appliquer un pourcentage}

\textit{Écrire le calcul et le résultat.}

\begin{multicols}{2}
  \begin{itemize}[label={$\bullet$}]
    \item 65 \% de 2250 = \dotfill \\ \Pointilles[1]
    \item 24 \% de  242 = \dotfill \\ \Pointilles[1] \columnbreak 
    \item 18 \% de  108 = \dotfill \\ \Pointilles[1]
    \item  7 \% de 3300 = \dotfill \\ \Pointilles[1]
  \end{itemize} 
  \end{multicols}

\subsection*{Ex3 - Calculer un pourcentage}

\textit{Écrire le calcul et le résultat.}
  
\begin{multicols}{2}
  \begin{itemize}[label={$\bullet$}]
    \item 165 par rapport à 686 : \dotfill \\ \Pointilles[1]
    \item  25 par rapport à 425 : \dotfill \\ \Pointilles[1] \columnbreak 
    \item 146 par rapport à 180 : \dotfill \\ \Pointilles[1]
    \item  29 par rapport à 290 : \dotfill \\ \Pointilles[1]
  \end{itemize} 
\end{multicols}

\subsection*{Ex4 - Compléter un pourcentage}

\textit{Quel pourcentage reste-t-il pour le bleu ?}

\begin{multicols}{2}
\begin{itemize}[label={$\bullet$}]
  \item 50 \% est vert.
  \item 20 \% est jaune.
  \item 13 \% est rouge.
  \item Le reste est bleu. 
\end{itemize} \columnbreak 
\Pointilles[6]
\end{multicols}

\newpage

\subsection*{problème 1}

Une pastèque est composée à 80\% d'eau. Une pastèque pèse 1450g. Quelle quantité d'eau est contenue dans une pastèque ? \\

\Pointilles[4]

\subsection*{problème 2}

Pour obtenir la couleur souhaitée, Jasmine doit mélanger : 

\begin{itemize}[label={$\bullet$}]
  \item 75 \% de vert.
  \item 15 \% de jaune.
  \item Le reste de blanc.
\end{itemize}

Quelles sont les quantités nécessaires de vert, jaune et blanc pour faire 20L de peinture  ? \\

\Pointilles[8]

\subsection*{problème 3}

Le collège Faubert est composé de 256 garçons et de 286 filles. Quelles sont les pourcentages de garçon et de fille au collège Faubert ? \\

\Pointilles[4]

\subsection*{problème 4}

Un élève très attentif pendant le cours de mathématiques compte le nombre de véhicules qu'il voit passer. Il compte 150 véhicules : 

\begin{multicols}{3}
\begin{itemize}[label={$\bullet$}]
  \item 70 voitures.
  \item 34 camions.
  \item 28 vélo.
  \item 15 trottinettes.
  \item 3 moto.
\end{itemize}
\end{multicols}

Quell est le pourcentage de camions ? \\

\Pointilles[4]

\end{document}